\documentclass[10pt,a4paper]{article}
\usepackage[utf8]{inputenc}
\usepackage{amsmath}
\usepackage{amsfonts}
\usepackage{amssymb}
\author{ }
\title{Menadan Atreii}
\author{}
\begin{document}
\maketitle
\section{Général}
L'empire de Bael Turath s'est effondré il y a plus de deux siècles, pratiquement du jour au lendemain, laissant de nombreux royaumes successeurs s'affronter pour le territoire autrefois uni. La dynastie des Méréides, disparue, du jour au lendemain. La capitale et les régions alentours ont été dévastés par les combats qui suivirent, la magie employée, et pour certains, une malédiction divine. Autrefois les joyaux de la couronne impériales, il s'agit aujourd'hui des terres de cendres, sur lesquels vivent des clans d'elfes plus ou moins nomades, au milieu des ruines des monuments à la gloire des empereurs de jadis.
Atrasis est l'ancienne capitale de la province d'Ister, au bord des terres de cendres. Il s'agit d'une des plus grandes villes de l'ancien empire, et une cité-état rivalisant avec les autres royaumes successeurs pour s'emparer de ces terres. Elle accueille des marchands du monde entier, mais aussi des érudits, et les désespérés du continent.
Comme dans tout l'ancien empire, les nouvelles ambitions paraissent creuses face à l'ombre des monuments mal entretenus, chaque prince passant plus de temps à défendre les frontières qu'à rebâtir ce qui a été détruit.
Au milieu de cela, un petit groupe d'aventuriers se forme, bien déterminé à en apprendre plus sur ce qui a causé la fin de l'empire. Leur histoire les mènera loin, probablement beaucoup plus que ce qu'ils avaient anticipés....
\section{Menadan Atreii}
Menadan est un orque d'un certain âge, mais a combattu depuis plus longtemps que beaucoup espère vivre.

Il est né il y a presque cinquante ans, fils de serviteurs des Atraxos. A l'adolescence, il choisit d'apprendre le métier des armes, et servit aux côtés d'Alexandre Atraxos, à l'époque le jeune héritier de la famille, dans sa campagne de Mérétrie. Pendant des années il le suivit de champs de bataille en champs de bataille, devenant son protecteur.

Lorsque qu'Alexandre hérita des titres et richesses de sa famille, il fut de moins en moins présent sur les champs de bataille, jusqu'à ne plus y aller après la naissance de sa première fille, Anna Atraxos. La protection de la jeune fille fut confiée à Menadan, qui ne l'a plus quittée depuis, presque toujours dans son ombre. Il a vu ses premiers pas, il a assisté à son premier sortilège, et au fil des ans, vu ses interrogations sur l'ancien empire devenir de plus en plus prenantes.

Il l'a protégée de leurs ennemis politiques tout comme des intrigues de la famille, et même de sa propre curiosité par moment. Il l'a accompagnée lors de son périple dans plusieurs anciennes provinces pour en savoir plus sur la fin de l'Empire.

Il n'a donc pas été étonné d'apprendre qu'elle pensait avoir trouvé une amulette qui pourrait lui donner des réponses, mais qu'il faudrait sans doute aller dans les terres de cendres pour les obtenir. Le vieil orque alla chercher quelques unes de ses propres connaissances pour leur proposer un travail, ce que certains acceptèrent. Il a ainsi contacté un éclaireur elfe, qui avait déjà travaillé avec certains de ses frères d'armes, ainsi qu'une gladiatrice plutôt prometteuse avec qui il s'entrainait de temps en temps.

Son objectif dans cet aventure est surtout de s'assurer qu'Anna s'en sorte bien, et si possible, qu'elle trouve des réponses à ses questions. Il ne fait pas vraiment confiance à la tiefline; Kallista, qui lui a ramené l'amulette.

Il est généralement extrêmement calme dans ses propos comme dans ses actes. Les deux seules choses qui peuvent le mettre hors de lui sont les menaces contre Anna, ou les insultes à son courage. Dans ces cas là, il se montre terrifiant.
\end{document}