\documentclass[10pt,a4paper]{article}
\usepackage[utf8]{inputenc}
\usepackage{amsmath}
\usepackage{amsfonts}
\usepackage{amssymb}
\author{ }
\title{Kallista}
\author{}
\date{}
\begin{document}
\maketitle
\section{Général}
L'empire de Bael Turath s'est effondré il y a plus de deux siècles, pratiquement du jour au lendemain, laissant de nombreux royaumes successeurs s'affronter pour le territoire autrefois uni. La dynastie des Méréides, disparue, du jour au lendemain. La capitale et les régions alentours ont été dévastés par les combats qui suivirent, la magie employée, et pour certains, une malédiction divine. Autrefois les joyaux de la couronne impériales, il s'agit aujourd'hui des terres de cendres, sur lesquels vivent des clans d'elfes plus ou moins nomades, au milieu des ruines des monuments à la gloire des empereurs de jadis.
Atrasis est l'ancienne capitale de la province d'Ister, au bord des terres de cendres. Il s'agit d'une des plus grandes villes de l'ancien empire, et une cité-état rivalisant avec les autres royaumes successeurs pour s'emparer de ces terres. Elle accueille des marchands du monde entier, mais aussi des érudits, et les désespérés du continent.
Comme dans tout l'ancien empire, les nouvelles ambitions paraissent creuses face à l'ombre des monuments mal entretenus, chaque prince passant plus de temps à défendre les frontières qu'à rebâtir ce qui a été détruit.
Au milieu de cela, un petit groupe d'aventuriers se forme, bien déterminé à en apprendre plus sur ce qui a causé la fin de l'empire. Leur histoire les mènera loin, probablement beaucoup plus que ce qu'ils avaient anticipés....
\section{Kallista}
Kallista est une tiefline, montrant haut et fort que quelque part dans ses ancêtres, certains ont pactisé avec diables ou démons. Elle est née il y a un peu moins de 20 ans, dans les faubourgs d'Atrasis, où ses parents passaient d'un travail mal payé à un autre. Depuis qu'elle a douze ans, avec la mort de son père, elle vit seule, le plus souvent dans les rues ou en occupant de vieilles bâtisses endommagées. Le seul souvenir de sa mère est une vielle amulette sans prétention, probablement le seul objet d'une quelconque valeur qu'elle ait jamais eu.

Ayant besoin d'argent, elle décidé il y a quelques semaines, de se résigner à vendre l'amulette. On lui conseilla de contacter des spécialistes dans les vieilleries impériales pour vendre ça, et elle finit par être dirigée vers une jeune femme, Anna Atraxos, une noble dont le passe-temps semble de collectionner ce genre de vieux trucs.

A sa grande surprise, la vielle amulette semble avoir de la valeur, et plutôt que de la lui acheter directement, la jeune femme lui proposa de l'engager pour essayer de comprendre réellement la valeur de cette amulette. Poussée un peu de curiosité et la perspective d'un salaire pour quelque temps, Kallista accepta.

Kallista est d'un caractère très pragmatique dans ce qu'elle entreprend, et préfère éviter de prendre des risques, en particulier des risques inutiles. Elle n'a par contre pas de problèmes avec la violence : certains ne comprennent que cela.
\end{document}