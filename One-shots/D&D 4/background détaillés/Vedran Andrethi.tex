\documentclass[10pt,a4paper]{article}
\usepackage[utf8]{inputenc}
\usepackage{amsmath}
\usepackage{amsfonts}
\usepackage{amssymb}
\author{ }
\title{Vedran Andrethi}
\begin{document}
\section{Général}
L'empire de Bael Turath s'est effondré il y a plus de deux siècles, pratiquement du jour au lendemain, laissant de nombreux royaumes successeurs s'affronter pour le territoire autrefois uni. La dynastie des Méréides, disparue, du jour au lendemain. La capitale et les régions alentours ont été dévastés par les combats qui suivirent, la magie employée, et pour certains, une malédiction divine. Autrefois les joyaux de la couronne impériales, il s'agit aujourd'hui des terres de cendres, sur lesquels vivent des clans d'elfes plus ou moins nomades, au milieu des ruines des monuments à la gloire des empereurs de jadis.
Atrasis est l'ancienne capitale de la province d'Ister, au bord des terres de cendres. Il s'agit d'une des plus grandes villes de l'ancien empire, et une cité-état rivalisant avec les autres royaumes successeurs pour s'emparer de ces terres. Elle accueille des marchands du monde entier, mais aussi des érudits, et les désespérés du continent.
Comme dans tout l'ancien empire, les nouvelles ambitions paraissent creuses face à l'ombre des monuments mal entretenus, chaque prince passant plus de temps à défendre les frontières qu'à rebâtir ce qui a été détruit.
Au milieu de cela, un petit groupe d'aventuriers se forme, bien déterminé à en apprendre plus sur ce qui a causé la fin de l'empire. Leur histoire les mènera loin, probablement beaucoup plus que ce qu'ils avaient anticipés....

\section{Vedran Andrethi}
Vedran est un elfe des terres de cendre, anciennement du clan Andrethi, qui parcourt les terres de cendre orientales. 

Il a été banni à 16 ans du clan, après avoir tué son frère dans une violente dispute. Il a alors parcouru les terres de cendres pour quelques décennies, en évitant scrupuleusement le parcours des Andrethi. Il a côtoyé beaucoup d'autres clans pour un temps, avant de reprendre la route. 

Après quelques décennies d'errance, il se dit qu'il pourrait essayer quelque chose d'autre, et commença à vendre ses services en tant que guide et garde aux caravanes qui parcourent parfois les terres de cendres. Cela lui parait toujours un peu étrange de travailler avec les N'wah, les étrangers, mais il s'est habitué à leurs coutumes.

Il a toujours respecté son bannissement, et compte bien continuer à le faire, même si beaucoup de choses ont changé : les terres de cendres dans leur ensemble sont chez lui maintenant.

Il a récemment été contacté par un orque au service de la maison Atraxos. Il avait déjà travaillé pour cette maison pour escorter et guider leurs caravanes, mais cette fois-ci, l'offre parlait d'une expédition, probablement vers les ruines impériales qui parsèment le paysage. Avec une certaine curiosité, il a décidé d'accepter l'offre, et donc doit se rendre au manoir des Atraxos pour discuter avec son employeuse de la mission.

C'est une elfe à l'humour souvent un peu caustique, qui apprécie beaucoup de passer ses jours sur la route et ses nuits prêts d'un feu de camp, dans les terres de cendres qu'il aime tant.


\end{document}