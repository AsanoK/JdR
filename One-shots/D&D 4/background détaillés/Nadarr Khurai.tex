\documentclass[10pt,a4paper]{article}
\usepackage[utf8]{inputenc}
\usepackage{amsmath}
\usepackage{amsfonts}
\usepackage{amssymb}
\title{Nadar Khurai}
\author{}
\begin{document}
\maketitle
\section{Présentation générale}
L'empire de Bael Turath s'est effondré il y a plus de deux siècles, pratiquement du jour au lendemain, laissant de nombreux royaumes successeurs s'affronter pour le territoire autrefois uni. La dynastie des Méréides, disparue, du jour au lendemain. La capitale et les régions alentours ont été dévastés par les combats qui suivirent, la magie employée, et pour certains, une malédiction divine. Autrefois les joyaux de la couronne impériales, il s'agit aujourd'hui des terres de cendres, sur lesquels vivent des clans d'elfes plus ou moins nomades, au milieu des ruines des monuments à la gloire des empereurs de jadis.

Atrasis est l'ancienne capitale de la province d'Ister, au bord des terres de cendres. Il s'agit d'une des plus grandes villes de l'ancien empire, et une cité-état rivalisant avec les autres royaumes successeurs pour s'emparer de ces terres. Elle accueille des marchands du monde entier, mais aussi des érudits, et les désespérés du continent.

Comme dans tout l'ancien empire, les nouvelles ambitions paraissent creuses face à l'ombre des monuments mal entretenus, chaque prince passant plus de temps à défendre les frontières qu'à rebâtir ce qui a été détruit.

Au milieu de cela, un petit groupe d'aventuriers se forme, bien déterminé à en apprendre plus sur ce qui a causé la fin de l'empire. Leur histoire les mènera loin, probablement beaucoup plus que ce qu'ils avaient anticipés....
\section{Nadarr}
Nadarr est un drakéide, servant Celle Qui Voit le Temps, Valara, la déesse des runes, de la civilisation et des présages. Son culte est apparu après la chute de l'Empire, appelant les souverains à ramener la paix sur les terres impériales. Deux branches ont émergé depuis, pour le moment encore unies, mais plus forcément pour très longtemps. L'une appelle à la réunification des terres impériales sous une même bannière, l'autre parle d'imposer la paix aux anciennes provinces.



Nadarr a éclot il y a une petite trentaine d'années dans les provinces orientales de l'ancien empire, et sa couvée a été élevée par le temple de Valara. Il a apprit les légendes sur l'empire et ce qui a pu mener à sa perte, mais a toujours été intrigué : si l'empire s'est effondré aussi rapidement, quels en étaient les signes avant-coureurs, et cela aurait-il pu être éviter?

Une fois son noviciat terminé, il prit donc les voeux de Sanctati, un gardien du temple. Et son serment fut de chercher à comprendre comment une civilisation si puissante pouvait tomber en si peu de temps. Il parcourt depuis les terres de l'ancien empire en quête de ce savoir, défendant sur sa route les fidèles et les temples de Valara.

Il a croisé la route d'une jeune femme, Anna Atraxos, une passionnée de la même période. Elle a accès à de nombreuses ressources, et Atrasis, où elle réside, permet de se rendre facilement dans les terres de cendres, où se cachent probablement les réponses.

Il n'a pas été tout à fait étonné de recevoir une lettre, ornée du sceau des Atraxos, mentionnant que la jeune femme recherchait des gens, elle avait trouvé un artefact qui pourrait bien cacher quelques éléments de réponse. Elle n'en disait pas plus dans sa lettre, mais la possibilité valait bien le coup d'aller en discuter. Le rendez-vous est à la maison de ville des Atraxos.

Il est passionné par sa quête, et peut parler longuement de la fin de l'empire, ainsi que des guerres entre les royaumes successeur. A part cela, il apprécie de parler de Valara et de son message de paix sur les anciennes provinces.
\end{document}