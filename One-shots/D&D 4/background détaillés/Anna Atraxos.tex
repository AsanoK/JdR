\documentclass[10pt,a4paper]{article}
\usepackage[utf8]{inputenc}
\usepackage{amsmath}
\usepackage{amsfonts}
\usepackage{amssymb}
\author{ }
\title{Anna Atraxos}
\begin{document}
\section{Général}
L'empire de Bael Turath s'est effondré il y a plus de deux siècles, pratiquement du jour au lendemain, laissant de nombreux royaumes successeurs s'affronter pour le territoire autrefois uni. La dynastie des Méréides, disparue, du jour au lendemain. La capitale et les régions alentours ont été dévastés par les combats qui suivirent, la magie employée, et pour certains, une malédiction divine. Autrefois les joyaux de la couronne impériales, il s'agit aujourd'hui des terres de cendres, sur lesquels vivent des clans d'elfes plus ou moins nomades, au milieu des ruines des monuments à la gloire des empereurs de jadis.
Atrasis est l'ancienne capitale de la province d'Ister, au bord des terres de cendres. Il s'agit d'une des plus grandes villes de l'ancien empire, et une cité-état rivalisant avec les autres royaumes successeurs pour s'emparer de ces terres. Elle accueille des marchands du monde entier, mais aussi des érudits, et les désespérés du continent.
Comme dans tout l'ancien empire, les nouvelles ambitions paraissent creuses face à l'ombre des monuments mal entretenus, chaque prince passant plus de temps à défendre les frontières qu'à rebâtir ce qui a été détruit.
Au milieu de cela, un petit groupe d'aventuriers se forme, bien déterminé à en apprendre plus sur ce qui a causé la fin de l'empire. Leur histoire les mènera loin, probablement beaucoup plus que ce qu'ils avaient anticipés....
\section{Anna Atraxos}
Anna est la fille aînée d'Alexandre Atraxos, un noble d'une ancienne famille impériale. Elle pratique la magie depuis qu'elle a dix ans, et se passionne pour l'histoire impériale, en particulier sa chute.

Elle est née dans le luxe et l'opulence d'une des maisons les plus nobles et riches d'Atrasis, et son père en était un général quand il était plus jeune. Elle est donc habituée à la présence de serviteurs, comme le vieux Menadan, un orque assigné à sa protection depuis sa naissance, et un ancien compagnon de son père. 

Elle a apprit la magie avec certains des meilleurs précepteurs de la ville, mais préférait, et de loin, les leçons sur l'histoire. Lorsque ses précepteurs finirent par la lasser,elle décida de parcourir certaines des anciennes provinces, espérant en apprendre plus sur Bael Turath. Au cours de ce voyage, elle fit la connaissance d'un drakéide, Nadarr Khurai, lui aussi intéressé par la chute de l'Empire.

A son retour, elle s'occupa un temps de certaines affaires de famille, avant de revenir à sa passion. En particulier, une jeune tiefline l'a contacté pour essayer de vendre une vieille amulette. Cette amulette est probablement un vieux sceau mémoriel, un artefact assez rare de la période impérial, stockant les archives et mémoires des nobles de l'époque.La plupart de ces objets ont disparu rapidement lors des premiers guerres entre royaumes successeurs. Et les dates gravées par l'artisan correspondent avec la fin de l'empire. Il faudrait réussir à comprendre comment utiliser ce sceau, mais une fois cela accompli, cela permettrait potentiellement d'en apprendre beaucoup plus sur cette période !

Anna a donc commencé à rassembler une petite équipe, d'abord pour en savoir le plus possible sur où et comment activer cet objet, puis pour se lancer dans une probable expédition dans les terres de cendres. Entre autre, la jeune tiefline et Nadarr acceptèrent, et Menadan contacta d'autres personnes pour servir d'escorte et de guide.

C'est une jeune femme organisée, qui peut toutefois se laisser emporter par sa passion pour l'histoire impériale. Dans ce cas, elle devient impossible à arrêter. 
\end{document}