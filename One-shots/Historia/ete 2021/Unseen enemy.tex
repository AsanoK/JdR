\documentclass[10pt,a4paper]{article}
\usepackage[utf8]{inputenc}
\usepackage[french]{babel}
\usepackage[T1]{fontenc}
\usepackage{amsmath}
\usepackage{amsfonts}
\usepackage{amssymb}
\author{ Antoine Robin}
\title{Unseen enemy - Historia}
\begin{document}
\maketitle
\section{Scénario}
\subsection{Acte 1 - Au nom de l'Église}
Les personnages sont accueillis par un des assistants de Meisel (Florian), un pinson portant une simple robe de bure, fermée par une broche en forme d'os entrecroisés. Celui-ci mène les personnages vers Meisel, qui va accueillir les PJs. 

Meisel congédie son assistant (on a besoin de lui dans l'aile ouest), accueille Raymond, et se présente aux autres personnages, et notamment son rôle à San Mastino : mettre fin à l'épidémie qui y sévit depuis quelques mois, par ordre du souverain pontife.

Il décrit ensuite la situation en ville, et la terrible épidémie qui sévit ici depuis quelques mois, avec des symptômes horribles. 

Il termine par ceci : 'Même si je pense que la prière est un excellent soin, et que l'aide des ancêtres est toujours discrète, je serai ravi de trouver un remède à ce mal, et le plus vite possible. La moitié des personnes affligées meurent, l'autre moitié s'en sort presque sans aide. Tout ce que nous pouvons faire c'est leur donner quelque chose pour la fièvre, les forcer à boire constamment. De l'eau et du miel vous voyez, comme ils ne peuvent plus manger. Nous avons besoin d'un remède, j'ai des gens qui y travaillent, mais nous ne comprenons pas : nous ne comprenons pas encore comment cette maladie se transmet ! J'aurais besoin que vous trouviez et convainquiez Gertrude de travailler avec nous.'

Il répond ensuite aux questions des PJs, notamment sur qui est Gertrude. Si on pose des questions sur Wegener, il ne cache pas vraiment son inimitié avec l'Occipital.
\subsection{Acte 2 - Fait ou fiction ?}
Les personnages arrivent rapidement à la cabane de la fameuse herboriste. Elle est un peu étrange, et est protégée par un pangolin de grande taille, qui cherche à les dissuader d'interrompre celle-ci. DC 20 (22 si ils mentionnent l'église ou sont visiblement armés, 17 si ils mentionnent la maladie) pour le convaincre de les laisser passer. Tenter de l'intimider va déclencher immédiatement un affrontement.

Si les choses dégénèrent, ne pas rendre le combat mortel ! Gertrude entend rapidement le bordel devant chez elle, et sortira pour avoir des explications.

Si l'approche s'est bien passée, les personnages seront invités à entrer chez elle, sinon, la discussion se fera sur le palier.

Si on lui demande si elle a répandu la maladie, elle répond que non.

Si on lui demande pourquoi Nicola Wegener pense qu'elle la répand, elle répond que piur l'église, elle est un bouc-émissaire facile : une sorte de sorcière, soignant les gens pour presque rien. Une menace pour l'autorité de l'Occipital !

Si on lui demande si elle connait un remède, elle répond qu'elle n'est pas loin, et invite les personnages à voir ses notes.

Si on parle de Victor Meisel, elle se referme, et indique ne rien vouloir faire avec l'Église, elle connait leur fonctionnement, et reconnait qu'elle pourrait même soupçonner l'église d'être derrière toute cette affaire : les hôpitaux pleins, leurs poches doivent être pleines comme jamais!

Si on continue, elle mentionne que l'épidémie a commencé près des docks, dans les quartiers les plus pauvres de la ville. Cela pourrait être un bon point de départ sur les origines de ce mal.

Les personnages peuvent ensuite aller se diriger vers les docks, où se trouve un nouvel hôpital géré par l'occipital. En plus de cet hôpital, les perosnnages peuvent avoir trois discussions avec les locaux : un enfant (une chouette extrêmement sale), qui est envoyé par sa mère, qui a entendu les questions des PJs. Il est clairement malade, ses yeux dorés tournant légèrement au gris. Il a trouvé un tissu bordeau dans la boue près de la première maison touchée, avec de fines bandes bleues. Un tel tissu vient sans aucun doute du quartier des tisserands, près du palais de l'occipital (DC16 Int). Le groupe peut garder le tissu si ils ramènent le remède dès que possible à l'enfant. En regardant chez les tisserands, les PJs apprennent qu'il s'agit d'une commande spécifique d'un homme de main de l'occipital, un alchimiste du nom de Joris de Witt. Le second témoin est une vielle chèvre croulante, un vieux devin se tenant sur un bâton de marche noueux. Il parle en énigme et en nonsenses, mais mentionne de sombres silhouettes s'introduisant chez les gens ces derniers temps. Il prévient les PJs que les silhouettes sont toujours là et vont les chercher. Enfin, une aubergiste (une poule) peut rejoindre une des deux autres conversations, et mentionne qu'elle a hébergé un groupe d'individus avec de grandes capes sombres, qui souhaitaient rester discrets, mais elle craint qu'ils n'aient un lien avec la maladie. Elle peut donner aux PJs une fiole avec encore quelques gouttes d'un liquide grisâtre qu'elle a trouvé dans leur chambre. Elle demande de l'argent contre ses informations.
\subsection{Acte 3 - la source}
Les personnages sont attaqués dans une ruelle par un groupe d'hommes de mains encapés. 

De retour chez Meisel, le bulldog passe par beaucoup d'émotions : il est déçu que Gertrude ne souhaite pas travailler avec l'église, et furieux, car il est persuadé de l'implication de Wegener dans cette triste affaire.

Toutefois, meisel était justement en train de recevoir la visite d'un ami à lui, Venceslao da Mira, chef inquisiteur de San Mastino, et loyaliste convaincu de Femore III. Il déteste par ailleurs également Wegener. Mais il ne pourra agir avec toute la force de l'inquisition que si des preuves irréfutables de l'implication de l'occipital lui sont amenées. Il aurait besoin de preuves tangible, et pense que si de telles preuves existent, elles seront dans le palais occipital. heureusement, il connait une route sûre, ainsi qu'un moment adapté pur entrer dedans en évitant les problèmes. Il pense que si il y a une affaire d'empoisonnement comme celle-ci, son assistant personnel, De Witt, un alchimiste de la confédération est probablement derrière tout cela. Il peut également assurer que la récompense sera excellente : les fonds de l'église sont immenses, et le pontife lui-même sera ravi que cela soit réglé.

L'infiltration est aussi facile que l'inquisiteur le mentionnait. Il ne faut qu'un test d'athlétisme DC14 pour passer le seul obstacle, une fenêtre.

Quand les personnages arrivent dans l'étude, elle est vide, à l'exception d'une discrète alcôve, où l'ibis étudie. Il est possible de discuter avec lui, mais il ne trahira pas son maître. test de persuasion DC24 est la seule façon de réussir à lui tirer les vers du nez. Si un affrontement dure plus de deux rounds avec lui, deux gardes arrivent, et les Pjs vont commencer à manquer de temps pour rassembler les preuves.

En fouillant de Witt, ils trouveront une clé de bronze, qui ouvre un coffre dissimulé contenant de nombreux papiers intéressant. La maladie est en effet artificielle, créée par de Witt sur les ordres de Wegener pour affaiblir le Saint-siège, qu'il juge incapable tant d'aider les pauvres de la ville, que de mener le continent à la guerre. L'épidémie lui permet de montrer à tous les problèmes de l'église actuelle !

La recette de l'antidote est aussi dans ces papiers, à la fois complexe et coûteuse à produire.

Ce que les Pjs font avec le remède peut changer pas mal de chose sur la conclusion.
\section{PNJs utiles}
\subsection{Victor Meisel}
Tarsus de l'Eglise, et intendant de l'hôpital de la ville, il s'agit d'un bulldog d'âge moyen, un érudit relativement calme. Sauf Envers Nicolas Wegener, qu'il ne peut pas supporter : il considère qu'il s'agit d'un politicien plus qu'un religieux, déteste son attitude, et de toute manière, répond directement à Sa Sainteté.
\subsection{Gertrude}
Une herboriste tatou relativement jeune. Elle est un peu étrange, regardant fixement la personne à qui elle parle et étant facilement surprise, et changeant complètement d'attitude près de son travail : elle devient très confiante et assurée.

En discussion, elle ne reste pas concentrée sur un sujet pendant long.

\subsection{Joris de Witt}
L'ibis est probablement effrayé par les Pjs lorsqu'ils le croisent. Il est habillé dans de trs riches vêtements, amis sans aucun bijou. Le seul accessoire à sa tenue est une paire de lunette posée sur son bec. 
\section{Lieux de l'intrigue}
\subsection{La ville de San Mastino}
Une grande ville au nord-ouest du Saint-Siège, sa richesse n'étant rivalisée que par la capitale elle-même. On l'appelle parfois 'la perle du nord'. Ses hauts bâtiments et ses rues étroites limitent le passage tant de gens que des rayons du soleil. C'est aussi une ville importante, sur le plan culturel, politique, et religieux.
\subsection{L'hôpital du temple}
Un lieu de soin aux malades, administré par l'église. A côté de celui-ci, une sorte de petite maison tranquille sert de demeure à Victor Meisel.
\subsection{Le cottage de gertrude}
Une cabane, vraiment, et ayant grand besoin de réparations urgentes. Une ruine, presque. 

Des plantes apparemment sauvages envahissent le terrain alentour, mais en s'approchant, on peut se rendre compte qu'il s'agit d'un jardin médicinal, même si possiblement le moins organisé de tout le Saint-Siège.

L'intérieur est également en désordre, avec des notes, des plantes, des pots, et des ouvrages divers un peu partout, en piles sans ordre apparent.
\subsection{Les docks de San Mastino}
Comme partout, le lieu le plus pauvre, criminel et sale de la ville. Les routes sont couvertes de boue et de déchets. Peut de gens sortent dans les rues, signes des ravages qu'a produit l'épidémie par ici.
\subsection{L'infirmerie de Wegener}
Une ancienne place de marché sur les docks, reconvertie en sorte d'hôpital de campagne, où de nombreux membres de l'église travaillent. Les commentaires (du personnel) sont à la limite d'être séditieux envers Weiseil et le Pontife Femore III, qui échouent à aider les pauvres, en restant loin du coeur de l'épidémie.
\subsection{Le palais occipital}
Si les couloirs de services sont propres et fonctionnels, les zones d'usage sont quand à elle baignée dans le luxe, la richesse, et l'esthétique de la confédération.  Le sol et en marbre, les peintures sont somptueuses, et l'architecture est clairement étrangère.

L'étude de l'occipital est une magnifique pièce, avec de beau panneaux de bois nobles, un épais tapis pourpre, et une odeur de parchemins et de cire à sceller.
\section{Documents de jeu}
\subsection{La lettre}
'Au nom de sa sainteté Femore III et des ancêtres bénis, vous être convoqués par l'Église à l'hôpital du temple avant la le prochain coucher du soleil. L'affaire est urgente. - Victor Meisel, humble servant de l'Église, Tarsus et intendant des hôpitaux sacré.'
Les personnages reçoivent un message en arrivant en ville, confirmant l'urgence de la situation.
\end{document}