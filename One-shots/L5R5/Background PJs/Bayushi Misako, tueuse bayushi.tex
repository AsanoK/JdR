\documentclass[10pt,a4paper]{article}
\usepackage[utf8]{inputenc}
\usepackage{amsmath}
\usepackage{amsfonts}
\usepackage{amssymb}
\title{Bayushi Misako, tueuse du scorpion}
\author{}
\date{}
\begin{document}
\maketitle
Misako est une bushi (combattante) du scorpion, déjà connue pour sa dangerosité, et appréciée pour cela de ses maîtres.

Elle a été nommée d'après une ancêtre, qui s'est fait connaître pour avoir battu à plate couture un maître de shôgi du clan du lion. De fait, la jeune combattante est une joueuse assidue de shôgi, en mémoire de son ancêtre. Elle y trouve un certain réconfort, et apprécie honnêtement de pouvoir battre quelqu'un sans risquer de le tuer, volontairement ou non.

Elle a en effet été formée à l'école des tueurs Bayushi, la principale école de combattants et d'assassins du clan du scorpion, dont les bons élèves doivent être capables de neutraliser un adversaire de toutes les manières possibles, y compris les moins honorables : disgrâce, meurtre, duel, poison, accident de chasse.... La spécialité de Misako réside plus dans le duel, mais elle connait les bases des autres méthodes.

Après plusieurs cas notoires, la jeune femme a acquis une réputation de tueuse : son premier vrai duel a terminé avec une jugulaire tranchée, le second par un bras tranché, et le dernier était de toute manière à mort, mais de l'avis de tous, une mort particulièrement salissante. Après ces exploits, et afin de limiter la pression que les autres clans commençaient à lui infliger, ses maîtres décidèrent de l'envoyer ailleurs : elle escorte depuis Bayushi Miya, une inquisitrice du scorpion, chargée de débusquer les traîtres à l'empire. Elle a prit cette décision avec une certaine satisfaction : après le second duel, elle a commencé à avoir du mal à tuer. Au lieu de concentration, avant un combat vient de l'appréhension, et des souvenirs des fois précédentes. Cela ne l'empêchera pas de faire son devoir, mais elle n'y prend pas goût.

Elle a profité de suivre l'inquisitrice pour continuer à travailler son point fort, son sens de l'observation. en tant que duelliste, cela lui permet d'étudier précisément ses adversaires. En tant que protectrice, et assistante par défaut de Bayushi Miya, cela lui a permit d'anticiper plusieurs problèmes, et d'aider l'inquisitrice dans ses devoirs. 




\end{document}