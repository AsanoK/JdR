\documentclass[10pt,a4paper]{article}
\usepackage[utf8]{inputenc}
\usepackage{amsmath}
\usepackage{amsfonts}
\usepackage{amssymb}
\title{Shosuro Kimiko}
\author{}
\date{}
\begin{document}
\maketitle
Shosuro Kimiko est une jeune femme extrêmement séduisante, ce qui est à la fois sa bénédiction, et sa malédiction.

Elle a été formée à l'école des infiltrateurs Shosuro, l'une des plus prestigieuses écoles de courtiers, mais aussi l'une des plus efficace école de shinobi.Elle n'est pas une grande spécialiste de l'assassinat ou du vol, mais sait se défendre et surtout, obtenir des informations sur les adversaires du clan.

Elle a été rapidement notée par ses maîtres pour sa grande beauté, une arme redoutable entre les mains de qui sait la manier. Elle a donc été formée en premier lieu au jeu de la cour, où cette bénédiction lui permet de servir au mieux son clan.

Shiro no Takamatsu est son second poste, après un passage par la cour impériale d'hiver. Elle s'y ennuie moins, car ses responsabilités sont ici beaucoup plus importantes, servant d'assistante au représentant du scorpion, Bayushi Taneda. En particulier, elle est chargée de collecter et de lui résumer les informations disponibles sur la cour et ceux qui y sont.

En plus d'être jolie, Kimiko est connue pour un humour, qui est pour elle un bon moyen de relâcher la pression de sa vie quotidienne. 

Les deux problèmes de Kimiko vont ensemble : étant tout à fait habituée à recevoir les avances et les révélations de sentiments des autres, elle a beaucoup de mal à révéler les siens. Or, depuis peu à shiro no Takamatsu est arrivé un jeune membre des familles impériales, Seppun Eiji. Il est, pour elle, en tout point parfait, toutefois, elle n'a pas encore osé l'approcher trop de peur de se ridiculiser. Son charme naturel ne semblant pas suffisant pour atteindre le jeune homme (mais plus que suffisant par ailleurs si l'on en juge par la pile croissante de lettres d'amour d'autres courtisans...), elle maudit de plus en plus cette faiblesse qu'elle a.

Une journée type pour elle commence relativement tôt : après s'être préparée rapidement, il faut qu'elle fournisse les informations dont Bayushi Taneda aura besoin dans la journée. Elle se prépare ensuite complètement pur apparaître à la cour, où elle passe la majeure partie de la journée, que ce soit pour discuter, intriguer, ou correspondre par lettres avec les courtisans et nobles présents. En soirée, il peut arriver qu'elle s'éclipse discrètement pour aller fouiner un peu, mais cela reste rare, le gain étant rarement très intéressant. Si elle peut, elle essaie fréquemment de passer du temps avec Seppun Eiji, mais n'a pas encore osé lui révéler ses sentiments, de peur que cela soit mal formulé ou interprété.
\end{document}