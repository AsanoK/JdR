\documentclass[10pt,a4paper]{article}
\usepackage[utf8]{inputenc}
\usepackage{amsmath}
\usepackage{amsfonts}
\usepackage{amssymb}
\title{Tsuruchi Ryôtarô, archer de la guêpe}
\author{}
\date{}
\begin{document}
\maketitle

Ryôtarô est un des rares samurai des clans mineurs à Shiro no Takamatsu. Il vient du clan de la guêpe, connu pour ses archers et traqueurs doués.

Son arrière grand-père, d'après lequel il est nommé, aurait appris certaines des techniques des courtiers de la grue, même s'il n'a jamais confirmé cette rumeur. En tout cas, il est vrai que cet ancêtre a transmis un certain nombre de chose liées aux cours de Rokugan. Cela explique peut-être le poste actuel de Ryôtarô, qui a été envoyé à la cour de Shiro no Takamatsu pour représenter le clan de la guêpe et les clans mineurs, et au besoin, escorter et servir le représentant principal.

Comme tout les Tsuruchi ou presque, il est un tireur doué, maîtrisant bien mieux le tir à l'arc que le sabre, dont il n'a que de vagues rudiments. Sa petite taille est un relatif avantage pour passer inaperçu, même si cela est d'une utilité relative lors de la vie de la cour. En effet, il a hérité de sa famille une petite stature, souvent dépassé d'une bonne tête par beaucoup de samurai, et de plus de deux par certains crabes!

Au delà de cela il est connu pour son franc-parler, dans une société qui valorise la subtilité. Cela finira par lui coûter cher, mais pour le moment, aucun accident grave ne s'est encore produit à cause de cela.

A part avec un arc, il est très à l'aise un pinceau à la main, privilégiant une calligraphie précise et concise, ou la peinture de paysages naturels, le plus souvent les montagnes non loin de la forteresse de son clan.

Le point du bushido qui l'a toujours étonné de voir suivi, et qu'il craint quelque part de ne pas pouvoir suivre est de service 'jusqu'à la mort' : il est terrifié de celle-ci, et craint de faire honte à sa famille et son clan si jamais il se retrouve devant une situation qui exigerait son sacrifice, car il ne pense pas pouvoir en être capable.

A la cour, il est le plus fréquemment à proximité de Tsuruchi Emon, le représentant des clans mineurs, et son oncle. Il le sert en étant présent à différentes occasions, généralement celles où son franc-parler ne feront pas trop de dégâts, et parfois sur des travaux d'enquêtes ou des parties de chasse.

sa journée typique commence tôt, par le tir de nombreuses flèches sur des cibles. Après un bain, il mange un bon petit-déjeuner, qui lui tiendra la journée, et part explorer les environs du château. Il revient dans l'après-midi, où il assiste aux évènements de la cour. S'il a le temps, il apprécie de finir la journée en faisant de la peinture ou répondre aux lettre en calligraphiant de belles réponses.
\end{document}