\documentclass[10pt,a4paper]{article}
\usepackage[utf8]{inputenc}
\usepackage{amsmath}
\usepackage{amsfonts}
\usepackage{amssymb}
\title{Asahina Sanae}
\author{}
\date{}
\begin{document}
\maketitle
Sanae est une shugenja relativement expérimentée du clan de la grue, elle communique avec les kami via ses invocations et leurs signes. Comme une majeure partie des Asahina, c'est une pacifiste convaincue, même si elle sait que des sacrifices seront nécessaire pour atteindre une paix durable.

Elle a été nommée d'après une de ces ancêtres, une diplomate de grand talent ayant permit il y a plusieurs siècles, d'éviter un conflit de succession au  sein de la grue, en convainquant les différentes factions qu'un autre candidat était le meilleur choix.

Elle a toujours ressenti la présence des esprits, même sans pouvoir communiquer avec eux. Avec sa formation de shugenja Asahina, elle peut désormais le faire, et ses devoirs impliquent purifications et négociations avec eux.

Toutefois, sa réputation ne vient pas de ses compétences, mais bien de son passé : lors de son gempuku, elle a assez gravement offensé un magistrat du scorpion, ce qui lui a valu des remontrances publiques. Aujourd'hui encore, les membres de sa famille ne lui font pas tout à fait confiance pour ne pas commettre d'autres impairs dans le futur.

Rejetant ce qui s'est passé à ce moment sur le scorpion, elle déteste ce clan : elle n'apprécie pas, et cela se voit fréquemment, d'être en compagnie d'un membre de ce clan. Par ailleurs, elle tend à considérer comme une insulte le fait que ces traîtres notoires et agresseurs fréquents soient considérés comme un clan majeur. Elle essaiera d'éviter un nouvel impair, mais si cela arrive, cela tombera probablement à nouveau sur eux.

Elle a récemment été demandée à Shiro no Takamatsu afin d'enquêter sur des signes étranges aperçus au château : des lampes se sont éteintes soudainement avant de se rallumer, de la nourriture a tourné dans le cellier.... Rien de séparément très inquiétant, mais le daimyo a préféré, devant la répétition, demander de l'aide aux shugenja Asahina.

En terme de passe-temps, elle apprécie énormément de lire, écouter, et raconter les nombreux contes de Rokugan. Ses préférés parlent généralement de noble sacrifice pour la paix, un thème récurrent dans sa famille, ou ont une morale drôle sur la vie de la cour.

Sa journée type à Shiro no Takamatsu commence à l'aube, par de la méditation dans le petit temple du château, avant de répondre aux éventuelles lettres et de discuter avec ses connaissances sur place. L'après-midi est généralement consacrée à l'étude ou à invoquer les esprits pour en savoir plus sur ces signes étranges. En soirée enfin, elle assiste fréquemment au différent évènements, même si elle préfère éviter les membres du scorpion.
\end{document}