\documentclass[10pt,a4paper]{article}
\usepackage[utf8]{inputenc}
\usepackage{amsmath}
\usepackage{amsfonts}
\usepackage{amssymb}
\title{Daidoji Hisao}
\author{}
\date{}
\begin{document}
\maketitle
Daidoji Hisao est un jeune courtisan du clan de la grue, et un élément prometteur de la famille Daidoji, si toutefois le destin lui laisse le temps de réaliser ces promesses.\documentclass[10pt,a4paper]{article}
\usepackage[utf8]{inputenc}
\usepackage{amsmath}
\usepackage{amsfonts}
\usepackage{amssymb}
\begin{document}
•
\end{document}

Il provient d'une branche relativement importante de la famille Daidoji, pouvant en réalité tracer son arbre généalogique jusqu'à un autre Daidoji Hisao, d'après qui il est nommé, se trouvant, d'après la rumeur, persistante depuis sa naissance, être un fils illégitime d'un empereur. Cette parenté a généralement dédié cette branche de la famille au jeu de la cour plus qu'à la défense militaire des intérêts du clan, même si certains ont pu y exceller.

Hisao a suivi en ce sens la tradition familiale, et a été formé au sein de l'école, peu connue à l'extérieur du clan, des maîtres espions Daidoji. Cette école forme des spécialistes de la récupération et de l'exploitation des informations sensibles : si les courtisans Doji forment des alliances, les daidoji cherchent à obtenir les secrets des adversaires de la grue. Étant jugé très prometteur par ses sensei, il a été nommé, malgré sa jeunesse en tant qu'assistant au hattamoto de Shiro no Takamatsu, Daidoji Teru. Celui-ci, beaucoup plus expérimenté, est donc devenu son kôhai, chargé de lui montrer les méthodes et techniques de l'école, et d'en faire un parfait outil pour son clan.

Ce poste lui permet de satisfaire sa curiosité presque maladive pour les secrets des gens : il a un sens pour trouver les secrets, et savoir lesquels ont une certaine valeur. Il a toujours été curieux, en particulier pour ce qui est dissimulé, peut-être un lien avec son homonyme ? Il n'est pas sévère, en particulier pour un samurai, il n'apprécie pas l'idée de tuer, même pour les bonnes raisons. Cela peut évidemment lui poser problème dans son rôle, la politique Rokugani pouvant se révéler littéralement mortelle.

L'ombre principale au tableau toutefois n'est pas dans son tempérament, mais bien dans sa santé : depuis tout jeune, sa santé est mauvaise, et ne s'améliore guère. La plupart du temps, cela l'empêche de faire trop d'efforts, mais certains jours, il peut être incapacité par de violentes quintes de toux, parfois sanglante. 

Aujourd'hui, sa journée typique est comme suit : il se lève relativement tard, prend un frugal petit-déjeuner, avant d'aller retrouver Daidoji Teru pour étudier les actions à entreprendre dans la journée. La majeure partie de celle-ci est ensuite consacrée à la récupération d'informations : beaucoup d'évènements de cour, tel que les séances officielles, mais aussi les différents festivals, des parties de jeu, et de nombreuses discussions derrière des portes closes. La majeure partie de cela ne lui apporte rien, ou peu de chose, mais de temps en temps, il tombe sur une pépite. Cela peut durer jusque tard dans la soirée, après quoi, il lui faut encore noter les informations obtenues, sous une forme codée, et essayer de recouper tout cela avec ce qui est déjà connu. Il se couche généralement parmi les derniers au château, souvent après l'heure du rat (23h à 1h du matin).


\end{document}