\documentclass[10pt,a4paper]{article}
\usepackage[utf8]{inputenc}
\usepackage[french]{babel}
\usepackage[T1]{fontenc}
\usepackage{amsmath}
\usepackage{amsfonts}
\usepackage{amssymb}
\author{ Antoine Robin}
\title{One-shot Mutant Year Zero}
\begin{document}
\maketitle
\section{L'arche}
\subsection{Connu des PJs}
Il y a une trentaine d'année, le monde a sombré. Le monde d'avant était plein de merveilles technologiques, mais celles-ci n'ont pas sauvé ses habitants : le chaos, et des conflits de plus en plus fréquents ont fini par faire sombrer leur civilisation, et ceux qui ont survécu au carnage subsistent péniblement dans les ruines.

Les personnages sont nés dans l'Arche, un refuge souterrain, abritant de nombreux mutants, comme eux. Il y avait plusieurs humains quand ils étaient petits, mais il ne reste que l'Ancien, un humain âgé, dont la parole gardait une grande influence sur l'Arche. 

Par prudence, les habitants de l'arche n'ont volontairement pas ouvert la porte principale de cet abri, jusqu'il y a quelques mois. cela n'avait pas empêcher des mutants plus entreprenant de se faufiler à l'extérieur via un réseau de tunnels mal cartographiés cela dit.

La culture de l'Arche était brutale, avec des conflits fréquents et pas toujours pacifiques entre différents groupes, aux frontières changeantes. Les mutants ont rapidement acquis des compétences utiles : se défendre, explorer les environs de leur abri, bricoler du matériel de fortune.....

Une fois la porte principale ouverte, les mutants purent plus facilement découvrir le monde extérieur, et ses habitants : des créatures mutées, des goules, mais aussi d'autres survivants, souvent les plus dangereux : entre les gangs mécaniques, les cultistes de l'apocalypse, et différents groupes de survivants, la Zone n'est toujours pas calmée.

L'Arche elle-même est tombée : un groupe de cultiste, menée par une psychopathe notoire du nom d'Amazielle a lancé l'assaut, tuant tout ceux qui restaient sur leur passage. Les personnages ont réussi à fuir par les tunnels, et se sont regroupées dans une sorte de grotte de béton, aux nombreux piliers. Entre les carcasses mécaniques de toutes sortes, ils vont pouvoir planifier leurs prochaines actions.
\subsection{Éléments inconnus}
Le programme ARCHE était un programme du ministère de la défense, basé sur un des abris souterrains de l'hexagone. L'objectif du projet était de créer des soldats capables d'opérer dans les milieux contaminés qui devenaient la norme des champs de bataille. Toutefois, l'effondrement du gouvernement et la ruine de la Zone ont causé l'échec du projet, et l'abandon de celui-ci par presque tous les scientifiques impliqués (sauf l'ancien, qui a maintenu le silence au sujet de la vérité).

La fin du monde a été une suite de conflits très violents un peu partout dans le monde, avec l'emploi d'armes nucléaires et des problèmes sociaux menant à es révolutions globales. Les marques de ces nombreux conflits sont toujours visibles : checkpoints, cratères, ruines éventrées....
\section{La Zone}
\subsection{La tour noire}
Une grande tour noire est non loin du point de départ des PJs. Elle a été plus grande, mais le dernier tier est tombé sur le reste, créant une sorte de labyrinthe de poutres métalliques.

Au pied de celle-ci, des lacs aux eaux troubles et insondables. La végétation qui pousse ici est déformée, et des mouvements suspects trahissent la présence de chose sous la surface.
\subsection{Les grandes ruines}
Au milieu d'une immense place, les ruines d'un grand bâtiment de pierre. Les fenêtres sont brisées, et les murs portent les traces de nombreux affrontements, mais la structure semble toujours solide. Le lieu est utilisé comme point de repère pour beaucoup et sera difficile à transformer en abri.
\subsection{Le puit et ses fortifications}
Une autre grande esplanade, celle-ci lourdement fortifiée par le passé. Au bout de celle-ci, une sorte de grand trou dans le sol, que les marchands appellent 'le puit'. Celui-ci semble disposer de nombreuses galeries, mais des défenses automatisées protègent encore partiellement l'endroit.

\subsection{L'autel}
Une autre grande tour, pleine celle-ci, même si sa partie supérieure est à peine un squelette de métal. Les cultistes de l'apocalypse considèrent qu'il s'agit d'un lieu sacré, où leur chefs peuvent observer la signification de l'apocalypse, et où les conflits se règlent dans le sang.
A ses pieds se trouve un réseau de tunnels que les cultistes utilisent fréquemment pour se déplacer hors de vue.

\subsection{Les trois types de tunnels}
Le premier type sont les anciennes lignes de transport : de grands tunnels avec des rails métalliques, parfois encore utilisables avec des véhicules spéciaux. Ils sont encore en bon état, et le principal risque est de tomber sur un autre groupe de survivants, qui utilisent les lieux pour se déplacer sans passer par l'extérieur.

Le second type sont les anciens égouts : la moisissure et l'humidité y sont la règle, et il faut être désespéré pour y rester longtemps. Mais c'est aussi le réseau le plus étendu, couvrant la majeure partie de la zone, y compris les zones abandonnées. Il est partiellement en ruine, mais avec de très nombreuses redondances

Le troisième est constitué des anciennes carrières courant sous la zone, de grandes cavernes creusées dans la roche, peu humides. certaines sections sont comblées par les ossements des habitants de l'ancien monde, mais la majeure partie est juste inconnue et non cartographiée. Quelques créatures y résident, mais les principaux dangers sont de se perdre, d'être noyé dans une galerie submergée, ou encore les effondrements, ce réseau étant le plus ancien et le moins entretenu, depuis bien avant la chute.
\subsection{La jungle}
Un ancien bâtiment isolé est maintenant complètement recouvert d'une épaisse végétation, et de nombreuses créatures rôdent dans ce qui reste. A l'intérieur, la végétation est toujours aussi présente, bloquant des couloirs et des pièces entières.

Le lieu est adossé à la rivière d'une part, au cratère d'autre part, et aux fortifications de l'esplanade, celles-ci séparant la jungle du puit.
\subsection{Le cratère Zero}
Une partie de la zone où aucun bâtiment ne se dresse, mais bien un cratère, de près de 15 mètres de profondeurs. Les tunnels passant là ont été détruits, même si certains sortent dans le cratère.
\subsection{La flèche}
Une flèche d'un très ancien bâtiment en pierre se dresse, là où auparavant il y en avait deux. Le bâtiment est sur une île, presque au centre de la zone. La flèche restante donne l'impression qu'un vent trop violent pourrait la faire tomber, mais a survécu à tout ce temps.
\section{Factions}
\subsection{Culte de l'apocalypse}
Pour ses membres, le jugement dernier est arrivé, et si eux l'acceptent, les infidèles y résistent encore. Ils s'attaquent aux bastions de civilisation et aux abris pour en tirer leurs propres ressources. Ce sont des pillards et le plus souvent de dangereux psychopathes. Ils recrutent des esclaves et des désaxés ou torturent leurs prisonniers jusqu'à les convertir.

Leur lieu le plus sacré est l'autel, où ils règlent leurs conflits dans le sang, et observent la destruction totale de la Zone.
\subsection{Bikers et roadsters}
De nombreux groupes différents vivent aux alentours de la Zone, des nomades, qui vivent et meurent à bord de leurs véhicules réparés avec les moyens du bord.

Ce sont des petits groupes, hautement mobiles, et qui n'hésitent pas à s'en prendre aux groupes isolés, et à marchander avec les autres. Ils sont limités par contre aux anciennes routes les plus importantes et notamment à la bande de routes qui ceinture la Zone.
\subsection{L'enclave}
Un des principaux groupe qui souhaite faire revenir le monde d'avant. Ils sont bien organisés, mais ont besoin de nombreuses ressources pour alimenter leur nombre. S'ils agissent principalement depuis le nord de la zone, leurs patrouilles passent un peu partout dans celle-ci, s'accrochant très fort avec les cultistes.
\subsection{Les goules}
Un des dangers qui rôde un peu partout dans les ruines, les goules forment des groupes généralement de petite taille. Ils se nourrissent de chair, et chassent en meute pour s'en procurer. Ils sont plus nombreux à sortir à la nuit tombée, le moment le plus dangereux étant le coucher du soleil.

Un nid particulièrement important est basé dans le sud de la Zone, entre l'autel et le puits.
\subsection{Les autres mutants}
Différents petits groupes existent dans les ruines, autour d'abris de fortune. leurs motivations et objectifs sont simples :survivre jusqu'au lendemain, même si cela implique le conflit avec les autres factions.
\section{Tables aléatoires}
Quand les PJs explorent un lieu, savoir si ils trouvent un abri décent:
\begin{enumerate}
\item La zone souffre de la pourriture, et aucun abri ne semble viable
\item Les bâtiments de la zone ne sont pas en bon état, et la recherche d'un lieu sûr ne donne rien.
\item Un lieu potentiel a été trouvé, mais il est déjà occupé : éliminer les occupants pourrait permettre de s'y installer, si la lutte ne détruit pas l'abri.
\item Un lieu a été trouvé, mais il s'agit du repère d'un danger important. S'attaquer à cet abri serait sans doute du suicide.
\item Un lieu acceptable a été trouvé : il manque quelque chose pour que cela soit idéal, mais cela pourrait tout à fait être une solution temporaire. Il peut manquer un point d'eau sûr, une position défendable, être à côté d'un danger existant, ne pas offrir un bon abri face aux éléments, ou encore être bien connu dans les environs.
\item Un très bon lieu a été trouvé, mais les mutants ont été remarqués, et un groupe adverse voir d'un mauvais oeil ce genre d'installation.
\end{enumerate}
\end{document}