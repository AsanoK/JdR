\documentclass[10pt,a4paper]{article}
\usepackage[utf8]{inputenc}
\usepackage{amsmath}
\usepackage{amsfonts}
\usepackage{amssymb}
\title{Sœur Denise, prêtresse}
\date{}
\begin{document}
\maketitle
\section{Histoire}
"Denise" est une prêtresse, vénérant la trinité, et aidant ses ouailles. Elle a été envoyée vers l'Alianais pour ancrer l'église dans les moeurs des colons, voire peut-être convertir les païens à la seule vraie religion.

C'est du moins ce qu'elle raconte quand on lui pose la question. En vérité, il s'agit de Blanche, une jeune femme qui a jusqu'ici vécu de vol et de contrebande en Ascellie. 

Un de ses receleurs (un prêteur sur gage assez riche du nom de Renaud) a essayé de la doubler, et elle l'a poignardé. Elle a pris la fuite au plus vite pour éviter la justice, et a décidé de quitter complètement la région pour aller en Alianais, sous un déguisement lui permettant de passer relativement au-dessus de tout soupçon.
\section{Personnalité}
Blanche est une femme pragmatique, dont la sécurité passe avant tout : elle a déjà tué, et n'hésitera pas à le refaire au besoin. 

Elle essaie par ailleurs de garder son secret le plus possible, prenant dans les faits le rôle de prêtresse, quand bien même elle n'en a guère l'éducation ni les habitudes.

Quand elle est surprise, elle peut jurer assez violemment, ce qui pourrait la trahir.
\section{Mécanique}
personnalité : méfiante
origines : urbain, bas-fonds
profession : criminelle
secondaire : religieuse
\subsection{Attributs}
\begin{itemize}
\item Agressivité : 4
\item Caractère : 3
\item Coordination 3
\item Empathie : 3
\item Logique :3
\item Résilience :1
\item Vigilance : 6
\item Forme : 1
\item Mémoire :2
\item Destin :4
\end{itemize}

\subsection{Compétences}
Langue natale : Ascellien

culture générale (Valleaux) 6, culture générale (Ascellie)2, culture générale (crime) 4; défense 4, corps à corps 1, athlétisme 3, discrétion 4, passe-passe 3, négociation 3, intimidation 4,poignarder 3, arme blanche (couteau)2, manipulation 2, fouille 3, observation 2, crochetage 2, mensonge 3, corps à corps 2, discussion 2, prières 3, liturgie 2, calme 2, connaissances (théologie) 2, culture générale (église) 2, symbolisme 1, lecture (Ascellien) 1.
\subsection{Matériel}
dague, vêtements robustes, sac à dos, tenue de prêtresse, symbole sacré (triangle ornementé), 1 semaine de rations.
\end{document}