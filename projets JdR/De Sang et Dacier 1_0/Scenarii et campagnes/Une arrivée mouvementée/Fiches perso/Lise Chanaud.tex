\documentclass[10pt,a4paper]{article}
\usepackage[utf8]{inputenc}
\usepackage{amsmath}
\usepackage{amsfonts}
\usepackage{amssymb}
\title{Alain Chanaud}
\date{}
\begin{document}
\maketitle
\section{Histoire}
Alain est herboriste depuis des années. Guérisseur en fait, puisqu'au delà des plantes, il s'occupe de tout patient arrivant chez lui.

Du moins, c'est ce qu'il faisait jusqu'il y a quelques mois. En effet, c'est là qu'une petite épidémie de vérole a frappé la région, laissant de nombreux morts dans son sillage, malgré ses efforts. Certains villageois amers ont alors commencé à raconter qu'il était à l'origine de cette maladie, l'accusant d'être un sorcier. Il a essayé d'ignorer ces ragots, mais ils n'ont fait que se renforcer, au point que certains commençaient à l'éviter.

il a donc décidé de partir, ne plus travailler pour des gens qui ne lui faisaient plus confiance, et se rendre utile ailleurs. Il a saisit certaines de ces affaires, une partie de son stock d'herbes, et est parti vers l'Alianais : il a entendu que la vie y était dure, et que ses compétences y seraient certainement très appréciées.
\section{Personnalité}
Alain est quelqu'un d'assez méthodique et de calme : il faudrait un cataclysme pour qu'il ne perde son calme.

Il est également fondamentalement gentil : en tant que soigneur, il oubliait souvent de réclamer un paiement aux plus démunis.
\section{Mécanique}
personnalité : calme
origine : rural, ferme
profession : guérisseur
secondaire : détendu
\subsection{Attributs}
\begin{itemize}
\item Agressivité :1
\item Caractère :3
\item Coordination :4
\item Empathie :5
\item Logique :4
\item Résilience :2
\item Vigilance : 2
\item Forme : 1
\item Mémoire :4
\item Destin :1
\end{itemize}
\subsection{Compétences}
langue natale : Ascellien
culture générale (comté d'Estiaire) 7, culture générale (agriculture) 3, culture générale (alianais) 2, profession (bricolage) 4, arme blanche (couteaux) 2, athlétisme 2, escalade 2, observation 3, herboristerie 4, organisation 2, médecine 3, chirurgie 1, connaissances(plantes) 4, fouille 1, négociation 4, discussion 6, calme 6, premiers soins 5, observation 2, corps à corps 2, manipulation 1.
\subsection{Matériel}
couteau, vêtements robustes, sac, herbes médicinales, bandages, aiguille et fil, 1 semaine de rations
\end{document}