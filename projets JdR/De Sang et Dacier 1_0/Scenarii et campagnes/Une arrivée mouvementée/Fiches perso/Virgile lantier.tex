\documentclass[10pt,a4paper]{article}
\usepackage[utf8]{inputenc}
\usepackage{amsmath}
\usepackage{amsfonts}
\usepackage{amssymb}
\title{Virgile Lantier}
\begin{document}
\section{Histoire}
Virgile est un bâtard : son père est le seigneur de Ferremont, sa mère une des servantes du château. Il a toutefois été au moins reconnu par feu son paternel.

Il a été formé comme un homme d'arme : combat, équitation, tactique... Et est un jeune homme en pleine forme.

Depuis la mort de son géniteur, il s'est rendu compte que ses perspectives d'avenir étaient limitées : ses trois demi-frères n'ont pas vraiment besoin de lui pour gérer le (modeste) domaine familial, et étant bâtard, il n'a aucune perspective de bon mariage.

Il a donc quitté le château de bonne heure, avec un gambison, son arme, son bouclier, et un cheval de selle pour se diriger vers l'Alianais, où il a de la famille éloignée : une cousine de son père est femme de banneret dans le nord, et il espère y faire valoir ses compétences et ce lien familial pour y vivre dans de meilleures conditions.
\section{Personnalité}
Virgile est quelqu'un de confiant dans ses capacités : il est relativement jeune, en forme, bien entraîné.... Il en vient parfois à être arrogant : il sait lire et est relativement érudit.

Il est secrètement l'amant d'Ilitia Valis. Les deux jeunes gens restent secrets à ce sujet pour éviter de causer des problèmes, notamment avec les parents de celle-ci, et au sein du convoi.
\section{Mécanique}
enfance : serviteurs
jeunesse : écuyer
adulte : homme d'arme
\end{document}