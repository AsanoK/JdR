\documentclass[10pt,a4paper]{article}
\usepackage[utf8]{inputenc}
\usepackage{amsmath}
\usepackage{amsfonts}
\usepackage{amssymb}
\title{Jaques}
\date{}
\begin{document}
\maketitle
\section{Histoire}
Jaques est, comme ses parents, un serf : un paysan lié à sa terre, n'ayant pas le droit de la quitter, et devant diverses corvées à son seigneur.

Il n'est toutefois pas un serf exemplaire : braconnage, dissimulation d'une partie de sa récolte.... Cela lui a plusieurs fois valu des problèmes : du pilori aux amendes, il a plusieurs fois eu affaire à la justice seigneuriale.

Après un conflit de plus avec le bailli (représentant du seigneur) de son village, il a décidé de prendre ses affaires et de partir avec sa femme et leurs fils. Ils sont partis en pleine nuit avec l'espoir de pouvoir vivre plus librement dans cette contrée lointaine.
\section{Personnalité}
Jaques est pragmatique : il ne braconnait et ne dérogeait pas aux règles par volonté, mais par nécessité. Il n'en est pas fier, mais n'en a pas honte non plus.

C'est quelqu'un avec une certaine volonté, que l'on peut qualifier de têtu.

Il déteste qu'on le méprise, ou que l'on méprise qui que ce soit d'ailleurs. De la même façon, il déteste les gens malhonnêtes.
\section{Mécanique}
personnalité : direct
origine : rural, ferme
profession : forestier
secondaire : criminel
\subsection{Attributs}
\begin{itemize}
\item Agressivité :4
\item Caractère :3
\item Coordination :4
\item Empathie :2
\item Logique :1
\item Résilience : 5
\item Vigilance : 4
\item Forme : 4
\item Mémoire :1
\item Destin :2
\end{itemize}
\subsection{Compétences}
langue natale : Ascellien
culture générale (marches alianaises) 5, culture générale (agriculture) 3, profession (bricolage) 2, arme blanche (couteaux) 5, athlétisme 6, escalade 2, observation 1, arme blanche(hache) 4, arme de traits (arc) 3, discrétion 5, culture générale (forêts) 4, culture générale (faune)2, observation 2, pistage 2, profession (chasse) 3, profession (bricolage) 2, défense 3, premiers soins 1, intimidation 2, négociation 1, pistage 2, poignarder 2, passe-passe 1, fouille 2, mensonge 1.
\subsection{Matériel}
hache, couteau de chasse, arme de tir, 1 semaine de ration, matériel de bivouac, gibecière, vêtements épais et chauds
\end{document}