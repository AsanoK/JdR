\documentclass[10pt,a4paper]{book}
\usepackage[utf8]{inputenc}
\usepackage[french]{babel}
\usepackage[T1]{fontenc}
\usepackage{amsmath}
\usepackage{amsfonts}
\usepackage{amssymb}
\author{ Antoine Robin}
\title{Projet MedFan 2}
\newcommand{\titre}{MF2}
\begin{document}
\maketitle
\tableofcontents
\chapter*{Introduction}
\section*{Qu'est-ce que \titre}
\titre est un Jeux de rôle, au cours du quel les joueurs vont pouvoir incarner des personnages d'un univers de \emph{Low Fantasy}. 
\section*{Intentions et objectifs}
Dans \titre, on joue:
\begin{itemize}
\item Dans un univers généralement de low fantasy : ici point d'orcs, d'elfes ou de nains, et peu de magie. Si il y en a, ils ne seront sans doute pas au premier plan.
\item Avec un MJ parmi les joueurs, mais ceux-ci peuvent garder un certain contrôle sur le déroulement de l'histoire, par des choix qu'ils peuvent faire.
\item Dans un univers basé sur la politique, les intrigues, les trahisons, bref, les relations, parfois difficile entre des personnages.
\item Dans un univers où les personnages, joueurs et non joueurs sont faillibles, imparfaits, voire franchement antipathiques pour certains.
\item Dans un univers où les jeux de factions, la réputation et l'influence sont importants, voire centraux à la narration.
\item Avec un système qui peut être violent, notamment contre les PJs : ceux-ci restent humains et donc fragiles.
\item A un jeu qui peut (si tout le monde est OK avec cela dans le groupe), accepter l'affrontement, social ou physique, entre PJs.
\end{itemize}
\section*{Roadmap de développpement}
\subsection*{Priorité maximale}
\begin{itemize}
\item Système de résolution de tests ; OK première version
\item Framework d'intrigues et de combats sociaux
\item Framework de duel
\item Réputation
\item Factions
\item Zone de jeu locale
\item Règles de propriétés, de domaines et d'entreprise
\end{itemize}
\subsection*{Priorité intermédiaire}
\begin{itemize}
\item Framework d'enquête
\item Framework de bataille rangée et de campagnes
\item Création de personnages
\item Récupération, physique, morale et sociale
\item Équipements
\item Zone de jeu régionale
\item Magie, religion et mysticisme
\item Boîte à outils du MJ : cultures, PNJs, rencontres, défis....
\end{itemize}
\subsection*{Priorité basse}
\begin{itemize}
\item Zone de jeu internationale
\item Maladies et infections
\item Exploration et voyages
\end{itemize}
\chapter{Règles de base}
\section{Tests}
Pour réaliser un test de compétence, le personnage lance 2d6, auquel il ajoute son score de compétence. On compare ensuite la somme de ces éléments à la difficulté du test : si le score est supérieur, le test est réussi, sinon, c'est un échec.

Un jet devrait toujours avoir des conséquences narratives : un échec doit avoir des conséquences, tout comme une réussite. Ainsi, il vaut mieux éviter les jets tentés 'par défaut' par un joueur, au cas où cela fonctionne.

\subsection{Conséquences}
Si les 2d6 d'un test sont identiques, le test va s'accompagner de conséquences, choisies par le joueur, avec l'accord du MJ. Si le chiffre indiqué sur les dés est impair, c'est une conséquence négative, sinon, c'est une conséquence positive.

Des propositions de conséquences, négatives ou positives sont proposées dans la description des différentes compétences.
\subsection{Difficulté des tests}
La difficulté d'un test va de 3 (trivial) à 15 (succès rare pour quelqu'un d'exceptionnel)
\chapter{Création de personnages}
\chapter{Règles avancées}
\chapter{Mener une partie}
\chapter{Lore}
\end{document}