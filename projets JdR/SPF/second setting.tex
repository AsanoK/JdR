\documentclass[10pt,a4paper]{book}
\usepackage[utf8]{inputenc}
\usepackage[french]{babel}
\usepackage[T1]{fontenc}
\usepackage{amsmath}
\usepackage{amsfonts}
\usepackage{multicol}
\usepackage{amssymb}
\usepackage{geometry}
\geometry{top=4cm, bottom=4cm, left=3cm, right=3cm}
\author{ Antoine Robin}
\title{Tabaya \\ {\Large Sous une Pluie de Feu}}
\begin{document}
\maketitle
\tableofcontents
\chapter{Introduction}
\section{Objectifs de design}
\begin{itemize}
\item Disposer d'un second setting, ficitf celui-là, pour SPF.
\item Travailler ce nouveau setting autant que possible
\item Inspirations principales : The Expanse, Dune, Mosul, conflits modernes en Syrie et au Donbass.
\item Doit fournir plusieurs opportunités de campagnes différentes pour SPF, avec des ambiances et des types d'unités très différentes
\end{itemize}
\chapter{Histoire}
\begin{multicols}{2}
\section{Histoire globale}
[WIP]
Objectifs narratifs : 

	• Créer une situation de conflits larvés entre plusieurs grandes factions, façon guerre froide
	• Justifier les jeux d'alliance et les importances politico-économiques des différentes factions majeures
	• Intégrer les technologies clés (voyage spatial, fusion, portails….) dans ces évolutions
	• Imaginer les changements sur le plan militaire des technologies, notamment de communication, de plus en plus développées (pas forcément plus complexes cela dit)
	• Imaginer les nouvelles formes de conflit générées par les évolutions politiques et technologiques : si le conflit en réseau fait XXIème, il pourrait tout à fait se maintenir, avec également l'ajout des affrontements spatiaux dans la balance.
	• Proposer des macro-factions , que ce soit des alliances, hégémonies ou autre, à la façon de l'OTAN et du pacte de Varsovie. Ces macro-factions jouent sur l'échelle stratégique, mais l'alignement avec l'une ou l'autre peut venir jouer sur les problèmes locaux de factions secondaires ou du moins locales. Les forces armées de ces macro-factions disposeraient de forces propres, et de commandements intégrés regroupant les troupes des factions constituantes par exemple, si on imagine un fonctionnement basé sur l'OTAN. Le degré d'importance des commandements unifiés/intégrés dépendrait sans aucun doute du degré de centralisation politique et des besoins militaires.
	• Questions à poser pour générer le setting : Quelles macro-faction, avec quels composants, pourquoi ces composants ont-ils évolués ainsi, quelle sont leurs importances relatives. Sur la section militaire, il faudra évaluer l'impact des technologies proposées et des changements culturels importants dû aux changements massifs de contexte. Le design des macro-factions va clairement être la clé de voute de cette conception de setting. 
	• Dans un premier temps, le plus important reste d'avoir une vague idée d'une macro-faction et de son implication dans le setting local. Les points essentiels : principaux éléments constitutifs, importance politique relative de ces éléments, quelques justifications pour ces changements, comment sont gérées les opérations militaires ainsi que le renseignement et la diplomatie. 
	• L'évolution des formes de combat va être relativement décorrélée de cette pensée sur les macro-factions, même si les macro-factions vont avoir leurs spécificités suivant leurs expériences propres des conflits : insurrections, affrontements spatiaux, coloniaux, conflits conventionnels, larvés….

\section{Technologies développées depuis le XXIème siècle}
\begin{itemize}

\item Portail spatiaux, des sortes de replis espace-temps, extrêmement coûteux à mettre en œuvre et à maintenir opérationnels. Maintenus hors du contrôle étatiques par différents traités, afin de garantir qu'ils ne soient pas visés lors d'un conflit armé potentiel. Le repli n'est pas parfait, le temps de traversé dépendant de la technologie utilisée (et des moyens mis en œuvre) et de la distance initiale entre les deux points à relier. Rareté : équivalent d'un réacteur nucléaire moderne. Des accidents graves ont pu se produire dans ces tunnels ou autour des portails eux-mêmes, avec les énormes quantité d'énergie impliquées.
\item Présence de quelques armes à énergie, rechargées avec des batterie carbone haute capacité. Les armes peu chères sont dites 'à éclat' ou 'à fusion' ,suivant si un matériau est brisé pour être projeté sous forme de fléchettes haute vélocité, ou si le matériau est fondu pour servir de projectile. C'est surtout le cas des armes militaires, les civils employant un mélange d'armes non-léthales, d'armes chimiques(à poudre) ou de marché noir.
\item  Énergie la plus courante, surtout sur un monde colonial : fusion nucléaire, les réacteurs étant bien miniaturisés sur les mondes primaires. Localement, d'autres sources d'énergie plus adaptées peuvent être déployées, des matériaux intelligents aux éoliennes par exemple. Les plus petits réacteurs peuvent alimenter des véhicules militaires de bonne taille, et la plupart des autres véhicules emploient des batteries plus ou moins standards (en tout cas les entrées-sorties sont standardisées au sein des deux alliances majeures).
\item  Les champs de force existent, mais restent rares, généralement réservé à un but strictement militaire, pour protéger des installations stratégiques. Au départ déployés pour le confinement des réacteurs à fusion puis des portails. A part dans ces deux cas, jamais actifs en permanence, à cause de la consommation énergétique très importante.
\item  Choix de véhicules sur la Tabaya : voiliers glissant sur le sable, différenciés par différents supports: coussins d'air, roues(rares), coques dédiées, foils…. Les véhicules rampant (sans traction aérienne) sont plus rares, et généralement utilisés pour des opérations strictement industrielles ou militaires. Les transports aériens sont utilisés, mais nécessitent de nombreuses autorisations et certifications pour voler en raison du temps. Des courses de volants, utilisant le vent de manière créative pour gagner du temps existent, et sont très populaires, même si les accidents sont fréquents.
\item Réseau(x) de communication très décentralisés, mais intégrés dans la vie courante, les appareils proches pouvant se joindre sous la forme de réseaux locaux, les différents réseaux des villes étant reliés entre eux par des installations gouvernementales. Les informations et contenus réseaux venant d'[alliance 1] sont plus rares, étant déployées à intervalle régulier. Cela signifie qu'il est possible de suivre des séries ou des émissions spécifiques, mais sans aucune des fonctionnalités live, ce qui explique bien souvent leur popularité moindre. Les séries locales par contre fonctionnent très bien, des émissions 'Real-life' aux discussions avec différentes personnalités ou influenceurs, en passant par de très nombreux formats de divertissement et d'informations.

\end{itemize}
\section{Tabaya}
Une planète colonisée suite à la découverte d'une technologie de trous de vers, aux alentours du milieu du XXIème siècle, permettant de lancer une sphère d'expansion spatiale importante, avec des explorations de plusieurs exoplanètes parfois très éloignées. Le trajet via les portails prend de quelques minutes à quelques jours, et les informations doivent transiter par coursiers, plus ou moins automatisés. Les colonies secondaires sont souvent dépendante technologiquement des colonies du cœur, qui disposent de la population beaucoup plus importante de la Terre.
La planète fait partie de ces colonies secondaires, le tunnel de saut prenant une journée complète à traverser, et n'étant pas traversable par les plus gros cargos. La majeure partie des deux vraies vagues de colonisation étaient constitué de réfugiés politiques, religieux ou économiques, n'ayant pas grand-chose à perdre aux confins de l'expansion humaine, et souhaitant vivre leur vie à leur façon.

Si plusieurs vagues d'immigration se sont succédées depuis la fondation de la colonie, il y a 120 ans, elles ont regroupés de très nombreux profils, issus de cultures vastement différentes : employés de [compagnie], réfugiés économiques, politiques ou religieux, exilés des colonies primaires, ou encore employés d'une corporation secondaire. Ce mélange important a été globalement réussi : les dangers de la planète et ses nombreux défis ont commencé à définir une culture planétaire commune, ce que les premiers états ont encouragés, pour se donner une légitimité.
Les premiers immigrés ont bâtis les premières villes, bien protégées contre les éléments, et surtout les tempêtes de sable. Ils occupent aujourd'hui la majeure partie des postes administratifs te des entreprises locales importantes. Toutefois, ils entrent de plus en plus en conflits avec les autorité de [compagnie] et de [Alliance 1], dont des représentants restent majoritairement aux commandes des postes-clés : les responsables des filiales de grandes entreprises sont tous des hors-mondes, tout comme de nombreux postes de la haute administration. Par ailleurs, les taxes sur l'import-export de ressources sont jugées très élevées, et venant d'une compagnie qui en gère essentiellement l'exploitation, sans nécessairement apporter de la valeur.

Cela conduit aujourd'hui à une instabilité politique croissante, alors que les différents états sont souvent jugés comme tributaires de [Alliance 1] au mieux, et de [compagnie]au pire, sans nécessairement que ces deux entités ne soient vues comme légitimes à imposer leurs vues sur la planète.
\end{multicols}
\chapter{État 1}
\begin{multicols}{2}
\section{Histoire}


\section{Société}
Le premier état historique, et disposant du meilleur accès aux stations orbitales est celui de [état 1]. C'est aussi possiblement le plus stable de la région, malgré des mouvement politiques de plus en plus revendicatifs. Il s'agit nominalement d'une démocratie représentative, mais de plus en plus d'habitant trouvent que ce qui est représenté sont les intérêts des hors-monde, que ce soit les corporations ou les alliances du cœur.

Les mouvements principaux de contestations sont les suivants: en premier lieu, le problème perçu de l'inégalité entre la colonie et les colonies primaires, dont des ressortissants occupent de nombreux postes administratifs ou corporatifs, et qui bénéficient généralement d'un meilleur niveau de vie de ce fait. Cette fracture est d'autant plus importante entre les villes et le territoire entre celles-ci, où les ressources de la planète sont exploitées par des locaux, mais pour les profits des hors-mondes. 

L'[état 1] commence par ailleurs à essayer de limiter l'influence des très puissants groupes d'entraide, des structures datant des débuts de la colonisation, où l'assistance que chacun pouvait espérer ne pouvait dépendre que de ses voisins si l'on était hors des structures corpo. Des groupes d'assistance se sont donc formés, pour fournir des soins, de l'aide lors des constructions, etc, et qui aujourd'hui sont parfois devenus de puissants groupes d'influence sur la politique locale, alors que plusieurs se sont regroupés. La presse corpo commence à parler de clans, pour les dénigrer, mais par endroit, le terme prend réellement, avec une certaine fierté. Attention toutefois, ces groupes sont partis d'une excellente intention, mais ne sont pas exempts de corruption, certains fonctionnant de plus en plus comme des cartels, contrôlant ce qui se passe, y compris illégalement sur des territoires fluctuant. Cela permet à l'[état 1] de commencer à déployer un arsenal législatif issu des colonies primaires, visant à lutter contre le crime organisé. Pour le moment, rien de spectaculaire, mais suivant les décisions gouvernementales, la situation pourrait très vite déraper : une répression importante, qui affecterait des 'clans' appréciés pourrait se transformer rapidement en problème politique majeur, avec un mouvement de protestation allant des actions organisées aux émeutes violentes.

La société est relativement morcelée, entre des réalités très différentes : les secteurs corpo et diplomatiques sont pratiquement au niveau de vie des mondes du cœur, bénéficiant de lourds investissements des corpo et des groupes politiques hors-monde. La sécurité y est assuré par des groupes extérieurs, et tout y est très contrôlé. Le reste des villes varie grandement, entre une classe moyenne naissante, basée sur des entreprises planétaires, qu'elle soit dirigée purement vers la planète, ou dans une optique d'export, et une concentration de la pauvreté, avec notamment les immigrés les plus récents, qui ne disposent pas forcément de contrats corpo, de soutiens, etc. et qui arrivent dans ces villes via les transports hors-monde. Hors des villes on trouve les exploitation corpo, parfois nomades, mais gérées de très loin par leur propriétaires : à l'exception de certains sites spécifiques, il s'agit en réalité de sous-traitants, et on y trouve rarement plus d'un ou deux contrats corpo. Les conditions de travail et de vie sont très dures sur ces exploitations, quel que soit leur but. Les plus anciennes et les plus stables commencent à devenir des communautés qui deviendront des villes, à partir des installations annexes de ces exploitations : hôtels, bars, magasins…. Enfin, certains montent leur propre communauté, basées sur des idéologies, des religions, des expériences sociales, ou encore sur l'espoir de trouver des points de ressources à proposer aux corpos en explorant le vaste territoire. La possibilité de devenir rapidement riche attire de nombreuses personnes, mais la majeure partie vont vivre une dure vie et mourir dans la misère. Certains se tournent parfois vers le banditisme, essayant de piller ou intimider les exploitations, ou de voler les convois de ressources à une corpo pour les revendre à une autre. Ces tentatives sont pénibles pour les corpo, mais pas encore assez coûteuse pour justifier des efforts trop importants pour éliminer le problème. Les forces de l'ordre essaient de réagir à ces problèmes, mais manquent souvent de capacité pour chercher les coupables au milieu du désert.

\section{Culture}
En premier lieu, c'est une culture d'entraide et de débrouille : ne pas apporter des compétences pratiques à la bonne marche d'un groupe, c'est être un boulet. Mentalité très présente hors des villes, et dans les clans. Savoir se défendre est considéré comme une base, tout comme les règles de l'hospitalité, qui sont sacrées. 


Elles sont globalement les suivantes : ne jamais refuser l'entrée à quelqu'un devant une tempête de sable ou une crise urgente; ne pas outrepasser sa bienvenue : on ne critique pas son hôte, ni ne reste trop longtemps. Si l'hôte mentionne que le problème est passé, l'invité doit partir. L'hospitalité implique aussi traditionnellement de nourrir son invité au moins comme soi-même, et de discuter avec lui. Il est impoli de refuser une invitation à discuter.

A part cela, la musique très commune, avec pas mal d'instruments à vent, soit utilisant la force des vents naturels, soit pensés pour l'évoquer dans les villes. Dans ce domaine, les artistes humains sont très largement préférés aux IA qui réalisent pourtant de bien meilleurs performances dans le coeur. L'accent est souvent mis sur les émotions liées à la nature : le désert, le vent, la mer, le vide spatial.

Au niveau sportif, les sports de glisse sont très populaires, et se réalisent soit sur l'eau, soit sur le sable des déserts. Les paris et compétitions sont toujours très suivis. Les sports d'équipes sont aussi très appréciés, notamment les jeux de balles libres (une sorte de jeu de balle au prisonnier, mais sur terrain fortement encombré).

Une spécificité en matière de communauté : la notion de couple très souvent remplacée par celle de maisonnée : les premiers arrivés sur la planète se sont regroupés par affinité, et les maisons avec un seul couple sont toujours rares. Les couples (ou autres structures) peuvent ne pas être très stable, mais les maisonnées tendent à le rester beaucoup plus. Les jeunes essaient souvent de former une maisonnée avec des amis proches (parfois de la famille d'un âge proche, mais ce n'est pas le cas général).

\end{multicols}
\chapter{Compagnie}
\begin{multicols}{2}

[Compagnie] est une corporation née au milieu du XXIème siècle, au lancement du développement des premières colonies solaires, afin de profiter des nouveaux marchés créés par les demandes de ces nouveaux habitats. Sa spécialité historique est le transport orbital, de fret et de passagers.

Après ce départ, la compagnie a continué en étendant ses activités sur tout le trafic spatial dans le système solaire, en augmentant sa proportions de bâtiments dédiés au transport inter-orbital. Une politique agressive en matière de réduction des prix, et la capacité à fournir à ses clients un service intégrant la totalité du transport de passagers ou de fret, d'un point A à un point B, au travers de différents orbites ou corps célestes.

Ces premières réussites, et l'explosion des besoins avec le développement des premiers portails, ont permis une expansion rapide, notamment envers les premières colonies extrasolaires, très dépendantes économiquement des échanges avec les colonies du système. Quelques décennies plus tard, et [compagnie] contrôlaient pratiquement le passage par certains portails, avec des contrats de quasi-exclusivité sur le ravitaillement de colonies comme [planète]. 

La division transport gère le transfert de passagers, que ce soit pour des vols planétaires locaux, un changement d'orbite, ou un changement de système. Elle opère une très grande variété de véhicules de tous types suivant les besoins, et possède de nombreuses filiales locales gérant les transports en commun de villes de taille très diverses.

La division Fret opère de même pour les ressources : matières premières ou traitées par exemple. C'est le cœur historique de business de [compagnie], et qui continue de rapporter très gros, malgré des problèmes réguliers de pertes totales ou partielles de cargaisons.
La recherche et innovation est une petite division, chargée d'optimiser les autres divisions, et fournir de nouvelles solutions aux problèmes rencontrés. Leurs principales activités sont aujourd'hui en structure navale, en optimisation de la logistique, ou encore en matière légale ou financière.

La compensation des pertes est là pour s'assurer que les pertes de la compagnie soient aussi faibles que possible, que ce soit en termes financiers ou d'images. Leurs activités impliquent de la gestion de crise, et peuvent aller de la reprise d'assaut d'un cargo piraté, au déploiement d'une armée d'avocat et de spécialistes en communication pour limiter les dégâts d'un scandale, à la revente au meilleurs prix d'actifs non-rentables. Deux compagnie de sécurité appartiennent majoritairement à cette division pour les crises dites 'chaudes', et recrutent principalement des anciens membres des forces armées d'[alliance 1].

La division finance et assurance comprend diverses banques, et gère les montages financiers sur lesquels se basent les autres divisions. Ils peuvent également fournir des conseils financiers, notamment à différents organisation étatiques ou non.

La division représentation est la dernière fondée, et a repris une partie du rôle de marketing qui était auparavant géré par chaque division, ainsi que la mise en place de plusieurs structures média généralistes, notamment à destination des colonies, qui ne peuvent profiter de toutes les features live sur les contenus produits dans les colonies primaires ou le système central.

Le développement matériel est chargé de développer et construire les différents vaisseaux de la flotte, et de fournir des vaisseaux à différents clients, civils ou militaires. Sa R\&D est connue pour être en pointe sur les procédés de production de grandes pièces d'équipements en 0-G, et l'industrialisation de pièces uniques. Son QG est dans les chantiers navals autour de Mars.

Enfin, la gestion des affaires coloniales, peu connue dans le système central, redoutée en dehors. Sous le couvert de services de consultation pour les régimes des colonies, il s'agit de la première source de revenu à l'échelle de [compagnie], même si cela est généralement dissimulé sous des montages financiers complexes et de nombreuses refacturations.  Sous différents prétextes, elle taxe la majeure partie des imports-exports sur les colonies qu'elle aide à administrer, et ce avec la bénédiction, ou au moins le désintérêt des autorité d'[alliance 1]. Par ailleurs, elle avance ses pions au sein des administrations locales pour 'optimiser les opportunités' = imposer plus ou moins ouvertement des contrats avec les autres divisions de [compagnie], avec une très forte agressivité en matière de business.
\end{multicols}
\chapter{Alliance 1}
\begin{multicols}{2}
\end{multicols}
\chapter{Campagne}
\begin{multicols}{2}
\end{multicols}
\chapter{Règles spécifiques à Tabaya}
\begin{multicols}{2}
\end{multicols}
\end{document}