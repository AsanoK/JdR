\documentclass{report}
\usepackage[utf8]{inputenc}
\usepackage{multicol}
\usepackage{geometry}
\usepackage{graphicx}
\usepackage{spf}
\usepackage{comment}
\geometry{top=3cm, bottom=3cm, left=2cm, right=2cm}
\title{Sous une pluie de feu}
\author{Antoine ROBIN }
\date{}

\begin{document}
\chapter{Scénario : il n'y a rien au-delà de la Volga}
\begin{multicols}{2}
\setlength{\columnsep}{0.5cm}
\setlength{\columnseprule}{0.5pt}

%scenario de découverte, devant durer quelques heures, avec une présentation des possibilités du jeu: combat, attrition, social, ambiance....
%Setting ? a définir. Bataille à Danzig au premier jour de la seconde guerre ?
\section{Synopsis et personnages}
\subsection{Le scénario}
Ce scénario d'introduction doit permettre aux joueurs comme au MJ de découvrir \diminutif. Il contient plusieurs exemples des obstacles et histoires que peuvent vivre les personnages de ce jeu. Dans l'enfer des ruines de Stalingrad, des tâches a priori simples peuvent devenir particulièrement dangereuses, comme s'approvisionner en eau ou en nourriture.

Les PJs appartiennent à la seconde compagnie du premier bataillon du 42ème régiment de fusiliers de la garde, 13ème division de fusiliers de la garde, sous le commandement du général Rodimtsev. Cette division a été envoyée dans la ville entre le 14 et le 16 septembre 1942, pour repousser la poussée initiale de la VIème armée allemande. Si son intervention a été décisive, les pertes seront très lourdes pour les gardes de Rodimtsev : sur environ 10000 hommes et femmes, un peu plus de 300 seront toujours en état de se battre à la fin de la bataille.

Après une introduction à la situation générale, désastreuse pour la 62ème armée de Chuikov, chargée de défendre la ville, les personnages vont traverser la Volga sous la menace des troupes allemandes qui se sont déjà emparées d'une bonne partie de la ville. Les personnages, faisant partie de l'avant-garde, vont devoir se battre pour établir une tête de pont dans les ruines de la ville, non loin de la gare principale et du magasin Univermag (lieu de reddition, le 31 janvier 1943, du récemment nommé maréchal Paulus).

Les personnages vont devoir se battre pendant une semaine sur cette position avant de devoir se replier sur l'univermag et la place rouge, marquant la fin du scénario. Il est tout à fait possible de jouer ce scénario indépendamment, ou comme un élément d'une campagne plus vaste.
\subsection{Personnages}
Vous pouvez au choix utiliser les personnages prétirés de ce livre, ou créer les vôtres, ou encore mélanger ces deux options.

Si vous souhaitez créer vos propres personnages, l'unité utilise les règles suivantes : il s'agit d'une unité d'infanterie, avec les spécialisation vétérans et tireurs d'élite (à ce moment, l'unité n'a pas encore son entraînement urbain, cela viendra dans les ruines de Stalingrad). Les personnages font partie du premier bataillon du 42ème régiment de fusiliers de la garde, appartenant à la 13ème division de fusiliers de la garde, partie de la 62ème armée chargée de tenir Stalingrad.

\encard{L'heure militaire}{Lors de l'écriture du scénario, toutes les heures sont écrites sur le format suivant : hhmm, un format assez courant dans les forces armées pour aller plus vite et limiter les risques de confusion. Par exemple, 20h, s'écrit 2000 et 7h15 du matin s'écrit 0715.}

Si vous souhaitez représenter une escouade plus spécifique, comme des troupes du génie, ou un groupe d'arme antichar, vous pouvez retirer la spécialisation tireur d'élite pour la remplacer par une autre, comme spécialisation antichar ou sapeur. 

\section{Introduction : les flammes de Stalingrad}
Le scénario démarre après la reformation de la division après la bataille de Kiev : aux vétérans de la 1ère brigade aéroportée ont été ajoutés de nombreuses recrues, et le matériel lourd perdu a été remplacé. Les armes légères viennent à peine d'être rendues, et la division a été déplacée jusqu'à l'arrière de la 62ème armée, sur la rive est de la Volga, au niveau de Krasnaia Sloboda (en face du centre-ville de Stalingrad, de l'autre côté du fleuve).

Les troupes ont été acheminées en train, avant de passer le début de la journée du 13 septembre à marcher jusqu'à un réseau de tranchées à environ 2 km à l'est de la Volga. Pour démarrer, vous pouvez lire ou paraphraser la description suivante:
\begin{quotation}
A 16h, le 14 septembre 1942, la 13ème division arrive en vue de Stalingrad. Les rumeurs vont bon train dans les rangs sur ce qui va se passer, mais la plupart ne pensent pas le combat imminent : les soldats viennent à peine de recevoir leurs armes. Au cours des deux jours de marche qui ont précédé, tous ont pu voir le haut panache de fumée noire s'élevant dans le ciel au-dessus de leur destination.

Le camarade Staline a été très clair quelques mois plus tôt : chaque pas en arrière est un pas de plus vers la Volga, dernière ligne de défense symbolique de l'union soviétique. Les officiers ont interdiction de se replier sans ordre sous peine de mort, et ces ordres doivent venir de très haut. Pour beaucoup, l'atmosphère est celle de la dernière chance : c'est à Stalingrad, la ville de l'homme de fer, que réside la dernière chance d'arrêter l'envahisseur fasciste et sa barbarie. C'est ici, sur les rives de la Volga, que se décidera la grande guerre patriotique !

Quand les personnages y arrivent, un réseau de tranchés  a déjà été creusé, avec de nombreux bunkers pour les accueillir. Le personnel logistique est arrivé un peu plus tôt et les troupes qui viennent de marcher pendant six heures peuvent manger un morceau. C'est chaud et comestible, ce qui est largement suffisant pour la plupart.

De nombreuses pièces d'artillerie et leurs servants tirent vers la ville régulièrement. Leurs insignes indiquent au moins 3 divisions : 196ème, 87ème et le 23ème corps blindé. Le bruit de leurs tirs vient couvrir les bruits de combat venant de l'autre côté de la colline, dans le centre-ville. L'artillerie de la division reçoit immédiatement l'ordre de former ses batteries à proximité et de commencer à ouvrir le feu sur l'ennemi, sous les vivats des troupes présentes.
\end{quotation}
 Le quartier-maître de la compagnie, Vassily Grigoryevich, passe s'assurer que tout le monde a déjà ses munitions, signe avant-coureur que l'unité risque de passer rapidement à l'action.

Les personnages vont avoir un peu de temps pour découvrir leurs camarades, et poser des questions sur la situation dans laquelle ils vont être plongés. Ils arrivent dans leurs cantonnements vers 1600 le 14 septembre.

Si les personnages essaient d'avoir des rumeurs sur ce qui se passe, vous pouvez utiliser celles-ci, ou en ajouter d'autres de votre cru:
\begin{itemize}
    \item La ville est déjà considérée comme perdue, et la division va devoir empêcher les troupes allemandes de traverser la Volga. Ils vont devoir renforcer et étendre ces tranchés pour espérer faire quelque chose. Mentionner trop ce genre de rumeurs peut amener un commissaire politique à sévir, surtout si elles sonnent 'défaitiste', la punition allant de corvées supplémentaires à l'exécution suivant le commissaire et sa vision des choses.
    \item La division va peut-être voir du combat dès le lendemain : il faudra traverser la Volga, et ça promet d'être moche. Un soldat a discuté avec quelqu'un qui s'est battu à Odessa et Sébastopol, et apparemment, en zone urbaine, on est tout le temps à proximité de l'ennemi.
    \item Certains disent que les gardes de Rodimtsev vont devoir remplacer les gars de la 131ème, qui se battent dans le secteur depuis deux semaines. Il faudrait repousser les fascistes hors de la ville par le sud dans ce cas.
\end{itemize}

Les personnages n'auront toutefois pas trop le temps de se préparer pour la suite : les officiers sont convoqués immédiatement pour recevoir leurs ordres. Vers 1630, les commandants de compagnie reçoivent les consignes et cartes pour gérer leurs opérations, et ils convoquent leurs responsables de pelotons pour 1700. A ce moment, les personnages devraient apprendre ce qui va les attendre : leur bataillon doit traverser dès ce soir la Volga pour préparer la zone de débarquement pour le reste de la division. Ils doivent traverser la Volga à 2000, suivis deux heures plus tard par le reste de la 13ème.
%TODO : ajouter carte préliminaire sur les positions allemandes.

L'organisation est sommaire : les personnages ont une heure de repos avant de se diriger vers la flottille de barques et de ferries qui doit transporter leur bataillon. Le sergent pourra rapidement apprendre auprès d'un officier de logistique en train de se casser la voix sur les différents groupes d'hommes que leur peloton doit traverser à bord d'un ancien canot de sauvetage (baptisé depuis 'la boite à chaussure'). C'est un groupe de marins de la flotte de la mer noire qui va les faire traverser : un sergent, le sergent Ivaniev, et une dizaine de rameurs. Ils doivent traverser avec le quartier-maître de leur compagnie, qui transporte avec lui plusieurs caisses de munitions et de matériel divers.

\encard{La flotille de la Volga}{Des embarcations de tout types ont été réquisitionnées pour transporter hommes et matériels dans Stalingrad, des barques aux ferries en passant par des bateaux de pêcheurs. Certains sont dirigés par leurs propriétaires, d'autres par des marins de la flotte de la mer noire, qui sont plus utiles ici après la chute de Sébastopol. Des radeaux de bois sont aussi utilisés, leur matériau réutilisé sur l'autre rive pour bâtir des bunkers.}

A 2000, l'ordre est donné par un coup de sifflet, il faut traverser la Volga ! Les rameurs et le sergent qui dirige la manoeuvre commencent à avancer la barque, qui bientôt se trouve dans le bras principal de la Volga, et se dirige vers les plages.

De l'autre côté, la nuit permet aux PJs de profiter de la vision de l'enfer devant eux. Vous pouvez lire ou paraphraser la description suivante :
\begin{quotation}
De l'autre côté de la Volga, des ruines éventrées brûlent à perte de vue. Des usines partent en flamme, et une épaisse fumée noire s'élève de grands réservoirs devant votre barque.

L'air est empli de l'odeur âcre de la fumée et de la poudre, et le bruit des pièces d'artillerie, déjà présent, est maintenant renforcé par celui des armes légères, qui se déchaînent au milieu des flammes.

Quelques tirs de mortiers commencent à tomber dans le fleuve, soulevant de grandes gerbes d'eau de plusieurs mètres, les plus proches vous arrosant copieusement. Au sifflement des mortiers s'ajoute bientôt le son des rafales de mitrailleuses tirées depuis l'autre rive.
\end{quotation}
La traversé va être la première épreuve du feu pour les PJs, qui pourront réagir à plusieurs évènements qui se déroulent à côté d'eux lors de la traversée :
\begin{itemize}
    \item un ferry voisin est en train de brûler après avoir été touché par un tir de mortier, et des soldats se jettent à l'eau pour échapper au naufrage : des tests de labeur peuvent permettre d'en hisser à bord, mais il faudra en premier lieu convaincre les marins de faire de détour pour les récupérer.
    \item Une rafale perdue laboure leur barque, blessant certains des marins et des passagers. On peut alors remplacer les marins, et essayer de soigner les blesser, ou au besoin remonter le moral/contraindre les marins à continuer.
\end{itemize}

Alors que les personnages finissent la traversée, vous pouvez lire ou paraphraser la section suivante :
\begin{quotation}
Alors que l'odeur de fumée commence à prendre le pas sur tout le reste, la rive occidentale de la Volga n'est qu'à quelques mètres : bientôt, le fond du canot racle contre la plage.

Alors que les balles traçantes continuent à voler, les soldats sortent de leurs embarcations pour se jeter au sol derrière le moindre couvert, avant de commencer à - enfin - retourner le feu à l'ennemi !

Votre peloton est proche du centre de cette plage, des embarcations de toute taille arrivant au fur et à mesure des deux côtés. Les abords abrupts de la Volga vous couvrent un peu, mais vous restez dangereusement exposés aux tirs, décidément trop nombreux.
\end{quotation}

C'est l'occasion pour les joueurs de subir leur premier combat ! A l'arrivée, la zone de débarquement est une petite plage le long du centre-ville. Plusieurs cratères d'obus forment des couverts, ainsi que des débris d'embarcation abandonnées. La plage n'est pas très large avant que les rives abruptes de la Volga ne forment un obstacle, ou un couvert, suivant le point de vue. Le principal problème pour les PJs sera un bâtiment au nord, qui est bien défendu par des troupes adverses. Leur commandant de peloton ordonne d'avancer vers la ville, tout en gérant le feu venant de leur flanc. Il y a deux menaces, dans tous les cas, le combat ne dure que trois tours de table (avant que le peloton ne traverse la plage pour arriver dans les décombres de la ville).

La première menace est une escouade d'infanterie dans la ruine d'un petit bâtiment à environ 80 mètres, essentiellement avec des fusils et des grenades. La dangerosité est de 2, et il faudra 3 réussites contre eux pour les faire se replier vers le bâtiment principal. La conséquence si cette menace n'est pas gérée est d'augmenter de 1 la dangerosité de la seconde menace. Tags : portée (200).

La seconde menace est un groupe de panzergrenadiers (fantassins) avec un nid de mitrailleuse au premier étage d'un bâtiment indiqué sur les briefing comme 'la maison des spécialistes'. Le niveau de menace est 3, et il faut 4 réussites pour les forcer à se mettre à couvert, ou détruire la mitrailleuse, et limiter le danger sur la zone de débarquement. Si cette menace n'est pas gérée, cela aura une influence sur le ravitaillement plus tard dans le scénario. Tags : portée (500).
\section{Partie 1 : avancer vers la gare 1}
Le peloton se rassemble dans les rues de la ville, et reçoit l'ordre d'avancer : un contre-ordre est tombé pendant la traversé, et le premier bataillon doit avancer pour reprendre la gare numéro 1 à l'envahisseur fasciste.

La place rouge, principal accès vers la gare, est encore à ce moment à peu près intacte, une grande étendue, seulement coupée la la fontaine de Barmaley, et quelques buissons ayant survécu au manque d'entretien de ces derniers mois, et aux bombardements des dernières semaines. Deux ou trois cratères ont retourné le pavé, et la première compagnie avance le long du nord de la place, vers l'univermag.

La première compagnie va s'arrêter pour prendre d'assaut l'univermag, apparemment, des fascistes sont retranchés dans les pièces intérieures. Les PJs peuvent traverser la place sans être trop inquiétés alors que le reste des unités prend d'assaut le bâtiment. 

Il faut ensuite passer un pâté de maison, le long de la rue Gogola avant de se retrouver face à la gare. Le peloton reçoit l'ordre de se placer rapidement dans les décombres des bâtiments : eux-même ne participeront pas à l'attaque, mais doivent en couvrir les flancs. Ce bloc contient entre autre une usine de clou, où le QG du bataillon va commencer à s'installer.

La première journée des PJs se passera donc dans ce pâté de maison, à couvrir sa façade nord. Ils pourront assister à la préparations de deux offensives contre la gare : la première se fait un peu après minuit, avec la majeure partie de la première compagnie, la seconde le lendemain dans la journée avec  un peloton de leur compagnie (la seconde).

Les personnages devraient arriver dans le pâté de maison le 14 septembre vers 2200, et commencer à le fortifier. Le lieutenant Chervyakov fait installer vers 2300 le QG du premier bataillon dans l'usine de clous, et gère l'organisation des combats à la gare, ainsi que la défense de ce bloc. L'ordre peut être donné aux PJs d'assister à creuser dans les décombres une partie d'une tranché de communication vers l'arrière du bloc et l'univermag (test d'endurance+labeur).

Les personnages peuvent choisir comment se préparer au cours des heures suivantes, avec les règles de repos. Il est possible de réaliser une seconde action de repos, au prix d'un point de stress supplémentaire le lendemain.


Le 15 septembre, les choses devraient être assez calmes pour l'unité dans ce pâté de maison : comme indiqué plus haut, une offensive se lance vers le milieu d'après-midi, où un peloton traverse l'avenue du communisme pour aller reprendre la gare. Ce peloton profite d'un recul des troupes allemandes, prélude à un barrage d'artillerie sur la position. A part cela, en début de soirée, les personnages peuvent se retrouver dans un combat avec une petite patrouille d'allemands, ainsi qu'un observateur d'artillerie: la première menace e a une dangerosité de 2, et abandonne après trois tours de table, et a les tags suivants : portée 150, discret 2. Si cette menace n'est pas gérée, l'artillerie adverse aura une bonne idée de la position défensive, ce qui les conduira à tirer plus tard dans la journée . La seconde menace est un observateur d'artillerie dans un bâtiment annexe de la gare, d'une dangerosité de 1, et les tags suivants : portée (vue), discret 4. Si cette menace n'est pas gérée, les tirs d'artillerie seront immédiats, et précis !

Le barrage d'artillerie, si il vise les PJs est le suivant : 1d3 PJs doivent réussir un test de couverture+chance SR 3, 4 si l'observateur d'artillerie est responsable. Les PJs peuvent bénéficier d'un bonus suivant leur protection, et ne peuvent être visés si ils se trouvent dans l'arrière du bâtiment ou dans le sous-sol.

A ce stade, les combats urbains alentours sont féroces : le reste de leur régiment est engagé dans un combat à mort avec les troupes allemandes, et on peut entendre le tonnerre de l'artillerie au nord, sur le Mamayev Kurgan, où le reste de la division se bat pour le sommet.

Un autre évènement, sans combat cette fois-ci se déroule le 15, le commissaire du bataillon, le camarade commissaire Denikin, ayant apparemment survécu. Or, celui-ci est une plaie notoire, n'hésitant pas à abattre les hommes quand il entend du 'défaitisme'. Si il disparaissait dans le combat urbain, le bataillon ne le pleurerait pas très longtemps. Il transmet toutefois un ordre utile : les déplacements d'unité doivent être évités autant que possible dans la journée, pour ne pas attirer les avions adverses. Il vaut mieux essayer de coller autant que possible les positions allemandes, et ne pas hésiter à lancer des raids la nuit pour les garder sur les nerfs.

Le 16 septembre, les combats vont devenir plus violents, plus vicieux. Deux évènements marquants pour les PJs : en premier lieu, on va leur demander de reprendre le bâtiment voisin, en formant un petit groupe d'assaut, soutenu par le reste de leur peloton. Une fois cet assaut mené, les PJs vont y découvrir une situation complexe : eux auront avancé au rez-de-chaussée, tandis qu'une escouade allemande tient le premier étage, et qu'un autre groupe de soldats soviétiques tient le dernier étage.

L'assaut du bâtiment commence par essayer de se rapprocher le plus possible de la position allemande (avec un test d'infiltration SR 3). Une fois cela fait, il faut combattre pratiquement au corps à corps le groupe d'allemands qui défend le bâtiment, étant surtout retranché dans la cuisine d'un appartement détruit. Cette menace a une dangerosité de 4, puis 2 si le test d'infiltration a été réussi, et le tag portée 150. Il est possible de tenter de tuer la sentinelle également, en réussissant un autre test d'infiltration SR 5, puis un test de mêlée SR 2. 

Le nettoyage du reste du bâtiment est un combat avec une menace : l'escouade de fantassins au premier étage. Ils ont une dangerosité de 3, ayant la bonne position pour eux, et finiront par se replier comme ils le peuvent après 4 réussites. Un test peut être tenté pour communiquer avec le groupe de soviétiques à l'étage, pour réduire leur dangerosité à 2. Si les PJs fouillent en détail le bâtiment, ils peuvent tomber sur la famille Nikolaiev dans le sous-sol du bâtiment.

Egalement le 16, les personnages vont pouvoir voir que pour certains, la vie continue dans les ruines : de certaines caves et abris sortent des civils quand les combats se calment. Ils cherchent à récupérer des choses dans les décombres, à se procurer de la nourriture et de l'eau, qui manquent terriblement. En particulier, les personnages pourront discuter avec Mr Kryuchkov, ou encore Alevtina Klimova, une gamine qui passe dans les décombres.

C'est vers 0200 le 17 septembre qu'apparaissent les premières neiges de Stalingrad : elles ne sont pour le moment pas importantes, et fondent rapidement au soleil le lendemain, mais elles indiquent que les températures baissent rapidement avec l'arrivée de l'automne.

\section{Combats pour la gare}
%17 -18 septembre dans la gare de Stalingrad : soutenir un assaut, participer à un autre, tenir la position, avant de devoir se replier.
Le 17 septembre commence tôt : un messager arrive pour donner l'ordre au peloton de reprendre la gare vers 3h du matin, l'escouade devant fournir un tir de couverture au groupe d'assaut, avant d'avancer à son tour dans le bâtiment.

Le tir de couverture est un défi de compétences, nécessitant 4 réussites avant deux échecs. En premier lieu, des tests de tirs SR 2 sont possibles, avec un avantage avec une arme ayant le tag barrage. Un test de tactique SR 3 peut également donner un maximum d'une réussite, en guidant le tir du reste de l'escouade. D'autres tests peuvent être entrepris avec l'accord du MJ, avec un SR de 3 ou 4 probablement.

Avancer vers la gare nécessite un test d'infiltration, de couverture ou d'athlétisme SR 2, pour ne pas être visé par l'ennemi dans les bâtiments voisins. Un personnage avec une arme barrage peut fournir un tir de suppression pour permettre aux autres de passer avec un SR 1, mais son SR pour traverser sans encombre devient 3.

A l'aube du 17 septembre, les personnages devraient donc être dans la gare ou à proximité immédiate, les forces allemandes se préparant à contre-attaquer le lendemain en début de matinée. Il faudra se préparer à cette offensive anticipée, avec les actions de repos. Les personnages pourront apprendre que le lieutenant senior Chervyakov a été gravement blessé et a été remplacé par son adjoint, le lieutenant Feodseyev. Si un peu d'action est préférée par les joueurs, un groupe d'infirmières essaie d'évacuer les blessés graves de la gare vers la Volga, et auront besoin d'aide : c'est un défi de compétence nécessitant 6 réussites avant 4 échecs, avec un SR de 2, et des compétences comme labeur, tir (avec une arme de barrage), athlétisme, tactique....

Vers 1200 le 17 septembre, les forces aériennes de l'armée rouge vont à la rencontre de leurs homologues allemands au-dessus de la ville, et après de longues minutes de combat tournoyant, un chasseur allemand tombe rapidement vers le sol comme une pierre, sous les hourras des soldats. Quelques instant plus tard, c'est un chasseur soviétique qui commence à descendre en flammes, avant de s'écraser sur un bâtiment des environs. Il est potentiellement possible de retrouver la pilote, décédée, et de rapporter sa mort à la hiérarchie, au prix d'un défi de compétence nécessitant 4 réussites avant 2 échecs, les tests se faisant à SR 3. En cas d'échec, un personnage au hasard subit une attaque de dangerosité 2 lors d'un duel de tirs en se repliant.

Vers 0900 le 18 septembre, les allemands lancent une attaque majeure sur la gare pour la reprendre. Ce combat dispose de trois menaces distinctes : le groupe d'assaut, le tank de soutien et les sapeurs. Si le groupe principal est vaincu, le combat se termine après un tour de table complet, le reste des troupes allemandes se repliant rapidement.

Le groupe d'assaut est la menace principale, de dangerosité 4 et nécessitant 5 réussites, et dispose de deux mitrailleuses. Chacune peut être détruite en prenant un risque, ce qui réduit la dangerosité de 1. Si la menace n'est pas gérée, les unités soviétiques vont commencer à se replier, augmentant de 2 la dangerosité de cette menace après 2 tours de table. Tags : portée 500. Ils seront repoussés après 7 réussites.

Les sapeurs sont là pour détruire les défenses soviétiques et ouvrir le chemin. Ils ont une dangerosité de 1, et nécessitent 3 réussites pour être vaincus. Si ils ne sont pas gérés après 2 tours de table, ils peuvent augmenter de 2 la dangerosité du groupe principal, dont les éléments avancés peuvent attaquer à la grenade et la baïonnette. Tags : portée 150, discret 2

Enfin, le tank de soutien est là pour fournir un tir d'appui avec son canon principal. Il vise en priorité les défenses soviétiques là où l'infanterie pourra avancer. Il dispose d'une dangerosité de 3, et nécessite 4 réussites pour être neutralisé (endommagé et en repli, ou en flammes, si jamais les PJs ont une arme antichar). Les tags de cette menace sont : défense 3+, portée 1500, tir de barrage 2, perforant 3. Si cette menace n'est pas gérée en trois tours, un PJ se battant contre la menace principale subit une attaque. 

Si les PJs repoussent l'assaut, ils peuvent se replier sans danger dans la foulée, en croisant au passage une autre unité soviétique qui se préparer à monter à l'assaut pour reprendre la gare. Sinon, un autre peloton arrive en renfort pour reprendre la position, mais un PJ au hasard subit une attaque de dangerosité 4.
\section{L'usine de clous}
%18 - 21 septembre à se battre à nouveau près de l'usine de clous, puis dans celle-ci. Pas mal de résistance aux assauts, problèmes de ravitaillement aussi.
Le 18 septembre devrait être plus calme pour les PJs, même s'il s'agit d'un des pires jours de leur unité, prenant et perdant plusieurs fois la gare. Les PJs, après leur repli, reçoivent l'ordre de Feodseyev de ramener du ravitaillement : les réserves de nourriture sont inexistantes, et celles de munitions disparaissent à vue d'oeil. L'eau est un problème constant, la majeure partie de celle-ci étant polluée par les métaux lourds et les débris d'origine inconnue. Ils doivent aussi transmettre au plus vite un rapport écrit sur la situation au centre-ville au QG de la 62ème armée. 

Une expédition pour le ravitaillement est une entreprise risquée : entre les tirs d'artillerie, les raids aériens, les tireurs. Mécaniquement, il s'agit d'un défi de compétences, nécessitant 8 réussites avant 4 échecs pour fournir dans un délai décent les munitions et rations nécessaires. Parmi les tests intéressant : orientation, infiltration, logistique, athlétisme, labeur ou encore conviction (pour convaincre les quartiers-maîtres de leur fournir ce dont ils ont besoin, voire de l'aide pour l'amener). Le trajet n'est pas évident : traverser les rues n'est pas une bonne idée, et il vaut mieux passer le plus possible dans les désormais nombreuses tranchées de communication entre les décombres et les bâtiments.

Le ravitaillement s'organise dans un réseau de tranchées et de bunkers sur et à proximité de la rive de la Volga : les munitions ont été dissimulées dans des sous-sols et abris souterrains, le reste est stocké dans des abris à l'air libre le long des tranchées du 'village', le nom donné par ses habitants au lieu. Le QG de Chuikov est un peu plus au nord, dans un bunker enterré où se trouve le général Chuikov et son staff. Un capitaine à l'air épuisé, du nom de Berejnoï écoutera leur récit et prendra leur rapport pour le communiquer au général.

En revenant prêt de l'usine de clous après leur long trajet, les personnages se rendent compte que la situation a évoluée sur le terrain : les bruits de combats semblent plus proches de l'usine de clous. En arrivant sur place, les membres de leur unité les informeront rapidement que la gare est tombée et que les fascistes contrôlent également les blocs au nord et au sud, et préparent sans doute l'assaut sur ce bloc. Le seul accès vers l'extérieur est une étroite tranchée de communication à l'arrière du bloc.

Les PJs sont assignés à défendre une partie du bloc pour la suite des opérations, le premier bataillon n'ayant pas reçu l'ordre de se replier, la seule solution consiste à s'enterrer. 

Lors de la nuit du 18 au 19, vers 0100 on donne l'ordre aux PJs de mener une reconnaissance sur un des blocs voisins, ce qui est un défi de compétences basé essentiellement sur l'infiltration, l'observation et l'orientation, le tout SR 3. Le défi nécessite 4 réussites avant 2 échecs pour être accompli avec succès. En cas de réussite, les personnages ont deux choses : en premier lieu la possibilité de voir que pas mal de troupes sont rassemblées, ce qui semble indiquer un assaut prochain, et une idée des positions de départ de l'assaut. Également, les personnages peuvent avoir l'option de tenter de capturer une sentinelle allemande. Dans ce cas, il faut réussir un test de mêlée SR3 pour le faire silencieusement. Le prisonnier doit ensuite être ramené vers le QG où il sera évacué vers l'arrière pour interrogation par le NKVD. En cas d'échec, certains personnages pourraient être attaqués (dangerosité 2), et surtout, ces informations ne seront pas disponible, ce qui rendra plus compliqué le lendemain.
%gestion du ravitaillement, avancée allemande

Le 19 septembre à l'aube, les tirs d'artillerie brisent le calme relatif pour une dizaine de minutes avant le début de l'assaut allemand pour tenter de s'emparer de l'usine et de son bloc d'habitation. Il y a deux menaces pour les PJs :

La première est un groupe de pionniers équipés pour l'assaut, avec des armes automatiques et un lance-flammes. Dangerosité 4, nécessitent 4 réussites pour être repoussés. Si ils ne sont pas gérés en trois tours, tous les personnages au rez-de-chaussé sont attaqués par ce groupe. Tags : portée 100

La seconde menace est un groupe de fantassins avec une mitrailleuse qui leur fournit un soutien. Dangerosité 3, et il faut 3 réussites avant 2 échecs pour les gérer. Si ils ne sont pas gérés, les personnages subissent barrage 2 pour le reste du combat. Tags : portée 800, barrage 1

Il est possible pour les PJs de rompre le combat en se repliant de leur bâtiment, vers l'intérieur du bloc. Dans tous les cas, à d'autre endroits du bloc, d'autres escouades soviétiques se sont repliées, abandonnant un bâtiment du bloc à l'ennemi. Il est peut-être possible de les repousser par un assaut, ou par un défi de compétences nécessitant 4 réussites avant 2 échecs (SR 4).

Dans la soirée la situation des troupes de la garde est mauvaise : même s'ils contrôlent encore le bloc, la pression est importante pour les quelques 150 hommes et femmes. Le lieutenant Feodseyev organise une défense en plusieurs lignes, de manière à repousser les fascistes s'ils arrivent encore à prendre pied. Les rapports des autres unités, au compte-gouttes, indiquent que les troupes adverses sont proches d'encercler la position.

Un groupe de soldats arrive également vers la position, une petite vingtaine de combattants mélangeant d'anciens cadets de l'école militaire, de marins de la 92ème brigade d'infanterie de marine, et des survivants de la 10ème division du NKVD. Ils se sont repliés depuis leurs positions défensives et ont rejoint celle-ci.
%combat dans les bâtiments, repli progressif des soviétiques

Au matin du 20 septembre, la situation est donc très compliquée : les troupes allemandes sont présentes sur trois côtés du bloc, et ont commencé à mettre une grosse pression sur celui-ci, un schéma qui va se maintenir lors de ce 6ème jour de combat. 

En premier lieu, en début d'après-midi, un groupe d'assaut va avancer, soutenu par un canon d'infanterie dissimulé derrière une barricade de décombres. Cela va leur permettre de raser les zones de résistance sur le côté nord du bâtiment. Si une sentinelle a été capturée plus tôt, chaque PJ peut réaliser un jet de son choix avec avantage au cours  du combat, ayant l'information sur l'axe et le nombre de combattants adverses. Ces informations sont apportées par un messager, qui prie pour pouvoir passer au retour.

Les PJs sont donc partis pour un autre combat, avec trois menaces:

La première est le groupe d'assaut, constitué d'une trentaine de fantassins, qui essaie de s'approcher. Dangerosité 3, puis 4 après 3 tours de table. Il faut réussir 5 tests avant d'en perdre 3 pour réussir à les faire se replier. En cas d'échec, c'est le front nord du bloc qui est pris par ces hommes, et chaque Pj subi une attaque immédiate alors qu'il essaie de se replier. Tags : portée 500

La seconde est leur soutien, constitué d'une vingtaine d'hommes aux fenêtres et d'une mitrailleuse dans un étage. La dangerosité est de 3, et il faut réussir 4 succès avant deux tours de table complets. En cas d'échec à les gérer, les PJs dans les étages subissent barrage 2. Tags : portée 800, barrage 2

Enfin, le canon d'infanterie est une menace de dangerosité 1, qui nécessite trois tests avant 2 échecs pour être battu. En cas d'échec, les personnages engagés contre la menace principale subissent un désavantage sur leurs tests de couverture.

Encore une fois, il est tout à fait possible de se replier vers le sud du bloc, toujours tenu par la garde. Dans ce cas, le nord du bloc tombe immédiatement, et chaque PJ subit une attaque immédiate du groupe d'assaut.

Si les PJs tiennent, le nord du bloc tient tant bien que mal, mais vont devoir se replier, en bon ordre toutefois. Ils auront le temps et l'occasion de préparer les défenses de l'usine, qui sert de bastion principal à la défense. En cas de victoire, ils gagnent donc une action de repos à utiliser.

Dans tous les cas, les munitions commencent à nouveau à manquer, ce qui va forcer les PJs à récupérer du matériel sur le champ de bataille (test d'infiltration SR2), ou à rationner différemment : test de logistique pour le chef d'escouade pour redistribuer les munitions et grenades efficacement avec les PNJs.
%combat dans les bâtiments, le ravitaillement devient critique, ainsi que la communication

A ce moment, les gardes ne sont plus très nombreux, et si quelques escouades tiennent des bouts des bâtiments adjacents, la plupart sont terrés dans les décombres de l'usine elle-même. Le plafond s'est effondré avant l'arrivée de la garde, sur les quelques machines abandonnées, et quelques tranchées ont été creusées pour communiquer avec le reste des anciennes positions soviétiques.


Le 21 septembre, les PJs vont se battre et devoir défendre l'usine de clous, bien protégée et préparée contre des assauts déterminés. Une centaine de combattants tiennent la position, et la défendent becs et ongles. Toutefois, les pertes commencent à s'accumuler, en plus de la fatigue.

Au cours du premier assaut, vers 0900 les défenses fournissent un avantage aux tests de couverture. Il est constitué d'une seule menace, un groupe d'assaut de dangerosité 3, qui est repoussé après 5 succès ou trois échecs. Si l'attaque n'est pas repoussée par les PJs, ceux-ci sont contraints de se replier un peu plus dans l'usine, perdant leur avantage défensif. Tags : portée 500

Pendant ce temps, les reste des gardes défend le bâtiment contre les assauts sur les autres fronts, et essaient de maintenir ouverte a tranchée de communication à l'arrière. Celle-ci est toutefois trop menacée et doit être abandonnée.


Le second assaut se fera beaucoup plus déterminé, avec un groupe de lance-flammes et un assaut de fantassins. Il démarre un peu plus tard dans la journée, vers 1300.

La première menace est le groupe de sapeurs avec des lances-flammes, d'une dangerosité de 1, puis 5 après le premier tour de table complet. Ils sont repoussés après 4 réussites ou 3 échecs. En cas d'échecs, la dangerosité du groupe d'assaut augmente de 2. Tags : portée 50, barrage 1

La seconde menace est le groupe de fantassins, qui utilise de nombreuses grenades à manches. Dangerosité 3, et nécessitent 6 réussites ou 4 échecs pour être repoussés. En cas d'échec, chaque PJ subit une attaque, puis les gardes réussissent à repousser l'assaut. Tags : portée 500

Ce second assaut se termine au corps à corps un peu partout, avec une sauvagerie rare. Dans la confusion, le lieutenant est tué, et remplacé par le lieutenant Dragan, qui commande la compagnie des PJs. Celui-ci rassemble les troupes qui lui restent, et prépare une sortie pour se replier, ayant finalement eu l'autorisation de Chuikov.
%A ce stade, les PJs devraient avoir été forcés dans l'usine de clous, où la défense continue à s'organiser avec une centaine de soldats. Le lieutenant Feodseyev meurt ce jour là lors d'un assaut, remplacé par le lieutenant Dragan, de la compagnie des PJs
\section{Percée et repli}
%ordre de se replier, et réalisation de la percée pour tenter de s'échapper : la percée s'annonce complexe, et se révèle meurtrière.

Alors que les bruits de combat s'estompent, et sont remplacés par les cris des blessés, le lieutenant Dragan passe d'escouade en escouade, signalant que Feodseyev a été tué, qu'il prend le commandement pour la suite, et qu'il vient de recevoir par radio l'autorisation de se replier. L'escouade doit se préparer à effectuer une percée d'ici la nuit vers l'est et la Volga, l'objectif étant de trouver un nouveau point facile à défendre pour la suite des opérations.

Les munitions se font rares, et la plupart des survivants vérifient que celles des morts et des fascistes ne seront pas laissées sur place. Certains préparent quelques surprises pour l'ennemi, et tous vérifient leur matériel.

Les personnages peuvent utiliser une action de repos (pour la dernière fois de ce scénario), avant que la tentative de percée ne soit tentée. Il sont moins d'une centaine à se préparer pour évacuer, certains transportant les blessés qui ne peuvent se déplacer.


%combat pour traverser le pâté de maison
Les semblant d'escouade sont organisés avec différents rôles : certaines doivent sécuriser l'arrière de la colonne, d'autres ouvrir le chemin, d'autres enfin, empêcher les troupes fascistes de refermer leur piège, en prenant les alentours de la tranchée de communication. Les PJs reçoivent ce dernier rôle, devant combattre au nord de la colonne principale dans la tranchée de communication, et la suivre.

Profitant de la nuit tombante, vers 1730, l'ordre est donné, par dignes pour rester le plus silencieux possible, de lancer la percée. 

Cet assaut est un autre combat, au cours duquel des renforts adverses vont essayer de les empêcher de se replier en bon ordre.

La première menace est une escouade d'infanterie dans les bâtiments au nord de la tranchée. Elle a une dangerosité de 3, qui devient 2 en combat rapproché. Il faut 4 réussites pour les chasser des bâtiments. Si ils ne sont pas gérés en 2 tours de table, ils pourront tirer sur la tranchée de communication, ce qui mettrait en grave danger la colonne de repli. Tags : portée 500

Après 1 tour de table, des renforts arrivent, avec une mitrailleuse qui s'installe sur les arrière des PJs pour les faucher si ils sortent. Dangerosité 3, nécessite 5 réussites, une prise de risque comptant pour 3, si jamais il y a 4 échecs, toutes l'escouade subit une attaque. Tags : portée 1200, barrage 2.

Enfin, après 2 tours de table, un tireur embusqué prend pour cible les PJs : dangerosité 4, nécessite 3 succès avant 3 échecs. Si il n'est pas géré, un personnage au hasard subit une attaque de dangerosité 6. Tags : portée 900, barrage 1, discret 3.

Les personnages, si ils sont en vie, peuvent ensuite courir vers la tranchée de communication, et sortir de l'étau des troupes allemandes. Cela ne signifie pas pour eux la fin de la bataille de Stalingrad, mais au moins auront-ils pu sortir de l'usine de clous.
Le scénario peut terminer ici, avec la détermination du sort futur des personnages toujours en vie. Sinon, cela peut servir de point de départ à une campagne de \nomjeu dans les ruines de Stalingrad : les combats présentés ici n'ont duré qu'une semaine, la bataille elle terminera le 2 février, 4 mois plus tard. 

En premier lieu, si vous préférez terminer ici le scénario, vous pouvez utiliser la table suivante ou déterminer par le moyen de votre choix ce qui arrive aux personnages dans le futur. 

\begin{enumerate}
   \item Le personnage est fait prisonnier par les allemands, qui ne l'abattent pas immédiatement. Il est envoyé en allemagne dans un des nombreux camps de concentration : privations, humiliations et violence pourront avoir raison de lui, ou il pourrait survivre à ce calvaire et être vu comme un traître en puissance par le NKVD. Certains ont été libérés, puis renvoyés immédiatement au front dans en tant que Strafniki : membre d'un bataillon pénal, assigné aux tâches les plus dangereuses.
    \item Le personnage meurt lors de la suite de la bataille de Stalingrad. Les lieux pour cela ne manquent pas : une ruelle en ruine, un bâtiment dévasté, ou des lieux plus célèbres, comme la maison de Pavlov, tenue pendant cent jours par des troupes de la 13ème. Vers le mois d'octobre, la majeure partie de la division se bat dans les usines Barrikady, l'usine de tracteurs ou encore la fabrique Lazur. 
    \item Le personnage meurt plus tard dans la seconde guerre mondiale, mais de manière héroïque, gagnant par ses actes une ou plusieurs récompenses à titre posthume. Parmi les possibilité : détruire de nombreux tanks allemands, prendre une position critique comme un pont et le tenir le temps que des renforts arrivent, mener un raid audacieux dans les lignes adverses, ou bloquer une contre-attaque violente des fascistes.
    \item Le personnage est blessé au combat, suffisamment pour être mis hors de combat pendant plusieurs mois. Si il revient au front, ce sera pour la fin de la guerre, probablement en 1944. Sinon, il peut être renvoyé à l'arrière pour y entraîner des troupes, travailler dans une usine ou être mis à la retraite, en fonction de ses capacités et de sa blessure.
    \item Le personnage survit à Stalingrad, mais aussi à la suite des opérations de la seconde guerre mondiale. S'il reste au sein de la 13ème division de la garde, il pourra même finir par participer à la bataille de Berlin, même si avec un rôle moins central qu'au cours de Stalingrad. 
    \item Le personnage survit à la seconde guerre mondiale en tant que héros décoré : au cours de la suite, il reçoit une récompense importante comme l'ordre de Lénine ou le titre de Héros de l'union soviétique. Il peut espérer de l'avancement ou une retraite confortable dans la vie civile une fois le troisième Reich vaincu.
\end{enumerate}

Sinon, il est tout à fait possible de continuer au-delà de ce scénario pour rejouer la campagne de Stalingrad ! 

En premier lieu, l'univermag ne tombe pas tout de suite aux mains des allemands, les troupes soviétiques tenant encore quelques temps ce bâtiment avant de se replier.

Un peu plus tard, un peloton, commandé par le lieutenant Pavlov, reprend une maison plus au nord  du centre-ville, devant la place du 9 janvier, derrière les lignes allemandes. Cette maison, connue sous le nom de 'maison de Pavlov' sera tenue cent jours par des troupes soviétiques qui se relaieront pour garder cette position derrière la ligne de front.

Parmi les autres possibilités : continuer le combat urbain en se repliant un peu plus vers la Volga au fur et à mesure, en ajoutant également le combat souterrain, alors que les deux camps tentent de contourner l'autre par les égouts de la ville, survivre au froid qui arrive sur la ville (et se procurer des tenues hivernales, et du combustible), ou se déplacer vers le nord pour se battre dans la grande zone industrielle, qui manquera de tomber au main des allemands vers la fin octobre, mais dans laquelle restera un îlot de résistance soviétique.

Dbut novembre, c'est l'opération Uranus, et l'encerclement des troupes allemandes, qui limiteront dès lors leurs opération offensives pour ne pas être débordées par leurs arrière. Cela ne signifie pas la fin des combats, les troupes soviétiques devant encore reprendre les ruines aux milliers de fascistes, ce qui durera jusqu'au 2 février, quand la dernière poche de résistance se rend.
\section{Récompenses et médailles}
Plusieurs récompenses et médailles peuvent être obtenues par des personnages au cours de ce scénario, suivant les actions entreprises par les PJs. Ces médailles peuvent être obtenues par tous après confirmation de l'information par un officier. Évidemment, au cours de la bataille, ce n'est pas une priorité, et il vaut mieux noter le fait d'arme pour le transmettre. Tous les officiers ne recommandent pas leurs hommes de la même façon, à vous de décider quelle(s) médaille(s) vos PJs peuvent recevoir.

La plus haute récompense pour un soldat est celle de héro(ïne)s de l'union soviétique. Elle s'obtient par des actes d'exceptionnelle bravoure, idéalement menant à de grandes destructions dans le camp adverse, ou une victoire soviétique. Elle implique une grande possibilité de sacrifice de soi au cours de la bataille, et est souvent décernée à titre posthume. Est comparable à la Medal Of Honor américaine ou la croix de Victoria Britannique. le sniper Vassily Zaitsev recevra la sienne pour avoir tué environ 200 allemands dont 11 snipers lors de la bataille de Stalingrad, et formé de nombreux snipers soviétiques à rôder dans les ruines (les 'lapereaux', Zatsev voulant dire lièvre).

Vient ensuite l'ordre de Lenine, un peu moins rare, et ancienne plus haute récompense atteignable. Elle est décernée pour des faits exceptionnels : la meilleure tireuse d'élite de l'union soviétique l'a reçu pour ses 309 victimes confirmées, dont plusieurs snipers allemands, l'immense majorité au cours de la bataille d'Odessa. Légèrement en dessous se trouve l'ordre de la bannière rouge.

L'ordre de la guerre patriotique, tout récent, est donné pour des actions de bravoure ayant causé de lourds dommages à l'ennemi : par exemple, détruire 3 chars moyens ou deux chars lourds adverses, ou trois avions pour la première classe. La seconde classe est plus commune, mais nécessite tout de même des dommages importants à l'ennemi. 325000 médailles de première classe et 950000 médailles de seconde classes auront été remises aux membres de l'armée rouge, des partisans et d'unités alliées engagées sur le front de l'est.

L'ordre de gloire est décernée pour des actes de bravoure spécifiques, le premier donnant droit à la seconde classe de l'ordre, le second acte donnant droit à la première classe, et le troisième acte de bravoure donne le titre de 'chevalier' de l'ordre. Pour un fantassin, les actes valables sont par exemple d'être le premier à entrer dans une position adverse en menant l'assaut, sauver les couleurs de l'unité de la capture, détruire 2 tanks avec une arme antichar ou 1 avec des grenades, 10 à 50 adversaires tués, capturer un officier adverse, et quelques autres. Elle vise à récompenser des actes qui ne méritent pas nécessairement une médaille plus importante, mais qui, répétés, sont exceptionnels. Seuls 2656 membres de l'armée rouge recevront le titre de chevalier.

Pour acte de bravoure devant l'ennemi, les personnages peuvent également recevoir la médaille 'pour le courage'. Pour des actes menant à une victoire, même locale, ils peuvent recevoir la médaille 'pour mérite au combat'. Ces deux médailles sont parmi les plus communes, ayant été chacune décernée plusieurs millions de fois lors de la seconde guerre mondiale.

Enfin, le 22 décembre 1942, est décernée à tous les soldats ayant participé à la bataille la 'médaille pour la défense de Stalingrad'.
\section{Changer de personnage au cours du scénario}
Au cours de l'affrontement, il est fort possible que certains personnages meurent, ou soient mis hors de combat pendant une longue période et doivent être évacués. Il est toutefois tout à fait possible de jouer un nouveau personnage.

En premier lieu, des réorganisations de l'unité font que de nouvelles têtes issues d'escouades décimées rejoindront sans doute l'escouade. Ils utilisent alors les mêmes règles de création de personnages que les PJs au début du scénario. Alors que l'organisation devient de plus en plus compliquée, et laissée  à l'appréciation des escouades elles-même, d'autres soldats, par exemple des troupes du génie peuvent aussi rejoindre, plus ou moins temporairement l'escouade.

Également, au vu du chaos de l'affrontement, de nombreuses unités se retrouvent sans officiers, voire sans sous-officiers, et peuvent alors être tentées de se mettre sous le commandement d'un groupe de gardes qui aurait gardé une certaine organisation. Parmi les possibilité : les cadets de l'école militaire de Stalingrad, durement touchés par un mois de combat, les troupes de marine dont une brigade a été déployée dans la ville, ou encore les troupes spéciales du NKVD (leurs officiers ne prendront pas leurs ordres d'un militaire, mais les troupes peuvent suivre la garde sur le terrain).

Enfin, il est tout à fait possible que des civils rejoignent l'unité, que ce soit pour les aider, défendre leur ancien lieu d'habitation, ou simplement pour survivre. Certains ont été organisés au début de la bataille en milice d'ouvriers et travailleurs, chargés de défendre leur ville et leurs usines.

En termes de jeu, les troupes du NKVD sont de l'infanterie avec les spécialisation forces de sécurité et entraînement urbain (cela fait plusieurs jours qu'ils se battent en ville). Les cadets de l'école militaire sont de l'infanterie avec les spécialisations unité d'honneur et entraînement urbain. Les milices ouvrières ou les habitants armés sont des unités de milice avec la spécialité troupes de garnison. Les troupes de marines de la flotte de la mer noire sont de l'infanterie avec les spécialisation troupes de marine et vétérans. Enfin, il est possible de jouer des membres d'une des unités rattachées à la 13ème en remplaçant sa spécialisation tireur d'élite par sapeur ou troupes de reconnaissances, ou encore remplacer infanterie par transmissions ou unité médicale.
\section{Fiches de personnel}
\subsection{Militaires de la 13ème division}
\paragraph{Brigadier Alexander Rodimtsev} Commandant de la 13ème division de fusiliers de la garde, vétéran décoré de la guerre civile espagnole, et officier des troupes aéroportées. Il a une réputation de héros avant le début de la guerre, et le comportement de sa division à Karkhov a été exemplaire, lui octroyant le rang de garde peu après. Au cours de la bataille, son Bunker est dans le 'village', sur la rive ouest de la Volga, non loin du secteur des usines.
\paragraph{Colonel Ellen} Commandant du 42ème régiment de fusiliers de la garde, sous le commandement de Rodimtsev. Il dirige le sud de la division lors de la bataille, depuis le magasin général Univermag, avant d'avoir l'autorisation de Chuikov de replier son QG vers le théâtre Gorki, à peine plus loin.
\paragraph{Lieutenant Senior Chervyakov} Commandant initial du premier bataillon du 42ème régiment de fusiliers de la garde, il sera blessé le 17 par une frappe aérienne de la Luftwaffe. C'est un homme énergique et parfois sévère, qui reconnaît que du temps peut être acheté par le sang de ses hommes, mais qui n'ira pas les sacrifier pour autant. 
\paragraph{Lieutenant Feodseyev} Adjoint de Chervyakov, qui le remplacera au commandement du premier bataillon pour les combats autour de la gare. Il est rapidement extrêmement fatigué par les contraintes du commandement dans ces conditions, et les rapports de pertes effarants dans ses troupes. Il sera tué le 21 dans l'usine de clous, après avoir stoppé l'élan des troupes allemandes.
\paragraph{Lieutenant Anton Dragan} Lieutenant de la seconde compagnie, celle des PJs. Il dirigera le repli de l'usine de clou après la mort de son supérieur dans les combats violents qui s'y déroulent le 21 septembre.
\paragraph{Caporal Kozhusko} L'ordonnance du lieutenant Dragan, les PJs pourraient avoir affaire à lui. Il est connu pour être un des 5 survivants en état de combattre du bataillon au 25 septembre.
\paragraph{Quartier-maître Vassily Grigoryevich} Un des officiers de la logistique du 42ème régiment de fusiliers de la garde, avec qui les PJs vont traverser la Volga.
\paragraph{Sergent Mariya Borovichenko}femme combattant depuis Kiev avec la 13ème division, et décorée pour sa bravoure au combat, alors qu'il s'agit d'un membre du corps médical. Ses exploits précédents ont déjà été mentionnés dans la presse, et elle a été décorée de l'ordre de Lénine, qu'elle porte toujours sur elle. Les personnages pourraient la croiser dans ses efforts héroïques pour évacuer les blessés graves de Stalingrad.
\subsection{Autre personnel militaire}
\paragraph{Lieutenant Klavdiya Nechaeva}
Pilote de Yak-1 appartenant à un escadron féminin de chasseurs. Elle meurt le 17 septembre dans le secteur de Stalingrad, au cours d'un combat que les PJs pourraient voir.
\paragraph{Sergent Ivaniev} Le sergent responsable de leur canot pour traverser la Volga. Son travail est de faire traverser les gardes à son bord et si possible de revenir vivant, et il ne compte pas rester plus longtemps que nécessaire sous le feu ennemi.
\subsection{Civils de Stalingrad}
\paragraph{ Mr Milan Kryuchkov} Un monsieur d'un certain âge, qui a toujours vécu à Stalingrad, quand on parlait encore de Tsaritsyn. Il est un fervent communiste, et pourra encourager les PJs à se battre contre les envahisseurs. Il cherche des choses à récupérer dans son ancienne demeure, et pourrait en proposer aux PJs. Il faudrait d'ailleurs faire attention à ce qu'il ne se prive pas trop pour les autres !
\paragraph{Alevtina Klimova}Une gamine, qui traîne dans les ruines des bâtiments pour essayer de trouver de la nourriture et des vêtements. La maison de sa famille était de l'autre côté de la gare, mais elle a été séparée avec les bombardements. Elle s'est aménagé un abri discret dans une cave à moitié ensevelie.
\paragraph{Famille Nikolaiev} Six personnes d'origine juive (la grand-mère maternelle, les parents et leurs trois enfants), qui vivent cachés dans une cave d'un bâtiment ) proximité de la gare. Ils habitaient dans un appartement du deuxième étage. Ils peuvent être convaincus de quitter cet abri pour un autre, voir de partir vers la Volga, ce qui les aiderai : si les personnages ne les aident pas, les allemands les feront sortir de leur abris pour les utiliser en tant que main d'oeuvre sacrifiable à l'arrière des lignes, voire les envoyer dans les camps.
\end{multicols}
%TODO : carte 1 : maison des 15-16 septembre (et position dans le pâté de maisons, et schémas d'opération)
%carte 2 : gare numéro 1 et schémas
% carte 3 : usine de clous et le pâté de maison complet + notes de de qui se passe
% documents divers : schémas de la progression en ville.
\end{document}