\documentclass[10pt,a4paper]{article}
\usepackage[utf8]{inputenc}
\usepackage[french]{babel}
\usepackage[T1]{fontenc}
\usepackage{amsmath}
\usepackage{amsfonts}
\usepackage{amssymb}
\author{ Antoine Robin}
\title{Marquise Aleksandra Viktoria Kolarova de la maison des maléfices}
\begin{document}
\maketitle
\section{Concept}
Troisième et plus jeune fille du prince Viktor-Kolarov de la maison des maléfices, et ainsi sous le feu des projecteurs depuis son plus jeune âge, d'autant que son tempérament ne l'incite pas toujours à la retenue. Elle alterne entre un statut de favorite des médias, et couverture de presse à scandale, et ce depuis le début de son adolescence.

Afin de canaliser cette énergie, et en restant cohérent avec les traditions familliales, elle vient de terminer ses classes au sein de l'académie des Lancers.
\section{Histoire}
Née au sein de la famille Kolarov, Aleksandra est la plus jeune des trois filles du prince, et porte le titre de marquise. 

Elle a passé une enfance relativement tranquille, essentiellement dans la demeure de vacance de la maison Kolarov, un petit palais dans un fjord magnifique.

Son adolescence a été relativement agitée : contrairement à ses deux soeurs, plus sage, elle a plusieurs fois été remarquée par la presse pour ses actions, souvent cavalières, dangereuses, ou simplement provocantes. Elle a de nombreuses fois éprouvé lourdement les nerfs de ses parents, et causé au moins deux démissions parmi le service de presse du palais. Malgré ou peut-être grâce à cela, elle a acquis à cette époque un esprit acéré pour étudier les news, et créer les siennes si besoin.

Juste avant la guerre, elle avait démarré des études à l'université de la baronnie, en science politique. Encore une fois, elle développa son statut de coqueluche de la presse, cette fois à scandale : les soirées étudiantes et diverses histoires de coeur ont fait couler beaucoup d'encre.

Puis la guerre a éclaté, elle-même quittant rapidement le domaine Hexen sous la protection de troupes de sa maison. Ce changement dramatique de situation l'a sérieusement secoué au début du conflit : elle avait perdu beaucoup de ses repères, sa vie ayant changé radicalement presque du jour au lendemain. Elle a par ailleurs perdu de nombreux condisciples et amis dans le conflit.

Plus récemment, son arrivée à l'Académie de Lancers de sa maison a été fortement médiatisée, servant de véritable vitrine de la propagande princière. Si ses deux soeurs participent plus au front intérieur, elle a décidé de servir sur les premières lignes, et de faire fructifier une telle opportunité de propagande. Son expérience en la matière lui a permis d'orchestrer la plupart des éléments ayant 'fuité' jusqu'à la presse sur son apprentissage à l'académie.
\section{Mécanique}
LL0 : leader, exemplar, executioner\\
LL1 : Vlad 1, exemplar 2 \\
LL2 : Vlad 2, vanguard 1\\
LL3 : Tortuga 1, vanguard 2

\section{Évolutions potentielles du personnage}
Un personnage dont le sort est irrémédiablement lié à celui de sa maison, de part les liens du sang. Cela va rendre très difficile tout éloignement majeur de leurs objectifs.

Toutefois, elle va avoir une expérience de cette guerre très différente notamment de ses deux soeurs, ce qui pourrait changer énormément sa relation avec elles et le reste de sa famille. C'est ce rapport au conflit en cours qui sera sans doute son principal axe d'évolution, suivant comment la guerre se déroule pour elle et sa maison.

A part cela, elle va aussi probablement voir changer sa relation avec la presse. Elle l'utilisait comme hobby étant plus jeune, et maintenant se met soigneusement en image pour sa maison. Dans le futur, cette relation pourrait changer : la presse pourrait se retourner contre elle, ou au contraire la soutenir dans ses changements.
\end{document}