	\documentclass[10pt,a4paper]{article}
\usepackage[utf8]{inputenc}
\usepackage[french]{babel}
\usepackage[T1]{fontenc}
\usepackage{amsmath}
\usepackage{amsfonts}
\usepackage{amssymb}
\author{ Antoine Robin}
\title{Concepts personnages}
\title{Svorost - concepts de personnages}
\begin{document}
\maketitle
\tableofcontents


\section{Officier - drake tank - Defender}
\subsection{Concept général}
Officier de carrière dans les troupes mécanisées. Pour lui ce conflit est plus sérieux que d'habitude, mais cela signifie surtout de nouvelles opportunités d'avancement avec l'accroissement des effectifs. Il est généralement froid et calculateur dans son approche du combat.
\subsection{Build technique}
Talents ll0 : heavy gunner, siege specialist, leader\\
équipement LL0 : heavy machine gun, charged blade, RPG\\
LL1 : drake 1 (argonaut shield et on remplace le RPG par le canon d'assaut), heavy gunner 2\\
LL2 : drake 2 : mech, équipé avec équipement identique, peut-être rempalcer la mêlée par les missiles à concussion. Heavy gunner 3.\\
LL3 : drake 3 : la heavy machine gun est remplacée par le leviathan. leader 2. Core du drake ?
\subsection{Background}
[Si hors Yakutsk] Issu d'une famille de petite noblesse, sa meilleure chance de carrière, de réussite, et de manière générale d'ascension sociale est l'armée. Il en a bien conscience, et d'un naturel calculateur, compte bien tirer parti de cet état de fait. Il ne se bat pas tant pour un idéal que pour lui-même et sa famille. Il a réussit à devenir pilote puis Lancer grâce à ses capacités, et ce avant le début de la guerre. Étant déjà pilote, il a depuis le début de la guerre formé des pilotes, avant de négocier une position de commandement sur la ligne de front.

[Si en Yakutsk] Ancien opérateur d'engins industriels, et un impliqué de la première heure dans le Parti, il a eu l'opportunité de piloter un mécha. Après une paire d'accrochage avec des troupes des Hexen, où il a réussit à revenir avec sa monture et une partie de ses camarades, il a endossé des responsabilités, qu'il apprécie. L'exercice du combat est pour lui un outil à plus d'un titre : cela permet évidemment de défendre le Parti, mais aussi lui apporte une reconnaissance qu'il n'avait jamais pu avoir auparavant, en tant qu'un des premiers officiers potentiels des Lancers.
\subsection{Évolutions potentielles}
Le premier axe d'évolution du personnage serait son rapport à la guerre, que celle-ci reste une opportunité, ou devienne plus personnelle suite aux évènements de la campagne.

Un second axe potentiel est l'effet du commandement, entre les pertes potentielles et la pression sur ses épaules. Il a les capacités intellectuelles pour gérer, mais peut-être pas la capacité à encaisser.

\section{Poster boy - tortuga - CQB support}
%Poster Boy !
\subsection{Concept général}
Un favori de la presse, dont les exploits son relayés presque dès son retour à la base. Il apprécie énormément cette attention, même s'il doit faire attention s'il souhaite avoir un instant à lui. Pour se faire mousser, il peut prendre des risques inutiles, ou en tout cas inconsidérés.
\subsection{Build technique}
Talents LL0 : vanguard, exemplar, juggernaut \\
équipement LL0: heavy melee weapon, shotgun, assault rifle\\
LL1:Tortuga 1 (on remplace l'assault rifle par le deck sweeper), exemplar 2\\
LL2: Tortuga 2 (on prend le mech, équipé avec les 2 CQB dédiés), vanguard 2\\
LL3: blackbeard 1, core +1 monture main (avec l'arme de mêlée tortuga 2), juggernaut 2
\subsection{Background}
[Hors Yakutsk] Un des premiers roturiers à devenir pilote, puis Lancer, ces deux exploits ont attiré sur lui de nombreuses attentions : celle de sa hiérarchie pour commencer, qui se méfie de lui, et celle de la presse, surtout, qui se passionne pour ses exploits et écrit sur ses actions dès que la censure le permet. Il est suivi avec une énorme attention, car il représente pour beaucoup une solution n'étant ni l'écrasement total par la noblesse, ni la sanglante révolution des Yakutsk

[Yakutsk]Un des rares Lancer de la république, les journaux de celles-ci l'adorent, car il représente la capacité de lutter à armes égales avec les oppresseurs. Il est tout autant suivi par les services de sécurité que par les reporters et correspondants de guerre. Il apprécie énormément la première attention, et a appris à vivre avec la seconde. 
\subsection{Évolutions potentielles}
Le premier point pourrait être sa relation à la propagande autour de lui : suivant les évènements dont il est témoin, il pourrait être tenté d'utiliser la tribune offerte par la presse pour dénoncer des agissements d'un régime ou d'un autre. Ou il pourrait choisir de maintenir sa position, et devenir réellement le poster boy de sa faction.

Par ailleurs, les relations avec le régime qu'il soutien pourraient être amenés à évoluer, suivant les actes des uns et des autres. Si on lui fait confiance, il sera tenté de répondre de même, et inversement, il pourrait perdre ses illusions.
\section{Noble - atlas/blackbeard - mêlée mobile}
%Build de cavalerie noble
\subsection{Concept général}
Une pilote de lancer issue de la noblesse, bercée des récits d'héroisme de la cavalerie mécanisée, elle a prit a coeur leur style de mobilité et de force de frappe. Elle peut paraître arrogante, mais c'est surtout une apparence due au style d'éducation qu'elle a reçu.
\subsection{Build technique}
\subsubsection{Version 1 : Blackbeard}
talents LL0 : duelist 1, executioner 1, skirmisher 1\\
mech LL0 : everest avec épée lourde, fusil d'assaut, arme tactique de close\\
LL1 : Blackbeard 1, Duelist 2; on remplace l'arme tactique par la hache tronçonneuse\\
LL2 : Blackbeard 2, Brawler 1; on équipe le blackbeard avec la nanocarbon, une chainaxe et un fusil d'assaut/fusil à pompe/CQB BB ? \\
LL3 : Atlas 1, exceutioner 2 ou skirmisher 2; On peut utiliser la zipline, mais on vise le Jäger Kunst 1 au LL4.
\subsubsection{Version 2 : Atlas}
talents LL0 : duelist, skirmisher, combined arms\\
mech LL0 : everest avec épée lourde, fusil d'assaut, arme tactique de close\\
LL1 : Atlas 1, duelist 2; on équipe le MGMS et le harpon (remplace le fusil)
LL2 : Atlas 2, skirmisher 2; Atlas avec arme tactique de close + harpon et très grosse mobilité avec le JK 1\\
LL3 : Atlas 3, skirmisher 3; on équipe la terashima blade.
\subsection{Background}
[hors Yakutsk] Une fille de relativement haute noblesse, qui a depuis son adolescence appris à manier sa monture, qu'elle dirige avec un talent naturel. Pour elle, la guerre est un moyen de faire ce pourquoi elle est née : servir sa maison et sa famille. Elle a une énorme confiance dans ses capacités et celle de sa monture, née d'une très longue pratique, et chasse en priorité les excellents pilotes adverses, augmentant régulièrement son tableau de chasse.

[Yakutsk] Issue d'une famille de petite noblesse militaire, elle a apprit à piloter dès son plus jeune âge. C'est une des raisons de sa survie : les forces de la république avait besoin d'un cadre de pilotes efficaces pour ne pas envoyer leurs recrues se faire tailler en pièce. Sa famille maintenue en otage par les forces de sécurité, elle n'a d'autres choix que d'obéir aux ordres, en espérant que des états de services suffisant permettent de les libérer, au moins sous condition.
\subsection{Évolutions potentielles}
Son background va énormément conditionner ses possibilités d'évolution.

Dans les deux cas, sa confiance pourrait être amenée à évoluer : elle pourrait tout à fait devenir un véritable fer de lance, ou remettre en question chacun de ses choix.

Aux côtés des maisons nobles, la perte des anciennes règles de cavalerie pourrait l'amener à douter, ou peut-être à embrasser ces nouvelles formes de guerre, à moins qu'elle ne tienne malgré tout à les maintenir coûte que coûte.

Au sein de la république, le point central sera sa relation avec cet état qu'elle hait. Au contact de ses nouvelles troupes, elle pourrait finir par comprendre leur lutte, et peut-être à accepter leurs idéaux, ou au contraire, refuser de servir cet état qui la menace à tout instant, elle et sa famille.
\section{Technicienne - swallowtail - Support}
%Idéaliste ?
\subsection{Concept général}
Technicienne de formation, elle a customisé son premier mécha à peine sortie de l'adolescence. Depuis la guerre, elle emploie ses capacités pour le conflit, et a récemment eu l'opportunité de monter dans ses créations, pour limiter la casse chez les pilotes qu'elle côtoie, plutôt qu'attendre le retour de leurs montures blessées.
\subsection{Build technique}
Talents LL0 : spotter, siege specialist, drone commander\\
Mech LL0 : everest avec mortier, fusil d'assaut, lame tactique.\\
LL1 : swallowtail 1, spotter 2; On prend les deux nouveaux systèmes\\
LL2 : swallowtail 2, spotter 3; On prend le mech avec le mortier et 2 oracles.\\
LL3 : Gorgon 1, drone commander 2; on regarde les 2 systèmes du gorgon 1. Autre option avec un lancaster ou barbarossa si on soutient une arme loading.
\subsection{Background}
Elle est née dans une famille d'ouvriers, et a réussi à accéder à de solides connaissances en mécanique grâce à de grands efforts de ses parents. A 15 ans, elle était capable d'entretenir et de modifier la plupart des véhicules, et a réussi à décrocher un job à l'entretien des montures de lancers.
\subsection{Évolutions potentielles}
En premier lieu, c'est un personnage sans expérience directe du combat. Cela risque d'être un traumatisme, qui pourrait changer sa vision du monde, et des technologies impliquées. Alors qu'elle a toujours considéré cela comme un sujet passionnant, cela pourrait devenir légèrement terrifiant, ou des outils nécessaires, suivant la façon dont tout cela se passe.

Par ailleurs, elle n'a pas franchement de grande vision politique, ou de convictions personnelles importantes. Cela pourrait changer dramatiquement, dans un sens ou dans un autre, ou encore la maintenir dans cet état par dégoût.
\section{Vétéran - Death's Head - Artillery}
%vétéran
\subsection{Concept général}
Un vétéran du conflit, ayant déjà perdu une partie de sa famille, ainsi que plusieurs parties de lui-même (littéralement dans le cas de son bras). Il a survécu au début du conflit dans l'infanterie, avant d'être noté pour devenir pilote. Si cela a augmenté sa capacité de destruction, il trouve aussi que les projectiles allant dans sa direction sont proportionnellement au moins aussi dangereux et beaucoup plus nombreux.
\subsection{Build technique}
Mech LL0:Evrest avec cyclone pulse rifle, assault gun et tactical melee weapon \\
Talent LL0 : crack shot, heavy gunner, skirmisher \\
LL1 : death's head 1, heavy gunner 2\\
LL2 : death's head 2, crackshot 2; passage en death's head avec le cyclone + vulture\\
LL3 : Barbarossa 1, heavy gunner 3; on va chercher les autoloadings du barbarossa pour maximiser l'impact du cyclone. Le rail est pas ouf, on ira pas le chercher a priori.\\
\subsection{Background}
[Hors Yakutsk]Vétéran des troupes d'infanterie, il a déjà perdu un frère au front, et deux membres de sa famille lors d'une frappe ennemie sur des cibles civiles. Si cela le motive quand il a un ennemi dans son viseur, cela a aussi renforcé sa détermination à survivre à toute cette histoire, et à se trouver du bon côté du conflit lorsque la poussière retombera. Il a déjà été blessé à plusieurs reprises, mais maintient de très bons états de service, suffisant pour gagner une place parmi les pilotes potentiels.

[Yakutsk]Un insurgé de la première heure, il a perdu une bonne partie de sa famille dans les contre-offensives de la maison Hexen, et s'il continue d'en vouloir grandement aux nobles de toutes sortes, il compte surtout leur survivre à tous. Il a plusieurs fois ramené un mécha tenant plus par chance que par réelle intégrité structurelle, et commence à avoir une certaine réputation de dur à cuire parmi les troupes de la jeune république.
\subsection{Évolutions potentielles}
Il est déjà désabusé par la guerre, mais peut-être qu'elle pourrait redevenir un problème personnel, et pas seulement une mauvaise épreuve à passer. Sinon, il pourrait contaminer par sa mauvaise volonté le reste de son escadron.

Sinon, ses relations avec sa hiérarchie pourraient devenir compliquées : il est efficace, mais ne cherche plus à faire de zèle. Par ailleurs, la guerre, ses buts et ses méthodes commencent à le dégoûter, et cela pourrait lui valoir de nombreuses inimitiés.
\section{Convaincu - Vlad - striker/controller}
\subsection{Concept général}
Un personnage convaincu par la justesse du combat mené par sa faction dans le chaos de Svorost. Il n'a pas besoin de propagande pour cela, et pourrait bien la relayer auprès de ses camarades. L'ennemi doit être vaincu pour restaurer l'ordre légitime sur Svorost, et rien ne l'empêchera.
\subsection{Build technique}
Talents LL0: executioner, nuclear cavalier, vanguard\\
équipements LL0 : heavy close, 2 shotguns\\
LL1 : Vlad 1, excecutioner 2; on prend la lance à la place d'un shotgun et le système dédié\\
LL2 : Vlad 2, excecutioner 3; on prend le mech, équipé comme l'everest avec le nailgun à la place du shotgun et le nouveau système.\\
LL3 : Vlad 3, nuclear cavalier 2; on prend la superheavy de close et le pieu.
\subsection{Background}
[Hors Yakutsk] Un noble, élevé dans la conscience de sa supériorité sur la plèbe. Le conflit sur Svorost est pour lui un problème à régler pour deux raisons : remettre les criminels de la république populaire à leur place, et mettre sa maison à sa juste place dans l'échiquier politique. Il est convaincu par sa cause, et dispose de bons arguments, qu'il a parfois entendu depuis qu'il est né.

[Yakutsk] Un militant de la première heure, plusieurs fois recherchés par les forces de l'ordre pour différents crimes : insurrection, assassinat, trafics divers.... Il est convaincu par l'usage de la violence pour soutenir la révolution, et ne voit aucun mal à l'employer contre les oppresseurs ou leurs marionnettes, pour libérer les opprimés de Svorost !
\subsection{Évolutions potentielles}
L'évolution la plus brutale serait d'être confronté à une opposition entre les idéaux de sa faction et ses méthodes. Cela pourrait remettre en question tout ce sur quoi se bâti sa confiance, et pourrait le retourner complètement.

Être confronté en personne avec les forces adverses pourrait aussi influencer ses opinions, semer le doute dans son esprit.
\end{document}