\documentclass[10pt,a4paper]{book}
\usepackage[utf8]{inputenc}
\usepackage[french]{babel}
\usepackage[T1]{fontenc}
\usepackage{amsmath}
\usepackage{amsfonts}
\usepackage{amssymb}
\author{ Antoine Robin}
\title{Personnage trône écarlate}
\begin{document}
\chapter{En compétition}
\section{???, combattant tiefelin}

\chapter{Recalés}
\section{Epanoui-à-l'ombre, leshy fongus}
Un leshy champignon, apparu dans les montagnes près de Janderhoff. D'un naturel plutôt enjoué et sympathique, il apprécie les matières organiques pourrissantes, avec une légère préférence pour la viande. Peu au fait des conventions sociales, une poubelle un peu pleine, ou les débris trouvés dans la rue peuvent lui servir de repas. Il a développé une certaine passion pour des nourritures fermentées cela dit, leur trouvant un goût unique : fromages et alcools peuvent donc trouver grâce à ses yeux, ainsi que les viandes sèches. Il n'apprécie pas le sel cela dit.

Il est arrivé récemment à Korvosa, et reste toujours intimidé par le bruit et l'agitation qui y règne, et préfère penser aux épaisses forêts d'où il vient. Pourquoi est-il à Korvosa et comment cela le relie à Lam ?

\subsection{Build}
Ascendance et héritage : fungus leshy \\
Background : aucune idée, peut-être pas important \\
classe et style de jeu : sans doute un ranger

\section{Vroklan, garde-oeuf des embermaws}
Vroklan est un kobold rouge appartenant à la tribus des embermaws. En tout cas, il leur appartenait avant de perdre sa charge. En effet, il faisait partie des gardes-oeufs de la tribus, chargés de protéger les oeufs de la dragonne rouge Tsazgatherax. Mais l'oeuf en sa garde a été dérobé ! Rejeté, et exilé par sa tribus, il a essayé malgré tout de retrouver cet oeuf, dont il a suivi la trace jusqu'à Korvosa. Il n'est pas à l'aise du tout dans les grandes rues de pierre, mais est déterminer à retrouver cet oeuf ! Il a cru entendre que Gaedrenn Lam serait lié à ce vol odieux, et compte bien lui poser la question.

Il n'est pas arrivé depuis longtemps à Korvosa, et reste un peu perdu parmi les peaux-lisses. En particulier, il souhaiterait trouver d'autres kobolds, pour retrouver une forme de communauté : la sécurité d'une antre bien protégée et d'une communauté nombreuse lui manque grandement.

\subsection{Build}
Ascendance et héritage : kobold caveclimber ou dragonscaled\\
Background : garde ?\\
Classe et type de jeu : fighter arme à deux mains (hast?) ou peut-être sword and board. Pas d'archétype trop évident, peut-être beastmaster? ou cavalier \\

\section{Aemon Venatius, médecin tiefelin}

Concept : un médecin tiefelin, d'origine chélaxiene. Un bon civil pas très habitué à toutes ces histoires d'aventures, étudiant à l'université de Korvosa la médecine. Il est aussi fier de son héritage chélaxien, que peu fier de son héritage infernal (mais en même temps, il pourrait vouloir en apprendre plus sur cela également). Il est confiant dans ses capacités intellectuelles, mais pas toujours le plus socialement adapté, ni très à l'aise lorsqu'il faut agir. 

Thème relativement low fantasy, avec une certaine polyvalence.

Ameon est issus d'une famille d'origine chélaxiene, récemment immigrée, habitant depuis dans le médian. Ses parents ne parlent presque jamais de la raison qui les a poussé à quitter l'empire, mais cela est peut-être lié à la marque de l'influence infernale sur leur fils. En tout cas c'est une de ses hypothèses.

Plongé dans la médecine depuis tout petit, avec l'officine d'apothicaire de ses parents, il a réussi à intégrer l'université de Korovsa, où il est actuellement étudiant. Il ne participe que peu aux traditions de son école, étant le plus souvent plongé dans les livres, ou occupé à en apprendre plus en disséquant des corps, pour essayer de mieux en comprendre le fonctionnement. Dans sa promotion il occupe un peu le rôle de mouton noir, un peu étrange, mais très appliqué dans ses études.

En plus de ses études, il conserve chez lui des flacons d'échantillons intéressants, encore une fois afin de mieux comprendre le fonctionnement de tout cela. 

Il a été plongé dans les problèmes avec Lam suite à une sombre affaire lié à un des cadavres disséqué : un témoin a juré avoir vu le jeune homme tuer l'homme pour découper son cadavre, avant de se rétracter. Même si son nom a été nettoyé depuis, il a enquêté sur cette affaire, et a pu remonter, grâce à certaines traces sur le corps, à la vieille pêcherie. Il compte bien en savoir plus, pour savoir pourquoi il a été accusé à tort.

Il parle doucement et calmement, sauf dans la panique, où il pourra monter le ton et la vitesse.

\subsection{Build}
Ascendance et héritage : Humain tiefelin \\
Background: accusé de meurtre \\
Classe et style de jeu : Investigateur légiste, sans doute le skill monkey\\


\section{Aretius Bromathan, noble de Korvosa}
Concept : un noble korvosan, relativement ambitieux et débrouillard. Il déplore l'orientation religieuse récente de son patriarche, et préfère les traditions militaires de sa famille. Il est un peu arrogant, notamment sur ses capacités martiales

Thème plutôt high fantasy, très polyvalent.

Aretius est issu de la famille Bromathan, ancienne famille noble, essentiellement connue pour son implication dans les forces militaires de la ville. Son patriarche actuel est cependant plus intéressé par le culte de Sarenrae, et a pu délaisser les traditions familiales. C'est du moins le point de vue d'aretius, qui envisage pour sa part une carrière dans une des organisations armées, si possible la compagnie du sable, mais la garde rouge pourrait faire affaire.

Il a apprit à se battre au sein de l'académie Orsini, avec un style qualifié d'utilitaire : peu de fioritures ou de spectacles, peu d'élégance même, mais des résultats indéniables. Cela le fait passer pour un rustre dans la bonne société de Korvosa, mais il n'en a cure. Il apprécie également de fréquemment sortir du manoir familial à la moindre occasion, pour aller faire la fête ou simplement se balader, ce qui a pu lui attirer quelques ennuis, et quelques contacts en ville, au-delà des hautes sphères de la ville (qu'il trouve quelque peu ennuyeuses).

C'est quelqu'un de confiant, voire arrogant dans ses capacités martiales, et sa capacité à se sortir le cul des ronces. Il peut avoir un langage à plusieurs niveaux, suivant ce qu'il pense de ses interlocuteurs: de très châtié à très vulgaire.

Il a été entraîné dans cette affaire de Lam suite à une affaire de chantage, subie par une bonne amie à lui : Lucia Albici Porphyria. Ayant appris de ces problèmes, il s'est proposé pour enquêter un peu à ce sujet, et soit contacter la garde, soit régler lui-même le problème. Quelques questions posées à un ami de la société céruléenne, et il envisage sérieusement de ruiner l'activité criminelle de ce triste sire, ce qui lui donnerait peut-être une certaine réputation.

\subsection{Build}
Ascendance et héritage : Humain, héritage à définir ?
Background : noble (pour un guerrier), écuyer (pour un roublard)
Classe et style de jeu : sans doute roublard ruffian, possiblement avec un archétype un peu spécifique.  Regarder le build avec l'archétype aldori swordmaster+roublard de ce guide : \begin{verbatim}
https://docs.google.com/document/d/1jksJ0jkRNdQ1r6veBdDpYY-4wuBRTsLmQlVMIX_j27U/edit#heading=h.dbdq7k20fx00
\end{verbatim}


\section{Visimar Dolgrin, artisan runiste nain}
Concept : un artisan des runes, peu enthousiasmé par les affrontements, mais ne reculant pas devant le test de ses oeuvres. Il est très propre sur lui, sauf en sortant de son atelier, et apprécie les civilités. 

Visimar est né à Korvosa, au sein du clan Dolgrin, des marchands de Janderhoff, son père et un de ses frères s'étant établis directement à Korvosa pour mieux négocier avec les habitants. Il est né dans les livres de compte, mais ce n'était pas ce qui l'attirait, et il a été formé comme apprenti runiste à la citadelle céleste.

En tant que forgeur de runes, il est surtout spécialisé dans les oeuvres d'art et les mécanismes intéressant, plus que dans les runes de bataille, les fendoirs, ou les armures pour ceux qui les manient. Il apprécie le calme des runes mineures, et de leur intrication complexe pour obtenir une belle pièce. 

Il est très récemment revenu à Korvosa pour y établir un petit atelier, modeste, mais prometteur, avec des commandes qui commencèrent à arriver. Il pouvait prendre inspirations sur les nombreuses cultures de la ville, et disposait d'une certaine liberté créative. Puis son atelier fut pillé par un groupe de cambrioleurs, malgré la protection très théorique de la société céruléenne. 

Ruiné par cette histoire, Visimar est bien décidé à régler une bonne fois pour toute le problème. Il n'aime pas la violence, mais reste un nain, et o ne lui marche pas sur les pieds sans manger un gnon!

Il est généralement très calme et poli, parlant assez doucement. Cela peut changer soit quand on s'en prend à son travail, ce qui lui inspire une sorte de mélange entre une colère sourde et du mépris, soit quand il parle de son travail, où la passion l'emporte rapidement sur la mesure.
\subsection{Build}
Ascendance et héritage : nain, à déterminer.\\
Background : artisan\\
Classe et style de jeu : inventor, innovation à étudier\\
\section{Amal Dolgrin, négociant sans scrupules}
Concept : un marchant en tout, distribuant avec largesse son trésor pour se faire des relations, et bâtir ensuite ses affaires. Repris de Kardum du Talion, avec un côté parrain assumé. Il a la particularité d'être légèrement allergique à la magie, rien de grave, mais ça le gratte.

Amal est un membre du clan Dolgrin, qui a toujours vécu à Korvosa, discutant avec des gens de toutes ascendances, et faisant affaire avec tous. Il est né dans les livres de compte de son clan, des marchands de Janderhoff, dont une branche s'est établie à Korvosa sur le long terme. Et cette éducation de marchand lui a donné un certain sens des affaires.

Ainsi, il essaie toujours de négocier son chemin au travers des problèmes, idéalement avec de meilleurs cartes en main que son interlocuteur. Il n'apprécie pas la violence physique, en tout cas trop proche de lui, mais sait se défendre au cas où.

Il a développé quelques relations dans la société céruléenne, en particulier sur le marché des docks, où il conduit certaines de ses affaires. Toutefois, il ne voit pas l'argent obtenue comme une fin, mais bien comme un moyen : pour lui, les relations, la réputation et un solide réseau de dettes contractées sont la véritable base du pouvoir et de l'influence.

Alors qu'il comptait se lancer complètement à son compte, il a eu maille à partir avec des hommes de Gaedren Lam, qui menacent de couler son début d'activité. Ce contre-temps fâcheux pourrait en plus lui faire perdre le peu de réputation que ses affaires louches pouvaient lui avoir parmi son clan. Une situation inacceptable qu'il compte bien régler, quitte à se salir les mains lui-même pour une fois.

Il se comporte en 'entrepreneur', prenant facilement des risques avec son argent en espérant des dividendes plus loin sur la route, quitte à parfois rogner avec la moral. Il est très généralement dans la légalité, même si parfois il peut s'autoriser quelques petites entorses.

En terme de comportement, c'est une anguille, qui préfère ne pas subir de conséquences de ses échecs, mais n'hésite pas aussi à récompenser tout ceux qui sont autour de lui, à s'intéresser à leurs problèmes : cela lui offre une certaine reconnaissance, ainsi que des opportunités parfois inespérées.
\subsection{Build}
Ascendance et héritage : nain, à déterminer\\
background : ???\\
classe et style de jeu : roublard ou alchimiste
\section{Renzi Coponisa, gardien de la communauté Ysoki}
Concept : un ysoki combattant, qui protège sa communauté. Il n'est pas toujours très sympathique envers les gens extérieurs, un peu méfiant dans ses rapports avec eux. Problème avec les grands espaces vides (agoraphobie légère).

Thème plutôt low fantasy/dark fantasy.

Renzi vient d'une des familles Ysoki de korvosa, établis depuis relativement longtemps dans le quartier de pointe nord, mais toujours un peu repliés sur eux-même. Il vient d'une fratrie de 14 : 6 frères et 8 soeurs, la plupart habitant toujours à Korvosa. Evidemment, lui connait la totalité des membres de sa famille, jusqu'à ses cousins issus de germain, ce qui parait improbable à tout autre. La plupart des membres de sa famille travaillent en tant que petits artisans : tailleurs, couturiers ou encore cordonniers,  avec une branche non-négligeable dans le bâtiment (notamment la menuiserie), et un de ses frères (et certains de ses enfants) aux docks. Il n'a jamais quitté les murs de Korvosa pour plus d'une heure.

Renzi lui-même a été apprenti d'un oncle couvreur, mais a découvert sa vocation dans les altercations de rue avec certains des criminels de la ville : d'un naturel plutôt sanguin, il réagissait très fortement à ces rixes, et a appris à se battre pour protéger sa communauté. Il a fini par arrêter son apprentissage pour travailler comme videur à la sirène poisseuse, puis finalement pratiquement à plein temps pour aider les siens quand ils ont des problèmes. Il ne s'intéresse pas aux décisions ou aux détails, il est là pour éviter que les autres ysoki ne soient blessés. Cela a pu lui attirer quelques problèmes avec la garde, qui n'apprécie pas de voir quelqu'un faire ce genre de choses à leur place. De temps en temps, il travaille comme chasseur de prime, si cela rejoint ses intérêts.

Son problème avec Lam vient de la disparition récente d'une de ses (nombreuses) nièces, qui aurait été ensuite aperçue non loin de sa pêcherie. Après quelques rapides questions auprès de ses connaissances en ville, il en a apprit un peu plus sur ce Lam, et n'a pas du tout apprécié ce qu'il a entendu. Il souhaite donc retrouver rapidement sa nièce, et n'aura pas franchement de considération pour le criminel au cours du processus.

Étant habitué à vivre en grande communauté urbaine, il risque les crises de panique dans les endroits déserts et surtout, trop ouverts. Il préfère de très loin les foules et bâtiments denses, voire même des tunnels, confortablement sombres et avec une solide voute rocheuse au-dessus de la tête, de préférence bien sec.

De part sa vocation, il se méfie un peu des étrangers à sa communauté, à moins d'avoir travaillé avec eux pendant un certain temps. Il se méfie en particulier de ceux qui ressemble à des fauteurs de troubles : petits truands hors de la société, aventuriers de passage.
\subsection{Build}
Ascendance et héritage: Ysoki deep rat\\
Background : Enfant disparu\\
Classe et type de jeu : guerrier reach basé sur la force. Archétype à déterminer.\\


\section{Alissa Dolgrin, forgerone naine}
Concept : une jeune naine, avec un équipement ornementé, intéressée par les notions d'aventures, une combattante bien entraînée, qui a quelque peu déserté pour aller casser la gueule de Gaedren Lam.

Elle est née à Janderhoff, au sein du clan Dolgrin, composé essentiellement de marchands. Elle-même n'ayant aucun talent pour le commerce et les négociations en général, elle a fini, au grand dam de ses parents, par être entraînée par une des écoles militaires de la citadelle céleste. Après cela, elle avait malgré tout une utilité pour les siens : les caravanes transportant leurs biens au travers de la Varisie, et vers Korvosa devaient être protégées, et elle fut donc employée pour cette tâche, dans l'idée de diriger à terme la défense de telles caravanes.

Elle a donc en quelques années traversé une bonne part de la Varisie, notamment la ville de Korvosa, où deux de ses cousins se sont installés pour faire affaire avec les humains. Elle même s'y est également installée récemment, afin de protéger le petit entrepôt et les deux ateliers.

C'est à ce moment là qu'elle a croisé la route de Gaedren Lam, un criminel local, mais aussi une plaie notoire pour les affaires, tentant d'extorquer toujours plus d'argent au clan Dolgrin. Alissa étant une forte tête, elle a décidé d'aller s'occuper directement de son cas plutôt que d'attendre que le problème se règle de lui-même.

Psychologiquement, c'est une tête brûlée, qui, quand elle a une idée, n'en démords que rarement. Elle tient ses positions très longtemps, trop longtemps parfois même, ce qui peut la déservir. C'est aussi une grande gueule, pas toujours très adaptée aux situations sociales. A part cela, elle apprécie le chant, même si son répertoire personnel contient plus de chants de marche que de berceuses.

Elle parle fort, de façon franche, et en abusant d'argot de Janderhoff. Elle ne cache pas vraiment non plus ses intentions ou ses opinions.

\subsection{Build}
Ascendance et héritage : naine, probablement des forges.\\
Background : guarde\\
Classe et style de jeu : guerrier hammer and board\\

\section{Septimus Coponisa, gros bras Ysoki}
\subsection{Concept}
Homme-rat chasseur de primes, avec des liens importants au sein des Ysoki de Korvosa. Un personnage plutôt polyvalent, avec un rôle de support de la première ligne, que ce soit à distance ou en combat rapproché. Peut-être une recherche d'archétype roublard au niveau 2?
\subsection{Lore}
Septième enfant d'une des famille Ysoki de Korvosa, les Coponisa, septimus n'a jamais quitté la ville. Cette famille travaille essentiellement dans le domaine des teintures, pas très riches, mais pouvant vivre correctement, malgré la fratrie de 13 à laquelle septimus appartient. 

En plus de ce nombre, la famille a des liens forts avec la plupart des autres Ysoki de Korvosa, les différents groupes se prêtant fréquemment assistance en cas de besoin, que ce soit pour des problèmes professionnels ou privés.

Septimus a commencé un apprentissage auprès d'un des couvreurs de cette communauté, mais ne l'a jamais fini. En effet, il est rapidement apparu que le jeune rat, et en particulier son approche directe de certains problèmes seraient mieux employés ailleurs.  Il a ainsi été engagé comme sécurité par les Restes, où il a appris à gérer les clients un peu récalcitrants.

En plus de cela, il a pu aider sa communauté en assurant également leur sécurité, parfois en se mettant à dos des groupes de brutes locales, mais il a pris rapidement l'habitude de les gérer. Cela ne lui a pas franchement valu d'amitiés chez les gardes, mais ce n'est pas un problème pour lui, il ne leur fait pas confiance pour régler efficacement les problèmes des Ysoki de la ville.

Son problème avec Lam est toutefois plus personnel que cette vocation : il a récemment été informé par sa soeur, tertia, que deux de ses enfants (des jumeaux), avaient disparus. Il a posé rapidement des questions à droite et à gauche, dont les réponses semblaient accuser le vieux truand. Sans trop s'inquiéter, Septimus compte bien récupérer ses neveux, avec pertes et fracas au besoin, ça ne lui posera aucun problème de conscience.

N'ayant jamais quitté les murs de la ville, il pourrait avoir des réactions inattendues face à de grands espaces, préférant de loin le confort d'un bâtiment, voire mieux, d'un souterrain. C'est également un mangeur de premier ordre, très actif, voire hyperactif au quotidien.


\subsection{Build}
Guerrier avec arme à allonge, sans doute avec le background enfant disparu. Arme = fauchard + étoiles de lancer.

Ou guerrier deux armes avec pick et light pick probablement. même background ?
\subsection{Roleplay}
Un homme-rat sous cocaine, parlant vite, avec une voix plutôt aigue.

Au niveau registre, il est généralement relativement poli, sauf quand il considère parler à des truands ou des fauteurs de trouble. Dans ce cas, il devient beaucoup plus vulgaire.

Il est agoraphobe, et souvent affamé (avec une légère prédilection pour le fromage).

Ses objectifs personnels sont de calmer autant que possible les troubles. A part cela, 
\subsection{Évolutions possibles}

\section{Vala/Mordran Dolgrin}
\subsection{Lore}
Un membre du clan Dolgrin, des marchands venus de Janderhoff. Ses parents se sont installés à Korvosa afin de négocier plus directement avec leurs clients, mais maintiennent beaucoup de liens avec la citadelle céleste.

Il/elle est donc né(e) dans les livres de comptes, les reçus, et les inventaires de caravanes ou d'entrepôts.

N'ayant aucun talent personnel pour l'art du commerce il/elle est allé à Janderhoff pour y suivre un apprentissage auprès d'un grand-oncle orfèvre. Il/elle y a appris de nombreuses techniques de travail des métaux, ainsi que les motifs typiques des citadelles célestes.

Après être revenu à Korvosa, il/elle y a ouvert un petit atelier, dans l'idée de vendre ses créations, et de profiter de l'atmosphère cosmopolite de la ville. En effet, cela fournir de nombreux autres motifs et techniques sur lesquels expérimenter. Les débuts de l'activité furent relativement tranquilles, le temps que le bouche à oreille fasse son travail, puis les commandes commencèrent à arriver plus nombreuses.

C'est un matin, en ouvrant l'atelier, qu'il/elle vit que tout était ravagé : les commandes en cours avaient été volées, les métaux précieux également, les meubles étaient détruits, tout comme une bonne partie des outils.

Le temple d'Abadar et les juges de la cité vendirent presque tout ce qui restait de son activité pour couvrir les dettes dues aux commandes annulées. Pendant ce temps, il/elle mena sa propre enquête sur ce cambriolage, et entendit parler d'un certain Gaedren Lam. Celui-ci avait au moins recelé une partie des commandes volées, la garde l'ayant légèrement amendé pour cela, mais sans plus. 

En se renseignant un peu plus, il apparut que Lam avait sans doute fait plus que profité du cambriolage, et en était probablement à l'origine. Il/elle fit appel à quelques faveurs parmi ses amis et sa famille pour utiliser quelque temps une forge et des matériaux, pour réaliser une armure et quelques armes, afin de faire payer directement le prix de son crime à Lam.
\subsection{Build}
Guerrier force, nain des forges, avec le background marchand ? ou peut-être artisan, si on change lore(guild).

Armé d'un maul (bec de corbin), et lourdement armuré(e), le tout fait main.
\subsection{Roleplay}
Un nain(ou une naine) très poli, respectueux, et peu aventurier dans l'âme. Au départ, ça reste un simple artisan, habitué à son confort, à ses habitudes, plus qu'au sang et à la violence des aventures.

Il apprécie peu la vulgarité, sauf quand il est vraiment énervé, comme par Lam : s'attaquer à lui passe encore, mais à ses affaires, à son ART ?! Jamais!

Un peu étrange pour un nain, avec un côté artiste assez prononcé, et des manières de citadin un peu bourgeois plus que d'aventurier.

Ses objectifs personnels seraient de reprendre, au moins partiellement, son activité et son art, au moins dans un premier temps.
\subsection{Évolutions potentielles}
En premier lieu, un certain enhardissement pourrait venir avec l'utilisation de ses armes et armures, plus que leur simple réalisation.

Dans les périodes moins orientées combat, il pourrait presque servir de semi-face, avec son côté bourgeois bien installé, plutôt que par l'intimidation comme beaucoup de personnages martiaux. Ses capacités de craft peuvent aussi être un atout.

Enfin, il va s'aguerrir avec le temps, ou se briser, suivant le degré de traumatisme de toute cette affaire.

Mécaniquement, on pourrait partir sur les archétypes suivants : 
\end{document}