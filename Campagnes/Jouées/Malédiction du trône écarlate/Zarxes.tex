\documentclass[10pt,a4paper]{book}
\usepackage[utf8]{inputenc}
\usepackage[french]{babel}
\usepackage[T1]{fontenc}
\usepackage{amsmath}
\usepackage{amsfonts}
\usepackage{amssymb}
\author{ Antoine Robin}
\title{Zarxes, guerrier tiefelin}
\begin{document}
\maketitle
\subsection{Concept}
Un jeune tiefelin, avec un certain don pour le combat, et une dent contre Lamm. Il est oscille entre idéaliste et plein d'espoir, ou cynique et réaliste. Il profite au combat de son apparence terrifiante pour gagner un avantage sur ses adversaires, et revient d'un premier travail de mercenaire dans le nord-est de la varisie.

\subsection{Background}
Zarxes a été trouvé un matin devant les portes du hall du racloir, encore un bébé, avec juste un pendentif portant son nom au coup. Il a passé une bonne partie de son enfance à y trimer, subissant par ailleurs les brimades de ses camarades pour son apparence étrange et menaçante.

Il s'en est enfui à 7 ans, et fut convaincu par Lamm de travailler pour lui à la place, afin de gagner son pain, et d'avoir un toit. Il se rendit vite compte qu'il avait échangé un enfer pour un autre, Lamm n'étant pas plus moral que les gardiens du hall, et les autres orphelins n'étant pas forcément plus sympathiques envers le jeune et chétif tiefelin. Par ailleurs, il n'était pas le pickpocket le plus doué de la bande, trop souvent maladroit. Mais tout cela l'a endurci : il a appris à subir des coups, et en grandissant, à en rendre.

Quand il pouvait, il s'échappait de la pêcherie pour aller à l'impasse du barde, où il appréciait de découvrir de nouveaux lieux par les récits qui en étaient faits par les voyageurs.

Cela continua quelques années, entre les problèmes avec la garde, les mauvais traitements de Lamm, et quelques amitiés parmi les orphelin, notamment un jeune aasimar, arrivé quelques années plus tard à la pêcherie, Naedris. Ils avaient le point commun de ne pas être complètement humains, et Zarxes a pris plus d'un coup pour protéger l'autre gamin.

Il a fini par s'enfuir à nouveau, profitant de la rencontre avec un nain, Tarvir Godrin, qui recherche des mercenaires dans Korvosa, que ce soit pour des compagnies organisées, des groupes d'aventuriers (notamment la société des éclaireurs), ou des compagnies commerciales par exemple. Zarxes a insisté lourdement auprès du vieux nain aigri pour que celui-ci l'engage sur un contrat, afin de quitter la ville. Après quelques semaines d'insistance, et sa patience érodée, Tarvir lui proposa un contrat, avec un groupe de mercenaires qui devaient aller protéger la petite ville de Trunau, au sud des terres de Belkzen, ainsi que la route commerciale entre la Varisie et ce qui reste de Dernier-Rempart. Il s'agissait sans doute de se débarrasser d'un gamin devenant franchement pénible, mais pour Zarxes, c'était l'opportunité de quitter Korvosa, et d'apprendre à se battre auprès de professionnels.

Il a donc passé les deux années suivantes entre les terres de cendres et Trunau, apprenant à se battre auprès de la compagnie de la bannière pourpre, un groupe de mercenaires relativement efficaces, à défaut d'avoir un sens aigu des affaires. Il a mis du temps à se faire à la vie de la compagnie, étant une forte tête sans grande confiance envers ses compagnons, mais il a fini par rentrer dans le rang, et a amélioré ses talents martiaux contre les pillards orcs. Il a apprit la discipline militaire, et la valeur du travail : suer lorsque c'est tranquille évite de saigner lorsque ça chie ! Il a aussi tatoué sur l'épaule le symbole de la compagnie, la fameuse bannière pourpre, portant un symbole de l'ancien empire Chelaxien.

Le contrat avec Trunau terminé, il a quitté la compagnie, au moins temporairement pour revenir à Korvosa, pour régler quelques affaires, notamment le cas de Lamm, et potentiellement les gardiens du hall du racloir également. Il a maintenant 17 ans, sait bien se battre, et souhaite éviter que d'autres orphelins se retrouvent dans son cas, pris entre les griffes du pire orphelinat de la ville, et les criminels comme Lamm.
\subsection{Build}
tiefelin humain (?), background ex-orphelin de Lam.

Guerrier arme à deux mains, partant sans doute sur un archétype d'hellknight armiger, et un build intimidation. Sinon, un archétype d'ensorceleur (lignage infernal ?) est une autre option, en jouant à l'épée bâtarde pour un build de Gith.
\subsection{Roleplay}
Un tiefelin encore assez jeune et idéaliste, en tout cas sur certains points : il rêve de monter dans la société, et espère pouvoir le faire. Si il est confronté à encore plus de discriminations alors qu'il a prouvé ses talents, il pourrait facilement être désabusé.

Il vient de revenir de Korvosa, mais connait bien la ville, ayant exploré la majeure partie de celle-ci depuis qu'il est haut comme trois pommes. Et s'il apprécie la vie de mercenaire, il est beaucoup plus à l'aise les pieds sur du pavé, des murs de pierre autour de lui.

Il est très confiant dans ses capacités, et pense que la loi de Korvosa, malgré ses problèmes, devrait être appliquée plus strictement. Sinon, il ne sert à rien d'essayer de l'améliorer.

Une fois le problème de Lamm résolu, il s'intéresse aux forces armées de la ville, ainsi qu'aux chevaliers infernaux. Il espère notamment avec eux en apprendre un peu plus sur ses origines, ou du moins apprendre à dominer cet aspect de lui-même.

Il reste assez méfiant des étrangers, à moins de ne pas avoir d'autres options. En tout cas, il partira du principe qu'on essaie de l'exploiter. Même s'il tendrait à perdre cette attitude après son temps comme mercenaire, elle reste bien ancrée.
\subsection{Evolutions possibles et objectifs personnels}
Pour le moment son objectif est plutôt simple : démolir Lamm, et possiblement le hall du racloir (il est moins sûr à ce sujet). Pour la suite, il envisage de l'ascension sociale via les chevaliers infernaux, espérant gagner le respect qu'il n'a jamais pu avoir. Il n'envisage pas l'échec, ce qui pourrait coûter cher à sa fierté. Enfin, si la bannière pourpre a besoin d'un coup de main en passant dans la région de Korvosa, il assistera ses anciens compagnons d'arme.

Comme indiqué dans la section build, il a plusieurs options d'évolution mécaniques, suivant comment se passe son approche des chevaliers infernaux.

Au niveau du roleplay, il y aura l'évolution de sa relation avec le groupe : il démarrera sans doute méfiant, mais pourrait s'ouvrir de différentes manières suivant la réception du groupe. 

En plus de cela, suivant la campagne, il y aura une évolution de ses objectifs personnels, en particulier sa dent contre certains orphelinats, qui pourrait évoluer de différentes façons.
\end{document}