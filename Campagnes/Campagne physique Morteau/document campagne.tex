\documentclass[10pt,a4paper]{book}
\usepackage[utf8]{inputenc}
\usepackage[T1]{fontenc}
\usepackage{amsmath}
\usepackage{amsfonts}
\usepackage{amssymb}
\author{ Antoine Robin}
\title{Campagne Morteau}
\begin{document}

\chapter{Arcs narratifs}
\section{Arcs primaires}
\subsection{Le dragon des montagnes céleste}
Au nord-est des hautes-terres, on trouve des montagnes célestes : celles-ci flottent pour une raison inconnue, au-dessus des falaises et de la mer. Elles sont depuis une dizaine d'année le territoire d'un dragon bleu, bien décidé à défendre son territoire contre les intrus. Les shungue l'associe depuis à la mort qui rôde, et essaient parfois de l'apaiser avec des offrandes, en particulier le groupe des Kotsoteka.
\paragraph{Boss} Le dragon bleu adulte (niveau 13)
\paragraph{Accroches} Les personnages croisent des kobolds servant le dragon. Ils devraient en déduire la présence de celui-ci dans la région. D'autres rencontres devraient ensuite les mener progressivement à son antre.
\paragraph{Rencontres}L'antre de la bête, en fait un donjon avec plusieurs rencontres de combats et d'exploration, culminant par un affrontement avec la bête elle-même. Une tentative d'enlèvement par des Kotsoteka, qui envisagent de donner ces prisonniers au dragon pour calmer sa violence. Une attaque du dragon sur le navire des PJs. Celui-ci vise le navire plutôt qu'eux, et se replie rapidement. Des drakes attaquent les PJs depuis la mer ou les cieux.
\subsection{Les traces d'une ancienne civilisation}
Les marais à l'ouest ont été conçus il y a bien longtemps, et sont maintenus par de puissants sortilèges, via des pierres gravées. Ces pierres font plus que modifier le climat, elles maintiennent d'anciens sceaux magiques visant à empêcher le retour d'une puissante liche.
\paragraph{Boss} Liche (niveau 12) ou dragon noir (niveau 11)
\paragraph{Accroches}
\paragraph{Rencontres} Morts-vivants s'en prenant aux défenses magiques de la région. Groupes de morts en patrouille ou en train de récupérer des ressources : captifs, ressources magiques, informations, créatures à réanimer.... Un donjon de 3 ou 5 rencontres pour trouver plus d'informations sur ce qui se passe. Un donjon massif de 7 rencontres pour affronter la cause de ces problèmes !
\subsection{Le culte du seigneur-démon}
Dans les montagnes et le désert au sud, on trouve de nombreuses tribus de Gnolls, et d'autres créatures qui chassent les hommes. Ces créatures sont dirigées par l'influence d'une matriarche lamia.
\paragraph{Boss}une matriarche lamia
\paragraph{Accroches}
\paragraph{Rencontres}
\section{Arcs secondaires}
\subsection{Une forêt primordiale}
Dans la jungle centrale, on peut trouver de très nombreuses créatures sauvages, y compris certaines qui ont partout ailleurs disparu depuis bien longtemps.

Celles-ci, si elles ne sont pas coordonnées ni fondamentalement maléfiques, peuvent être des menaces importantes pour un groupe qui s'aventurerait sous la canopée.

Plus qu'un vrai arc narratif, il s'agit plus des menaces importantes dont les personnages pourraient être mis au courant par des natifs ou des habitants, et qui pourraient faire l'objet d'une traque spécifique.
\paragraph{Boss}Deinosuchus (niveau 9), T-rex (niveau 10)
\paragraph{Accroches}Rencontre avec un chasseur, qui vient de réchapper des griffes d'une créature particulièrement dangereuse, lançant ainsi une mini-quête de chasse. Mise au défi par des locaux de traquer une telle créature pour prouver leur valeur.
\paragraph{Rencontres}  Rencontres de combat avec ces créatures, souvent dans leur domaine, leur antre. Une créature ancienne et connue des locaux aura sans doute un territoire de chasse bien maîtrisé.
\subsection{Habitants des profondeurs}
Les eaux de la région sont le foyer d'une tribu de sahuagins, qui lancent des raids récurrents contre les villages côtiers, Port-Anastasie étant rapidement ajouté à la liste des cibles potentielles. Ils sont évidemment plus adeptes au combat maritime, mais peuvent mener des raids un poil plus vers l'intérieur des terres.
\paragraph{Boss}Chef sahuagin
\paragraph{Accroches}Le navire de personnages, ou un village où ils passent est attaqué par un groupe de raid, qui s'empare de ressources et/ou de captifs.
\paragraph{Rencontres}Première rencontre face à un groupe de raid léger. Affrontement face à des patrouilles en mer. Assaut d'un avant-poste pour déterminer où se trouve leur base principale. Attaque de cette base principale, et affrontement contre leur chef.
\subsection{Fanatisme aveugle}
Une troisième expédition, et certaines suivantes, apporte de nouveaux arrivant prônant une vision de la religion plutôt extrême. Ils détestent rapidement les natifs pour leur mode de vie, et sont partisans de les affronter pour que la colonie survive.

Le gouverneur a des capacités limitées pour les gérer vu leur nombre et leur statut de citoyens de la colonie. Ce qui signifie que les PJs vont probablement devoir choisir un camp dans cette histoire.
\paragraph{Boss}Chef des nouveaux arrivants
\paragraph{Accroches}Un petit groupe d'entre eux se prépare à résoudre une différent avec les natifs par les armes. Le différent est un malentendu, mais cela ne les arrêtera pas : leur manière de vivre est supérieure à celle de ces sauvages.
\paragraph{Rencontres} Affrontements ou négociations pour limiter la violence, ou au contraire en faire partie. Cela va de petits groupes revanchards, à une foule en colère, en passant par un affrontement contre le chef de ce groupe religieux, qui est plutôt dangereux, et surtout, parfaitement extrême dans ses convictions. Beaucoup de rencontres sociales, potentiellement en triangle avec des PNJs plus 'normaux' : les PJs discutent avec quelqu'un d'autre et ils interviennent, ou les deux camps débattent et les PJs servent d'arbitre. 
\subsection{Du sang dans les bois}
Si les relations avec les Shungue s'enveniment, ils sont suffisamment près de la colonie pour lancer des attaques contre celle-ci : des raids contre le bétail, raser de fermes, ou potentiellement, s'en prendre à la colonie directement. Tenter de les poursuivre est aussi dangereux, vu leur capacité à utiliser le terrain à leur avantage.
\paragraph{Boss} Chef de groupe Shungue
\paragraph{Accroches} Des colons ont été tués par les natifs : ce sont des armes qui ont fait ça, sans aucun doute, et les émissaires ont été exécutés également. 
\paragraph{Rencontres} Des affrontements contre des patrouilles ou groupes de chasse se dirigeant vers les fermes et villages annexes à la colonie. Un raid contre Port-Anastatsie, les Pjs arrivant juste après ou juste avant. Des escarmouches violentes si les PJs essaient de poursuivre. Récupération d'informations sur ce qui se passe, découverte d'un rassemblement de chefs qui pourrait être critique. Affrontement avec un chef hostile, et négociation avec les autres.
\subsection{La sombre tapisserie}
L'influence de la sombre tapisserie est très forte dans la région : d'anciens autels parsèment la jungle, à demi-enfouis, des créatures étranges et improbables hantent les côtes, et les locaux peuvent admettre vénérer des esprits des eaux parfois cruels.

A mesure que les personnages découvrent des traces de cette influence, cette influence se rend compte de leur présence. Au point de finir par envoyer des chasseurs à ses trousses.

Dans un lac, des créatures étranges étudient les personnages et leurs actions
\paragraph{Boss}Un aboleth, puissant seigneur aquatique.
\paragraph{Accroches} Un autel à demi-enterré dans la jungle montre des traces de sanglants rituels à une déité inconnue. Les villageois ont abandonné un bébé près de la plage pour les esprits des eaux.
\paragraph{Rencontres} Groupe de skum en maraude, chiens de Tindalos en poursuite, aboleth dans un lac sombre, cultistes cherchant désespérément un sacrifice.

\subsection{Vol de bétail}
Dans la colonie, le bétail disparait peu à peu, pris par des créatures inconnues. Certains ont monté la garde, mais sans grand succès. Un ours-hibou a fait le coup, emportant moutons et chèvres, mais aussi un cheval au fil des semaines !

Son antre est un peu plus loin dans la jungle (2 hexagones), et il va falloir réussir à le pister pour aller le chasser. L'attendre est aussi une possibilité, mais il est rapide et très discret.

Sinon, il hante les forêts de la région à la recherche de concurrents et de proies.
\paragraph{Boss}Un ours-hibou
\paragraph{Accroches} Des fermiers se rassemblent pour demander de l'aide au gouverneur, ou des traces de sang peuvent être visibles par les PJs. Sinon, on peut voir un attroupement de paysans dans un pré non loin de Port-Anastasie.
\paragraph{Rencontres} L'ours-hibou, dans son antre, plus des rencontres d'explorations. Son antre est un mini-donjon avec potentiellement d'autres créatures, comme une meute de rats, ou des choses de ce type. 
\subsection{La chasse commence}
Près des montagnes au sud, les villageois ont peur : nombe d'entre eux sont tués par une créature féroce, qui laisse des cadavres dans son sillage.
\paragraph{Boss}un rakshasa
\paragraph{Accroches} Un cadavre est découvert dans la jungle, ou des villageois terrifiés viennent en parler.
\paragraph{Rencontres}Des serviteurs du rakshasa, envoyés pour tuer ces gêneurs, la créature elle-même, qui vit au village ! 
\section{Arcs tertiaires}
\subsection{Une mystérieuse disparition}
Une jeune femme a disparu de chez elle. Sa famille fait partie des premiers colons. Elle est parti avec un jeune homme des Jupe, qui la courtise depuis plusieurs semaines.
\paragraph{Accroches} Un père inquiet et furieux demande de l'aide aux PJs, ou la rumeur enfle rapidement dans les tavernes. Sinon, un local venu négocier peut donner l'information, et le père veut récupérer sa fille.
\paragraph{Rencontres} Affrontement avec le jeune homme et/ou son groupe suivant comment la situation se passe. Rencontres sociales pour convaincre l'un ou l'autre des camps.
\subsection{Devoir de mémoire}
Un cadavre de natif est retrouvé dans l'antre d'une créature quelconque. On peut distinguer des traces des marques qui indiquent à quel village il appartient, et à côté de lui, un objet étrange.
Si celui-ci est ramené, les villageois seraient ravis, et pourraient également faire leur deuil.
\subsection{Un naufrage inquiétant}

Un navire s'est échoué sur la côte, et son contenu pourrait être pillé. Toutefois, son ancien équipage, sans repos, et revenu à la vie, et défendra son navire comme de son vivant !
Mini-donjon de 3 rencontres, dont une avec l'ancien capitaine du navire, qui est secrètement responsable du sort de son équipage.
\subsection{Conflit de voisinage}
Deux familles de colons n'arrivent plus à se supporter : un conflit de voisinage qui commence à dégénérer de plus en plus en violence. Si les PJs pouvaient régler le problème, cela simplifierai a vie de tout le monde.

Un arc très social avec possiblement un affrontement non-léthal.
\subsection{Bâtir un temple}
Un jeune prêtre d'une divinité mineure veut faire bâtir un temple à celle-ci. Il a l'accord du gouverneur, mais aucune ressources à disposition.

Si les PJs l'aident dans son entreprise, il pourra leur fournir des soins et une assistance spirituelle. Cela peut consister à récupérer des matériaux, ou encore convaincre des colons de l'aider.
\subsection{Épreuve de chasse}
Un groupe de natifs souhaite mettre à l'épreuve les PJs afin de les accepter. L'épreuve consiste en premier lieu en une chasse contre une créature dangereuse, qu'il faudra traquer puis tuer et enfin, ramener au village. 

Une fois cela fait, une cérémonie devrai améliorer grandement l'opinion des natifs quand aux PJs.
\section{Arcs personnels}
\subsection{Pourchassé}
Un personnage porte un mystérieuse amulette, dont les pouvoirs vont s'éveiller au fur et à mesure de la campagne. 

Mais il n'est pas le seul à s'intéresser à cette amulette : des assassins arrivent bientôt pour le tuer et s'en emparer. AU bout d'un moment, leur commanditaire pourrait même se déplacer jusqu'à Port-Anastasie pour une confrontation finale.
\subsection{Les problèmes familiaux}
Un personnage est venu avec un frère ou une sœur, qui va avoir des problèmes : financiers, personnels, judiciaires.

Il va donc falloir le/la sortir de différents faux pas. La contrepartie sera de profiter de ses contacts plus ou moins réglo, que ce soit au sein de la colonie ou même des tribus locales.
\subsection{Intérêt local}
Un personnage semble fasciner certains natifs. Il apparait, si des questions sont posées aux bonnes personnes, qu'un individu correspondant à son apparence est mentionné dans une prophétie locale. Evidemment, les détails de cette prophétie sont à découvrir par ailleurs.
Elle pourrait correspondre à plein de chose, peut-être une future intégration de ces natifs avec la colonie?
\chapter{Les différentes zones}
\section{Les jungles des basses-terres}
Thèmes :ruines ensevelies, anciens secrets, magie sourde.... Chants d'oiseaux, bruit de pluie,  rugissements dans la brume.


Des jungles épaisses et peu peuplées, elles sont habitées de très nombreuses créatures, et des gardiens hantent ces bois, les protégeant contre les menaces.

Quelques villages subsistent, généralement en ne prenant qu'un strict minimum de ressources. Il s'agit souvent de pêcheurs-chasseurs-cueilleurs, formant de toutes petites communautés.

Certaines de leurs légendes mentionnent les anciens dieux, des entités lointaines, aux buts mystérieux. Ces entités sont liées à la sombre tapisserie, et pourraient prêter l'oreille à leur supplications de temps en temps.

Les animaux sont très variés : des dinosaures aux tigres, en passant par toutes sortes de créatures plus ou moins communes. En plus de cela, on peut croiser les quelques habitants, des leshys, des esprits des bois.

Les recoins sombres sont aussi infestés de skums, disparaissant dans leurs repaires humides en cas de menaces.

\subsection{Rencontres aléatoires}
Iruxi, kobolds, ours, ours-hibou, arboréens, leshy, léopard, tigres, guenaude verte, raptors et autres dinosaures, natifs(shungue, Iruyok), skum, hydre, feux follets, serpents, nixie, revenants, gobelins

sables mouvants, crues rapides, pluies tropicales
\begin{enumerate}
\item Une patrouille de kobolds chasse dans la région, et tombe sur les PJs. Ils appartiennent aux Pehkwi Tuhka, une tribu de kobold ayant récemment juré allégeance au dragon bleu. Suivant le niveau, la patrouille est plus ou moins importante
\item Un shungue réalise le test du chasseur : il doit traquer seul et revenir avec un trophée d'une puissante créature. Il se déplace vers le sud, mais peut discuter et partager son feu avec des étrangers, si ceux-ci le comprennent.
\item les personnage sont suivis par un gardien arboréen. Celui-ci est plus curieux qu'agressif, et cherche à obtenir des informations.
\item Suivant la saison, les personnages croisent un ours, ou sa tanière. Une mère avec des petits serait de loin le plus dangereux, se rencontrant au printemps. Cela peut sinon être un ours-hibou.
\item un léopard ou un tigre tends une embuscade aux personnages, essayant d'en tuer un pour s'en emparer et se barrer avec.
\item une meute de raptors s'en prends aux personnages : ils ont faim et chassent en meute.
\item Une guenaude verte s'emploie à tromper les aventuriers qui passent trop près de sa hutte. Elle cherche à les attirer sur le territoire d'une créature dangereuse, et à les attaquer au même moment.
\item une hydre a fait son antre dans la région, empoisonnant les environs et devenant progressivement une menace pour les créatures et habitants
\item Les personnages arrivent dans un village de natifs, quelques huttes sur pilotis abritant au total une cinquantaine de personnes. Ceux-ci sont plutôt curieux de l'arrivée d'étrangers.
\item Une marre abrite un grand serpent, qui s'en prend à ces proies nouvelles.
\item des feus follets hantent une vieille construction en pierre. Ils deviennent agressifs si on s'en approche. La vieille construction abrite 
\item une patrouille de skums, revenant vers leurs tanières sous-marines. Ils peuvent transporter des captifs ou des ressources. A moins qu'il ne s'agisse d'un éclaireur ou deux, isolés.
\item un ancien autel, à demi-enfoui dans la jungle, porte des symboles étranges, mais difficiles à regarder. Des offrandes récentes y ont été posées : quelques fruits et un bol contenant un peu de sang séché.
\item Un magnifique étang, soigneusement dissimulé par une communauté de leshys vivant là, crées au départ par des druides de la région.
\item Une source d'eau pure est gardée par une nixie. Celle-ci approche les PJs pour leur demander de l'aide, peut-être avec une hydre non loin, ou une autre créature problématique ?
\item des ruines ensevelies sont un ancien tombeau, renfermant toujours des gardiens : des revenants des cairn défendent l'endroit et ses trésors.

\end{enumerate}
En plus de cela, des rencontres mineures peuvent être mises en place à bas niveau avec des rats géants ou des milles-pattes.
\subsection{Lieux notables}
\begin{description}
\item[Camp des Pehkwi Tuhka] Le village principale de cette petite tribus de kobold, qui vénère maintenant le dragon bleu des montagnes célestes. le camp lui-même est à proximité de la côté, dans une petite baie. Ils sont hostiles envers les autres humanoïdes, dans leur interprétation de la volonté du dragon. celui-ci est représenté dans leur camp par des peintures de dragon un peu partout. En terme de combat, c'est un donjon de trois rencontres : patrouille extérieure, gardes du mur, renforts dans le camp, avec le chef, et le shaman et sa créature (à déterminer).

\item[L'antre de l'ours-hibou] Un ours-hibou vit dans la région, et son antre se trouve ici. Elle est assez grande, sous les racines d'un très grand arbre. La première salle contient des groupes de rats, attirés par l'odeur de la charogne, et la salle du fond peut contenir la bête elle-même.
\end{description}
\section{La côte carstianne}
La région tout entière est bordée par l'océan, qui est probablement aussi dangereux que les terres. Il recèle de nombreux habitants, pas tous accueillant envers les étrangers.

Ce sont des eaux chaudes tropicales, avec des coraux, des algues colorées, et une très grande variété de poissons et de créatures. En plus des nombreux animaux, des villages se trouvent un peu partout le long de la côte, et des sahuagins hantent les profondeurs.

La côte nord est particulièrement traitresse, avec des falaises abruptes et des rochers sous-marins dangereux. Le bassin au sud de port-Anastasie est beaucoup plus facile à explorer, avec de grandes plages, peu de pièges naturels. Enfin, au niveau de la couronne écarlate, c'est intermédiaire : des récifs de corail impressionnant peuvent se révéler traîtres, mais sans les courants violents du nord. Ses eaux abritent par contre une population de sahuagin, ainsi que des créatures dangereuses.


\subsection{Rencontres aléatoires}
crocodiles, dragon tortue, anguilles et murènes, guenaude marine, kobolds,  poulpe géant, pterosaures, reefclaw, sahuagin, requins, elasmosaures, crabes géants, orques et dauphins, drake marins, draugr, hippocampes, méduses, calamars, tritons, natifs (suivant la région).

tempêtes, courants, tourbillons
\begin{enumerate}
\item une patrouille de chasseurs sahuagins, montés sur des requins, attaque le navire ou le groupe.
\item des créatures vivent dans un récif que les PJs passent, et elles se sentent menacées. En particulier si les Pjs cherchent des ressources.
\item Un navire échoué recèle des créatures qui en ont fait une antre, ou des marins maudits, relevés d'entre les morts.
\item un groupe de sahuagin revient vers la mer, chargés de pillage et de captifs. Un de ceux-ci pourrait leur donner des informations sur la région.
\item un groupe de kobolds semble d'être naufragé, et recherche son chemin. ils sont très méfiants des PJs, mais peuvent ne pas attaquer suivant comment on s'y prend avec eux.
\item Une créature a été influencée par un objet dans un navire naufragé, et a changé de comportement : il peut garder cet objet, ou attaquer à vue les créatures des environs.
\item un village de pêcheurs, vide. Des traces de luttes sont visibles par endroits, mais aucun corps.
\item 
\end{enumerate}
\subsection{Lieux spécifiques}
\begin{description}
\item[Autel ancien] un petit temple simple, et en ruine. Les gravures sont difficiles à lire, plus difficiles à comprendre, mais parlent sans doute de sacrifices. Si les personnages restent trop longtemps, un groupe de natifs peut s'en prendre à eux pour les sacrifier aux divinités du lieu. Ou une créature s'en prend à eux?
\item[Avant-poste sahuagin] Un lieu secret et bien défendu, c'est un village dont les habitants lancent des raids sur toute la région. Il faudra sans doute trouver ses défenses extérieures pour déterminer son emplacement exact.
\item[]
\end{description}
\section{Les hautes-terres}
Les hautes-terres sont principalement des plaines herbeuses, traversées de petites rivières dont les berges sont souvent boisées.

La majeure partie des habitants sont des Shungue, qui traquent les nombreux animaux vivant dans les plaines.

La côte est très abrupte, avec de nombreuses falaises escarpées, et est particulièrement traîtresse à naviguer. Les natifs privilégient de petites embarcations à voiles et à rame, pour pouvoir traverser les zones les plus mauvaises. C'est au large de ces côtes qu'on trouve les montagnes célestes, habitat du dragon bleu.

En plus des villages et forts, quelques autres créatures vivent dans la région : des rocs et oiseaux-tonnerres vivent dans le nord-ouest notamment.

\section{Les montagnes Célestes}

\section{Les marais}
A l'ouest de Port-Anastasie, on trouve une grande zone marécageuse, ancien royaume inondé par magie. Si le plus grand lac de la région cache de noirs secrets, les marais aux alentours ne sont pas en reste, dissimulant d'anciens temples et bâtiments.

C'est ici le repaire d'une puissante liche, jusqu'ici maintenue scellée par de puissantes défenses magiques, mais celles-ci commencent à s'éroder.

De nombreuses créatures vivent là : gobelins, ogres, trolls, mais aussi une force croissante de morts-vivants de plus en plus
 organisés.
 \subsection{Rencontres}
 chauves-souris géantes
 stryges
\section{Les montagnes}
Plus au sud en longeant la côte, on trouve une chaîne de montagne relativement importante. Elle marque la limite nord des [culture principale 2], et abrite plusieurs villages et avant-postes. Les Aarai appellent la chaîne de montagnes la couronne écarlate.

Les animaux dans la région sont nombreux, vu le peu d'habitants, et souvent de bonne taille : trolls, ogres, oiseaux-tonnerres....

Une passe permet de traverser directement la chaîne de montagnes, en longeant la rive orientale du lac Aruyo. Sinon, les passages les plus faciles sont la mer, et l'endroit où les montagnes la rejoignent. Ces endroits sont aussi un peu moins dangereux que le reste des montagnes, des patrouilles passant régulièrement par là.


\subsection{rencontres}
au nord: gorilles
au sud:

\section{Les terres arides d'antakapari}
Les terres arides d'Antakapari sont une grande étendue de sable et de savanes au sud de la chaîne de la couronne écarlate, et s'étend sur des centaines de kilomètres dans chaque direction. 

Ces terres sont traversée par le fleuve Elus, lien majeur pour les tribus Aarai, en particulier, la clé du pouvoir royal de la tribus bleue.

Certaines sections de ces terres sont cependant contaminées par une forme de magie étrange, qui a modifié l'apparence des terres, leur donnant un aspect torturé.

Si les terres arides abritent de nombreuses créatures classiques dans ces climats (lions, hyènes ...), les désolations abritent des monstres changés par la magie des lieux : manticores, ettins....
\subsection{Rencontres}
Chimère, manticore
\chapter{Port-Anastasie, ses habitants, son évolution}
\section{Description des lieux}
\section{Habitants}


\chapter{Cultures actuelles et anciennes}
\section{Les shungue}
%inspiration : apaches, peut-être avec une pointe de maori sur certains thèmes.
Au nord, et principalement dans les hautes-terres, mais aussi dans la partie nord de la jungle, on trouve les Shungue, un peuple arrivé depuis peu dans la région après y avoir émigré du nord. 


\subsection{Organisation sociale}
Ils forment des groupes regroupés en plusieurs fédérations plus ou moins en conflit : les Yamparika(hautes terres), Jupe(dans la jungle),  Kotsoteka(présence légère sur des îles dans la région), Kiowa et Nokoni (les deux derniers sont plus au nord). Les différents groupes d'une même fédération peuvent avoir des conflits, mais ceux-ci sont le plus souvent réglés sans grande violence. En revanche, les raids entre fédérations sont beaucoup plus courants.

Un groupe est divisé en villages, le plus souvent formé d'une famille étendue. Les jeunes gens vont fréquemment se marier hors de leur village natal, voire hors de leur groupe, ce qui est plus rare, plus dangereux, mais la réussite de ce genre d'entreprise est source de prestige.

Les villages d'un même groupe s'entraident pour ériger et entretenir des forts de bois (Arwa), qui servent de protection en cas de conflit d'importance. Les villages en eux-même sont déplacés après quelques années dans un endroit, afin de laisser la zone se développer seule à nouveau, en attendant de peut-être y réimplanter un village.

Au sein d'un village, le chef est le plus souvent un chef de raid aguerri, ou un grand chasseur. Il est conseillé par un shaman, qui interprète la volonté des esprits. Ces shamans peuvent fréquemment servir de contre-pouvoirs, et sont le principal vecteur de discussions entre groupes, échangeant des nouvelles.

\subsection{Histoire, mythes et légendes}
Les shungue vénèrent de nombreux esprits, qui semblent correspondre  à des personnifications de certains plans : le trompeur est ainsi lié à l'enfer, la destructrice au abysses, le néant à l'Abaddon, le protecteur au paradis, le conseiller aux champs-Élysée....

Ils ont ainsi de nombreuses légendes sur ces esprits, leurs nombreuses apparences, et comment ils peuvent se manifester parmi les hommes, pour les aider, observer, ou s'en prendre à eux.

Selon eux, le monde est né d'un œuf primordial, dont est sorti un immense serpent, et dont les mortels habiteraient l'ancienne coquille. Le serpent lui-même est une représentation courante dans les motifs claniques, symbole du cycle de création et destruction.

\subsection{Culture}
La chasse, la cueillette et l'artisanat sont les trois activités principales des Shungue. Les jeunes gens sont fréquemment les chasseurs, et sont accompagnés de quelques chasseurs expérimentés pour leur apprendre les ficelles. En grandissant, ceux qui n'ont pas un grand talent pour la chasse partent cueillir ou réaliser les objets nécessaire au village, suivant les besoins. Les femmes enceintes ou ceux qui s'occupent des enfants sont la principale source de savoir sur l'artisanat.

Vêtements, décorations et tatouages sont généralement avec des motifs géométriques ou des représentation très stylisées du sujet. Les vêtements sont le plus souvent réalisés en cuir, décorés de plumes, d'épines.... 

Courtiser un membre d'un autre village, groupe ou fédération est une entreprise qui doit servir à montrer son talent : la court doit se faire en toute discrétion, sous peine de se faire rouer (plus ou moins gentiment) de coups par les parents de celui ou celle qui est approchée. L'intéressé(e) doit donc se faufiler, souvent à plusieurs reprises, vers un village qui n'est pas le sien, en évitant les groupes de chasse. Cela se fait fréquemment lors du printemps, avant la saison des raids. Si la personne recherchée est intéressée, elle finit par s'enfuir avec son soupirant pour rejoindre sa communauté.

Les héritages présents sont les suivant : humains, nains, halfelins, mais aussi quelques tengus, et des hommes-lézards près de la jungle.
\subsection{Réaction aux étrangers}
Hors d'un village, la réaction dépendra essentiellement du responsable de patrouille, et de ce qu'il peut penser de ces étrangers. Hors conflit ouvert, ou hostilité de la part de ceux-ci, les deux réactions les plus probables seront de se faire discret et d'observer ou évacuer, ou la curiosité, et une approche prudente.

Plus près d'un village, cela peut être plus dangereux si ils ne sont pas accompagnés. Auquel cas, les intentions des étrangers peuvent être mal comprises, surtout si c'est un étranger seul : on peut penser qu'il cherche à séduire quelqu'un. Il risque alors d'être roué de coups (douloureux, mais probablement pas mortel).

Sinon, des étrangers seront sans doute menés vers le chef ou le shaman, suivant qui a le plus d'influence dans le village. Les shamans ont une bonne connaissances de ce qui se passe dans la région, ce qui est un peu moins vrai pour les chefs.

\section{Au sud : les Aarai}
%inspiration : maasai, zulu, égypte ancienne (?)
\subsection{Organisation sociale}
Les aarai forment une hégémonie locale : au départ des éleveurs nomades, ils ont formé quelques cités qui peuvent imposer leur loi aux peuples les plus proches.

Leur reine contrôle les différentes tribus du territoire, dont la majeure partie sont encore très nomades : les villes sont surtout les lieux de rassemblement et d'échanges des tribus. La plupart sont relativement récentes, et servent également de garnisons pour l'entraînement des troupes, ainsi que de centre administratif.

Les tribus se différencient par leur couleur : les rouges, au nord sont ceux que les PJs verront le plus, avec les bleus, maison royale.

La société Aarai est essentiellement matriarcale : les femmes gèrent les troupeaux, les affaires, et dirigent les opérations militaires. Les hommes sont des chasseurs, des bergers ou des artisans essentiellement. Les femmes militaires sont moins nombreuses que les hommes, mais ce sont le plus souvent elles qui mènent la troupe, et servent d'unité de choc à la guerre.

Les relations avec les autres cultures sont souvent adverses : la plupart des villages du sud de la jungle par exemple, leur verse un tribut annuel de peur d'être conquis. Dans ces conflits, les captifs sont ramenés dans les villes, et travaillent pour les tribus à l'édification de celles-ci. Ils forment alors la caste la plus basse socialement : les esclaves. La plupart d'entre eux le reste jusqu'à un édit de pardon, que les souverains émettent assez souvent : dans ces cas, une proportion des captifs sont relâchés, choisis au hasard parmi leur population.
\subsection{Mythes et légendes}
\subsection{Culture}
Les deux genres pratiquent énormément de danses, tant pour des buts rituels que pour le loisir. Les danses rituelles sont souvent communes aux tribus, ou du moins assez proches, les danses de loisir sont par contre plus sujettes aux modes du moment.

Le vêtement le plus commun pour les civils est une sorte de grande toge ample, très confortable, teinte dans les couleurs de la tribu, plus ou moins intensément suivant leur richesse. Les militaires ont rarement plus qu'un pagne et une coiffe.

On trouve de nombreux héritages parmi les Aarai : humains, mais aussi orques, quelques gobelins, elfes et Khajit.


\subsection{Réaction aux étrangers}
La monarchie Aarai essaiera sans doute de satelliser des groupes étrangers, comme ils l'ont déjà fait dans la région : ils vont se renseigner sur la force en présence, avant de demander une reconnaissance de leur puissance, ou de réaliser une démonstration de force.

Plus directement avec les PJs, la plupart des Aarai seront intéressés pour négocier avec eux, et en savoir plus sur leurs voyages et explorations.

La monarchie pourrait prendre très directement deux actions en lien avec les PJs : tester leurs capacités, probablement par une attaque surprise, dont l'origine serait ensuite niée, et demander de guider une mission diplomatique, probablement par la mer, en direction de Port-Anastasie.

\section{Les anciens rois : Iruyok}
%inspiration : maya, aztèques
\subsection{Histoire}
Il y a cinq cent ans, le royaume des Iruyok dominait la région, depuis ses nombreuses cités-états, plus ou moins liées entre elles par un réseau d'alliances.

Celui-ci s'écroula pour des raisons encore mystérieuses il y a environ quatre cent ans. A ce moment, une puissance liche, Mictantecutli régnait sur la plupart des cités, et une alliance des autres et d'une force militaire de l'empire Laorai mis fin à ce royaume.

Les cités encore debout après le conflit ne purent se relever, et furent progressivement abandonnées, pour laisser place à des villages plus modestes.
\subsection{Restes}
De nombreuses communautés vivent dans la jungle avec cette culture. De leurs anciennes cités, il ne reste que des histoires de lieux hantés et d'horreurs sans nom. 

Aujourd'hui, beaucoup en appellent aux entités de la sombre tapisserie pour leur protection. Celles-ci étaient déjà invoquées du temps de la splendeur de leur royaume, des sacrifices étaient même organisés en leur nom. Aujourd'hui cela est plus secret : les sacrifices sont des abandons, ou des meurtres en pleine nuit, et les cérémonies sont plus rares, mais les divinités sont toujours mentionnées.

Avec le temps, leurs terres furent envahies de tout côtés : du nord les groupes shungue arrivèrent, au sud, les nomades remontèrent jusqu'aux montagnes et demandent aujourd'hui un tribut aux villages voisins. Et à l'ouest, l'empire Laorai s'est encore étendu là où était autrefois de nombreuses cités florissantes. La majeure partie de la population est en fait assimilée dans ces trois groupes, et seuls une minorité reste indépendante, dans le bassin du fleuve Iranok.
\section{Dans le lointain : l'empire Laorai}
%inspiration : chine période royaumes combattants
Il est potentiellement possible de croiser une expédition Laorai dans les jungles, en se dirigeant très à l'ouest. 

Si une telle expédition est croisée, elle sera sans doute à la recherche d'informations sur la région, pour s'assurer qu'aucune menace ne pèse sur leur frontière, ou encore pour envisager de piller ce que cachent certaines de ruines de la région.

Ces expéditions sont formées probablement de cavaliers mobiles, et d'un petit corps d'infanterie de métier.

Il y a également un monastère à la racine des montagnes, bien fortifié et protégé. Il abrite une bibliothèque d'une rare valeur, en particulier aussi loin de toute ville. Celle-ci se spécialise particulièrement dans l'histoire de ce continent. ce monastère est celui de Liang-Kai.

\chapter{Gérer l'histoire, le tempo}
\section{Donner des objectifs aux combats}
Quelques idées d'objectifs alternatifs, au-delà de la simple chasse de créature ou du meurtre simple de l'opposition.
\begin{enumerate}
\item Un captif est au main des adversaire, mais il peut mourir lors de l'intervention des PJs. Il peut être un allié, neutre, ou une autre forme d'adversaire. Il pourrait servir à donner des informations intéressantes sur les environs et/ou un arc narratif aux PJs.
\item Plutôt que le simple meurtre, il est peut-être possible, surtout pour un animal, de s'assurer qu'il quitte son territoire actuel, ou du moins ne menace plus les colons par exemple. Il a pu être déplacé par un évènement, craindre pour ses petits, ou manquer ses proies habituelles pour une raison ou une autre : blessure(s), ou fuite des proies.
\item un des camps essaie de mettre les voiles (littéralement?), et laisse donc une petite arrière-garde, qu'il faut vaincre rapidement pour envisager d'affronter le groupe principal.
\item Il faut neutraliser un petit groupe d'éclaireurs avant d'alerter le groupe principal. Une autre version pour forcer les PJs à agir un peu différemment.
\item Les personnages traquent une cible en mouvement, ou celle-ci traque les PJs lors de leur exploration.
\item L'adversaire cherche surtout à réaliser une démonstration de force pour dissuader les PJs de continuer plus loin. Très crédible avec une créature sauvage, qui cherche avant tout à défendre son territoire, ou des créatures conscientes qui ne souhaitent pas vraiment le combat.
\item 
\end{enumerate}

\end{document}