\documentclass[10pt,a4paper]{book}
\usepackage[utf8]{inputenc}
\usepackage{amsmath}
\usepackage{amsfonts}
\usepackage{amssymb}
\author{ Antoine Robin}
\title{Adran : terres des empires déchus}
\begin{document}
\maketitle
\tableofcontents
\chapter{Présentation générale}
\chapter{Arkhosia, l'empire des dragons}
Arkhosia était un ancien empire dirigé par des dragons, notamment le vénérable Arkhosis, un dragon rouge extrêmement puissant et ambitieux, dont le territoire s'étendait sur une majeure partie d'Adran. 
\section{Histoire}
Arkhosia est née des ambitions et des expériences du grand dragon Arkhosis. Dans son antre, il infusa d'énergie draconique des oeufs de différentes races,jusqu'à réussir à obtenir ce qu'il voulait : une nouvelle espèce stable pour lui servir de serviteurs. Mais sa mégalomanie ne s'arrêta pas à créer une nouvelle espèce, et bientôt, ses serviteurs partirent à la conquête du continent pour y soumettre les royaumes qu'ils rencontraient.

Il ne leur fallu pas longtemps pour imposer la loi du dragon sur une bonne part du continent, les troupes nouvellement vaincues étant rapidement ajoutée aux légions impériales.


\section{L'ancienne société d'Arkhosia}
Au sommet de l'empire étaient les dragons : Arkhosis lui-même, son clan, et les autres groupes de dragons qui le servaient. Venaient ensuite un grand nombre de serviteurs dévoués ou efficaces de l'empire, souvent des drakéides, mais aussi des membres d'autres races ayant fait preuve de leur talent.

Le coeur des armées arkhosiennes étaient toujours formé de ses drakéides, mené par certains des dragons qui s'étaient soumis. Autour de ces troupes de chocs s'organisaient les forces recrutées lors des conquêtes impériales.

L'administration impériale était elle souvent confiée à des membres des peuples vaincus, soigneusement surveillés par les agents des dragons et d'Arkhosis.
\section{Kar Adar, le royaume des esclaves rebelles}


\chapter{Bael Turath, flambeau de civilisation}
Entre le début du XIème siècle et la fin du XVIIème, l'empire de Bael Turath règnait sans partage sur le sud du continent, avant de s'effondrer aussi vite qu'inexplicablement.
\section{Histoire de Bael Turath}
Bael Turath est né de l'expansion rapide de la tribu de Turath, dans les monts d'Isther. Sous la direction de la dynastie des Méréides, ses anciens pillards et montagnards ont rapidement soumis les régions environnantes, avant de fonder un empire aussi puissant qu'Arkhosia en son temps.

L'empire s'agrandit rapidement en laissant les peuples vaincus conserver leur culture, parfois même leurs dirigeants. Pour beaucoup, la seule différence après la conquête était que leurs impôts allaient dans les coffres impériaux.

Au cours de son presque millénaire d'existence, l'empire connu plusieurs périodes de crises, généralement de successions, mais aussi des phases de rebellions internes. Malgré cela, l'empire se releva généralement rapidement de ces crises. La chute de l'empire fut donc aussi surprenante que brutale, même si ses détails restent flous encore aujourd'hui. Les érudits rivalisent encore aujourd'hui, plus de quatre siècles plus tard, sur les raisons et mécanismes de cette chute.

Quoiqu'il en soit, l'empire s'est effondré en quelques semaines : sans nouvelles de a dynastie impériale, l'administration s'écroula rapidement alors que de nombreux officiels et gradés s'emparèrent de territoires pour leur propre compte. En peu de temps, la paix impériale avait cédé la place à des conflits entre seigneurs de guerre locaux.
\section{Les terres de cendres}
Les conflits suivant la chute de l'empire ravagèrent l'ancien coeur de celui-ci : entre les mages de guerres, les armées en marche, et pour certains, une malédiction divine, cette région autrefois fertile est devenue un désert de cendres, habité de tribus nomades généralement hostiles aux étrangers. 

Les elfes de cendres et de nouvelles tribus turannes se battent dans les ruines de Bael Turath, escortant ou pillant les caravanes commerciales qui traversent l'ancien centre du monde civilisé.
\section{Les royaumes successeurs}
Lors de la chute, de nombreux militaires ou administrateurs civils profitèrent de leur position pour s'emparer de terres et de troupes locales et en devinrent les nouveaux maîtres. 

Ces nouveaux domaines sont depuis en conflit quasi-permanent, même si les plus puissants sont très stables depuis le temps. Certains espèrent clamer une légitimité impériale et cherchent à envahir leurs voisins, d'autres considèrent que la chute de l'empire était aussi normale qu'inévitable, certains enfin, se sont rebâtis sur les nations pré-impériales, et souhaitent conserver leur indépendance autant que possible. On trouve des états féodaux, des cités-états, mais aussi une république, des royaumes...
\chapter{TBD, l'aspirant}
\chapter{Espèces d'Adran}
\section{Drakéides}
\section{Tiefelins}
\section{Humains}
\section{Elfes}
\section{Nains}
\section{Les déformés}
Une siècle avant la chute de Bael Turath, ses mages développèrent de nouveaux rituels, destinés à modifier les chairs, à les sculpter pour répondre à des besoins spécifiques. Dans un premier temps, cela se répandit dans les arènes de tout l'empire, où des combattants professionnels firent face à des créatures hideusement déformées. 

Avec le temps, les usages se multiplièrent, et le talent des mages se fit plus précis. Aux premières monstruosités succédèrent des créatures plus stables, et plus spécialisées. Rapidement, cet usage se répandit à certaines des divisions militaires de l'empire, qui firent un grand usage de cette main d'oeuvre étrange. 

Les déformés 'originaux', qui furent soumis aux rituels directement, étaient ou sont toujours fréquemment des endettés, des serfs souhaitant quitter leur condition, ou encore des prisonniers de droit commun, suivant les provinces impériales ou les royaumes successeurs. Pour certains, il s'agissait d'un moyen de s'affranchir de dettes, en se vendant pour ces expériences, pour d'autres, se fut une contrainte, imposée par la loi.

Avec la chute de Bael Turath, certaines connaissances furent perdues, mais pas toutes, et nombre de royaumes successeurs continuent de faire grand usage de cette magie. Même ceux qui interdisent la création de nouveaux déformés acceptent généralement les individus déformés de naissance : nombre des déformations les plus stables sont transmises à la descendance, et avec le temps, des lignées distinctes ont même commencé à émerger. Certains haïssent leur condition de déformés, que ce soit pour leur apparence, leur identité sans histoire, ou les préjugés négatifs dont ils sont très souvent victimes. D'autres au contraire considèrent que c'est une forme de bénédiction, n'ayant pas d'histoire, ils ont tout à bâtir. D'autres enfin se considèrent comme des citoyens de leurs nations avant tout, et rejettent toute insinuation du contraire.

Mécaniquement, les déformés regroupent de nombreuses races : gobelins, orques, goliaths, hobgobelins....

\section{Les [caprins et minotaures]}
\section{Les Tabaxi}
\section{Les [aviaires]}
\section{Les hybrides}
Les hybrides entre différentes espèces existent parfois, mais ils sont plutôt rares, et ont la particularité, en tant qu'hybrides, d'être stériles. 
\chapter{Cultures d'Adran}
\section{Elfes des cendres}
\section{Turans}
\section{Drakéides}
\section{Adari}
\chapter{Religion(s) en Adran}
\section{Bael Turath et ses royaumes successeurs}
\section{Panthéon draconique}
\section{•}
\end{document}