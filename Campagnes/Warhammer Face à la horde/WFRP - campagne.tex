\documentclass[10pt,a4paper]{book}
\usepackage[utf8]{inputenc}
\usepackage[french]{babel}
\usepackage[T1]{fontenc}
\usepackage{amsmath}
\usepackage{amsfonts}
\usepackage{amssymb}
\author{ Antoine Robin}
\title{Warhammer - campagne}
\begin{document}
\chapter{Arcs narratifs}
\section{Arc principal : la bête d'Utingen}
\subsection{Synopsis}
\newcommand{\nomadversaire}{Kull}
Dans le sillage de la tempête du chaos, les personnages vont devoir sauver le middenland : dans la foulée d'Archaon, de nombreuses bandes de guerres et créatures en maraude ravagent toujours la région, et doivent être affrontées. En particulier, un chef de guerre du chaos, \nomadversaire, commence à rassembler de plus en plus de forces sous sa bannière maléfique, et vise la ville de Waldenheim en particulier.


L'acte 1 de cette campagne va faire découvrir aux PJs l'existence et l'ampleur de la menace devant eux : ils vont affronter des bandes en maraude, tant des hommes-bêtes que des guerriers du nord. Dans cet acte, les bandes de guerre du chaos sont relativement désorganisées, et \nomadversaire cherche à prouver sa puissance à celles-ci ainsi qu'aux dieux.

Au cours du second acte, les personnages vont devoir essayer de limiter la capacité des hommes-bêtes à se regrouper en une formidable harde : il va falloir les empêcher de mettre la main sur un ancien objet maudit, une relique des dieux sombres longtemps protégée par un clergé impérial, mais aussi limiter les effets de leurs raids autant que possible, tout en convainquant la société impériale du danger.

Lors du troisième acte, les forces du chaos se déchainent à nouveau, et les PJs sont au cœur des évènements : les hordes se jettent sur les murs impériaux pour les mettre à bas, tandis que les traitres tentent d'ouvrir les portes. Et il faudra bien affronter le maître de cette horde pour y mettre fin !
\subsection{Acte 1 : la horde se rassemble}
\subsection{Acte 2 : une course mortelle}
\subsection{Acte 3 : le siège de Waldenheim}
\section{Arcs secondaires}
\subsection{Les affres du crime}
Deux groupes criminels dirigent les affaires illégales à Waldenheim, et elles sont en conflit depuis longtemps : cela va de la rixe au vol de ressources, mais jusqu'ici, n'avait pas été trop violent. 

C'est un arc en trois parties:

Dans la première, les personnages sont engagés par un de ces deux groupes pour retrouver un de leur chargement qui a disparu vers les wastelands. 

Dans le second, un criminel récemment exécuté aurait avalé son dernier butin avant de mourir, et une sorte de course aux trésors se lance entre différents groupes.

Enfin, dans le troisième, le monde criminel est secoué de morts étranges. Suffisamment étrange en tout cas pour que les chasseurs de sorcières s'y intéressent. Les personnages pourront être engagés par l'une ou l'autre des factions.
\subsection{Le culte du chaos}
Un culte du chaos a infiltré Waldenheim, corrompant plusieurs citoyens relativement importants.

Dans un premier temps, les personnages peuvent être amenés à travailler pour ce culte, qui leur propose de travailler pour eux : il s'agit de s'assurer de la sécurité de membres du culte, puis une mission pour récupérer une sainte relique, qui est dans un village menacé par les groupes en maraude.

Plus tard, les personnages pourront assister à une tentative de meurtre ou d'enlèvement commis par des membres du culte, ce qui devrait les mettre sur leur piste.

Si les PJs commencent à remonter la piste, le magister prendra un risque : il fera proposer aux PJs un contrat, qui sera un piège pour s'en débarrasser, le contrat se déroulant dans un domaine un peu éloigné de la ville.
\subsection{Les familles von Hauptberg et Oberstein}

Deux familles importantes de la ville, qui sont régulièrement en conflit sur de nombreux sujets.

Les personnages pourraient sans doute être recrutés dans la maisonnée de l'une ou l'autre de ces familles, ce qui pourrait fournir de nombreuses occasions de découvrir les lieux et de travailler.

Une option est de faire rencontrer un membre d'une des maisons dans la première mission, qui pourrait leur proposer un travail.

Plus tard, le conflit entre les deux familles peut s'envenimer : duels de justice (truqués), affrontements plus ou moins violents, et une tentative d'assassinat (perpétrée par une autre faction ?). Bref, les personnages ne devraient pas trop manquer de travail à accomplir pour leur patron.

L'espionnage est une possibilité, avec la corruption d'un PJ pour obtenir des informations par un autre de leurs contacts.
\section{Arcs personnels}
\subsection{La vengeance : Tara}
Un personnage pourrait avoir une raison personnelle d'en vouloir à \nomadversaire : peut-être son village a été rasé par ses troupes, peut-être qu'un de ses proches en est un captif ?

Dans tous les cas, ce personnage sera très motivé par l'arc principal de la campagne, et des rappels de cela pourront être faits.
\subsection{Chasseur de sorciers : Dylan}
Le personnage de Dylan est suivi par un chasseur de sorcière, qui souhaiterai en savoir plus sur son passé et ses compétences.

Suivant comment la campagne évolue, cela peut signifier un conflit avec ce chasseur de sorciers, ou à l'inverse une proposition de travail pour cette organisation.
\subsection{La mort glorieuse : Théo}
\subsection{Aucune idée : JB}
\subsection{L'héritage familiale : nico ?}
La famille d'un personnage connait l'emplacement d'une relique dans la région, ou du moins, les indices pour y mener. La récupérer permettrait sans doute de bénéficier de l'objet, et pourrait être une ambition personnelle d'un PJ, surtout si celle-ci a été perdue récemment par exemple.

Cela peut aussi potentiellement être un territoire si le personnage est noble, ou une entreprise pour un personnage du peuple. Enfin, cela peut aussi être un titre par exemple de noblesse, pour une famille déchue de celui-ci.

Il faudrait alors laisser des options pour le personnage de récupérer ce bien ou titre d'une manière ou d'une autre.
\subsection{Le vieil ami}
Un personnage est arrivé dans la région à la recherche de quelqu'un : un viel ami, une connaissance, un membre de sa famille.... Toujours est-il que retrouver quelqu'un au milieu de la Drakwald, et pendant une période de guerre, ne sera pas de tout repos.
\section{Arcs tertiaires}
\subsection{Une menace sur le village}
Cet arc va ouvrir la campagne, en proposant une enquête/traque.

Le Bailli du village voit les personnages arriver, et souhaite leur proposer un travail : depuis quelques temps, les bois grouillent de créatures dangereuses, et il souhaiterai, si possible, que les aventuriers aillent vérifier ce dont il s'agit. En particulier, le meunier a affirmé que des créatures rôdent près de chez lui.

En effet, des hommes-bêtes cherchent des victimes à tuer dans la région, et le moulin, un des bâtiments les plus visibles de la région, attire facilement leur convoitise.

Cela peut se traduire par une première attaque par des ungors, qui voyaient cela comme une cible facile. Leurs traces peuvent ensuite être remontées vers une petite grotte voisine, ou une troupe un peu plus importante fait rôtir une bête quelconque. Si les traces ne peuvent être suivies, des rumeurs sur la 'vieille caverne', où des charbonniers auraient entendu des bruits étranges et vus des traces inquiétantes.
\subsection{Nuit sanglante (Aventure commerce)}
Peut arriver sur tout trajet des PJs sur une route.

Alors que les personnages sont forcés de s'abriter dans une petite auberge par une tempête, un groupe de cultistes de Tzeentch vient de s'en emparer, et prépare une cérémonie pour sacrifier ses anciens occupants. Des PJs peu suspicieux pourront sans mal être ajoutés à celle-ci, et satisfaire leur seigneur noir.
\subsection{ça a le goût de poulet}
Un riche marchand de Marienburg est récemment arrivé en ville, et cherche à manger des plats locaux, en particulier les viandes qu'il n'aurait jamais testé auparavant. Les aubergistes de Waldenheim sont ravis de lui faire goûter leurs spécialités, mais les citadins commencent à rapporter des disparitions, et beaucoup sont persuadés que ce marchand en est la cause : ce serait un cannibale !

Quand les personnages arrivent pour s'en débarrasser, payés par la populace, ils découvrent son corps, comme mangé de l'intérieur par quelque chose d'innommable. Pour beaucoup d'habitants, les PJs sont alors coupables des disparitions ! Les personnages vont devoir trouver rapidement la cause du problème avant d'être lynchés.
\chapter{Déroulement de la campagne}
Les personnages commencent la campagne à Utingen, un petit village de la Drakwald, à deux jours de marche de Waldenheim. 

La première mission va les mettre sur les traces d'une menace pour le village, qui sera ensuite effectivement attaqué par les forces de \nomadversaire. Arc tertiaire : Une menace sur le village. 

Au retour de cela, les personnages vont pouvoir entendre d'autres bruits inquiétants : des trompes et tambours de guerre résonnent dans les bois, auquel répondent des hurlement étranges et horribles : Utingen est menacé. Arc principal acte 1 : la horde se rassemble.


\chapter{Dramatis personae}
\section{Utingen}
\section{Waldenheim}

\chapter{Lieux importants}
\section{Waldenheim}
Waldenheim est une ville relativement fortifiée à la frontière entre le middenland, le nordland et les désolations de Marienburg. Elle appartient au Middenland.

Elle protège l'entrée de la grande route du nord dans la Drakwald.
\chapter{Profils utilisés}

\end{document}