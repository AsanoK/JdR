\documentclass[10pt,a4paper]{article}
\usepackage[utf8]{inputenc}
\usepackage[french]{babel}
\usepackage[T1]{fontenc}
\usepackage{amsmath}
\usepackage{amsfonts}
\usepackage{amssymb}
\usepackage{multicol}
\author{}
\newcommand{\paquetage}[6]{
\ttfamily
\begin{center}
\fbox{\large{\textbf{#1}}}
\normalsize

#2

\noindent\makebox[\linewidth]{\rule{0.8\columnwidth}{0.2pt}}
\textbf{Attributs \& Compétences:}
\begin{multicols}{2}

Base:\\
#3 \columnbreak 

Complet:\\
#4


\end{multicols}

\noindent\makebox[\linewidth]{\rule{0.8\columnwidth}{0.2pt}}
\textbf{Équipements}
\begin{multicols}{2}

Base:\\
#5 \columnbreak 

Complet:\\
#6

\end{multicols}

\end{center}
\rmfamily
}
\newcommand{\role}[4]{
\ttfamily
\begin{center}
\fbox{\large{\textbf{#1}}}\newline
\normalsize
#2
\noindent\makebox[\linewidth]{\rule{0.8\columnwidth}{0.2pt}}
\textbf{Attributs \& Compétences:}\newline
#3 
\noindent\makebox[\linewidth]{\rule{0.8\columnwidth}{0.2pt}}
\textbf{Équipements}\newline
#4
\end{center}
\rmfamily
}
\title{Equipement de l'imperium}
\begin{document}
\maketitle
\tableofcontents
\section{Nouveaux types d'unité}
\subsection{Armes}
%11/7
%mechanicus
\paquetage{Adepta sororitas}{Les soeurs de bataille, bras armé de l'ecclésiarchie et porteuses de Sa Volonté. Elles s'attaquent avec ferveur aux ennemis de l'empereur}{+1 combativité, +2 tir(au choix), +1 couverture, +2 connaissances techniques(culte impérial), +2 athlétisme, +2 discipline, +1 conviction, +1 tactique }{+2 tir(au choix), +1 tir(au choix), +1 tactique, +1 conviction, +1 discipline, +1 orientation}{armure énergétique légère(tag : défense 3+, outil(athlétisme)), pistolet bolter}{bolter}
\paquetage{Aeronautica Imperialis}{Les forces aériennes impériales, fournissant un appui aérien aux troupes de l'astra militarum. peut aussi représenter les chasseurs spatiaux de la flotte impériale.}{+1 combativité, +2 maintenance(aeronautica), +2 logistique, +2 discipline, +2 labeur ou pilotage(aeronautica) ou tir(armes de véhicule), +2 orientation, +1 observation, +1 couverture}{ +2 tir(arme de poing), +1 observation, +1 athlétisme ou pilotage(aeronautica) ou tir(armes de véhicule), +1 tir(arme de véhicule), +1 conviction, +1 interrogation}{chasseur ou bombardier, pistolet}{matériel d'entretien (tag : outil - maintenance (aeronautica))}
\paquetage{Mechanicus}{Les unités sous le commandement du clergé de Mars, disposant d'un accès au chant du dieu-machine(la noosphère). Se battent férocement pour les objectifs de l'omnimessie.}{+1 sang-froid, +2 tir(au choix), +1 tir(au choix), +1 discipline, +1 logistique, +2 couverture, +1 athlétisme, +1 orientation, +1 labeur, +1 mêlée}{+1 couverture, +1 logistique, +1 tir(au choix), +1 tir(au choix), +1 couverture, +1 mêlée, +1 tactique}{implants bioniques importants (Tag : défense 6+, outils(athlétisme ou mêlée ou observation)), accès à la noosphère, pistolet au radium}{fusil galvanique}
%aeronautica imperialis
%adepta sororitas
\subsection{Spécialité}
%4/3
%Cavalier/moto
%MBT
%Transport
%purification
\paquetage{Aviateurs}{L'unité pilote des aéronefs, et est constituée des pilotes et de leurs personnel de soutien : maintenance, administratif}{+1 pilotage(aeronautica), +1 tir(arme de véhicule), +1 orientation}{+1 tir(arme de véhicule) ou +1 observation, +1 tactique, +1 orientation}{Si l'unité n'en a pas déjà, un aéronef(véhicule avec le tag 'volant')}{carte de la zone d'opération(outil : orientation)}
\paquetage{Cavaliers}{Unité de cavalerie : chaque soldat dispose d'une monture. Généralement employé pour des tâches d'écrantage, de reconnaissance et d'exploitation. Peut également représenter une unité à moto ou à vélo dans certains cas.}{+2 pilotage(monture), +1 maintenance(monture), +1 survie}{+1 pilotage(monture), +1 maintenance(monture), +1 mêlée ou tir(au choix)}{Monture par personnage suivant l'unité}{arme de poing}
\paquetage{Char d'assaut}{ L'unité est aux commandes de puissants chars de combat, ayant pour mission de servir de fer de lance blindé aux troupes alliées. Ils disposent souvent d'un armement impressionnant en plus d'un excellent blindage.}{+1 pilotage(blindé), +2 maintenance(blindé) ou tir(appui ou artillerie), +1 logistique}{+1 maintenance(blindé), +1 pilotage(blindé), +1 tir(au choix)}{Un véhicule blindé par escouade, arme avec le tag combat rapproché}{boite à outils (outils : maintenance)}
\paquetage{Purification}{Cette unité apporte sa foi avec elle, et brûle les hérétiques de toute sorte par les flammes. Peut également représenter une unité spécialisée dans l'usage des armes à flammes ou des armes chimiques.}{+2 tir(lances-flammes), +1 couverture, +1 athlétisme}{+1 maintenance(lance-flammes), +1 mêlée, +1 tir(lances-flammes)}{arme lance-flammes}{arme de combat rapproché.}
\section{Nouvelles règles}

\subsection{Munitions}
Dans SPF, les munitions ne sont pas gérées de manière détaillée : on ne compte pas chaque grenade, chaque chargeur, chaque obus. A la place, certaines règles peuvent demander un test de munition. Il s'agira généralement d'un test de logistique, pour s'assurer que le personnage a fait attention à ses munitions, ou à un moyen d'en récupérer, mais suivant les circonstances, d'autres tests peuvent être demandés. 

Si ce test est demandé par le MJ sans venir d'une autre règle, le ou les PJs affectés gagne un PP.
\subsection{Tags}
3 nouveaux tags sont ajoutés, que ce soit pour les besoins des nouvelles règles ou des nouveaux équipements.
\begin{description}
\item[Impact]Quand un équipement avec impact est employé pour un test réussi lors d'un défi de compétence, le test compte pour deux réussites.
\item[Instable] à chaque utilisation de l'équipement pour un test, lancer un d6, sur 2+, rien ne se passe, sur 1, l'équipement surchauffe ou représente un danger pour son porteur. Dans ce cas, le porteur subit une blessure.
\item[Munitions abondantes]un personnage utilisant un équipement avec munitions abondantes dispose d'un avantage automatique sur les tests de munitions.
\item[Utilisation limitées]après chaque test réalisé avec cet équipement, le personnage doit réaliser un test de munitions.
\end{description}
\section{Armes}
De nombreuses armes sont utilisées par les forces impériales déployées sur la zone de guerre Tulea, en fonction de leurs unités respectives.
\begin{description}
\item[autocanon]arme d'appui polyvalente pouvant engager l'infanterie ou les véhicules légèrement blindés. Tags : portée 300/5500, perforante 2, AP, zone, barrage, déploiement strict 1.
\item[baïonnette]de nombreux modèles existent au sein de l'astra militarum, mais la baïonnette reste un de ses plus grands symboles. De nombreuses armes disposent d'emplacement pour l'accueillir. Tags : poing (couteau) ou épaule (sur monture), combat rapproché, précise (sur monture).
\item[Batterie multi-missiles]une batterie installée sous les ailes d'aeronefs impériaux afin de leur donner une grande capacité d'attaque au sol. Tags : véhicule, portée 500/1500, zone, perforante 2, encombrante.
\item[bolter ]arme d'infanterie sous la forme d'un fusil surdimensionné. Tire des projectiles autopropulsés explosifs capables de transformer ce qu'ils touchent en charpie. Arme de base de l'adeptus astartes et de l'adepta sororitas, cela reste une rare rarissime dans les rangs de l'astra militarum, à part chez certains officiers. Tags : épaule, portée 150/400, perforante 2, gênante.
\item[bolter lourd]arme d'appui, version surdimensionnée du bolter. Relativement commun comme arme secondaire des véhicules de l'astra militarum, il est également parfois utilisé par des équipes d'armes lourdes. Tags : soutien, portée 500/1200, automatique 500, encombrante, défense 6+, déploiement strict 1, barrage, perforant 3.
\item[canon laser]arme antichar très puissante, montée sur certains véhicules et parfois en arme d'appui au sol. Tags : appui, portée 500/1200, AP, perforante 4, encombrante, défense 6+, déploiement strict 1
\item[Carabine à fléchette]Une arme du mechanicus, tirant un déluge de fléchettes métalliques à haute vélocité. Employé par les unités de skystalkers. Tags : épaule, portée 200/400, perforante 1, gênante, automatique 200, barrage.
\item[Carabine laser] version raccourcie du fusil laser classique, souvent fournie à des unités blindées ou pensées pour le combat rapproché. Dispose d'un emplacement standard pour baïonnette. Tags : épaule, portée 75/200, munitions abondantes, automatique 75, combat rapproché.
\item[épée énergétique]armes de corps à corps aussi rares que puissantes, elles génèrent un champs énergétique qui détruit presque tout ce avec quoi il rentre en contact. Il permet de traverser les armures comme du beurre. Généralement un privilège d'officiers. Tags : poing, combat rapproché, précise, perforante 4.
\item[épée tronçonneuse]une arme de corps à corps relativement commune, en particulier chez les sous-officiers et certains officiers. Sa lame tronçonneuse transforme les chairs en charpie sanglante. Tags : poing, combat rapproché, précise, perforante 2.
\item[fuseur]une arme antichar redoutable faisant surchauffer sa cible, jusqu'à faire fondre la plupart des blindages, et la totalité des armures d'infanterie. Elle perd rapidement de son efficacité avec la distance, et requiert donc un grand courage de son utilisateur. Tags : épaule, encombrante, AP, portée 20/50, perforante (5 sous 20m, 2 entre 20 et 50m).
\item[fusil à plasma]comme toutes les armes à plasma, celle-ci peut se révéler dangereuse pour celui qui l'emploie. Il s'agit ici de la version fusil, parfois employé par des troupes d'élite de l'astra militarum, à qui l'on confie ce genre de reliques. Tags (tir normal) :poing, portée 120/300, perforante 3, gênante. Tags (tir surchargé) : poing, portée 35/75, perforante 3, gênante, AP, instable.
\item[fusil à pompe]parfois utilisé par l'astra militarum, mais plus fréquemment par les escouades d'abordage à bord des vaisseaux de la flotte. Redoutable à courte portée, il dispose en plus d'un port pour baïonnette standard. Tags : épaule, portée 75/150, automatique 25, munitions abondantes.
\item[fusil d'assaut ]certaines unités préfèrent utiliser des armes à munitions solides. Elles sont en particulier plus communes parmi les forces paramilitaires impériales, et se retrouvent donc aussi fréquemment aux mains d'éléments criminels ou rebelles.Certains disposent d'un emplacement standard pour baïonnette. Tags : portée 100/300, gênante, perforante 1, automatique 100.
\item[fusil de sniper] arme de précision utilisée par de nombreux sniper de l'astra militarum. Peut être employé avec diverses munitions spéciales, quand celles-ci sont disponibles. Tags :épaule, portée 400/800, gênante, précise, déploiement 1, perforante 2
\item[Fusil laser]arme standard de l'astra militarum, basée sur des cellules énergétiques standard, qui peuvent être rechargées n'importe où, dans un feu dans le pire des cas. Une arme robuste et fiable, même si peu puissante. Peu être surchargé. Dispose d'un emplacement standard pour baïonnette Tags(tir normal) : épaule, portée 100/300, munitions abondantes, automatique 75, gênante. Tags (tir surchargé) : épaule, portée 150/300, perforante 1, gênante.
\item[grenade frag]représente de nombreux modèles de grenades utilisés par l'astra militarum, ayant pour points communs d'être lancés à la main, et de disposer d'une grande zone d'effet. Tags : lancer, portée 20/30, zone, combat rapproché.
\item[grenade krak ]grenade antichar standard de l'astra militarum, elle est sensé permettre à un fantassin de s'attaquer à un véhicule blindé ennemi, ou à certaines créatures. La faible zone d'effet, et la portée limitée rendent l'exercice très périlleux. Tags : lancer, portée 15/25, AP, perforante 3, combat rapproché, utilisations limitées.
\item[Griffes bioniques]Un implant utilisé par les unités de Pteraxii du mechanicus. Tags : poing, précise, perforante 1, combat rapproché.
\item[lance-flammes]une arme spéciale relativement commune, fonctionnant au promethium, qui a la particularité de coller et de brûler dans presque toutes les conditions, y compris sous l'eau. Elle est favorisée par les unités anticipant des combats urbains ou de tranchées. Tags : lance-flammes, portée 50, gênante, zone, incendiaire, perforant 1, utilisations limitées.
\item[Lance-flammes lourd] arme d'appui assez rare en raison de sa portée courte et de son poids important, qui limite son emploi par l'infanterie. Tags : lance-flammes, portée 70, gênante, zone, incendiaire, perforant 2, utilisations limitées, déploiement 1.
\item[lance-grenade]arme d'appui couramment utilisée par l'astra militarum, utilise le plus souvent des grenades  fragmentations, mais les grenades krak sont également compatibles. Tags(fragmentation) : portée 375/500, zone; Tags(krak) : portée 300/400, AP, perforante 3, utilisations limitées, imprécise.
\item[lance-missile]Arme d'appui polyvalente, permet à l'infanterie de la garde d'affronter des groupes importants avec des munitions à fragmentation, ou des véhicules blindés avec les projectiles antichars. Parfois surnommés baises-tanks. Tags(frag) :soutien, utilisations limitées, portée 250, perforante 1, déploiement 1, encombrante, zone. Tags(antichar) : soutien, utilisations limitées, portée(200), AP, perforante (3), encombrante, déploiement 1.
\item[Lance explosive]Lance employée par la cavalerie de l'astra militarum, avec un tête explosive pour permettre de s'attaquer à des adversaires bien protégés. Tags(tête explosive):poing, AP, perforante 3, précise, impact (en charge), combat rapproché. Tags(normal) : poing, perforante 1, précise, impact(en charge), combat rapproché.
\item[longlas ] fusil de précision basé sur le fusil laser standard. Il est alimenté avec des cellules d'énergie 'pleine bourre' devant être rechargées après chaque tir, de même le canon peut nécessiter un changement régulier. Tags : épaule, portée 400/1000, précise, déploiement 1, gênante, perforante 2
\item[mitrailleuse ]arme d'appui, peu répandue dans les rangs de l'astra militarum, mais qui est déployée par de nombreuses autres organisations : milices, FDP ou encore groupes criminels. Elle fournit une très bonne cadence de feu, même si elle manque de puissance d'arrêt contre de nombreuses cibles. Tags :soutien, portée 400/1000, automatique 400, encombrante, défense 6+, déploiement 1, barrage, perforante 2.
\item[Missile hellstrike]Missile déployé sur des Valkyrie pour mener des opérations d'attaque au sol. Tags : véhicule, portée 800/2500, munitions limitées, perforante 4, AP, encombrante.
\item[mortier léger ]une arme d'appui très commune parmi les équipes d'armes lourdes de l'astra militarum. Elle permet de fournir un appui d'artillerie légère aux troupes voisines, généralement très rapidement. Tags : artillerie, indirect, zone, portée (800), imprécise, encombrante, déploiement strict 1
\item[mortier lourd]une version plus lourde du mortier. Elle est beaucoup moins fréquente, à part dans des unités de sièges, et certaines unités d'artillerie. artillerie, indirect, zone, portée(3000), imprécise, encombrante, déploiement strict 2
\item[multilaser]arme d'appui laser automatique, le multi-laser est le plus souvent monté sur des véhicules, parfois également sur des trépieds comme arme d'infanterie. Tags : soutien, portée 500/1200, automatique 500, encombrante, défense 6+, déploiement strict 1, barrage, perforante 1.
\item[pelle de tranchée]la fameuse 9-70 est une pelle généralement pliable, servant à réaliser de nombreux travaux de terrassements. Ses bords bien affutés en font également une bonne arme de corps à corps. Tags : poing, combat rapproché, outil (labeur pour les tests liés au terrassement).
\item[pistolet/revolver]certains officiers ou sous-officiers de la garde emploient des armes de poing à projectile solide de tous types. Au sein de la croisade Calidane, les plus communs sont des pistolets Tulean Syr-12, assez lourds. Tags (moyens) : poing, portée 30/75, munition abondante, combat rapproché. Tags(modèle lourd) : poing, portée 25/50, perforante 1, combat rapproché.
\item[pistolet bolter]version pistolet du bolter, il est parfois utilisé par des officiers, notamment par les commissaires, qui sont nombreux à essayer de se procurer une arme aussi dangereuse. Tags : poing, portée 30/75, perforante 2, combat rapproché, automatique 30.
\item[pistolet laser] pistoler standard de l'astra militarum, comme son cousin le fusil il est avant tout fiable. Tags : poing, portée 30/75, munitions abondantes, combat rapproché
\item[pistolet plasma]antique arme de poing, à la technologie mal maîtrisée. Dispose de deux modes de tir : normal ou surcharge. La surcharge fait prendre des risques à celui qui l'emploie. Tags (tir normal) :poing, portée 25/60, perforante 3, combat rapproché. Tags (tir surchargé) : poing, portée 35/75, perforante 3, combat rapproché, AP, instable.
\end{description}

\section{Armures}
\begin{description}
\item[bouclier d'assaut]utilisé par les troupes d'abordage de la marine impériale et par certains régiments focalisés sur le combat rapproché. Il permet d'augmenter la protection sans compromettre l'utilisation du fusil. Tag : défense 5+.
\item[armure carapace ]armure généralement réservée à des troupes d'élite, elle fournit une bien meilleur protection que la flak. Tags :défense 4+.
\item[armure flak]armure standard de l'astra militarum. Permet à un fantassin d'espérer survivre quelques temps, et se produit en masse dans de très nombreuses usines de l'Imperium. Tags : défense 5+.
\item[armure flak légère]armure utilisée par l'infanterie légère et la plupart des unités blindés. Elle est moins lourde et encombrante que la flak, permettant une meilleur mobilité. Tags : défense 6+.
\end{description}

\section{Véhicules et montures}
Véhicules et montures utilisent les règles de véhicule.
\begin{description}
\item[basilisk ]pièce d'artillerie autotractée, équipée d'un canon trembleterre et d'un bolter lourd de coque. Tags : vitesse : 30/10, équipage (pilote, mécano), transport(6, à découvert), fiabilité 3+, défense 4+, blindé.
\item[cheval ]monture parfois utilisée par certains régiments de l'astra militarum, ainsi que certains officiers. Tags : vitesse 20/10, fiabilité 3+, défense 6+, équipage(cavalier), transport(1).
\item[chimère ]véhicule de transport le plus commun de l'astra militarum. Dispose d'une arme de tourelle (souvent un multilaser), un bolter lourd de coque et plusieurs meurtrières pour les fantassins. Tags : vitesse : 80/30, équipage (pilote, chef de char, artilleur, mécano), transport (12), blindé, défense 4+, vision 6, fiabilité 2+.
\item[leman russ ]char de combat principal de l'astra militarum. Existe avec de nombreuses versions d'armes et de tourelles. Tags : vitesse 60/30, blindé, défense 3+ (5+ dans l'écoutille ouverte), vision 5, fiabilité 2+.
\item[marauder]bombardier lourd de l'aeronautica imperialis, dispose de nombreuses armes défensives ainsi que d'une forte soute de bombes dans sa version classique. Tags : 2000, défense 3+, blindé, équipage (pilote, navigateur), fiabilité 2+, volant.
\item[Mukaali ] monture utilisée par certains régiments de Tallarn, en particulier pour des patrouilles de longue durée. Tags : vitesse 15/15, fiabilité 2+, défense 6+, équipage (cavalier), transport(2), outil(survie).
\item[sentinelle de reco ]marcheur de reconnaissance léger et mobile. Souvent équipé d'un multi-laser, mais parfois d'autres armes d'appui (autocanon, canon laser). Tags : vitesse 40/40, défense 4+, équipage (pilote), fiabilité 3+.
\item[sentinelle lourde]version blindée de la sentinelle, employée souvent pour des opérations de reconnaissance en force. Tags : vitesse 25/25, défense 4+, blindé, équipage (pilote), fiabilité 3+.
\item[thunderbolt]chasseur atmosphérique lourd pouvant être déployé pour des missions de supériorité aérienne ou d'attaque au sol suivant l'équipement embarqué. Tags : vitesse 2500, défense 4+, blindé, équipage (pilote), fiabilité 2+, volant.
\item[Valkyrie]aéronef d'assaut et de soutien rapproché utilisé par de nombreuses unités de l'astra militarum. Elle est bien armée, et peut transporter une escouade jusqu'à la zone de guerre, ou les déployer par grav-chutes. Tags : vitesse 300, volant, fiabilité 2+, blindé (tant que les portes sont fermées), défense 4+, équipage (pilote, copilote), transport (10).


\end{description}

\section{Équipements divers}
\begin{description}
\item[Auspex]scanner de proximité existant en plusieurs versions. Permet de détecter une présence ennemie à proximité. Également utilisé par les unités de sapeurs. Tags : outil (perception), utilisations limitées(test d'opération : auspex).
\item[barbelés] des kilomètres de barbelés sont souvent déployés autour des positions défensives de l'astra militarum, pour gêner les mouvements de l'infanterie. Tag : piège, munitions abondantes.
\item[charge de démolition] charge explosive, souvent employée par les unités de sapeurs pour détruire les positions fortifiées ou opérer du terrassement rapide. Tags : piège, zone, perforante 3, AP, encombrante, utilisations limitées. 
\item[Chausse-trappe mordien] piège basique fait à partir d'une douille de bolt. Très facile à produire, et pouvant causer des dommages importants. Tags : piège, perforant 2, utilisations limitées, munitions abondantes.
\item[grav-chutes]système antigravité sous forme de harnais de saut, permettant de déployer de l'infanterie depuis une haute altitude. Matériel standard des unités aéroportées de l'astra militarum. Tags : outils (opération : matériel de saut)
\item[Masque à gaz/respirateur] de nombreuses variantes sont utilisées dans l'imperium autant pour lutter contre les attaques au gaz que pour les opérations courantes sur certains champs de bataille pollués ou hostiles. Tags : défense 2+ (contre attaques aux gaz et polluant aériens).
\item[medikit] pas vraiment standardisé, mais relativement commun au sein de l'astra militarum. Permet d'appliquer les premiers soins sur le champs de bataille. Tags : outil(premiers soins).
\item[mine antichar] mine pouvant endommager ou détruire un véhicule adverse. Souvent utilisé sur les routes d'accès évidentes. Tags : piège, perforante 3, AP, utilisations limitées.
\item[mine antipersonnel] mine déployée en masse sur les champs de bataille pour limiter la mobilité de l'infanterie adverse. Tags : piège, zone, perforante 1, munitions abondantes.
\item[Radio] très communément employée par la plupart des régiments avec un niveau technologique décent, la radio permet la transmission des ordres et informations. Tags : outil(tactique).
\item[Stims] produit permettant à un garde de disposer d'un boost temporaire d'énergie. L'utilisation régulière est proscrite par le Medicae. Permet de réaliser une seconde action lors d'un tour. Tags : instable, utilisations limitées, instable.
\item[Système de visée]Modificateur à une arme, ses tags s'appliquent à une arme du personnage (pas d'un véhicule. Tags : précise, déploiement 1.
\end{description}
\end{document}