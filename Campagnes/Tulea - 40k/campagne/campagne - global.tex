\documentclass[10pt,a4paper]{book}
\usepackage[utf8]{inputenc}
\usepackage[french]{babel}
\usepackage[T1]{fontenc}
\usepackage{amsmath}
\usepackage{amsfonts}
\usepackage{amssymb}
\author{ Antoine Robin}
\title{Campagne de Tulea}
\begin{document}
\section*{Ce document}
Ce document a pour but de structurer les éléments de la campagne de jdr par discord 'croisade calidane', et en particulier ce qui concerne la campagne de Tulea. Il s'agit de préparer les évènements globaux, afin que les joueurs des différentes unités puissent en profiter.

Il devra être accessible et modifiable par d'éventuels autres MJs qui souhaiteraient développer ce concept par la suite, ainsi que servir de référence pour la suite : une archive des évènements, ainsi qu'une structure globale réutilisable.

Les différentes campagnes joueurs ne seront pas dans ce document, mais dans leurs sous-dossiers respectifs, tout comme les éléments très spécifiques liés à leurs unités.
\chapter{La situation}
\section{Ruche Solus et Idenis}
Les deux ruches les plus proches, elles ont vécus des situations très différentes pourtant, partiellement à cause de leurs différences culturelles.

Solus a été un des foyers de la rébellion, est est majoritairement sous contrôle des insurgés. Ils contrôlent environ 65\% du volume total de la ruche, sans compter la sous-ruche. Au début de la campagne, les forces impériales contrôlent la forteresse Draconis, bastion inquisitorial dans le système, ainsi que le PC Arbites Alpha, immédiatement adjacent. Ces deux installations sont régulièrement coupées du reste des forces impériales par les troupes hérétiques, mais les défenseurs ont pu être appuyés par plusieurs bataillon du 12ème régiment de la maison Chiron. A part cela, le domaine des maisons Chiron et Palaxas sont relativement sécurisés, et pourraient servir de tête de pont pour une contre-offensive impériale. Le général Nathaniel Chiron a établi son QG dans le fort Lazarus, bâti sur le mur extérieur de la ville; et disposant de moyens de communications planétaires.

En face les forces hérétiques dans la ruche sont divisées en de nombreux groupes, souvent sans leader clairs. Cela ne doit pas diminuer leur dangerosité, tous ces groupes travaillant tout de même pour le magister Kycius. Deux groupes plus importants, mais a priori rivaux sont à noter : la maison renégate Clomara, ancienne maison mineure, et la maison renégate Arenis. Les premiers ont corrompu leurs forces armées, qui forment aujourd'hui l'élite des forces insurgées, mieux équipées et formées que les hordes sanguinaires de cultistes. Les seconds eux ont poussé leurs ouvriers à la sédition, et forment la puissance majoritaire, beaucoup des groupes de cultistes suivant leurs plans. Les forces des Clomara mènent les efforts pour diriger des troupes contre la ruche Idenis, tandis que les Arenis, moins bien organisés, essaient de réduire les positions impériales dans la capitale. Les insurgés ont beaucoup appris en deux mois de combats violents, et ont pu souvent s'équiper un peu mieux. Les troupes Tuléannes ont appris à se méfier des embuscades, des faux messages radios, et des tentatives d'infiltrations depuis toutes les directions. Dans tous les cas, ils manquent relativement de matériel lourd, en particulier de pièces d'artillerie. Si les Clomara semblent vénérer Slaanesh, les Arenis ne semblent pas avoir d'allégeance particulière.

A Solus, les forces impériales ont pu sécuriser de très nombreux réfugiés dans les premières semaines de guerre, mais cela pourrait se retourner contre elles : certains insurgés se font passer pour ces réfugiés, et surtout, ils imposent un stress important sur un ravitaillement limité. De nombreux ont été conscrits pour l'effort de guerre, principalement aux travaux de terrassement et déblaiement.

Idenis a réussi elle à repousser les insurgés, en petit nombre et mal organisés. En effet, ses puissantes milices ouvrières (en particulier minières) sont restées fidèles à l'Empereur, et ont réussi à s'organiser rapidement pour briser l'élan initial. Cela fait de la ruche le principal bastion impérial sur la planète, parfaitement opérationnel pour recevoir un flux de renforts, et fournir l'effort de guerre en matériel. Par ailleurs, plusieurs unités de milice sont en train d'être reformées en régiments de campagne par le général Chiron, avec la bénédiction de l'administratum. Elle n'est toutefois pas totalement sécurisée, quelques poches insurgées existant encore, majoritairement dans la sous-ruche. La menace est ici à l'extérieur de la ruche, avec des troupes ennemies menées depuis Solus contre les forces impériales. 
\section{Ruche Aventus}
Aventus est tombée dès la première nuit aux mains des forces du chaos, étant, d'après les informations, le coeur principal de la rébellion. Ses marais pollués risquent de former un obstacle majeur à toute avancée impériale. Il semblerait que le magister de l'ennemi soit à Aventus, ce qui pourrait expliquer sa chute rapide.

Les relevés orbitaux vont rapidement montrer que la ruche est plus active que jamais : le magister sait que si il souhaite la victoire finale, il va devoir surclasser les troupes impériales qu'il sait arriver, et n'a donc pas chômé pour se préparer. Les hordes de cultistes ont été organisées, des défenses ont été bâties, et surtout, une partie des installations industrielles de la ruche ont été reconverties pour bâtir des armes. De nombreuses unités hérétiques ont commencé à bouger de la ruche vers les autres, en particulier Austris et Solus.

Le magister en particulier voue son culte à Khorne : les troupes d'Aventus risquent de faire preuve d'une violence rare lorsqu'elles engageront les forces impériales.
\section{Ruche Austris}
A Austris, les forces impériales ont recommencé à gagner du terrain, sous la direction du maréchal Kowle : ses troupes personnelles, ainsi que plusieurs régiments des fdp sont engagées dans de violents combats de rue dans toute la ruche, ayant notamment réussi à sécuriser les docks ainsi que le front de mer. Des groupes de combats improvisés impliquant les civils toujours fidèles à l'imperium sont très communs, et malgré de lourdes pertes, progressent. Les combats se finissent fréquemment au corps à corps, où l'expertise de certains corps de métier, en particulier ceux liés aux industrie marine, se révèle très utile. 

Ici les forces impériales manquent cruellement de matériel, notamment de matériel lourd : blindés, artillerie.... 

Un nombre important de combat a également lieu sous l'eau pour le contrôle des points clés des voies d'eaux, qui pourraient permettre des infiltrations jusqu'au coeur des positions des deux factions.

Dans l'état actuel des choses, les forces impériales ont le dessus, et devraient le garder tant que des renforts n'arrivent pas d'Aventus.
\section{Ruche Nessus}
%artillerie créé ici
Nessus a fini par tomber après plusieurs semaines de combat. L'astroport et le fort adjacent ont notamment tenu pendant 6 semaines de combats, avant que leurs défenses ne soient envoyées en enfer par le feu des canons rebelles. Cela a permit l'évacuation par les airs de dizaines de milliers de ruchards vers Idenis et Austris. Evidemment il y a les nantis, mais aussi de nombreux ouvriers loyalistes qui ont pu rejoindre les positions impériales. Pour le moment, les forces impériales ne savent pas trop quoi en faire, mais ils disposent souvent soit de connaissances en fonderie, soit d'une grande motivation à se battre pour l'Imperium.

Dans tous les cas, la reconquête de Nessus ne sera pas un cadeau à reprendre pour les forces impériales : les renégats se sont aguerris au combat urbain, et, grâce aux usines encore actives, disposent d'un atout majeur : des pièces d'artillerie lourde. Le fort Garro notamment est tombé après trois jours de barrage par des pièces de très gros calibre. Les scans orbitaux ne révèleront pas la position de ces pièces, qui pourraient apparaître plus tard dans la campagne aux mains des renégats.
\chapter{Plan impérial}
Les FDP et forces loyalistes au sol essaient pour le moment de consolider leurs positions : Idenis doit être protégée, les défenses impériales à Solus renforcées, et si possible, Austris purgée avant l'arrivée de renforts ennemis.

Artemenes va sans doute tenter d'intercepter au moins une partie des renforts d'Aventus, probablement la colonne se dirigeant vers Austris, ce qui pourrait permettre de sécuriser la ruche et de récupérer un bon approvisionnement en nourriture.

En plus de cela, la majeure partie de ses forces seront sans doute engagées dans de violents combat sur le double front de Solus et Idenis : Idenis peut servir de point de relai avec l'orbite, et Solus reste une cible stratégique majeure pour sa capacité industrielle titanesque.
\chapter{Plan hérétique}
Le magister a un contrôle relativement souple sur ses troupes, mais il a un plan. En premier lieu, il compte faire tomber la ruche Solus, et transformer la capitale en enfer sur terre pour les forces impériales. Il cherche également à dépêcher des troupes pour faire tomber la forteresse Draconis, et s'emparer de ce qu'elle peut receler.

Les mouvements vers Austris et Idenis ne sont pour l'instant que des diversions pour occuper et garder les forces impériales sur la défensive. Pendant ce temps, il fait bâtir de nouvelles machines de guerres démoniaques à Nessus et Aventus, profitant de connaissances hérétiques qu'il a pu acquérir auprès des techno-hérétiques de Baroda.

Mais surtout, son principal espoir est l'arrivée prochaine d'un groupe de renforts envoyés par le Praedicator : un groupe de combat de ses fer-liés est en route, ainsi qu'un petit nombre de space marines du chaos. Toutefois, cela le presse également à être en bonne position, car il ne peut pas se permettre de paraître faible, où il risque d'être trahis par ses alliés. Ce groupe doit arriver un mois après les forces impériales, et sera constitué d'un croiseur de classe Slaughter, dont la mobilité devrait lui permettre de traverser un blocus impérial.
\chapter{Déroulement des évènements}
\section{Phase d'ouverture : débarquement orbital}
\subsection{front de Solus}
Déploiement de troupes impériales:
\begin{itemize}
\item 7ème necromundan
\item 3ème régiments de chasseurs de Trask
\item 2 régiments de troupes de choc cadiennes
\item 2 régiments d'infanterie Tuléanne
\item Soeurs de bataille de l'ordre de la lance grise
\end{itemize}

Leur objectif est d'arriver en renfort pour soutenir les forces loyalistes et tenir leur tête de pont pour reconquête ultérieure.
Ce front devrait être acharné, avec beaucoup de combat urbain violent. Par ailleurs, des défenses antiaériennes ont été remises en marche par les rebelles, et sont une menace constante pour le renforcement et le ravitaillement des forces impériales. Le transporteur \emph{Crusaders' hope}, transportant 3 régiments d'infanterie, sera lourdement endommagé lors de sa descente, avec de lourdes pertes pour les forces à l'intérieur. Les forces du méchanicus subiront aussi des dégâts importants lors de leur descente. 

Il faudra rapidement pour les forces impériales, réussir à neutraliser ces défenses de manière à augmenter les efforts pouvant être déployés ici : frappe orbitale localisée, soutien aérien, redéploiement de réserves, ravitaillement....
\subsection{Front Idenis}
Sur ce front, deux types de déploiement : les troupes chargées de sécuriser les abords de la ruche, et celles déployées en tant que troupes de réserves pour la suite de la campagne, qui auront une priorité basse pour les opérations de débarquement.

Ce front devrait être à l'avantage des forces impériales, même si des raids violents pourront se faire sentir. L'enlisement partiel est a priori inévitable.

En premier lieu, le groupe de combat chargé de sécuriser la voie vers Solus:
\begin{itemize}
\item 1er régiment de reconnaissance lourde de Calanth
\item 2 régiments de creuseurs de Roanne
\item 27ème régiment d'infanterie légère de Tallarn
\item 2 régiments de chasseurs de Trask
\item 14ème d'artillerie Morachéenne
\end{itemize}

Les troupes de réserve:
\begin{itemize}
\item 2 bataillons de kasrkins
\item 2nd régiment blindé vespasien
\item 12ème Royal de volpone 
\item 5 régiments d'assaut Zhannites (brigade du seigneur général)
\item 2 régiments de skitarii
\item 2 régiments d'infanterie morachéenne
\end{itemize}
Ces troupes servent de réserve stratégique et seront déployés sur le ou les fronts qui nécessiteront de nouvelles troupes pour faire pencher la balance.
\subsection{Front Austris}
L'effort principal sera d'intercepter les renforts ennemis en route vers la ruche, même si une partie des forces impériales ira dans la ruche elle-même. Opérations basées sur la ville secondaire d'Arala. Ravitaillement à prévoir pour ces troupes hors des villes majeures.

Les forces impériales ici auront un avantage certain : il leur suffit de ralentir les forces ennemies, qui risquent fort de tomber à court de ravitaillement au milieu du désert.
\begin{itemize}
\item 140ème régiment léger de Tallarn
\item 2 régiments de faucons harakoni
\item 2 régiments de creuseurs de  Roanne
\item 1er régiment blindé vespasien
\end{itemize}
\subsection{Front Nessus}
Les forces impériales vont se limiter à quelques opérations de reconnaissance, renseignement, et peut-être sabotage dans la région, afin de limiter les capacités ennemies à agir librement. aucun régiment ne devrait être déployé à plein temps.

Front totalement aux mains des forces hérétiques, les troupes impériales opérant ici devant agir dans les pires conditions.
\subsection{Front Aventus}
Lors de la fin de la phase d'ouverture, des troupes impériales doivent débarquer sur le plateau d'Oureti pour commencer à engager l'ennemi à Aventus. Le principal risque sera l'approvisionnement, la sécurisation du spatioport sera donc une priorité. Le déploiement de troupes disposant de capacité de combat en milieu pollué est souhaitable.

Ce front sera compliqué dès la phase d'ouverture : les forces ennemies sont nombreuses et bien retranchées, en particulier dans la ruche elle-même où des cultistes fous pourront profiter de chaque recoin pour lancer des embuscades face aux troupes impériales. Les troupes de krieg vont être comme à la maison. L'environnement en lui-même est un problème également : les étendues polluées vont se révéler un problème pour toutes les unités engagées sur ce front, hérétiques ou loyalistes.
\begin{itemize}
\item 2 régiments de troupes de choc cadiennes
\item 2nd régiment de grenadiers Atecan
\item 2 régiments de ligne de Krieg
\item première cohorte de skitarii, second clade d'Agripinaa.
\end{itemize}
\subsection{Front orbital}
Le groupe de combat impérial devrait avoir le contrôle total de l'espace au cours de cette phase, disposant des seuls vaisseaux du système, hors quelques vaisseaux civils mineurs.

Le contre-amiral Fenk va déployer les transporteurs lourds pour débarquer leur cargaison à Idenis, ainsi que sur le plateau d'Oureti et dans le désert près d'Austris. Le \emph{Pride of Saint Jus} va être déployé en orbite haute pour couvrir les vecteurs d'arrivée, et le \emph{Litany of Glory } va fournir un soutien direct aux troupes lors de leur débarquement par le feu de ses batteries navales, et le déploiement d'une vague de chasseurs et de bombardiers.

Dans un premier temps, tout soutien sera impossible au-dessus de Solus, en raison de ses puissantes défenses orbitales.
\section{Continuation}


\chapter{Dramatis Personae}
\section{Forces Tuléannes}
\subsection{Gouverneure Merity Chiron}
Gouverneure planétaire, évacuée de Solus à Idenis.
\subsection{Général Nathaniel Chrion}
Commandant Loyaliste à Solus et Idenis
\subsection{Maréchal Kwole}
Commandant loyaliste à Austris
\section{Forces impériales}
\subsection{Seigneur général militant Artemenes}
Commandant du groupe impérial de renforts
\subsection{Contre-amiral Fenk}
Vétéran de la flotte gothique et commandant de l'escadre impériale : 2 croiseurs et plusieurs escadrons de destroyers.
\section{Insurgés}
\subsection{magister Kycius}
Leader des forces rebelles et instigateur de la nuit de sang.
\end{document}