\documentclass[10pt,a4paper]{book}
\usepackage[utf8]{inputenc}
\usepackage[french]{babel}
\usepackage[T1]{fontenc}
\usepackage{amsmath}
\usepackage{amsfonts}
\usepackage{amssymb}
\author{ Antoine Robin}
\title{Campagne sororitas}
\begin{document}
\chapter{Ordre de la lance d'argent}
Une mission militante menée par la palatine Mirya. Ses objectifs sont doubles : en premier lieu soutenir la croisade dans ses opérations contre les ennemis de l'humanité, et dans un second temps, sécuriser deux lieux importants pour le couvent sur la planète : la basilique de Saint Icarus, dans la ruche Solus (et les reliques qu'elle devait contenir/protéger), et le couvent local de la lance d'argent, également dans la ruche Solus, dont l'état actuel est inconnu.

Mirya a fait pression pour que les soeurs soient déployées dans la première vague des forces impériales auprès de la ruche Solus, afin de pouvoir mener à bien au plus vite ses objectifs sacrés.

\chapter{Front de Solus}
La force de l'adepta doit donc être déployée par deux moyens: l'avant-garde sera en drop pods, tirés par le \emph{Vigilant Eyes}, le vaisseau inquisitorial, le reste du contingent arrivant en Thunderhawks. 

En particulier, les dominions devraient faire partie de la vague en drop pods, qui ne devraient pas trop subir le feu des défenses automatisées, mais qui vont arriver dans les lignes d'affrontement. Les thunderhawks pour leur part vont subir de lourdes pertes par les défenses automatisées, qui vont ajouter un autre objectif de cibles à neutraliser.

La première mission sera de survivre à l'arrivée et de préparer l'arrivée des thunderhawks, avant de devoir se diriger vers les défenses automatisées.
\end{document}