\documentclass[10pt,a4paper]{book}
\usepackage[utf8]{inputenc}
\usepackage[french]{babel}
\usepackage[T1]{fontenc}
\usepackage{amsmath}
\usepackage{amsfonts}
\usepackage{amssymb}
\author{ Antoine Robin}
\title{Campagne Meewan}
\begin{document}
\maketitle
\chapter{Personnage(s)}
\chapter{Unité : Primo-cohorte, manipule epsilon, 2nd clade de Agripinaa}
Une unité agressive de l'adeptus mechanicus, déployée par les maîtres du monde-forge Agripinaa pour soutenir les efforts de la croisade contre les hereteks. Le redéploiement du contingent n'est pas tout à fait un mal, le technoprêtre dominus à la tête des 3 cohortes ayant validé ce choix de la part du commandement de la croisade : cela pourrait permettre de récupérer certains des secrets industriels des Tuléans. Les deux objectifs principaux du dominus sont d'explorer une crypte de données à Aventus et si possible de détruire toute capacité de production d'artillerie à Nessus. Si le plan d'attaquer Aventus a été validé, pour le moment, le commandement impérial n'a pas déployé de forces contre Nessus.

Les cohortes 2 et 3 sont ainsi conservées en réserve à Idenis pour de futures opérations, tandis que la première a été affectée, suite à un lobbying important, au front d'Aventus

La première cohorte est la formation légère du clade, avec de nombreuses unités de pteraxii et de dragons, et un fort parti de rangers.
\chapter{Front d'aventus}
\section{Prise du spatioport}
La cohorte va devoir se déployer depuis le plateau d'oureti, afin de menacer le spatioport de la ruche Aventus, dont la capture devrait permettre de déployer des renforts plus importants. Cette étape va impliquer deux difficultés : un milieu extrêmement pollué en premier lieu, avec des déchets toxiques et/ou radioactifs, et des forces adverses bien ancrées, prêtes à tendre de nombreuses embuscades aux troupes impériales. Il s'agira principalement de mutants de types variés, avec quelques troupes de cultistes plus nombreuses mais profitant moins bien du terrain.

L'opération va donc suivre le schéma suivant:
\begin{enumerate}
\item débarquement orbital par navettes rapides, sous le feu, ou en largage pour les unités de pteraxii. Ces derniers vont devoir sécuriser des points clés.
\item progression dans le décors de déchets radioactifs, au milieu des embuscades ennemies. Ici les pteraxii vont avoir un rôle double de réserve et de reconnaissance.
\item Bataille du spatioport, avec de violents affrontements face à des troupes hérétiques beaucoup plus nombreuses, et disposant de quelques bons équipements. Il faudra briser le périmètre extérieur, puis nettoyer l'intérieur du complexe.
\end{enumerate}

Si vraiment rien ne se passe comme il faut pour Meewan, il est possible que les forces impériales n'arrivent pas à prendre le spatioport, et doivent alors s'enterrer dans les terres polluées pour attendre un changement de la dynamique de la bataille, sans doute avec des renforts. Cela peut vouloir dire d'autres opérations pour d'autres PJs, ou une tentative de pari par le magos dominus par exemple, pour tenter de changer la situation.

Il me faudra créer:
\begin{itemize}
\item une carte de la région, et notamment des étendues polluées
\item une carte du spatioport afin d'illustrer le problème principal
\end{itemize}


\end{document}