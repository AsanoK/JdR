\documentclass[10pt,a4paper]{book}
\usepackage[utf8]{inputenc}
\usepackage[french]{babel}
\usepackage[T1]{fontenc}
\usepackage{amsmath}
\usepackage{amsfonts}
\usepackage{amssymb}
\author{ Antoine Robin}
\title{Campagne vespasienne}
\begin{document}
\maketitle
\chapter{Première brigade blindée vespasienne}
La brigade dispose de nombreux types de véhicules, divisés en 4 bataillons.

Les deux premiers forment la ligne de bataille principale, avec des compagnies de Leman Russ (et ses nombreuses variantes.

Le troisième est formé de l'infanterie d'appui, avec plusieurs chars lance-flammes en soutien.

Enfin, le quatrième est le bataillon d'appui, fournissant des unités d'artillerie motorisée, de reconnaissance et de maintenance. Un escadron antichar dédié peut également s'y trouver afin de couvrir les unités du régiment ou des forces impériales à proximité.
\chapter{Front d'Austris}
Le front d'Austris se basera sur la ville secondaire d'Arala, entre les ruches Aventus et Austris. C'est là que se déploient les forces impériales de ce front, sous le commandement du général Barka des creuseurs de Roanne. La ville doit être tenue pour empêcher les forces hérétiques de progresser vers Austris, où les forces loyalistes repoussent progressivement l'ennemi.

En particulier, la première BBV doit servir de poing blindé pour cette force d'intervention : ils doivent être l'arme pour rejeter l'ennemi dans le désert, grâce à leur mélange de résistance et de mobilité. 

Pour leur troupe antichar, il y a aura pas mal de préparation et de repos, avant l'avancée adverse majeure, que les Tallarn devraient repérer et engager. La première phase de l'offensive ennemie devra être encaissée, avant que les combats ne se dispersent dans la région. Si tous ne se passe pas trop mal pour les tallarn et cette unité, le front ne devrait pas être trop difficile à sécuriser, malgré quelques mauvaises surprises : des tanks marcheurs et d'autres abominations mécaniques déployées par l'ennemi dans le désert.
\end{document}