\documentclass[10pt,a4paper]{book}
\usepackage[utf8]{inputenc}
\usepackage[french]{babel}
\usepackage[T1]{fontenc}
\usepackage{amsmath}
\usepackage{amsfonts}
\usepackage{amssymb}
\author{ Antoine Robin}
\title{Campagne Ceta}
\begin{document}
\maketitle
\chapter{Personnage(s)}
\chapter{Le 140ème de Tallarn}
Le 140ème des raiders de Tallarn est un régiment léger de l'astra militarum. 

Il dispose de deux bataillons d'infanterie légère, d'un bataillon de reconnaissance et d'un bataillon de cavalerie.

Les bataillons d'infanterie forment la majorité des effectifs et de la puissance de feu, disposant de nombreuses armes d'appui et de très bons tireurs d'élite.

Le bataillon de reconnaissance est le plus polyvalent, disposant des rares véhicules du régiment, en particulier quelques transports ainsi que des sentinelles. Il dispose également d'une compagnie montée, et de nombreuses escouades de spécialistes.

Enfin, le bataillon de cavalerie est la force de frappe mobile, chargée d'infliger le coup de grâce à l'ennemi. Il dispose pour cela de nombreux cavaliers ainsi que d'infanterie montée ou mécanisée.
\chapter{Front d'Austris}
\section{Déploiement}
Le régiment va être déployé en écran dans la vallée de l'ionet, avec pour mission de repérer les avancées adverses vers la ruche Austris. En cas de problèmes, les autres régiments du corps d'armée (en particulier les blindés vespasiens et les 2 régiments de faucons harakoni) devraient être en mesure de réagir rapidement pour stopper l'ennemi. La dernière ligne de défense de ce front consiste en deux régiments de creuseurs de Roanne, un défendant la base d'opération d'Arala, et fortifiant les routes probables, le second se dirigeant vers Austris afin de soutenir les forces loyalistes sur place.


Le bataillon de cavalerie en particulier va avoir comme mission de servir d'écran à l'écran : patrouilles aux marges et en avancée du dispositif, ainsi que du harcèlement sur les unités adverses. 
Après quelques jours/semaines plutôt calmes, la colonne adverse va avancer dans le désert en direction de l'unité. Dans ce cas, les missions vont devenir des opérations de harcèlement pour ralentir et pénaliser au maximum les forces adverses, avec des raids long portée contre le ravitaillement ennemi par exemple. Au bout d'un moment, il sera peut-être aussi possible de mener une opération contre un officier supérieur de l'ennemi, ce qui porterait sans doute un coup fatal à l'avancée hérétique dans le secteur, et une probable victoire sur le front d'Austris, et un redéploiement vers Aventus selon toute probabilité.
\chapter{Dramatis personae}
\section{140ème de Tallarn}
\subsection{Sergent Ibn Faris}
Un type plutôt sympathique, notamment avec son escouade : le mot pour rire, une figure un peu paternelle, le gars pas chiant.
\section{Lieutenant Homa Khaledi}
Lieutenant du peloton de harcèlement, relativement jeune, mais déjà très efficace. Elle est généralement très professionnelle, mais ne supporte pas qu'on utilise mal son unité.
\end{document}