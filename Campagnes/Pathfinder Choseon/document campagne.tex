\documentclass[10pt,a4paper]{book}
\usepackage[utf8]{inputenc}
\usepackage[french]{babel}
\usepackage[T1]{fontenc}
\usepackage{amsmath}
\usepackage{amsfonts}
\usepackage{amssymb}
\author{ Antoine Robin}

\title{Garde Royale : guide de campagne}
\begin{document}
\maketitle
\tableofcontents
\chapter{Arcs narratifs}
\section{Arc primaire : querelle de succession}
L'arc principal de la campagne se fera en trois actes, tous orienté en fonction de l'état de santé du souverain, et des problèmes importants de successions qui s'ensuivent.

Le premier acte est un acte de présentation du setting, de découverte. Les personnages croisent les différentes factions politiques et ont l'occasion d'en apprendre un peu plus sur celles-ci.

L'acte deux est le moment où l'action démarre vraiment : le roi révèle avoir une grave maladie, et les questions de successions se posent de manière pressante : toutes les factions commencent à placer leurs pions et jouer leurs cartes les plus subtiles.

L'acte trois démarre à la mort du roi, probablement de sa maladie. Cela déclenche les pires troubles, avec une vraie menace de guerre civile.

La première faction importante est la plus traditionaliste, et soutient le premier prince héritier dans son accession au trône. Cette faction est essentiellement constitué autour du prince lui-même et d'un noyau de hauts fonctionnaires.

La seconde faction correspond aux réformateurs, menés par le second prince, et qui aurait sans doute la préférence du roi. La faction se constitue autour du prince cadet, ainsi que plusieurs fonctionnaires montant, qui ont un contrôle effectif de certains bureaux importants.

La troisième faction est celle du général Kim Mae-Jun, qui remporte une victoire importante contre les Hans, et est universellement reconnu comme un officier très compétent. Il voit la princesse comme sa clé vers le trône. Autour de lui, quelques officiers, et un certain soutien populaire dû à ses récentes victoires.

La quatrième faction est celle de la reine : celle-ci réalise la précarité de sa position et de celles de ses enfants, en particulier si un de leurs demi-frères monte sur le trône. Elle essaiera donc de s'en emparer pour eux. Pour se faire, elle profite de la richesse et de l'influence de sa famille, qui a depuis longtemps des connections dans la noblesse du royaume.

La cinquième faction, mais qui ne vise pas franchement le trône est celle d'une révolte paysanne dans les provinces du nord. Ils espèrent profiter du chaos de la succession pour faire sécession efficacement. Ils sont soutenus par certains administrateurs venus du nord, venus d'anciennes familles aristocratiques de la région.
\subsection{Acte 1 : le puissant royaume de Choseon}
Les intrigues dans cet acte sont relativement peu nombreuses, étant plus de l'ordre de l'accroche scénaristique avec la présentation de certaines factions.

Avant de lancer la cérémonie, les différentes factions suivantes doivent être au moins vaguement présentées :
\begin{itemize}
\item Le premier ministre qui soutient l'héritage par l'aîné
\item Le prince cadet, qui s'entend beaucoup mieux que son frère avec le roi
\item Des nouvelles du général Kim Mae-Jun et de ses victoires dans l'est.
\item La présence du troisième 
\end{itemize}

Cet acte amène à une cérémonie officielle importante, au cours de laquelle démarre l'acte 2. Cette cérémonie doit être foreshadowed au cours de l'acte.
\subsection{Acte 2 : la maladie du roi}
Cet acte démarre vraiment la course à la couronne : au cours de la cérémonie qui clôture l'acte 1, le roi fait un malaise. Rien de grave, dit le palais, mais la situation précaire n'échappe à personne. 

Différentes factions vont donc essayer de se positionner le mieux possible pour s'emparer de la couronne. La première à agir sera celle du prince héritier, qui rentrera à la capitale pour s'enquérir de la santé de son père. A ce stade, les autres factions amassent de l'influence et battent le rappel des troupes.
\subsection{Acte 3 : une succession douloureuse}
\section{Arcs secondaires}
\subsection{Des disparitions mystérieuses}
Dans les bas-fonds de Daegu, des disparitions inquiétantes sont notées. D'abord très discrètes, car visant essentiellement des esclaves et le bas peuple, leur nombre finit par être important, et les gardes de la ville s'inquiètent des corps qu'ils trouvent dans les caniveaux, au point d'hésiter à quitter leurs postes de gardes.

Il s'agit d'un groupe de vampires 'sauvages'(vrykolakas, ou spawns de vampires normaux) : l'ancien est malin et prudent, s'attaquant à des victimes isolées, et disparaissant rapidement. Ses deux rejetons par contre, sont complètement sauvages et tuent de manière beaucoup plus fréquente, sans prendre leurs précautions dans le choix des victimes et le timing.

Les personnages devraient d'abord trouver les deux rejetons, ce qui devrait considérablement limiter le nombre de victimes. Puis, l'ancien pourra chercher à se venger, ou finira par lui aussi commettre des erreurs importantes, ce qui déclenchera une seconde chasse par les PJs.

On a déjà trouvé trois victimes, deux d'entre elles trouvées non loin du mur, la troisième dans une arrière-cour. Les trois présentent des lacérations importantes, notamment au niveau de la gorge. En inspectant de manière plus détaillé la dernière, il apparait qu'une bonne part de son sang a disparu. Les deux autres aussi, mais c'était mois évident au vu de leur état. Les lieux où les corps ont été trouvés sont peu utiles, sauf l'arrière-cour, où il est possible de remonter une piste vers un réseau de tunnels de contrebande qui passent sous le mur pour arriver non loin d'une auberge, point de départ de la chasse des deux spawns. Pour des personnages de niveau 1-2, il s'agit de vampire spawns (p321 bestiaire1), en deux affrontements différents (avec un indice après le premier pour trouver le second, avec un test ???). Si les Pjs ne trouvent pas les indices, le second va se mettre en chasse et tomber sur les PJs la nuit suivante, ou dans la soirée, quand ils seront groupés. Niveau 1 : 120xp; niveau 2 :80 par, les deux peuvent être affrontés.

Kan Jung-Nam : maître vampire, noble local

\subsection{Le monastère de Kaejong}
Ce monastère pourra intervenir de plusieurs façons dans la campagne :
de part son importance politique, ses archives utiles, ou encore quelques menaces qui peuvent essayer de s'en prendre à ce lieu.

Il a été bâti pour enfermer et maintenir scellé à jamais un mal ancien. Ce rôle a été oublié depuis longtemps par la majeure partie du public, mais les prêtres importants de Maegu le savent, ainsi que le roi. 

Leur ennemi ici est un groupe de cultistes qui pensent pouvoir réveiller ce mal ancien pour ensuite profiter de ses pouvoirs pour leurs besoins personnels.

Ces cultistes pourront commencer par inciter des bandits de la région à s'en prendre au monastère, sans grand effet, mais cela pourrait préoccuper les moines. Ils doivent également pratiquer leurs rites impies, impliquant potentiellement des sacrifices humains.

Leur conspiration principale sera par contre de profiter du chaos de leurs autres tentatives pour avancer un de leurs hommes à l'intérieur du monastère, qui sera ensuite attaqué pour libérer la créature. Les personnages pourront être envoyés pour les arrêter.
\subsection{La frontière orientale}
Une des raisons de quitter la capitale : les personnages pourront être envoyés transmettre des ordres et surveiller ce qui se passe sur la frontière orientale avec le Han : le conflit avec l'un des royaumes voisin empire rapidement, et le maréchal Mae-Jun mène les opérations.

Une première possibilité serait d'aller lui transmettre un ordre royal, et en profiter pour enquêter sur des problèmes au sein de l'armée de l'est.

Alors que le conflit empire, les personnages pourraient devoir y retourner afin de continuer une enquête, ou de suivre un fugitif qui tente de passer la frontière.

Egalement, l'un des royaumes successeurs de l'empire Han a envoyé un groupe d'espion. La première piste sera la capture d'un agent de liaison, qui faisait le trajet d'un côté à l'autre des montagnes pour rendre compte à ses maîtres. Il faudra remonter sa piste, pour finalement arriver jusqu'à un petit groupe d'administrateurs qui reçoivent un paiement en échange d'informations, notamment sur les places-fortes de la frontière.

Premier élément lié à cet arc : les personnages, allant dans l'est, vont croiser une forte patrouille, appartenant au capitaine Zeng, qui aura un rôle à jouer plus tard au cours de cet arc.
\subsection{Le premier traître}
Au cours de l'acte 1, un des ministres royaux va s'exposer à des accusations de trahison. 

La première phase de cet arc sera une enquête sur des problèmes de crime organisé dans la capitale, enquête qui va se retrouver à éclabousser les bottes d'au moins un ministre du royaume. La seconde phase sera la construction d'un dossier d'accusation solide par le BRI, en récupérant des preuves des agissements du ministre incriminé. Enfin, cet arc devrait se terminer avec l'arrestation, et la gestion de certains problèmes politiques liés à sa chute.

\subsubsection{Première partie : le crime organisé}
Les personnages vont recevoir un nouveau briefing avec leur cheffe. Celle-ci a reçu des rapports de plusieurs magistrats locaux, indiquant que certains des criminels de Daegu sont plus subtiles et organisés que la majeure partie, et les miliciens n'arrivent pas à leur mettre la main dessus, possiblement corrompus.

En effet, un groupe criminel s'est organisé, avec des liens allant jusque chez les Hans. leurs spécialités sont le jeu, la prostitution, ainsi que les trafics en tout genre. Ils sont dirigés par Ku seung-min, un nain relativement âgé, envers qui les criminels de la ville sont très respectueux. Il gère via différents lieutenants ses affaires, essayant de garder pour lui et ses opérations, le profil le plus bas possible. Son nom en réalité est presque inconnu en ville, à l'inverse de ceux de ses lieutenants.

Trois pistes sont proposées aux PJs, deux par leur supérieure : les docks, une salle de jeu clandestine, et un des gangs qui gère une partie des affaires de la maison de plaisir de Kil-Ae.

Les docks ont eu leur bonne part d'affaires de crimes, en particulier des problèmes de territoire entre groupes. Ici, c'est une elfe, du nom de Ra So-Yi, qui gère un gang, mais elle a des problèmes de territoire, celui-ci étant disputé par un groupe de criminels locaux. Elle gère principalement des affaires de contrebande, et a vu ses bateaux attaqués par des petits criminels, ses hommes menant des représailles. Elle est une guerrière, maniant avec efficacité deux couperets de bouchers.

La salle de jeu, dans un quartier de classe moyenne, est dissimulé dans un sous-sol. Les gardes ont eu vent de son existence, le BRI, à cause de cette enquête, leur a demandé de ne pas intervenir immédiatement, afin de tenter de remonter dans l'enquête. Le patron est un goblin particulièrement retord, nommé Sok. Son entreprise paie ses taxes comme une taverne, mais les gardes pensent que des tables de jeu existent et celles-ci sont illégales. Sok est un mage correct

Le gang de la maison de plaisir fait surtout du racket à la protection, calmant pas mal de troubles dans ce quartier un peu chaud. Pas connu du BRI, Kil-Ae pourra se poser la question. Ils travaillent pour un humain, Hwa Kyun-Jin, qui recrute des gens relativement efficaces et discrets. Ses gars peuvent filer des coups de mains aux autres lieutenants de Ku. Jin est un alchimiste avec une arquebuse.

Chaque lieutenant connait les informations sur le patron, mais ne les divulguera pas forcément, sauf à se retrouver personnellement dans la merde. Certaines informations pourraient par ailleurs être parcellaires, afin d'inciter les Pjs à faire au moins deux des gangs, avant de pouvoir raider la demeure de Ku.
\subsubsection{Seconde partie : dossier d'accusation}
Une fois Ku arrêté ou tué, la patronne du BRi va demander aux PJs de bâtir un dossier d'accusation contre le ministre.

\subsection{Réarmement}
Les personnages sont envoyés enquêter sur un chariot d'armes qui a été intercepté par la garde de Maegu. Il faudrait remonter la piste de celui-ci pour savoir à quoi il correspond, et vers qui il se dirige.

Une enquête en plusieurs parties, essentiellement en acte 1.

Les commanditaires de ces marchandises sont les groupes criminels de la capitale, qui cherchent un avantage sur leurs rivaux. Evidemment, ils ne souhaitent pas voir la garde regarder de trop près dans leurs affaires.

Première étape d'enquête : déterminer vers qui vont ces armes, avec deux indices principaux, que sont le chariot lui-même, ainsi que son conducteur. L'interrogatoire peut se faire de différentes manières, et inspecter le chariot et surtout ce que les gardes peuvent dire, peut donner une idée d'où il allait. Il faut ensuite poser des questions sur place (le quartier près du fleuve), pour être guidé, plus ou moins volontairement, vers un chef de gang local, Pak Megujin. Celui-ci ira sans doute à la confrontation contre les PJs, qui pourront trouver une seconde cargaison d'armes aux mains de ses hommes.

Rencontre 1 : interrogatoire : épreuve d'influence; Chun Woo-Sung, conducteur de charrette; diplomatie 17, intimidation 14, duperie 15, autres compétences 18. découvrir DC 15, permet de remarquer qu'il porte des marques de fouet, probablement un esclave en fuite. Cela permet de réduire de 2 les tests d'intimidation contre lui. Seuil 6, doivent réussir en 4 rounds. Réussite pour savoir qu'il devait livrer à quelqu'un avec un bandana rouge près du navire 'la mouette'. En cas d'échec, il faudra discuter avec la garde, qui aura un avis sur le fait que cela devait aller vers les quais. xp = 100 en cas de réussite, 80 sinon. 6 réussites

Rencontre 2 : affrontement avec une première bande de brigands; 3 adversaires utilisant les règles des ruffians (p208 GMG); XP = 120xp; butin : 1 objet niveau 1 (pied-de-biche), 2 consommables niveau 1 (1 griffe d'ours-hibou et potion de soin mineure), 15 po

Rencontre 3 : le chef de bande : objectif:150XP au total, rencontre sévère (120 par joueur); 2 grave robbers et 1 bandits. Le bandit essaie de démoraliser ses adversaires, les grave robbers commencent par lancer leurs bombes avant d'avancer vers les PJs. butin = 1 objet niveau 2 (Parchemin sacré d'arme +1), 1 consommable niveau 2 (bronze bull pendant) et 10 po. Aucun butin gagné, rencontre évitée.

province de Daejong +trouver le nom du village où aller.

Rencontre optionnelle : un gamin des rues voit les PJs et s'enfuie en courant. Il est au courant que le gang sera intéressé par l'info et cours rapidement : (p156GMG), en utilisant les obstacles : foule, fruit cart, et wooden fence. XP = 50.

Il devrait y avoir deux pistes différentes pour la suite : la première consisterait à remonter la piste jusqu'au fournisseur, celui-ci faisant parti d'un groupe criminel ayant un contact dans l'armée du centre à la Forteresse de Hwaseong.

La seconde piste serait de suivre le chemin des armes précédentes, jusqu'à une province voisine où elles étaient acheminées et vendues. Le trajet vers le village (à 5 jours dans les montagnes), se fait sans problèmes de logistique, mais un groupe de bandit va tenter sa chance à attaquer le convoi non loin de la zone. RENCONTRE A FAIRE(rencontre 4) Une fois sur place, quelques questions pourront sans doute amener les personnages à discuter avec la patriarche de la famille Paek, les nobles locaux. Ceux-ci initialement nient, voire tentent de mettre cela sur le dos des bandits : il explique qu'il a tenté de former une milice, étant lui-même un ancien militaire, mais que l'opposition du magistrat local, peut-être corrompu, a mis un terme a cette démarche. Une observation attentive peut montrer des cibles et poteaux d'entrainements non loin de chez lui. Cela devrait se terminer sur un assaut contre les bandits, potentiellement avec un autre contre la magistrat local, et un défi de compétence pour trouver les bandits.

Village d'imje : lieu infesté par les bandits


Rencontre 4 : prisonner (1, 40xp), 4 commoners (-1, 20xp) : rencontre à 100xp (normale). Butin  = 10 po, healing potion, torche éternelle. Réalisé\\
Rencontre 5 (chef des bandits) : tracker (niveau 3, 80xp), 1 commoners (-1, 20xp), prisonnier(40xp), 1 guard dog (bestiaire 102, 20xp). Rencontre à 140xp. Butin : crying angel pendant, 15po, arme +1(épée longue)\\
Rencontre 6 (magistrat local et ses gardes) : harbormaster reskiné (niveau3, 80xp), 2 bodyguards (80) :rencontre à 130xp. Butin : élixir de compréhension (conso 2), chien d'onyx), 15 po\\
Rencontre 7 (Général Paek) : mage for hire (niveau 3, 80xp), cherchera à se défendre si coincé. \\

Récompense de quête : 150xp pour avoir résolu le tout (en comptant les bandits)
\section{Arcs tertiaires}
\subsection{Les statues du sanctuaire d'ibarae}
Un sanctuaire en ville semble avoir de plus en plus de statues. Mais personne ne semble trop savoir d'où elles viennent, et puisque personne n'est vraiment dérangé, la question de ce dont il s'agit reste entière. Le sanctuaire est la demeure d'un grig mage, qui apprécie de créer des petites statuettes, que les fidèles du sanctuaire amènent parfois chez eux. Quand il apprécie quelqu'un, il créé une petite statue pour cette personne, représentant par exemple quelque chose qui lui est précieux, ou lui-même, ou encore un de ses rêves.

Mais une cloche, brisée depuis au moins 200 ans, et conservée dans un recoin du sanctuaire a été récemment volée. Et cela a courroucé la petite fée, quicherche maintenant les coupables, et a enchanté certaines de ses créations pour aller la chercher, d'abord dans le quartier. Et comme la petite fée est joueuse, elle en profite également pour jouer des tours.

Ces petites statues pourraient bien commencer à déplacer des choses la nuit, et jouent des tours de manière générale avec les habitants du quartier.

En réalité, la cloche a été prise par quelqu'un qui apprécie énormément le sanctuaire, mais n'y passe que peu, et qui a donné la cloche à un artisan pour voir s'il est possible de réparer celle-ci, ou à défaut de la reproduire. Celui-ci a oublié la cloche, ou ne progresse pas car le problème est très complexe.

\subsection{Le manoir maudit}
Dans un quartier riche de la ville, la demeure d'un ancien administrateur tombe en ruine depuis quelques décennies, par absence d'occupants et d'entretien. Récemment acquis par de nouveaux propriétaires, le lieu semble par ailleurs hanté, ou maudit : un ouvrier a été tué sur place, et de nombreux autres refusent d'y aller, même menacé par le fouet.

Les premiers soupçons qui pourraient se montrer sont des soupçons de nécromancie : des traces de celle-ci peuvent être trouvées non loin du corps. Des parchemins de validation détruits sont aussi trouvables non loin. Mais des tests de PJs, ou des questions aux nécromanciens de Daegu via le BRI ne donneront rien : aucun nécromancien assermenté n'a été là au cours des dix dernières années, et aucune créature nécromantique n'existe dans le manoir. Par ailleurs, un spécialiste pourra reconnaître que les marques des parchemins sont incohérentes.

Des recherches plus poussées pourront révéler l'existence de passages secrets dans le manoir, l'un d'eux menant à une sorte de porte, apparemment scellée magiquement. A ce moment, les PJs ne devraient pas pouvoir l'ouvrir. Un trajet dans les archives, et ils pourraient apprendre que ce genre de porte peuvent être liées au culte des cinq héritiers, et que plus de détails sur leur fonctionnement peuvent sans doute être trouvés au monastère de Kaejong, à deux jours de la capitale. 

Une première porte dans les passages secrets est défendue par un trio de diablotins, qui observent les PJs et tentent de les effrayer. (récompense = 90xp). La seconde porte, pour le moment impossible à ouvrir, est elle défendue par un diable barbu affaibli (45pvs, CA, attaques, sauvegardes, etc -2; 2 dégâts de moins, 4 si c'est une attaque limitée. p88 bestiaire). Il bénéficie d'une illusion pour passer pour un spectre affamé.

La porte principale, magiquement verrouillée, ne s'ouvre que quand des cadrans à l'image de portions du ciel sont alignés pour former une image du ciel que l'on peut observer au moment du solstice d'hiver, moment privilégié du culte pour ses rituels les plus sombres.

Là-bas, les PJs (en plus de quêtes annexes locales), pourront apprendre les indices nécessaire pour résoudre l'énigme de la porte scellée. Celle-ci révèlera une ancienne chambre rituelle liée au culte, apparemment avec un gardien, que les Pjs devront défaire avant se pouvoir dissiper le problème d'origine.
\subsection{La partie de chasse}
Les personnages sont invités à participer à une partie de chasse : l'idée est de passer une ou deux journée à traquer un tigre, et d'en profiter pour discuter tranquillement avec la personne qui les invite.

Cet arc a été introduit lors de la session 8, avec l'annonce d'une audience royale. Le roi a entendu parler de leur efficacité sur leurs enquêtes, et souhaite rencontrer les magistrats. Il pourra en profiter pour jauger les différents personnages, et d'autres nobles pourraient bien être introduits à cette occasion. Il faut encore détailler ce qui devrait se passer lors de cette première partie de chasse.
\subsection{Le fugueur}
Un gamin apparait à la maison de Kisaeng où travaille Kil-Ae. Il s'est enfui de chez lui, et a réussi à entrer dans l'enceinte du complexe, où il a été retrouvé au petit matin, endormi dans un buisson. Les gardes n'ont pas spécialement envie de gérer cela, et le gamin implore presque un personnage qui passe à sa portée, pour être aidé. 

Sa famille, des paysans pauvre de la province voisine, ne peut pas vraiment élever ses enfants, et il a souhaité partir au plus vite, plutôt que de continuer à avoir faim. Il est ambitieux, et surprenamment malin. Suivant l'aide, ou le désintéressement des PJs, il pourrait devenir un personnage récurrent de la campagne.

\subsection{•}
\subsection{Le festival du dragon}
Les festivités du nouvel ans se préparent dans la plus grande effervescence : il s'agit pour les grandes villes de s'attirer la présence et les faveurs des dragons !

Les jeunes dragon, ou imugi sont assez communs, mais en ces temps de crises, la présence d'un grand dragon (yong), pourrait rassurer la population, ou peut-être ajouter un nouveau joueur à l'échiquier politique.
\subsection{Le mort sans repos}
Le bureau d'investigation est chargé d'une affaire privée, mais sérieuse : ils doivent se débarrasser du mari d'une puissante ministre. Toutefois, celui-ci est déjà mort il y a deux semaines, mais s'est relevé pendant la nuit suivante, et rôde maintenant dans leu demeure.

\section{Arcs personnels}
\subsection{Shin myun-woo : une guenaude au palais}
Une des nobles du royaume est en réalité une guenaude, qui essaie de mettre en place un couvent au sein de l'administration et de l'aristocratie de Choseon.

Il s'agit de la mère de Shin, et elle profitera du chaos de la succession pour lancer l'Appel sur certaines de ses filles, qu'elle compte bien placer au mieux par la suite.

Shin myun-woo est le personnage de Tara.
\subsection{Gang Ye-mong : le clan Yi}
Ayant été au service de la famille Yi, on pourra lui faire confiance pour protéger les jumeaux royaux au milieu du chaos, à défaut de protéger la reine elle-même. Elle va donc être très liée à l'intrigue principale, avec un intérêt personnel dans celui-ci. On peut même imaginer un affrontement entre elle et un membre de sa famille qui garderait toujours les Yi.

Au-delà de ça, un trajet au niveau du quatrième vieillard pourrait bien se révéler une aventure intéressante suivant la saison au cours duquel il est entreprit et du chemin choisit. Cela pourrait se révéler être une pause détente agréable pour le reste de l'équipe, notamment après une inspection du nord, où un risque de rébellion gronde.


\subsection{Kil-ae : le symbole du chaos}
En premier lieu, l'arc principal de ce personnage sera sans doute lié au monastère de Kaejong, ainsi qu'à ce qu'il enferme. On peut imaginer que l'influence du démon qui y est scellé a pu se répandre, que ce soit dans une famille qui aurait un lien avec son enfermement, ou une autre qui serait lié à son culte (qui se réveille et tente de libérer son maître).

Dans tous les cas, cela pourrait passer par une marque étrange, que quelqu'un au courant de cette histoire pourrait reconnaître comme étant le symbole de ce démon : un bien sombre présage évidemment. Cela pourrait mener les moines à surveiller de près la tieffeline, voir à lui poser directement des questions. Cet arc reste à détailler, mais va inclure la révélation de ses origines, et de l'influence qu'elle porte en elle, pour probablement finir par un affrontement avec la créature.

De manière plus secondaire, il est fort possible que certaines intrigues se déroulent dans la maison de thé où elle réside : ????
\subsection{Lim Jun-yon : L'ambition familliale}
La défense de sa famille risque d'être un moteur important pour ce personnage : celle-ci a de grands projets pour monter dans les sphères sociales, en particulier sa belle-mère. Il y aura notamment le mariage de sa soeur, militaire au sein d'une famille importante (le clan Yi, le clan song, un grand général ????), mais aussi des plans d'achats et d'expansion en général (avec sa demi-soeur déjà mariée et sa belle-mère). 

Tout ceci devenant bien évidemment de plus en plus risqué alors que le chaos va se répandre dans le royaume avec la maladie royale : s'allier à une faction deviendra alors un danger mortel, et Lim ne va pas manquer de problèmes pour les protéger.

Sur un ton plus léger, certains de ces évènements familiaux pourraient servir à détendre l'atmosphère pour une ou deux sessions.
\subsection{Nam Ji-hyo : ???}
\chapter{Règles maisons}
\section{Armes à feu}
Deux types d'armes à feu sont surtout présentes à Choseon :La lance de feu (seungja), constituée d'un canon au bout d'un long manche, tirant une charge de shrapnels dans une direction. Mortelle à courte portée .L'arquebuse,ou jochong, permettant un tir plus précis à longue portée, est un ajout plus récent venant des armées Hans.

\flushleft
\begin{tabular}{c c c c c c c p{0.1\textwidth} p{0.2\textwidth}}
Nom & Prix & dégâts & portée & recharge & enc & mains & groupe & traits \\
Seungja & 15po & d8P & ligne 35ft & 2 & 2 & 2 & gun martial & attachée (bâton), mortel(d10), misfire \\
Jochong & 20po & d10P & 60ft & 2 & 2 & 2 & gun martial & mortel(d10), misfire\\
\end{tabular}


\paragraph{Nouveau trait d'arme : misfire}
Lors d'une attaque avec une arme pouvant misfire, si le dé est un 1 naturel, le tir fait long feu et ne part pas. Cela implique de nettoyer l'arme avant de la recharger. Mécaniquement, cela oblige à utiliser l'action 'gérer un long-feu' avant de pouvoir tirer ou recharger.
Action :
Gérer un long feu : 1 action
Si le personnage réussi un test de crafting DD15 ou de lore(armes à feu) DD10, il peut à nouveau recharger celle-ci et tirer avec. Sinon, il faudra retenter cette action.

\section{Langues}

Les langues utilisées dans le livre de base ne seront pas utilisée dans la campagne : pas de nain, d'elfe, de commun, ou autre langue dépendant de l'ascendance.
En remplacement les règles suivantes:
\begin{itemize}
\item Tout les personnages parlent le Choseon à la place du commun, et remplacent toutes les langues qu'ils devraient obtenir par une langue de la liste ci-dessous.
\item En terme de règle, savoir parler une langue et savoir l'écrire nécessite de dépenser deux langues : une pour l'oral, une pour l'écrit. 
\end{itemize}


Liste des langues parlées : Choseon, Han, Junkan, Ihlan, Jeju (dialecte du nord de Choseon, assez différent)

Liste des langues écrites : Choseon, Choseon classique, Han, Han classique (langue des érudits), Junkan, Ihlan, Jeju, et différents codes secrets(à définir).

\chapter{PNJs}
\section{La cour royale}
\subsection{Le roi Song Su Yun}
Le roi de Choseon depuis 23 ans. Il est gravement malade, mais a à peine commencé à ressentir les premiers symptômes de sa maladie. Ceux-ci vont faire l'objet de toutes les spéculations après l'accueil triomphal du général Kim Mae-Jong. En effet, au cours de celui-ci, le roi va, de manière publique, être indisposé (un malaise). Cette maladie va ensuite s'aggraver au cours de l'acte 2, avant de causer sa mort au cours de l'acte 3.
\subsection{La reine Yi Il Sho}
La seconde épouse du roi Yun, issue du puissant clan Sho. Issue d'un clan puissant, elle est la mère des deux plus jeunes enfants du roi. Elle craindra, au cours des actes 2 et 3 qu'un des demi-frères de ses enfants ne monte sur le trône et ne se débarrasse de ces deux jeunes concurrents potentiels, et essaiera de saisir le pouvoir pour les protéger. Il est peut-être possible de négocier avec elle, mais il faudra qu'elle se sente suffisamment en puissance pour avoir des garanties.
\subsection{Le prince héritier Song Cheol Shin}
Militaire, a passé plusieurs années dans le sud du pays à diriger des opérations de la flotte Bleue. Il considéré par les traditionalistes comme l'héritier évident de part son aînesse. Il considère de même cela comme une évidence, et aura du mal à accepter d'être 'floué' par son jeune frère. Le convaincre de laisser tomber risque de s'avérer difficile, et il resterait sans doute un problème majeur pour le nouveau souverain. 
\subsection{Le prince Song Cheol Ahn}
Un érudit plutôt calme, il s'entend très bien avec son père, et a le soutien d'une bonne partie des réformateurs, qui saluent son esprit éclairé. Il ne s'attend pas spécialement à hériter, et sa nomination en tant qu'héritier présomptif lors de la maladie de son père le surprendra, en plus de le faire paniquer : il se rend bien compte de l'instabilité de la situation, et des dangers qu'il court personnellement. Il est peut-être possible de le convaincre d'abdiquer en faveur d'un autre candidat, s'il vient à penser que cela serait le mieux pour le royaume, mais serait aussi sans doute le meilleur choix.
\subsection{Capitaine Kyo Yong-Chol de la garde royale}
Un soldat de métier, un ancien de l'armée de l'ouest, recruté dans la garde royale il y a 15 ans. Il sert fidèlement le roi depuis.
\subsection{Administratice On Ji-Hye du bureau royal d'investigation}
Récemment nommée à ce poste, après des postes plus subalternes dans l'administration de la capitale. Elle a envie de faire ses preuves à ce poste très proche du roi. Elle a été notée pour son ascension rapide dans les rangs de l'administration, après un concours particulièrement bien réussi. Elle vient de la famille On, soutiens traditionnels du roi, et une des plus anciennes familles aristocratique du royaume.
\subsection{Général Kim Mae-Jong}
Général de l'armée de l'Est, et vainqueur du royaume de Wei au début de la campagne. Son accueil triomphal à Daegu va voir la révélation de la maladie royale. S'il vient à penser que son intervention peut ramener l'ordre dans le royaume, il le fera. Le risque étant que cette intervention se fasse en son nom, en raison de ses soutiens, tant à la cour que dans l'armée. Il serait sans doute possible soit de le convaincre qu'un autre candidat peut maintenir l'ordre, soit de le sortir avec une démonstration de force, politique ou militaire.
\subsection{Sung Ji-Yung, marchande Ysoki}
Une marchande d'âge moyen, que l'on trouve souvent au palais : elle fournit le palais avec de nombreuses matières premières et objets d'import, de tout le pays et au-delà. C'est du moins la justification officielle de sa présence là-bas, dans la réalité, son réseau commercial opère comme un efficace réseau de renseignement, notamment dans l'ancien empire des Hans.

Elle suit, comme d'autres, la présence des PJs et leur efficacité, les invitant à passer dans sa boutique pour discuter un peu plus longuement. Elle les recevra pour discuter, mais c'est surtout un de ses fils qui gère la boutique.
\section{Les rues de Daegu}
\subsection{Kan Jung-Nam, vampire ancien}
Un vampire relativement ancien, qui a changé à plusieurs reprises d'identités pour brouiller les pistes. Il est relativement subtil dans son approche, et limite pour sa part ses repas. Ses deux rejetons, encore très jeunes, eux ne se contrôlent que peu par contre.

Il ne se considère pas comme mauvais, et pourra peut-être chercher un accord avec les PJs si il est cerné (de part ses services rendus au royaume notamment). Il dispose d'informations importantes sur ce qui se passe au palais de part son âge, notamment il a des soupçons voire des informations sur les agissement d'autres créatures moins maîtrisées que lui : il soupçonne la présence d'une guenaude, voire de plusieurs, et connait l'histoire du démon du monastère de Kaejong.


Après la première rencontre avec les PJs, il est possible qu'il s'intéresse de plus près à eux, et contrôle de beaucoup plus près ses deux rejetons. 
\section{Les montagnes des 4 vieillards}
\subsection{Capitaine Zeng}
Une officière de l'armée royale, qui a participé à la récente victoire du général Mae-Jun. Elle a été envoyée patrouiller au nord pour s'assurer que d'autres incursions ne risquaient pas de passer, et limiter les problèmes de bandits, qui deviennent communs : des déserteurs, souvent Hans, se font bandits.
\chapter{Éléments de setting}
\section{Familles nobles}
\subsection{Le clan Yi}
En vérité un groupe de familles nobles apparentées, il est relativement puissant, ayant fourni au cours des derniers siècles plusieurs maréchaux et de nombreux administrateurs, ainsi qu'entre autre, la reine actuelle, Yi il sho. C'est une influence importante dans le le royaume, contrôlant de nombreuses terres dans les ports du sud.
\subsection{la famille Jien}
Famille noble récente, d'origine Han, elle n'a fournit qu'un seul administrateur, qui est toujours à sa tête : Jien jeong-hun. Celui-ci est un réformateur assumé, et le domaine de sa famille est une villa plutôt modeste, en bordure de la capitale.
\subsection{ la famille Paek}
Famille plus ancienne que fortunée, elle ne fournit pas d'officiel à chaque génération, et a plusieurs fois été proche de repasser au sein du peuple à cause de cela. Elle reste une famille plutôt ancienne dans la capitale, ayant fournit notamment de nombreux officiers à l'armée (aujourd'hui plusieurs des fils de la famille servent au nord et à l'ouest).
\chapter{Déroulement de campagne}
\section{Acte 1 : niveau 1-5 ?}
\subsection{niveau 1}
Les personnages, pour leur première enquête, vont être envoyés sur une histoire de trafic d'armes : la garde d'une des portes a trouvé un chariot rempli d'arme quittant la ville, et a arrêté son conducteur. Le responsable du BRI demande de trouver rapidement pour qui étaient ces armes et quel était l'objectif de leur trafic. Trouver leur origine sera l'étape suivante, pour éviter que ce genre d'objets ne se retrouve partout (mais il devrait manquer un indice important à ce moment pour ce faire). 
\emph{Arc secondaire réarmement, suite à prévoir avec de nouveaux indices qui arrivent au BRI}
Après cette introduction, alors que les gens iront se reposer par exemple, Lim va avoir des nouvelles de sa famille : sa belle-mère ayant finalisé l'achat de plusieurs propriétés, notamment une seconde fabrique de soie juste en-dehors de la ville. 

Pour ce niveau 1, la suite devrait être la suivante : 3 combats et 2 obstacles autres. Ceci se répartira entre une session de transition à la capitale puis une session de trajet jusqu'au village où les mène leur première enquête, ou alors la totalité de leur expédition dans les montagnes, suivant les besoins.

Cette session de repos verra donc plusieurs informations:
\begin{itemize}
\item En premier lieu pour Lim des achats par sa famille de nouvelles propriétés industrielles (arc perso)
\item Pour Gang Ye-Mong, une discussion avec le seigneur Yi Il-Sung, frère de la reine et intendants des messagers royaux. Celui-ci déplore que Gang n'ait pas été affecté à la sécurité de sa soeur, malgré ses demandes (lien avec l'arc principal).
\item Quelques disparitions sont signalées dans le quartier de Minae. Les personnages peuvent décider d'y aller avant ou après leur enquête en cours. (arc secondaire des disparitions mystérieuses).
\item Une grande victoire est annoncée dans l'ouest : les troupes du royaume de Wei de l'ancien empire Han ont été écrasées par le général Nam Sung-Min et son armée de l'est. La capitale commence à s'attendre à accueillir le général victorieux, probablement dans quelques mois, si le Wei demande la paix (arc secondaire de la frontière orientale, et anticipation du pivot acte 1/2 de l'arc principal, au cours de cette cérémonie).
\end{itemize}

loot niveau 1 :40 po,healing potion (conso 1), comprehension elixir (conso 2), torche éternelle, chien d'onyx, arme +1, crying angel pendant (conso 2). A fournir dans les 3 prochaines rencontres:
10-15po, 1 objet permanent et un conso par rencontre.

déjà obtenu : 10po, crowbar, healing potion, owlbear claw
\subsection{Niveau 2}
Après l'introduction du setting et des personnages eux-même, ce niveau devrait marquer l'arrivée dans le coeur de la campagne. En particulier, il faudrait que les personnages rencontrent le roi, pour commencer à intervenir dans la politique du royaume.

En plus de cela, plusieurs des arcs narratifs personnels devraient s'amorcer à ce moment, ainsi que des arcs tertiaires ou secondaires, voire des quêtes annexes pures.

Après leur retour du village d'Imje, les personnages vont pouvoir se reposer quelques temps à la capitale (actions de repos), et vont recevoir courriers et nouvelles (autant d'amorces de quêtes).
\begin{itemize}
\item Ye-mong : le seigneur Yi Il-Sung, le frère de la reine, a entendu des rumeurs de complots dans le palais royal. A part cela, la vie de caserne peut être mouvementée dans la garde royale. Des nouvelles de sa famille également : apparemment tout va bien dans le sud-est, avec les victoires du général Kim.
\item Jun-Yon : des nouvelles du quartier de Minae, apparemment aucune nouvelle disparition n'a été signalée, malgré des efforts renouvelés pour vérifier cela. Par ailleurs, la surveillance du tunnel n'a rien donné. Du côté familial, de nouvelles affaires ont été achetées, sa belle-mère lui offre deux nouvelles tenues, et lui annonce que pour son mariage, des négociations préliminaires sont en cours avec la famille Hwan, et leur héritier, Man Song-Jin. Sa belle-mère peut en tout cas proposer une rencontre dans un des parcs de la capitale, celui d'Ichon par exemple.
\item Myun-woo : sa soeur aînée, Shin Kyun-ok prend le relai de leur père, qui est reparti dans leur domaine du nord pour y régler quelques problèmes, dû à de récents orages. L'autre soeur aînée, Chul-soo est aussi de retour momentanément à la capitale.
\item Ji-hyo : Elle remarque la présence des statues du sanctuaires d'Ibarae, et subit comme Myun-woo l'embuscade de Chul-so. 
\item Kil-Ae : Est abordée par des gens louches qui semblent s'en prendre à elle pour son ascendance. Si elle est en difficulté, un garde de sa maison de plaisir (l'inspiration du poète) interviendra pour intimider les fauteurs de troubles. Egalement, une jeune noble récemment arrivé en ville sème le désordre dans les soirées, étant très souvent très alcoolisée, et cassant pas mal de choses. Son père paie, mais cela n'enlève rien au problème. La jeune noble s'appelle Yi Su-jin, du puissant clan Yi (il s'agit d'une nièce du seigneur Yi Il-sung).
\item quête générique 1 : prêt du manoir de la famille Shin se trouve le sanctuaire d'Ibarae, et ses nombreuses statuettes. De nouvelles semblent s'être ajoutées à celles qui existent déjà.
\item quête générique 2 : Une jeune femme est apparue au palais, sans aucun souvenir, certains membres de la garde royale s'occupent d'elle.
\item Convocation royale :A la fin de l'été, le roi convoque les personnages pour une audience. Il a entendu parler de leurs actions dans sa capitale et dans son royaume, et souhaite rencontrer ces magistrats prometteurs (et leur suite).
\end{itemize}

PNJs à préparer et détailler:
\begin{itemize}
\item Man song-jin
\item Yi su-jin
\item Shin Chul-so et Shin Kyun-ok
\item le roi, Song su-yun
\item une jeune femme sans mémoire
\end{itemize}

Butin niveau 2 : 88po, 2 objets de niveau 3, 2 objets de niveau 2, 2 consommables de niveau 1, 2 et 3. 
\begin{itemize}
\item rapière +1 (item niveau 2)
\item flute de maestro (item niveau 3). https://2e.aonprd.com/Equipment.aspx?ID=260
\item glamorous bluckler (item niveau 2 : https://2e.aonprd.com/Equipment.aspx?ID=734)
\item kit de détective : https://2e.aonprd.com/Equipment.aspx?ID=679
\item parchemin de sort niveau 2(conso 3) :
\item 2 parchemins de sort niveau 1 (conso 1) :
\item 2 fioles de silversheen (conso 2) :https://2e.aonprd.com/Equipment.aspx?ID=134
\item drakeheart mutagen (onso 3) : https://2e.aonprd.com/Equipment.aspx?ID=688
\end{itemize}

\subsection{Niveau 3}
Total : 2 items niveau 4, 2 items niveau 3, 2 conso lvl4, 2 conso lvl3, 2 conso lvl 2, 150 po.

4: arme magique striking +1 (pour qui ?), Sturdy shield
3: kit d'alchimiste (étendu), robe miroir 
conso 4 : potion de peau d'écorce, bombe d'éclats cristallins modérée
conso 3 : parchemin lvl 2, mutagène de juggernaut
conso 2 : 2 oils of potency
\section{Acte 2 : niveau 6-10?}

\section{Acte 3 : niveau 11 - 15 ?}
\chapter{Notes de séance}
\section{Session 0}
Présentation du setting, et réflexion sur le groupe. Sont choisis :
\begin{itemize}
\item Un tengu alchimiste (nam ji-hyo)
\item Une barde tieffeline (kil-ae)
\item Une duelliste changeline (shin Myun-woo)
\item Une elfe aasimar oracle (Lim Jun-yon)
\item Un guerrier kitsune (gang ye-mong)
\end{itemize}
La création des détails mécaniques et des backgrounds sont laissés aux joueuses en attendant la session 1.
\section{Session 1}
\paragraph{Date}12.12.2020
\paragraph{Déroulement} Introduction par personnage sur leur début de journée, dans le milieu de l'automne. Elles reçoivent leur première enquête : du trafic d'armes intercepté par la garde dans la capitale (à la porte princière). Elles y arrivent dans l'aprem pour y interroger le conducteur, qui révèle ce qu'il sait (un contact et employeur à Hwaseong, et un lieu de dépôt sur les quais du yeongsam). Elles se dirigent ensuite vers les quais, où elles sont attaquées par un groupe de truands, qu'elles surclassent avec difficulté. Gain total : 220xp.
\paragraph{Améliorations}
\begin{itemize}
\item améliorer la préparation des combats : profils des PNJs et préparation des rencontres plus sérieuse, avec le calcul d'xp notamment. Ici, l'équilibrage était douteux
\item trouver plus de musique pour tenir 2-3h au total, l'ost de my country new age étant trop courte.
\item continuer à essayer de plus narrer la situation
\item modifier l'arc secondaire réarmement pour le rendre plus important que juste un groupe de truands essayant de dépasser ses concurrents.
\item préparer les PNJs plus sérieusement : portraits et noms de la capitaine et de la supérieure des investigratices.
\end{itemize}
\section{Session 2}
\paragraph{Date}11.01.2021
\paragraph{Déroulement}
Interrogatoire du brigand s'étant rendu, celui-ci leur indiquant rapidement la direction de son chef de gang. Les PJs se préparent et se rendent à son repaire avec quelques gardes pour obtenir des réponses. Une discussion un peu tendue (et pleine de mauvaise foi), leur donne les informations qu'ils étaient venues chercher : les armes devaient être livrées dans un village de la province de Daejong, à quelques jours de marche vers le nord-est. total xp : 340.
\paragraph{Axes d'améliorations}
\begin{itemize}
\item accélérer le mouvement : anticipation des problèmes techniques, meilleure préparation de l'enchainement et des descriptions.
\item musiques à tester plus sérieusement d'ici la prochaine séance
\item mieux préparer les rencontres hors combat
\end{itemize}

\section{Session 3}
\paragraph{Date}01.02.2021
\paragraph{Déroulement}
Les personnages ont commencé par récupérer de leur enquête précédente et des combats qui en avaient découlé. Ils sont ensuite retourné au palais, pour obtenir plus d'informations sur le village d'Imje et la famille Paek, et signaler leur probable départ prochain.

Leur supérieure les informe à ce moment d'une seconde affaire, à traiter à leur convenance, des personnes ayant disparus dans un quartier du nord de la ville, le long des remparts (arc secondaire des disparitions inquiétantes).

Ils ont donc enquêtés rapidement sur cette seconde affaire, dans l'idée de voir ce qu'il en était avant de partir pour les montagnes. Une journée d'enquête leur a permis d'obtenir pluseurs informations : une créature semble chasser la nuit et se nourrir de sang, à l'intérieur des murs de la ville. Celle-ci a une odeur étrange, ce qui leur a permis de trouver un tunnel de contrebande utilisé par la créature pour se déplacer (le spawn n'en a pas besoin, mais cela lui évite de devoir faire attention aux rondes sur les murs). Ils ont aussi pris contact avec un noble local, habitant à l'extérieur et semblant connaître bien le quartier (Kan Jung-Nam, le maître vampire?).

La session se termine alors que les personnages envisagent de partir, la créature de la ville devant faire plus attention à cause de la garde, plus vigilante que jamais, et connaissant l'emplacement du tunnel.

xp total : 490 (+150 pour l'avancée sur l'enquête).
\paragraph{Améliorations}
La plus notable à faire serait de disposer (et fournir) une carte de la ville de Maegu, afin que les joueuses s'y retrouvent plus facilement.

En dehors de cela, les règles étaient peu nécessaires sur cette session, mais la préparation était bonne, malgré les deux embranchements possibles de la campagne.

Au retour des personnages de la montagne, il va falloir faire rentrer les PJs dans le jeu politique, peut-être au cours d'une soirée ou d'un autre évènement officiel, ou simplement en les faisant rencontrer certains des protagonistes de l'arc principal.
\section{Session 4}
\paragraph{Date}22.02.2021
\paragraph{Préparation} Avant de partir, les personnages devraient avoir quelques nouvelles : la belle-mère de Lim a finalisé l'achat de nouveaux ateliers de soie. Le frère de la reine pourra essayer de discuter avec Gang. Par la suite, les personnages devraient quitter la capitale, pour 4 jours de voyage. Ils croiseront la capitaine Zeng au cours de leur trajet, pour ensuite arriver au village d'Imje :arc narratif du trafic d'armes, rencontres 4 à 6 + xp quête.
\paragraph{Déroulement}
Après quelques préparatifs pour le voyage, le groupe part pour le village d'Imje. Sur la route, ils croisent la patrouille de la capitaine Zeng, puis sont attaqués par un groupe de bandits qui souhaitaient s'emparer de leurs marchandises. Le combat a été facilement remporté par les PJs. (rencontre 3 de cet arc fait + loot et xp donnés). A part cela les personnages ont eu quelques nouvelles : la famille de Lim a étendu son domaine commercial en rachetant différents fabricants, et le frère de la reine pu discuter avec Gong de sa position actuelle.

xp total : 590
\paragraph{Notes et remarques}
Il faudrait faire attention à ces séance trop lourdes, ou peut-être les avoir plus longues pour faire autre chose que de l'affrontement. Avoir une préparation par séance est plus facile que de planifier trop en avance une grande séance.
\section{Session 5}
\paragraph{Date}19.03.2021
\paragraph{Préparations}
Les personnages commencent en arrivant à Imje, et joueront sans doute leur enquête dans ce village, entre le patriarche, le magistrat, et les bandits de la région. Cela pourrait impliquer 1 à 3 combats suivant leurs actions (voir l'arc narratif du trafic d'armes, rencontres 5-6). Devrait mener au niveau 2 ou à ses portes. Pour varier un petit peu les arcs narratifs, il serait sans doute possible de leur donner des nouvelles en lien avec d'autres arcs, peut-être plus sur le retour.. 

L'arrivée à Imje se fera sans plus de problèmes, et un serviteur de la famille Paek va venir les chercher pour les mener au patriarche, dans sa demeure (qui a connu des jours meilleurs, mais profite d'une magnifique vue sur la vallée, au bord d'une rivière). L'ancien général aura droit à un test pour identifier la mascarade, auquel cas, il demandera directement aux personnages ce qu'ils sont venus faire ici. Il essaiera alors de les convaincre de l'aider et de le débarrasser du magistrat local. Si il rate ce test, il sera sans doute accusé de trahison, et essaiera de gagner du temps en expliquant le problème. Si il est contraint au combat, ou accusé de trahison, il se jettera sur les PJs malgré son grand âge. Si les personnages agressent l'ancien général, un groupe assez important de paysans montera jusqu'à sa demeure (prévenus par les serviteurs qui s'enfuient dès le début du combat). Ils sont nombreux, mais pas très doués, et prennent les personnages pour des bandits, et essaieront de les tuer : 8 commoners mal équipés.
\paragraph{Déroulement}
Les personnages commencent par arriver à Imje, où un serviteur les amène voir l'ancien général, qui se dit curieux. En particulier, il leur fait remarquer qu'ils sont nombreux et parlent bien pour des marchands vers une région aussi perdue. Les PJs révèlent leur rôle, et le général admet qu'il a passé la commande d'armes. Il explique aussi les raisons qui l'ont poussé à faire cela, et les PJs décident de passer l'éponge.

Ils repartent ensuite, en voulant enquêter au passage sur le magistrat corrompu et le problème des bandits. 4 d'entre eux arrivent chez le magistrat en uniforme pour enquêter directement sur le problème, et demandent des comptes. Le magistrat se défend en plaidant une faible compétence basiquement (il déclare travailler avec l'armée à ce sujet, et pensait donc ne pas avoir besoin de prévenir plus haut - il travaille effectivement avec l'armée, mais donne ces infos aux bandits aussi). Kil Ae vole un coffret dissimulé sous une latte de parquet [DETERMINER SON CONTENU]. 

La session se termine à ce moment, alors que plusieurs personnages sont installés chez le magistrat, et que Kil Ae a pu s'emparer d'un élément sans dote utile.

gain de 150xp d'histoire pour avoir eu l'histoire derrière la commande d'armes.
\paragraph{Notes et remarques}
Il aurait sans doute fallu mieux préparer l'enquête contre le magistrat ainsi que ses réactions, mais l'improvisation à ce sujet était plutôt bonne.
\section{Session 6 : lutte contre la corruption}
\subsection{Préparation} cf préparation séance précédente
\subsection{Déroulement}
Tout en discutant des actions à entreprendre et en allant à l'auberge du village, les personnages ouvrent le coffret découvert pour y trouver des lettres ainsi qu'une statuette de chien d'onyx (objet niveau 2). Les lettres semblent incriminer le magistrat local( un elfe du nom de Tan Kwang-Jo).

Le lendemain matin, les personnages sont attendus à la sortie de l'auberge par un groupe de brigands, qui les attaquent, et sont défaits. Leur cheffe (une homme-rat) est capturée après avoir été grièvement blessée. (loot : crying angel pendant, 15po, rune d'arme +1 (permet de transformer une arme au choix en arme magique +1), et 160xp). L'interrogatoire révèle que le magistrat a envoyé un messager vers le camp des bandits pour les prévenir de la présence des PJs.

Ceux-ci partent alors vers la capitale de province, où ils fournissent les éléments incriminant envers le magistrat au gouverneur, qui promet de mener une enquête rapide sur ce problème. (récompense histoire : 100xp pour avoir résolu cet arc narratif).

Avec cela, les personnages passent niveau 2, montée de niveau qui se fera entre les sessions 6 et 7 du coup, en séances individuelles.
\subsection{Remarques}
Il ne faudra pas oublier le loot de la rencontre du magistrat, probablement en récompense de quête à l'arrivée à la capitale.\\
La conclusion de l'arc (ce qui est arrivé au magistrat) pourra être mentionné plus tard dans la campagne.
\section{Session 7 : retour à la capitale}
\subsection{Préparation}
Début de la session : ellipse sur le retour à la capitale. Récompense : elixir de compréhension + 15po.

Nouvelles des affaires en cours : le magistrat de Dinae n'a rien vu de neuf : le tunnel est resté inutilisé, et aucune nouvelle disparition n'a été signalée, y compris par ses collègues. Il ne sait pas ce que l'équipe a fait, mais les en remercie.

Période de repos, au cours de laquelle les personnages vont pouvoir avoir de nouvelles amorces de quêtes, ainsi que réaliser des actions de repos [revoir les actions de repos].

Le déroulement de la séance devrait dépendre pas mal des volontés des joueuses, avec également les éléments préparés dans la section niveau 2 du déroulement prévu de la campagne.

Les dernières 45 minutes de la session seront une convocation royale pour une audience à la fin de l'été : le roi a entendu parler de leurs actions, et souhaite les remercier pour celles-ci. Il propose également de participer à une partie de chasse deux semaines plus tard, au nord de la capitale, dans la forêt de Palhyeon-ry, où il espère bien chasser un tigre ou deux, et souhaiterai ajouter les PJs à sa suite.
\subsection{Déroulement}
Après leur retour à la capitale, les personnages ont décidé de continuer leur enquête sur les disparitions du quartier de Dinae. Après avoir consulté les archives du BRI, ils ont trouvé des traces d'une vieille enquête, 50 ans plus tôt, sur des cadavres trouvés dans une situation proche. Le coupable de l'époque était apparemment un boucher, qui avait été trouvé allant déposer un corps dans une ruelle. (le boucher assistait le vampire, qui a pu soutenir sa famille depuis par sympathie pour ce serviteur loyal). L'ancien magistrat (et ancien ministre des finances, Man Jun-Yeong) leur a donné de nombreuses informations sur cette affaire, et a apprécié d'être consulté sur ce genre de sujet. Il a contacté la supérieure des PJs le lendemain pour en savoir plus, et suivre un peu les activités du bureau. Il pourrait devenir un personnage récurrent intéressant, avec un pied dans la politique du palais, et une bonne expérience d'enquêteur.

Parmi les informations qu'il a pu donner, l'existence du culte des 5 fils(ou 5 enfants), un ensemble de sectes vénérant un seigneur démon, et que le BRI chasse depuis longtemps. Lien avec l'arc narratif du monastère et de Kil Ae.

Par la suite, les personnages sont allé inspecter le tunnel de contrebande passant sous les remparts, et utilisé par le vampire. Apparemment, celui-ci n'a pas utilisé le tunnel depuis l'intervention des PJs (et pour cause, Kam Jun-nam a fait beaucoup plus attention à sa progéniture, qui doit apprendre à se contrôler). Ils n'ont pas appris grand-chose de plus à ce sujet, à part le fait que la créature soit apparemment plus calme ou discrète.

Après cela, les personnages sont allés se coucher, apprenant plusieurs éléments:
\begin{itemize}
\item Présence des deux sœurs de Myun-woo, et départ de son père vers le nord.
\item Cadeau de la belle-mère de Jun-Yon, et début des négociations avec la famille Man (pour Man Song-Jin)
\item Insultes envers Kil Ae pour son ascendance
\item Plus de statues au sanctuaire d'Ibarae
\item Vie de la garde pour Ye-Mong (pas encore la jeune femme amnésique)
\end{itemize}
La fin de la séance a été marquée par une discussion avec leur supérieure, qui a été contactée par l'ancien ministre des finances à leur sujet, et par un messager royal, pour une audience se déroulant la semaine suivante.
\subsection{Remarques}
Bonne improvisation, notamment pour le personnage de Man Jun-Yeong. La préparation par arc a permis pas mal de souplesse à ce niveau, notamment dans les informations qu'il a pu donner.

Il faudrait réfléchir sérieusement à comment récompenser les efforts des joueuses dans l'enquête sur la créature. Peut-être leur donner l'un des jeunes vampires, relativement faibles, et surtout, peu importants pour leur père au regard de sa sécurité. Il pourrait sinon brouiller les pistes en créant un nouveau vampire?

Oubli de musique sur la majeure partie de la séance !

Faire peut-être plus attention à proposer du jeu à tout le monde (Morgane a pas forcément eu beaucoup d'occasion de s'exprimer, faire attention à cela et/ou lui proposer du jeu propre).
\section{Session 8 : the plot thickens}
\subsection{Planification}
Objectifs :
\begin{itemize}
\item Récompenser les joueuses pour leur investissement sur les vampires (si possible)
\item Lancer un ou deux autres arcs narratifs dans l'attente de l'arc royal, possiblement un lié à Nam Ji-yo
\item Avoir un downtime?
\end{itemize}

\subsubsection{Récompense pour l'enquête}
L'idée serait de donner un os à ronger aux PJs pour leur donner l'impression que cet arc narratif est terminé. Sans doute via un affrontement avec un des rejetons de Kan Jung-nam, peut-être même un nouveau-né, créé d'un esclave local, dont la famille gagnerait en statut social via son influence. 

Le combat pourrait être amené par une alarme sonnée dans le quartier de Dinae au cours de la nuit, deux gardes ayant été agressés. Les PJs sont menés par les gardes à une cour, où le rejeton se nourrit, penché sur sa victime.

Si les PJs enquêtent, ils apprendront que l'esclave travaillait effectivement de l'autre côté du mur, sur les chantiers publics, notamment les réparations du mur. Sa famille n'a rien soupçonné, et est légitimement terrifiée des gardes royaux : la seule implication de So pourrait les faire condamner à mort (Kan devrait intervenir a pasteriori pour régler sa part du contrat).

\subsubsection{Nouveaux arcs narratifs}
\begin{itemize}
\item L'arc narratif du manoir hanté pourra être introduit, avec l'ordre de mission d'On Ji-Hye à ce sujet.
\item Les statues du sanctuaire d'Ibarae pourraient commencer à semer un peu de désordre dans le quartier où vivent les Shin. Par exemple, la nourriture du petit-déjeuner a été méchamment épicée pendant la nuit.
\item Une demande de rendez-vous pourrait être faite par le petit fils de Man jun-Yeong ? Une rencontre dans un des parcs de la capitale, afin de faire connaissance en toute bonne fois. D'autres PNJs intéressants pourraient faire également leur apparition à cette occasion, peut-être liés à d'autres arcs importants.
\end{itemize}
\subsubsection{Downtime}
activités possibles:
\begin{itemize}
\item crafting
\item Gagner des ressources
\item Signer des maladies/se reposer
\item Créer de faux documents
\item Subvenir à ses besoins
\item investir
\item retraining
\item leadership
\item Autres ? Quêtes personnelles?
\end{itemize}
Des évènements se trouvent dans le DMG.25 pour animer un downtime.
\subsection{Déroulement}
Enquête assez longue sur les vampires, avec des questions à la famille vivant là où vivait le boucher jugé il y a une cinquantaine d'année, mais aussi aux magistrats des quartiers de Daegu, et aux archives du BRI.

Les PJs ont entendu des cloches d'alarme dans la nuit, se rendant compte rapidement qu'il s'agissait du quartier de dinae. Là-bas, le magistrat local les a informé que les gardes avaient été attaqués par une créature, sans doute la même que pour leur enquête, et que celle-ci avait été coincée dans une court, où les PJs l'ont détruit.

Le corps de la dernière victime, ainsi que de la créature, ont été ramenés au palais pour étude remise au lendemain.

Au petit matin au manoir des Shin, la nourriture a été retrouvée avec des épices qui n'avaient rien à faire dans un petit-déjeuner. Les statues du sanctuaire d'Ibarae viennent de commettre leur première 'blague' envers les PJs.
\subsection{Remarques}
Il va falloir voir au niveau du rythme des sessions, peut-être mettre une séance plus action pour la suivante.
\section{Session 9 : retour à l'action}
\subsection{Préparation}
\subsubsection{Objectifs}
Après deux séances plutôt calme, on pourrait viser quelque chose de plus actif, éventuellement un affrontement et une poursuite?

Sinon, on va continuer à poser de nouveaux arcs narratifs pour la suite. L'arc principal avancera sans doute lors de la session suivante, avec l'audience royale et la partie de chasse.
\subsubsection{Planification}
Myun-Woo et Ji-Hyo vont rapidement apprendre que le problème d'épice est étrange, le manoir voisin ayant souffert du même problème. Dans tous les cas, elles ne devraient pas trop avoir le temps de s'en occuper.

Si à un moment les PJs se reposent, on va pouvoir utiliser les informations reçues par les PJs préparées dans la description de l'acte 1 et de son niveau 2.

Nouvelle mission, priorité basse : le manoir hanté. Celle-ci devrait occuper les joueuses également quelque temps, avec deux combats et un début d'enquête.
\subsection{Déroulement}
Après un rapide downtime, les PJs ont reçu l'enquête du manoir hanté.

Elles ont commencé leur travail, trouvant facilement le souterrain. Les diablotins ont été longs à vaincre, mais ont été battus. Elles ont découvert la porte principale, qu'elles n'ont pas essayé d'ouvrir, mais elles ont identifié le mécanisme d'ouverture, et comptent revenir après l'audience pour enquêter sur ce problème, et essayer de trouver la bonne date pour ouvrir la porte normalement.
\subsection{Remarques}

Mieux ficeler les indices dans les phases d'enquête : le trajet de l'ouvrier tel que décrit était pas ouf par rapport au scénar.
\section{Session 10 : Audience royale}
\subsection{Préparation}
\subsubsection{Objectifs}
Faire rencontrer certains des personnages importants de la campagne en une séance, probablement très exposition. En cas de besoin, on peut proposer une partie de chasse pour faire rencontrer d'autres personnages, notamment les princes ? Personnages à rencontrer : le roi, au moins un des princes, la reine et ses deux enfants, Sung Ji-Yung.

ça va pas être les sessions les plus passionnantes, mais cela permettra de mettre en branle le plot principal.

Il vaudrait mieux préparer aussi un ou deux plots secondaires pour enchaîner si cette audience va très vite. 
\subsubsection{Planification}
Préambule : quelques notions d'étiquette :
\begin{itemize}
\item Lorsqu'on s'adresse au roi, on l'appelle 'votre majesté'
\item Les pétitionnaires doivent s'avancer en ne regardant pas le souverain, et s'incliner à 50 pas. Si invités par le roi à s'approcher, ils continuent et s'arrêtent en s'inclinant à 20 pas.
\item Ne connaissant pas les phrases rituelles de la cour, on n'attends pas de leur part qu'ils y prennent part, mieux vaut limiter ses interventions.
\item Il ne faut ni interrompre le roi, ni un membre de la famille royale, ni parler si l'on y est pas convié par une question ou une requête directe.

\end{itemize}
Première période : l'audience royale. Au cours de celle-ci, le roi souhaite en premier lieu rencontrer et jauger les personnages, dont il a récemment entendu parler par plusieurs canaux. Il ne pose pas forcément de questions techniques sur leurs enquêtes, mais est là pour exprimer son appréciation de leurs efforts. Par ailleurs, il compte inviter les deux nobles à une partie de chasse le mois suivant, dans les montagnes. Une courtière inconnue des PJs (la mère de Shin myun-woo en réalité) va profiter de la question de la sécurité dans les montagnes pour demander plus de troupes pour l'armée de la bannière blanche à l'est.

Autour de l'audience, les personnages devraient rencontrer la reine et ses deux enfants, le prince héritier (aîné), ainsi que la maitresse espionne du roi. La maîtresse espionne arrivera dans l'antichambre peu après eux, prétextant une discussion avec l'intendant, et en profitant pour poser quelques questions, et indiquer aux PJs sa boutique : le rat et la fortune.

Le prince héritier sera croisé lors de la sortie des PJs, le roi et le prince commençant apparemment une discussion relativement grave sur le sujet de la défense maritime.

Enfin, la reine et ses deux enfants sont présents lors de l'audience, les deux enfants causant quelques troubles à un moment.


Cette audience toutefois ne devrait pas occuper toute la séance. Dans ce cas, les PJs ont potentiellement plusieurs options : leur enquête en cours notamment, qui devrait peut-être les mener au monastère de Kaejong. Sinon, ils pourraient tout aussi bien aller dans la boutique de Sung Ji-Yung, où ils pourront rencontrer un de ses fils.

Si c'est le cas, une attaque de tigre (p53 bestiaire) pourrait potentiellement rythmer le trajet, sans être d'une originalité exceptionnelle. Sinon, les personnages pourront simplement, après quelques jours de voyage, arriver au monastère, où ils seront reçus par l'abbesse elle-même, qui va demander une certaine persuasion pour accepter de leur donner l'accès à ces archives. Si la persuasion directe ne marche pas, elle a possiblement une épreuve à leur proposer : à déterminer rapidement, mais ça fera gagner de l'xp.

\subsection{Déroulement}
50 po au total, rapière +1, le kit de détective ont été données comme récompense, ainsi que la flute de maestro.

Mi Eun-Ju (mère de Shin) a profité de l'occasion pour lancer une pétition sur les troupes à l'est, pour augmenter les effectifs.

Les PJs ont commencé par le combat contre le Barbasu dans les tunnels, qu'ils ont vaincu avec difficulté. Après cela, l'audience royale s'est déroulée normalement, avec la mère de Shin faisant une intervention sous son déguisement habituel (plus jeune et jolie que sa vraie forme).

Après cela, les personnages sont allés au monastère de Kaejong, à 1 jour et demi de trajet, où ils ont rencontré l'abbesse, qui leur a proposé une épreuve pour mieux les observer et s'assurer qu'ils puissent accéder aux archives. Ils ont réussi cette épreuve, arrivant au sommet d'une montagne voisine où coule une source d'eau bénite. La session s'est terminé alors qu'elles partageaient le repas des moines, avec l'accord de pouvoir accéder aux cryptes du monastère où sont enfermées les informations.
\subsection{Remarques}
\section{Session 11 : archives du monastère}
\subsection{Préparation}
\subsubsection{Objectifs}
Donner aux joueurs des indices sur le culte des 5 héritiers. Notamment le nom de l'entité qu'ils vénèrent. Par ailleurs, je souhaiterai insister sur la menace représentée par ce culte et ses capacités. L'ouverture des archives pourrait bien être une occasion pour eux de tenter quelque chose, peut-être une attaque contre le monastère, avec un petit groupe qui s'infiltre directement vers les archives. On pourrait tenter un ou deux combats.
\subsubsection{Planification}
Scène d'introduction avec l'ouverture des archives de la garde royale. le processus est évidemment magique et lourdement protégé par un impressionnant réseau de sortilèges, qu'il faut désactiver. 

Puis les PJs vont pouvoir faire des recherches, ce  qui va leur prendre une bonne journée et demi. C'est ce délai qui va permettre aux cultistes de se préparer à l'assaut. Alors que les personnages trouvent leur information, une sorte d'ombre apparait, alors que la cloche d'alarme résonne dans le monastère. Les PJs vont pouvoir affronter cette créature avec l'aide de l'abbesse. Pendant ce temps, un groupe de cultistes progresse au travers du monastère, et servira de seconde rencontre non loin des archives.

Premier combat : une paire de hell hounds, l'abbesse en gère un. 100px.

Second combat : 3 bodyguards et un pridespawn (p297 bestiaire). 130xp

Bonus d'histoire : 150xp pour les informations.

Informations : le ciel recherché pour ouvrir la porte étrange serait sans doute celle de la veille du solstice d'hiver, jour sacré pour le culte. Les cultistes vénèrent Indrasenan, le destructeur banni, un puissant seigneur des enfers, autrefois un dieu, jeté des cieux par le maréchal céleste, Guang. Ces créatures à 3 têtes et aux multiples bras a juré sa vengeance sur la création toute entière.

Le Pridespawn n'attaquera pas Kil Ae, reconnaissant l'aura diabolique qui émane de celle-ci.
\subsection{Déroulement}
La séance s'est déroulée comme prévu : les personnage ont eu l'information qu'elles souhaitaient dans les archives, et ont réussi à se débarrasser du serpent infernal (hellhound reskinné), puis des attaquants mortels.

Les combats ont été très violents, mais les Pjs commencent à réfléchir aux implications de l'attaque : 
\begin{itemize}
\item Les cultistes sont relativement confiants pour attaquer une cible de manière aussi visible.
\item Les cultistes ont eu vent d'une vulnérabilité du monastère, ce qui peut laisser supposer un traître à la court par exemple.
\end{itemize}
\subsection{Remarques}
La session s'est très bien passée. peut-être profiter un peu plus de la préparation pour préparer plus de descriptions, notamment des adversaires quand il y en a, etc.
\section{Session 12 : retour du monastère}
\subsection{Préparation}
\subsubsection{Objectifs}
Donner des éléments aux PJs sur l'attaque qu'elles viennent de subir : plus de détails sur l'assaut en général, les intuitions de l'abbesse notamment. Le loot aussi, avec 18 po au total, un parchemin de sort lvl 2, 2 fioles de silversheen

Le retour à la capitale également, alors que la prochaine session verra la partie de chasse. Le retour en lui-même ne devrait pas être très complexe cela dit, mais on va commencer à bosser sur la suite des évènements dans la capitale. Le niveau 3 me paraitrait un bon moment pour amener le plot central.

Peut-être amener une intrigue secondaire ou tertiaire dans l'histoire ?
\subsubsection{La séance}
Première scène, avec l'abbesse qui s'assure que l'assaut a bien été repoussé, et qui prend des décisions pour défendre son monastère et envoyer un message au palais pour faire venir leurs mages afin de repenser au moins partiellement les défenses arcaniques de l'endroit.

Ellipse pour le retour à la capitale, qui amène à la scène 2, où les PJs présentent le message de l'abbesse au roi, qui l'accepte, et annonce l'envoi de plusieurs mages de sa cour sur place, et remercie les PJs pour leur travail à lutter contre cette secte, et l'aide qu'elles ont pu donner aux moines pour repousser l'assaut.

Puis, une phase de repos rapide pour le retour des PJs en ville. Une petite statuette de pierre apparaitra à Ji-Hyo en pleine nuit, essayant de placer un encrier en équilibre instable pour que celui-ci ne tombe. Elle vient du sanctuaire d'Ibarae et pourra y être suivi.

Si les PJs décident d'aller dans l'autel souterrain de la secte, ils vont devoir affronter ses gardiens : 2 envyspawns (80xp). Le lieu contient par ailleurs une statue du seigneur des enfers, avec trois dagues sacrificielles, les restes d'une ancienne victime, morte depuis longtemps. Loot : 20 po, une fiolde de drakeheart mutagen, 2 parchemins de lvl 1.


\subsection{Déroulement}
butin : 8 po et le glamorous buckler ont été donnés. Attention, stuff pas équilibré pour le moment, à voir pour la suite.

Séance plus ou moins comme prévu : les personnages ont quitté le monastère avec la lettre pour le roi. 

Le retour a été accompagné par la remise d'un rapport à leur supérieur, et la reprise de l'enquête sur le sanctuaire de la secte.

Avant cela toutefois, les PJs ont découvert les shikigami du sanctuaire d'Ibarae, qui ont fait passer leur message : leur vieille cloche a disparue, et il faudrait la retrouver.

Devant l'urgence de leur situation, les PJs ont décidé d'aller gérer la secte en premier lieu, mobilisant un groupe de garde pour sécuriser leurs arrières tandis qu'ils se dirigent vers les souterrain. La séance s'est terminée devant la porte scellée, les PJs s'apprêtant à l'ouvrir sans difficulté.
\subsection{Remarques}
Il va falloir planifier un peu mieux la suite de cet arc de secte, même si la suite immédiate risque plutôt d'impliquer la partie de chasse avec le roi.

Egalement, la guenaude que Ye-Mong avait détectée comme changeforme pourrait changer d'apparence (guenaude de sang), laissant une ministre morte, et possiblement des changement dans le gouvernement en réponse.
\section{Session 13 : une partie de chasse}
\subsection{Préparation}
Objectifs : faire avancer le scénario vers le pivot de l'acte 1 à 2, en amenant les plot points lors de la partie de chasse royale.

Scène 1 : affrontement contre les gardiens du repaire de la secte, face à 2 envyspawns. Cette rencontre, et la fouille du sanctuaire secret. Butin :  parchemin de dispel magic, 2 parchemins de command. Mutagène drakeheart. 

Scène 2 : Début de la partie de chasse. le roi discute très rapidement avec le prince héritier, à bonne distance, et du plan de la chasse. Les personnages vont faire parti des chasseurs du roi, qui l'accompagnent pour l'assister dans sa chasse.

Scène 3 : Traque d'un tigre, alors que le roi est accompagné par le prince cadet, lui prodiguant de nombreux conseils sur la façon de traquer avec des chasseurs. Combat contre le tigre, avec le prince et le roi. PROFILS A FAIRE : roi et prince héritier : noble et advisor respectivement, avec des arcs.

Scène 4 : Sur le retour de chasse, discussion avec Kan Jung-nam, qui souhaite discuter avec eux, car il a l'impression qu'il y a une créature au palais. Il n'a que peu d'indices, mais souhaite demander aux PJs en qui il a relativement confiance, d'être vigilant. Il peut mentionner que ses gardes ont surpris un chargement étrange se dirigeant vers le palais, avec des ingrédients de rituels maléfiques. Il ne sait pas qui devait réceptionner ce chargement(cela est une excuse, la réalité c'est la perte pour lui de deux agents dans des circonstances très étranges).

Scène 5 : De retour au camp de base, le roi leur demande de participer à la sécurité du triomphe du général, qui se déroulera deux semaines plus tard, et qui sera annoncé à son retour. Cela va déclencher les évènements menant à l'acte 2.
\subsection{Déroulement}

Séance plus courte que prévue, les PJs ayant 'juste' géré le sanctuaire, sans aborder la partie de chasse.

Infos obtenues dans le sanctuaire : présence probable d'un traître à la cour, avec une note indiquant de le joindre pour avoir l'emplacement du prochain.

Paek Sujin : déguisement de la mère de Shin, réellement décédée quelques semaines plus tôt, et officiellement décédée dans très peu de temps, quand la mère de Shin va changer de rôle pour mieux se dissimuler.
\subsection{Remarques}
\section{Session 14 : vraie partie de chasse}
\subsection{Préparations}
Scène 2 : Début de la partie de chasse. le roi discute très rapidement avec le prince héritier, à bonne distance, et du plan de la chasse. Les personnages vont faire parti des chasseurs du roi, qui l'accompagnent pour l'assister dans sa chasse. Musique tradi coréenne

Scène 3 : Traque d'un tigre, alors que le roi est accompagné par le prince cadet, lui prodiguant de nombreux conseils sur la façon de traquer avec des chasseurs. Combat contre le tigre, avec le prince et le roi. PROFILS A FAIRE : roi et prince héritier : noble et advisor respectivement, avec des arcs. Musique de 3 royaumes

Scène 4 : Sur le retour de chasse, discussion avec Kan Jung-nam, qui souhaite discuter avec eux, car il a l'impression qu'il y a une créature au palais. Il n'a que peu d'indices, mais souhaite demander aux PJs en qui il a relativement confiance, d'être vigilant. Il peut mentionner que ses gardes ont surpris un chargement étrange se dirigeant vers le palais, avec des ingrédients de rituels maléfiques. Il ne sait pas qui devait réceptionner ce chargement(cela est une excuse, la réalité c'est la perte pour lui de deux agents dans des circonstances très étranges). Musique tradi coréenne - retour

Scène 5 : De retour au camp de base, le roi leur demande de participer à la sécurité du triomphe du général, qui se déroulera deux semaines plus tard, et qui sera annoncé à son retour. Cela va déclencher les évènements menant à l'acte 2. Musique de red cliffs
\subsection{Déroulement}
Découverte de la partie de chasse, avec presque 300 personnes en incluant de nombreux gardes et serviteurs. trois groupes de chasse principaux : le roi, l'héritier et le beau-frère du roi. 

Affrontement difficile contre le tigre traqué par le roi, qui a blessé assez sérieusement celui-ci.

Retour au camp de base, discussions avec des PNJs pour leur tirer les vers du nez, notamment Kan Jung-Nam, qui les a contacté, mais a été identifié comme n'étant pas tout à fait celui qu'il prétend être. Les PJs planifient de regarder de plus prêt dans cette direction. Peut-être une révélation rapide de son rôle dans l'histoire ? Pas nécessairement, mais à étudier.
\subsection{Remarques}
PENSER A LEUR FILER AU MOINS XP POUR L'AFFRONTEMENT AVEC LE TIGRE !!! Plus la récompense royale pour l'avoir sauvé du tigre.
\section{Session 15 : préparation de sécurité}
\subsection{Préparations}
Donner XP tigre, et récompense royale (tirée du butin de leur lvl)

Au camp de base, le roi leur demande de participer à la sécurité du triomphe du général, qui se déroulera deux semaines plus tard, et qui sera annoncé à son retour. Cela va déclencher les évènements menant à l'acte 2. Musique de red cliffs

Downtime : à voir avec les joueuses e qu'elles souhaitent faire, si elles souhaitent faire quelque chose.

Enquête rapide sur les cloches : 
-Indice A : permet de suivre la cloche vers un quartier spécifique de la capitale. Des témoins ?
-indice B : permet d'identifier un symbole de profession chez le 'voleur'. Un objet qui est resté dans le sanctuaire d'Ibarae.
-Indices A et B devraient permettre de trouver le voleur, qui essaie de fuir les autorités : il a volé la cloche pour la vendre à un collectionneur, qui était prêt à payer cher pour une cloche sacrée, même brisée.
-En remontant la piste du collectionneur plus tard, les PJs pourront apprendre que celui-ci l'a faite restaurer, et serait en réalité ravi de la donner au sanctuaire, ravissant les esprits du lieu.

Préparations sécurité du triomphe militaire.
\subsection{déroulement}
butin : 50po; robe mirroir et bombe d'éclats cristallins modérés.

Début à la partie de chasse, où les PJs fouillent la tente de Kan Jung Nam et obtienne une lettre d'un de ses hommes, qui parle de la disparition d'un de ses hommes, juste avant la partie de chasse. Elle mentionne un soupçon d'effort coordonné contre lui et ses hommes. Il a fait arrêter un des contrebandiers 'au hasard' pour couvrir le fait de donner l'information, et peut-être déclencher une réaction. La fouille de la tente a été perçue comme une insulte par Kan Jung Nam, qui va améliorer sa sécurité, et est toujours suspecté par les Pjs, car il parait 'off', se déplaçant trop souplement pour sa corpulence affichée.

A part cela, les Pjs ont profité d'un petit temps de repos pour travailler également sur d'autres enquêtes, notamment celle de la cloche du sanctuaire d'Ibarae, les PJs ayant pu apprendre que celle-ci, si elle a suivi un chemin classique, a pu passer par un petit marché spécialisé, non loin des quais de la ville

Kil Ae a pu convaincre la gérante de son établissement de la laisser tenter sa chance et avoir une place lors de prochaines soirées riches à la maison de thé. Elle a pu faire des jalouses à ce sujet.

Enfin, les PJs ont été récompensé pour leur récents efforts, et ont reçu une récompense financière, mais aussi matérielle, et une nouvelle mission : participer à la sécurité de la parade militaire en l'honneur du général Kim Mae Jong. La séance s'est terminée lors de leur briefing à ce sujet, alors que les PJs ont appris également que leur seconde suspecte a été découverte, morte, le matin même, et que dès la fin de la parade, cela sera leur tâche principale.
\subsection{remarque}
Donner 100xp supplémentaires pour leurs progression sur les diverses enquêtes.

Faire une carte de Daegu.

\section{Session 16 : parade victorieuse}
\subsection{Préparatifs}
Objectifs : une session relativement charnière, puisqu'elle doit marquer le passage de l'acte 1 (intro) à 2(politique). Le prétexte est la parade militaire du général victorieux, qui est nommé maréchal. Quelques troubles (mineurs) sont à attendre, mais rien de grave, du moins, rien à part la santé du roi.
Les PJs doivent assurer la sécurité à proximité de l'estrade royale.

Mise en place de la sécurité:
La garde royale carbure à plein régime dès l'aube, pour une parade occupant essentiellement l'après-midi. Alors que les ouvriers terminent l'estrade royale à côté de la porte des princes, les gardes s'assurent que tout se passe dans le bon ordre, et que rien ne peut menacer sa majesté : les maisons voisines sont fouillées, les passants sont surveillés. Les PJs participent à cela, avec un rôle de vérification que rien ne touche l'estrade.

Puis, vers le milieu de journée, la foule commence à grandir le long des rues, des troupes de l'armée du centre viennent renforcer les rangs de la garde municipale pour maintenir ouverte la rue.Une petite marchande reste dans le coin, et un de ses fils est assigné aux PJs comme assistance militaire. La tension monte gentiment sur la garde royale, alors que les princes et la reine prennent place, suivi par le roi et la plupart des ministres.

La parade démarre alors, ouverte par les éclaireurs du col de Mungyeong, unité d'élite de l'armée blanche de l'est. Ils sont suivis par le général, qui passe une première fois à cheval devant la tribune royale, et qui doit être nommé maréchal plus avant dans les festivités, quand une bonne part de ses forces seront rassemblées sur la place d'arme. Son passage occasionne quelques troubles, alors qu'un petit groupe tente de franchir le cordon de garde pour se plaindre de cet officier.

Viennent ensuite les différentes unités de l'armée blanche, chacune avec un contingent de troupe menée par son étendard : la cavalerie ouvre la marche, puis l'infanterie, les tireurs, et enfin, l'artillerie.

A ce moment, un PJ attentif pourrait remarquer que le roi ne semble pas très opérationnel.

Les trophées forment la suite du cortège : des douzaines de bannières prises à l'ennemi, ainsi que les files de prisonniers de guerre qui seront vendus comme esclaves. Le général fait également parader les symboles de commandement du général vaincu, capturés avec son camp.

Enfin, les héros de guerre défilent en groupes par type d'unité, et chacun est appelé pour recevoir sa récompense devant la tribune royale. le dernier est le général, qui reçoit le titre de maréchal, ainsi qu'un dragon mineur en tant que monture. C'est à ce moment que la situation devient chaotique : le roi s'effondre, retenu avec peine par ses gardes rapprochés. 

Le premier problème des PJs va être de se frayer un chemin dans la foule et les courtiers qui commencent à paniquer. Il faudra sans doute jouer des coudes, voir dégainer, avec un risque de mouvement de foule. 

Une fois arrivées sur l'estrade, elles pourront voir les gardes du corps former les rangs et rassembler la garde royale pour évacuer au plus vite le roi vers le palais. Le roi ne semble pas saigner et est conscient, mais a l'air perdu.

Les troupes royales ainsi formées vont devoir progresser au plus vite dans la ville, quitte à se frayer brutalement un chemin au travers de la foule en cas de besoin (avec des conséquences dramatiques pour la suite).

idée : avoir une tentative d'assassinat, organisée par le vampire(?) contre le maréchal, un peu plus tard dans la journée bien chargée des PJs

\subsection{Déroulement}
350xp
Dans l'ensemble les choses se sont passées comme prévue : le défilé s'est passé sans grand heurt, les PJs apprenant que le maréchal a possiblement des cadavres dans le placard (figurativement parlant). Le malaise du roi est arrivé avec une relative surprise, et l'équipe a pu limiter au maximum les dangers pour la population, en participant à l'évacuation du roi jusqu'au palais sans grand troubles supplémentaires. Ils commencent à découvrir que la situation pourrait bien devenir explosive, voire très explosive.
\subsection{Remarques}
\section{Session 17 : conséquences}
Situation initiale : la capitale est en émoi, alors que le roi se remet de son malaise. Des informations sur la situation vont commencer à diffuser à la court, mais pas immédiatement. Pendant ce temps, le BRI continue ses missions de sécurité, avec notamment l'enquête sur la mort de l'administratrice qu'ils soupçonnaient.

Nouvelles et rumeurs sur le roi et la politique:
\begin{itemize}
\item Le fils cadet est au chevet de son père, le fils aîné est parti rapidement vers le monastère de Kaejong (vrai) 
\item Le clan de la reine commence déjà à rassembler des soutiens (partiellement faux : la reine le fait, pas son frère).
\item La court royale ne fait pas confiance au prince héritier (faux, il a des soutiens)
\item Le roi va bien en réalité, et la garde royale va bientôt arrêter ceux qui commencent à bouger pour rassembler des soutiens (faux)
\item C'est sans doute un empoisonnement, peut-être le travail d'espions Han (faux)
\end{itemize}

Ses appartements sont très calme, le corps, déjà bien décomposé étant partiellement dissimulé sous le parquet (deux serviteurs s'occupant de nettoyer le lieux l'ont trouvée là). Les serviteurs n'avaient pas accès à la pièce.

Indices de la scène : le corps est peu décomposé, et a été presque entièrement écorché. Il n'y a pas de trace de sang sur place toutefois. Un certains nombre d'affaires personnelles ont été posées en vrac dans un coin de la pièce. Une broche semble étrange parmi tout cela : c'est une broche masculine (appartenant à sa prochaine victime/son corps supplémentaire). Les serviteurs qui ont trouvé le corps peuvent indiquer qu'ils l'ont trouvé la veille, alors qu'on leur avait ordonné d'aller vérifier si tout allait bien pour elle, car elle n'était pas présente pour ses devoirs de la court. Un garde royal les accompagnait et peut confirmer leurs dires. Si on les interroge suffisamment loin d'un garde royal, les serviteurs mentionnent qu'ils pensent que cette aile du palais est sans doute maudite, plusieurs serviteurs étant morts récemment après y avoir été affectés. En cherchant bien, on peut trouver une sorte de poussière noire sous le lit, qui peut être analysée comme de l'Onyx.

Si les PJs en le trouve pas, ils seront rapidement informés par des serviteurs que des glyphes magiques ont été trouvés derrière certains des meubles de la pièce. Ceux-ci correspondent à des rituels de nécromancie et de divination. 

L'objectif de cette session est qu'au maximum, les PJs puissent détecter la présence d'une guenaude sanglante au palais, mais PAS qu'ils puissent la trouver : celle-ci est encore clairement un adversaire trop important.

En plus de cette enquête, plusieurs évènements peuvent se produire au palais et en ville:
\begin{itemize}
\item Un messager du palais transmet un message : la maîtresse espionne du roi souhaite rencontrer les PJs en ville, dans sa boutique. Celle-ci souhaite discuter de la situation, si les PJs ont bien compris la dangerosité de celle-ci (elle ne mentionnera pas les risques avec le clan de la reine, connaissant la loyauté potentielle de Ye Mong). Elle peut faire assigner un de ses fils au groupe des PJs en renfort, il appartient à l'armée du centre, en tant qu'artilleur. Elle ne révèle pas clairement son rôle, mais ne laissera pas forcément de doutes à ce sujet : elle a besoin d'agents efficaces sur le terrain, et elle pense pouvoir hijack les PJs sur ce point.
\item Les PJs tombent sur un groupe d'habitants de Dargu, qui souhaitent demander officiellement l'ouverture d'une enquête pour corruption contre une des ministres au palais, qui travaille avec différents groupes criminels. Il faudra sans doute demander une enquête préliminaire.
\item La supérieure des PJs leur offre une promotion de 2 rangs, effective immédiatement, et ainsi, le service d'une seconde équipe d'inspecteurs juniors pour la réalisation des enquêtes préliminaires et des follow-ups.
\end{itemize}
\section{Déroulement}
Démarrage dans le vif du sujet, avec l'exploration des quartiers de Paek Su-Jin (la suspecte décédée), qui leur a pris un certain temps.
Infos obtenues : le cadavre était sous le plancher, il y avait des runes rouges sur un mur derrière une étagère, de la poudre d'onyx, des documents et une broche en forme de dragon dans un tas. 

Avec ces informations, les PJs ont commencé à élaborer plusieurs hypothèses sur la situation, et Sung ji-Yung a souhaité discuter avec eux pour leur proposer un accord. Son fils accompagne donc le groupe, et les deux partis échangent leurs informations. Il ne me semble pas que les PJs aient bien l'info de son statut de maîtresse espionne, elle pourra jouer à ce sujet. On pourra aussi s'en servir pour potentiellement relancer une piste envers les joueurs.

Les PJs ont ensuite pû faire l'autopsie du corps : celui-ci est décédé il y a un moment déjà, et n'a plus de peau. Les conclusions temporaires des PJs sont les suivantes : une créature (oui), a pris la place de Paek Su-Jin (oui), puis de Kam Jung-Nam (non). Ils ont l'info qu'il existe bien des créatures capables de prendre l'apparence de leur victime via leur peau, et qu'elles se nourrissent de sang.

Enfin, la séance s'est arrêtée alors qu'une alarme sonne au palais, et que de la fumée se répand : les runes ont été touchées, et se révèlent être un piège.

Les PJs passent aussi au niveau 4 : il faudra sans doute s'arranger pour leur donner très vite le butin lvl3 manquant et préparer le butin lvl4 pour la suite
\section{Remarques}
Préparer des utilisations potentielles des ysoki, préparer le butin lvl 4 rapidement, et planifier un peu plus en amont les arcs narratifs de l'acte 4 pour commencer à gentiment les démêler avec les PJs.


\section{Session 18}
\subsection{Préparation}

\chapter{Profils de la campagne}

\section{Chef des cultistes (lvl 6)}
Alignement : chaotique mauvais \\
Attributs : FOR +2, DEX +3, CON +4, SAG +2, CHA +5, INT +4\\
Perception : +11\\
Skills : intimidation +15, duperie +15, diplomatie +13, discrétion +11\\
Items à définir\\
AC 23, VIG +11, VOL +17, REF +14\\
PVs : 80\\
Speed 25\\
Attaque : +13, 2d4+7\\
Sorts à définir. DC 23, bonus +15
\section{Ministre corrompu (lvl 7)}

\chapter*{Annexes}
\section*{Listes aléatoires}
\paragraph{Noms masculins (à garder à jour)}
\begin{verbatim}

Hwan Song-Jin
Ch'o Kyung-Mo
Man Jun-Yeong
Ok Woo-Sung
Nae Ji-Hu
Mun Byung-Ho
Sung Joon-Ho

Tongbang Seon
\end{verbatim}
\paragraph{noms féminins (à garder à jour}
\begin{verbatim}
T'ae Han-Bi
Uh Ah-Hyun
Ka Ha-Sun
Paek Sujin
P'yo Hee-Ae
Mi Eun-Ju
Ryong Gri-Na
Pi Young-Ah
An Yong-Suk
\end{verbatim}
\paragraph{Caractéristiques marquantes pour des PNJs}
\begin{itemize}
\item Très poli.e, trop même
\item Parle tout bas
\item Fascination pour une caractéristique spécifique d'un des PJs.
\item A le mauvais oeil
\item une jambe de bois ou prothèse mécanique. Si noble, la magie est aussi possible
\item Jeune marié.e
\item cheveux étranges : une mèche de cheveux a une couleur étrange, ou peut-être une décoloration?
\end{itemize}
\paragraph{Noms de boutiques}
\begin{itemize}
\item Chez grand-père Guang : armurerie très en vue de la capitale
\item Le marché des curiosité : enseigne d'objets étranges et parfois magiques.
\end{itemize}
\paragraph{Musiques utilisées}
\begin{itemize}
\item \begin{verbatim}https://www.youtube.com/watch?v=72JKkL6qSeg&list=PLQxxaVZxxtV3IftEeLXr_9Lul6Yi7iujh \end{verbatim}
\item \begin{verbatim}https://www.youtube.com/playlist?list=PL0c26ZhRmZb0TiUl169gUCrnUDDk7XKN8\end{verbatim} valide, peut-être trop de chants
\item \begin{verbatim}https://www.youtube.com/playlist?list=PLQeZIeLOTDJJ0cZccfI9vHFSgHF98qpCE\end{verbatim} valide, peut-être trop de chants
\item ost du film hero : très efficace, et suffisamment variée pour ne pas poser de soucis une fois en boucle sur une session de 2h30.
\end{itemize}

\end{document}
