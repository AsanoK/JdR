\documentclass[10pt,a4paper]{book}
\usepackage[utf8]{inputenc}
\usepackage[french]{babel}
\usepackage[T1]{fontenc}
\usepackage{amsmath}
\usepackage{amsfonts}
\usepackage{amssymb}
\author{ Antoine Robin}
\title{Campagne Pathfinder - Garde Royale}
\begin{document}
\documentclass[10pt,a4paper]{book}
\usepackage[utf8]{inputenc}
\usepackage[french]{babel}
\usepackage[T1]{fontenc}
\usepackage{amsmath}
\usepackage{amsfonts}
\usepackage{amssymb}
\author{ Antoine Robin}
\title{Garde Royale : guide de campagne}
\begin{document}
\maketitle
\tableofcontents
\chapter{Arcs narratifs}
\section{Arc primaire : querelle de succession}
L'arc principal de la campagne se fera en trois actes, tous orienté en fonction de l'état de santé du souverain, et des problèmes importants de successions qui s'ensuivent.

Le premier acte est un acte de présentation du setting, de découverte. Les personnages croisent les différentes factions politiques et ont l'occasion d'en apprendre un peu plus sur celles-ci.

L'acte deux est le moment où l'action démarre vraiment : le roi révèle avoir une grave maladie, et les questions de successions se posent de manière pressante : toutes les factions commencent à placer leurs pions et jouer leurs cartes les plus subtiles.

L'acte trois démarre à la mort du roi, probablement de sa maladie. Cela déclenche les pires troubles, avec une vraie menace de guerre civile.

La première faction importante est la plus traditionaliste, et soutient le premier prince héritier dans son accession au trône. Cette faction est essentiellement constitué autour du prince lui-même et d'un noyau de hauts fonctionnaires.

La seconde faction correspond aux réformateurs, menés par le second prince, et qui aurait sans doute la préférence du roi. La faction se constitue autour du prince cadet, ainsi que plusieurs fonctionnaires montant, qui ont un contrôle effectif de certains bureaux importants.

La troisième faction est celle du général Kim Mae-Jun, qui remporte une victoire importante contre les Hans, et est universellement reconnu comme un officier très compétent. Il voit la princesse comme sa clé vers le trône. Autour de lui, quelques officiers, et un certain soutien populaire dû à ses récentes victoires.

La quatrième faction est celle de la reine : celle-ci réalise la précarité de sa position et de celles de ses enfants, en particulier si un de leurs demi-frères monte sur le trône. Elle essaiera donc de s'en emparer pour eux. Pour se faire, elle profite de la richesse et de l'influence de sa famille, qui a depuis longtemps des connections dans la noblesse du royaume.

La cinquième faction, mais qui ne vise pas franchement le trône est celle d'une révolte paysanne dans les provinces du nord. Ils espèrent profiter du chaos de la succession pour faire sécession efficacement. Ils sont soutenus par certains administrateurs venus du nord, venus d'anciennes familles aristocratiques de la région.
\subsection{Acte 1 : le puissant royaume de Choseon}
Les intrigues dans cet acte sont relativement peu nombreuses, étant plus de l'ordre de l'accroche scénaristique avec la présentation de certaines factions.

Cet acte amène à une cérémonie officielle importante, au cours de laquelle démarre l'acte 2. Cette cérémonie doit être foreshadowed au cours de l'acte.
\subsection{Acte 2 : la maladie du roi}
Cet acte démarre vraiment la course à la couronne : au cours de la cérémonie qui clôture l'acte 1, le roi fait un malaise. Rien de grave, dit le palais, mais la situation précaire n'échappe à personne. 

Différentes factions vont donc essayer de se positionner le mieux possible pour s'emparer de la couronne. La première à agir sera celle du prince héritier, qui rentrera à la capitale pour s'enquérir de la santé de son père. A ce stade, les autres factions amassent de l'influence et battent le rappel des troupes.
\subsection{Acte 3 : une succession douloureuse}
\section{Arcs secondaires}
\subsection{Des disparitions mystérieuses}
Dans les bas-fonds de Daegu, des disparitions inquiétantes sont notées. D'abord très discrètes, car visant essentiellement des esclaves et le bas peuple, leur nombre finit par être important, et les gardes de la ville s'inquiètent des corps qu'ils trouvent dans les caniveaux, au point d'hésiter à quitter leurs postes de gardes.

Il s'agit d'un groupe de vampires 'sauvages' : l'ancien est malin et prudent, s'attaquant à des victimes isolées, et disparaissant rapidement. Ses deux rejetons par contre, sont complètement sauvages et tuent de manière beaucoup plus fréquente, sans prendre leurs précautions dans le choix des victimes et le timing.

Les personnages devraient d'abord trouver les deux rejetons, ce qui devrait considérablement limiter le nombre de victimes. Puis, l'ancien pourra chercher à se venger, ou finira par lui aussi commettre des erreurs importantes, ce qui déclenchera une seconde chasse par les PJs.
\subsection{Les sept sceaux de ???}
Un des plus grands sanctuaires de la ville a été bâti pour enfermer et maintenir scellé à jamais un mal ancien. Ce rôle a été oublié depuis longtemps par la majeure partie du public, mais les prêtres importants de Maegu le savent, ainsi que le roi.

Leur ennemi ici est un groupe de cultistes qui pensent pouvoir réveiller ce mal ancien pour ensuite profiter de ses pouvoirs pour leurs besoins personnels.

\subsection{La frontière orientale}
Une des raisons de quitter la capitale : les personnages pourront être envoyés transmettre des ordres et surveiller ce qui se passe sur la frontière orientale avec le Han : le conflit avec l'un des royaumes voisin empire rapidement, et le maréchal Mae-Jun mène les opérations.

Une première possibilité serait d'aller lui transmettre un ordre royal, et en profiter pour enquêter sur des problèmes au sein de l'armée de l'est.

Alors que le conflit empire, les personnages pourraient devoir y retourner afin de continuer une enquête, ou de suivre un fugitif qui tente de passer la frontière.

\subsection{Le monastère de gaideon}
Ce monastère pourra intervenir de plusieurs façons dans la campagne :
de part son importance politique, ses archives utiles, ou encore quelques menaces qui peuvent essayer de s'en prendre à ce lieu.
\section{Arcs tertiaires}

\section{Arcs personnels}
\subsection{Personnage de Tara : une guenaude au palais}
Une des nobles du royaume est en réalité une guenaude, qui essaie de mettre en place un couvent au sein de l'administration et de l'aristocratie de Choseon.

Il s'agit de la mère du personnage de Tara, et elle profitera du chaos de la succession pour lancer l'Appel sur certaines de ses filles, qu'elle compte bien placer au mieux par la suite.

\subsection{Personnage de Chloé}
\subsection{Personnage de Dorine}
\subsection{Personnage de Diane}
\subsection{Personnage de Morgane}
\chapter{Règles maisons}
\section{Armes à feu}
\chapter{PNJs}
\section{La cour royale}
\section{Les rues de Daegu}
\chapter{Éléments de setting}

\end{document}
\end{document}