\documentclass[10pt,a4paper]{book}
\usepackage[utf8]{inputenc}
\usepackage[french]{babel}
\usepackage[T1]{fontenc}
\usepackage{amsmath}
\usepackage{amsfonts}
\usepackage{amssymb}
\author{ Antoine Robin}

\title{Garde Royale : guide de campagne}
\begin{document}
\maketitle
\tableofcontents
\chapter{Arcs narratifs}
\section{Arc primaire : querelle de succession}
L'arc principal de la campagne se fera en trois actes, tous orienté en fonction de l'état de santé du souverain, et des problèmes importants de successions qui s'ensuivent.

Le premier acte est un acte de présentation du setting, de découverte. Les personnages croisent les différentes factions politiques et ont l'occasion d'en apprendre un peu plus sur celles-ci.

L'acte deux est le moment où l'action démarre vraiment : le roi révèle avoir une grave maladie, et les questions de successions se posent de manière pressante : toutes les factions commencent à placer leurs pions et jouer leurs cartes les plus subtiles.

L'acte trois démarre à la mort du roi, probablement de sa maladie. Cela déclenche les pires troubles, avec une vraie menace de guerre civile.

La première faction importante est la plus traditionaliste, et soutient le premier prince héritier dans son accession au trône. Cette faction est essentiellement constitué autour du prince lui-même et d'un noyau de hauts fonctionnaires.

La seconde faction correspond aux réformateurs, menés par le second prince, et qui aurait sans doute la préférence du roi. La faction se constitue autour du prince cadet, ainsi que plusieurs fonctionnaires montant, qui ont un contrôle effectif de certains bureaux importants.

La troisième faction est celle du général Kim Mae-Jun, qui remporte une victoire importante contre les Hans, et est universellement reconnu comme un officier très compétent. Il voit la princesse comme sa clé vers le trône. Autour de lui, quelques officiers, et un certain soutien populaire dû à ses récentes victoires.

La quatrième faction est celle de la reine : celle-ci réalise la précarité de sa position et de celles de ses enfants, en particulier si un de leurs demi-frères monte sur le trône. Elle essaiera donc de s'en emparer pour eux. Pour se faire, elle profite de la richesse et de l'influence de sa famille, qui a depuis longtemps des connections dans la noblesse du royaume.

La cinquième faction, mais qui ne vise pas franchement le trône est celle d'une révolte paysanne dans les provinces du nord. Ils espèrent profiter du chaos de la succession pour faire sécession efficacement. Ils sont soutenus par certains administrateurs venus du nord, venus d'anciennes familles aristocratiques de la région.
\subsection{Acte 1 : le puissant royaume de Choseon}
Les intrigues dans cet acte sont relativement peu nombreuses, étant plus de l'ordre de l'accroche scénaristique avec la présentation de certaines factions.

Avant de lancer la cérémonie, les différentes factions suivantes doivent être au moins vaguement présentées :
\begin{itemize}
\item Le premier ministre qui soutient l'héritage par l'aîné
\item Le prince cadet, qui s'entend beaucoup mieux que son frère avec le roi
\item Des nouvelles du général Kim Mae-Jun et de ses victoires dans l'est.
\item La présence du troisième 
\end{itemize}

Cet acte amène à une cérémonie officielle importante, au cours de laquelle démarre l'acte 2. Cette cérémonie doit être foreshadowed au cours de l'acte.
\subsection{Acte 2 : la maladie du roi}
Cet acte démarre vraiment la course à la couronne : au cours de la cérémonie qui clôture l'acte 1, le roi fait un malaise. Rien de grave, dit le palais, mais la situation précaire n'échappe à personne. 

Différentes factions vont donc essayer de se positionner le mieux possible pour s'emparer de la couronne. La première à agir sera celle du prince héritier, qui rentrera à la capitale pour s'enquérir de la santé de son père. A ce stade, les autres factions amassent de l'influence et battent le rappel des troupes.
\subsection{Acte 3 : une succession douloureuse}
\section{Arcs secondaires}
\subsection{Des disparitions mystérieuses}
Dans les bas-fonds de Daegu, des disparitions inquiétantes sont notées. D'abord très discrètes, car visant essentiellement des esclaves et le bas peuple, leur nombre finit par être important, et les gardes de la ville s'inquiètent des corps qu'ils trouvent dans les caniveaux, au point d'hésiter à quitter leurs postes de gardes.

Il s'agit d'un groupe de vampires 'sauvages'(vrykolakas, ou spawns de vampires normaux) : l'ancien est malin et prudent, s'attaquant à des victimes isolées, et disparaissant rapidement. Ses deux rejetons par contre, sont complètement sauvages et tuent de manière beaucoup plus fréquente, sans prendre leurs précautions dans le choix des victimes et le timing.

Les personnages devraient d'abord trouver les deux rejetons, ce qui devrait considérablement limiter le nombre de victimes. Puis, l'ancien pourra chercher à se venger, ou finira par lui aussi commettre des erreurs importantes, ce qui déclenchera une seconde chasse par les PJs.


\subsection{Le monastère de Kaejong}
Ce monastère pourra intervenir de plusieurs façons dans la campagne :
de part son importance politique, ses archives utiles, ou encore quelques menaces qui peuvent essayer de s'en prendre à ce lieu.

Il a été bâti pour enfermer et maintenir scellé à jamais un mal ancien. Ce rôle a été oublié depuis longtemps par la majeure partie du public, mais les prêtres importants de Maegu le savent, ainsi que le roi. 

Leur ennemi ici est un groupe de cultistes qui pensent pouvoir réveiller ce mal ancien pour ensuite profiter de ses pouvoirs pour leurs besoins personnels.

Ces cultistes pourront commencer par inciter des bandits de la région à s'en prendre au monastère, sans grand effet, mais cela pourrait préoccuper les moines. Ils doivent également pratiquer leurs rites impies, impliquant potentiellement des sacrifices humains.

Leur conspiration principale sera par contre de profiter du chaos de leurs autres tentatives pour avancer un de leurs hommes à l'intérieur du monastère, qui sera ensuite attaqué pour libérer la créature. Les personnages pourront être envoyés pour les arrêter.
\subsection{La frontière orientale}
Une des raisons de quitter la capitale : les personnages pourront être envoyés transmettre des ordres et surveiller ce qui se passe sur la frontière orientale avec le Han : le conflit avec l'un des royaumes voisin empire rapidement, et le maréchal Mae-Jun mène les opérations.

Une première possibilité serait d'aller lui transmettre un ordre royal, et en profiter pour enquêter sur des problèmes au sein de l'armée de l'est.

Alors que le conflit empire, les personnages pourraient devoir y retourner afin de continuer une enquête, ou de suivre un fugitif qui tente de passer la frontière.

Egalement, l'un des royaumes successeurs de l'empire Han a envoyé un groupe d'espion. La première piste sera la capture d'un agent de liaison, qui faisait le trajet d'un côté à l'autre des montagnes pour rendre compte à ses maîtres. Il faudra remonter sa piste, pour finalement arriver jusqu'à un petit groupe d'administrateurs qui reçoivent un paiement en échange d'informations, notamment sur les places-fortes de la frontière.


\subsection{Réarmement}
Les personnages sont envoyés enquêter sur un chariot d'armes qui a été intercepté par la garde de Maegu. Il faudrait remonter la piste de celui-ci pour savoir à quoi il correspond, et vers qui il se dirige.

Une enquête en plusieurs parties, essentiellement en acte 1.

Les commanditaires de ces marchandises sont les groupes criminels de la capitale, qui cherchent un avantage sur leurs rivaux. Evidemment, ils ne souhaitent pas voir la garde regarder de trop près dans leurs affaires.

Première étape d'enquête : déterminer vers qui vont ces armes, avec deux indices principaux, que sont le chariot lui-même, ainsi que son conducteur. L'interrogatoire peut se faire de différentes manières, et inspecter le chariot et surtout ce que les gardes peuvent dire, peut donner une idée d'où il allait. Il faut ensuite poser des questions sur place (le quartier près du fleuve), pour être guidé, plus ou moins volontairement, vers un chef de gang local, Pak Megujin. Celui-ci ira sans doute à la confrontation contre les PJs, qui pourront trouver une seconde cargaison d'armes aux mains de ses hommes.

La seconde étape, un peu plus tard, sera d'identifier d'où viennent ces armes, apparemment militaires dans leur origine. Cette enquête se déroulera à Hwaseong, où il faudra trouver des preuves de la corruption de certains militaires. 
\section{Arcs tertiaires}
\subsection{Les statues du sanctuaire d'ibarae}
Un sanctuaire en ville semble avoir de plus en plus de statues. Mais personne ne semble trop savoir d'où elles viennent, et puisque personne n'est vraiment dérangé, la question de ce dont il s'agit reste entière.

Il s'agit d'un esprit plutôt bénin (une fée en termes de règles), qui se façonne de la compagnie, et les anime parfois pour s'amuser. 
\subsection{La partie de chasse}
Les personnages sont invités à participer à une partie de chasse : l'idée est de passer une ou deux journée à traquer un tigre, et d'en profiter pour discuter tranquillement avec la personne qui les invite.

\subsection{Le festival du dragon}
Les festivités du nouvel ans se préparent dans la plus grande effervescence : il s'agit pour les grandes villes de s'attirer la présence et les faveurs des dragons !

Les jeunes dragon, ou imugi sont assez communs, mais en ces temps de crises, la présence d'un grand dragon (yong), pourrait rassurer la population, ou peut-être ajouter un nouveau joueur à l'échiquier politique.
\subsection{Le mort sans repos}
Le bureau d'investigation est chargé d'une affaire privée, mais sérieuse : ils doivent se débarrasser du mari d'une puissante ministre. Toutefois, celui-ci est déjà mort il y a deux semaines, mais s'est relevé pendant la nuit suivante, et rôde maintenant dans leu demeure.

\section{Arcs personnels}
\subsection{Personnage de Tara : une guenaude au palais}
Une des nobles du royaume est en réalité une guenaude, qui essaie de mettre en place un couvent au sein de l'administration et de l'aristocratie de Choseon.

Il s'agit de la mère du personnage de Tara, et elle profitera du chaos de la succession pour lancer l'Appel sur certaines de ses filles, qu'elle compte bien placer au mieux par la suite.

\subsection{Personnage de Chloé}
\subsection{Personnage de Dorine}

\subsection{Personnage de Diane}
\subsection{Personnage de Morgane}
\chapter{Règles maisons}
\section{Armes à feu}
Deux types d'armes à feu sont surtout présentes à Choseon :La lance de feu (seungja), constituée d'un canon au bout d'un long manche, tirant une charge de shrapnels dans une direction. Mortelle à courte portée .L'arquebuse,ou jochong, permettant un tir plus précis à longue portée, est un ajout plus récent venant des armées Hans.

\flushleft
\begin{tabular}{c c c c c c c p{0.1\textwidth} p{0.2\textwidth}}
Nom & Prix & dégâts & portée & recharge & enc & mains & groupe & traits \\
Seungja & 15po & d8P & ligne 35ft & 2 & 2 & 2 & gun martial & attachée (bâton), mortel(d10), misfire \\
Jochong & 20po & d10P & 60ft & 2 & 2 & 2 & gun martial & mortel(d10), misfire\\
\end{tabular}


\paragraph{Nouveau trait d'arme : misfire}
Lors d'une attaque avec une arme pouvant misfire, si le dé est un 1 naturel, le tir fait long feu et ne part pas. Cela implique de nettoyer l'arme avant de la recharger. Mécaniquement, cela oblige à utiliser l'action 'gérer un long-feu' avant de pouvoir tirer ou recharger.
Action :
Gérer un long feu : 1 action
Si le personnage réussi un test de crafting DD15 ou de lore(armes à feu) DD10, il peut à nouveau recharger celle-ci et tirer avec. Sinon, il faudra retenter cette action.

\section{Langues}

Les langues utilisées dans le livre de base ne seront pas utilisée dans la campagne : pas de nain, d'elfe, de commun, ou autre langue dépendant de l'ascendance.
En remplacement les règles suivantes:
\begin{itemize}
\item Tout les personnages parlent le Choseon à la place du commun, et remplacent toutes les langues qu'ils devraient obtenir par une langue de la liste ci-dessous.
\item En terme de règle, savoir parler une langue et savoir l'écrire nécessite de dépenser deux langues : une pour l'oral, une pour l'écrit. 
\end{itemize}


Liste des langues parlées : Choseon, Han, Junkan, Ihlan, Jeju (dialecte du nord de Choseon, assez différent)

Liste des langues écrites : Choseon, Choseon classique, Han, Han classique (langue des érudits), Junkan, Ihlan, Jeju, et différents codes secrets(à définir).

\chapter{PNJs}
\section{La cour royale}
\subsection{Le roi Song Su Yun}
Le roi de Choseon depuis 23 ans. Il est gravement malade, mais pour le moment n'en a pas conscience
\subsection{La reine Yi Il Sho}
La seconde épouse du roi Yun, issue du puissant clan Sho
\subsection{Le prince héritier Song Cheol Shin}
Militaire, a passé plusieurs années dans le sud du pays à diriger des opérations de la flotte Bleue
\subsection{Le prince Song Cheol Ahn}
Un érudit plutôt calme, il s'entend très bien avec son père
\section{Les rues de Daegu}
\chapter{Éléments de setting}
\section{Familles nobles}
\subsection{Le clan Yi}
En vérité un groupe de familles nobles apparentées, il est relativement puissant, ayant fourni au cours des derniers siècles plusieurs maréchaux et de nombreux administrateurs, ainsi qu'entre autre, la reine actuelle, Yi il sho. C'est une influence importante dans le le royaume, contrôlant de nombreuses terres dans les ports du sud.
\subsection{la famille Jien}
Famille noble récente, d'origine Han, elle n'a fournit qu'un seul administrateur, qui est toujours à sa tête : Jien jeong-hun. Celui-ci est un réformateur assumé, et le domaine de sa famille est une villa plutôt modeste, en bordure de la capitale.
\subsection{ la famille Paek}
Famille plus ancienne que fortunée, elle ne fournit pas d'officiel à chaque génération, et a plusieurs fois été proche de repasser au sein du peuple à cause de cela. Elle reste une famille plutôt ancienne dans la capitale, ayant fournit notamment de nombreux officiers à l'armée (aujourd'hui plusieurs des fils de la famille servent au nord et à l'ouest).
\chapter{Déroulement de campagne}
\section{Acte 1 : niveau 1-5 ?}
Les personnages, pour leur première enquête, vont être envoyés sur une histoire de trafic d'armes : la garde d'une des portes a trouvé un chariot rempli d'arme quittant la ville, et a arrêté son conducteur. Le responsable du BRI demande de trouver rapidement pour qui étaient ces armes et quel était l'objectif de leur trafic. Trouver leur origine sera l'étape suivante, pour éviter que ce genre d'objets ne se retrouve partout (mais il devrait manquer un indice important à ce moment pour ce faire). 
\emph{Arc secondaire réarmement, suite à prévoir avec de nouveaux indices qui arrivent au BRI}
\section{Acte 2 : niveau 6-10?}

\section{Acte 3 : niveau 11 - 15 ?}
\chapter{Notes de séance}
\section{Session 0}
Présentation du setting, et réflexion sur le groupe. Sont choisis :
\begin{itemize}
\item Un tengu alchimiste
\item Une barde tieffeline
\item Une duelliste changeline
\item Une elfe oracle ?
\item Une guerrière ????
\end{itemize}
La création des détails mécaniques et des backgrounds sont laissés aux joueuses en attendant la session 1.
\section{Session 1}
\paragraph{Date}12.12.2020
\paragraph{Déroulement}
\paragraph{Améliorations}
\chapter{Profils de la campagne}

\end{document}
