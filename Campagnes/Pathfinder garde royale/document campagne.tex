\documentclass[10pt,a4paper]{book}
\usepackage[utf8]{inputenc}
\usepackage[french]{babel}
\usepackage[T1]{fontenc}
\usepackage{amsmath}
\usepackage{amsfonts}
\usepackage{amssymb}
\author{ Antoine Robin}

\title{Garde Royale : guide de campagne}
\begin{document}
\maketitle
\tableofcontents
\chapter{Arcs narratifs}
\section{Arc primaire : querelle de succession}
L'arc principal de la campagne se fera en trois actes, tous orienté en fonction de l'état de santé du souverain, et des problèmes importants de successions qui s'ensuivent.

Le premier acte est un acte de présentation du setting, de découverte. Les personnages croisent les différentes factions politiques et ont l'occasion d'en apprendre un peu plus sur celles-ci.

L'acte deux est le moment où l'action démarre vraiment : le roi révèle avoir une grave maladie, et les questions de successions se posent de manière pressante : toutes les factions commencent à placer leurs pions et jouer leurs cartes les plus subtiles.

L'acte trois démarre à la mort du roi, probablement de sa maladie. Cela déclenche les pires troubles, avec une vraie menace de guerre civile.

La première faction importante est la plus traditionaliste, et soutient le premier prince héritier dans son accession au trône. Cette faction est essentiellement constitué autour du prince lui-même et d'un noyau de hauts fonctionnaires.

La seconde faction correspond aux réformateurs, menés par le second prince, et qui aurait sans doute la préférence du roi. La faction se constitue autour du prince cadet, ainsi que plusieurs fonctionnaires montant, qui ont un contrôle effectif de certains bureaux importants.

La troisième faction est celle du général Kim Mae-Jun, qui remporte une victoire importante contre les Hans, et est universellement reconnu comme un officier très compétent. Il voit la princesse comme sa clé vers le trône. Autour de lui, quelques officiers, et un certain soutien populaire dû à ses récentes victoires.

La quatrième faction est celle de la reine : celle-ci réalise la précarité de sa position et de celles de ses enfants, en particulier si un de leurs demi-frères monte sur le trône. Elle essaiera donc de s'en emparer pour eux. Pour se faire, elle profite de la richesse et de l'influence de sa famille, qui a depuis longtemps des connections dans la noblesse du royaume.

La cinquième faction, mais qui ne vise pas franchement le trône est celle d'une révolte paysanne dans les provinces du nord. Ils espèrent profiter du chaos de la succession pour faire sécession efficacement. Ils sont soutenus par certains administrateurs venus du nord, venus d'anciennes familles aristocratiques de la région.
\subsection{Acte 1 : le puissant royaume de Choseon}
Les intrigues dans cet acte sont relativement peu nombreuses, étant plus de l'ordre de l'accroche scénaristique avec la présentation de certaines factions.

Avant de lancer la cérémonie, les différentes factions suivantes doivent être au moins vaguement présentées :
\begin{itemize}
\item Le premier ministre qui soutient l'héritage par l'aîné
\item Le prince cadet, qui s'entend beaucoup mieux que son frère avec le roi
\item Des nouvelles du général Kim Mae-Jun et de ses victoires dans l'est.
\item La présence du troisième 
\end{itemize}

Cet acte amène à une cérémonie officielle importante, au cours de laquelle démarre l'acte 2. Cette cérémonie doit être foreshadowed au cours de l'acte.
\subsection{Acte 2 : la maladie du roi}
Cet acte démarre vraiment la course à la couronne : au cours de la cérémonie qui clôture l'acte 1, le roi fait un malaise. Rien de grave, dit le palais, mais la situation précaire n'échappe à personne. 

Différentes factions vont donc essayer de se positionner le mieux possible pour s'emparer de la couronne. La première à agir sera celle du prince héritier, qui rentrera à la capitale pour s'enquérir de la santé de son père. A ce stade, les autres factions amassent de l'influence et battent le rappel des troupes.
\subsection{Acte 3 : une succession douloureuse}
\section{Arcs secondaires}
\subsection{Des disparitions mystérieuses}
Dans les bas-fonds de Daegu, des disparitions inquiétantes sont notées. D'abord très discrètes, car visant essentiellement des esclaves et le bas peuple, leur nombre finit par être important, et les gardes de la ville s'inquiètent des corps qu'ils trouvent dans les caniveaux, au point d'hésiter à quitter leurs postes de gardes.

Il s'agit d'un groupe de vampires 'sauvages'(vrykolakas, ou spawns de vampires normaux) : l'ancien est malin et prudent, s'attaquant à des victimes isolées, et disparaissant rapidement. Ses deux rejetons par contre, sont complètement sauvages et tuent de manière beaucoup plus fréquente, sans prendre leurs précautions dans le choix des victimes et le timing.

Les personnages devraient d'abord trouver les deux rejetons, ce qui devrait considérablement limiter le nombre de victimes. Puis, l'ancien pourra chercher à se venger, ou finira par lui aussi commettre des erreurs importantes, ce qui déclenchera une seconde chasse par les PJs.

On a déjà trouvé trois victimes, deux d'entre elles trouvées non loin du mur, la troisième dans une arrière-cour. Les trois présentent des lacérations importantes, notamment au niveau de la gorge. En inspectant de manière plus détaillé la dernière, il apparait qu'une bonne part de son sang a disparu. Les deux autres aussi, mais c'était mois évident au vu de leur état. Les lieux où les corps ont été trouvés sont peu utiles, sauf l'arrière-cour, où il est possible de remonter une piste vers un réseau de tunnels de contrebande qui passent sous le mur pour arriver non loin d'une auberge, point de départ de la chasse des deux spawns. Pour des personnages de niveau 1-2, il s'agit de vampire spawns (p321 bestiaire1), en deux affrontements différents (avec un indice après le premier pour trouver le second, avec un test ???). Si les Pjs ne trouvent pas les indices, le second va se mettre en chasse et tomber sur les PJs la nuit suivante, ou dans la soirée, quand ils seront groupés. Niveau 1 : 120xp; niveau 2 :80 par, les deux peuvent être affrontés.

Kan Jung-Nam : maître vampire, noble local

\subsection{Le monastère de Kaejong}
Ce monastère pourra intervenir de plusieurs façons dans la campagne :
de part son importance politique, ses archives utiles, ou encore quelques menaces qui peuvent essayer de s'en prendre à ce lieu.

Il a été bâti pour enfermer et maintenir scellé à jamais un mal ancien. Ce rôle a été oublié depuis longtemps par la majeure partie du public, mais les prêtres importants de Maegu le savent, ainsi que le roi. 

Leur ennemi ici est un groupe de cultistes qui pensent pouvoir réveiller ce mal ancien pour ensuite profiter de ses pouvoirs pour leurs besoins personnels.

Ces cultistes pourront commencer par inciter des bandits de la région à s'en prendre au monastère, sans grand effet, mais cela pourrait préoccuper les moines. Ils doivent également pratiquer leurs rites impies, impliquant potentiellement des sacrifices humains.

Leur conspiration principale sera par contre de profiter du chaos de leurs autres tentatives pour avancer un de leurs hommes à l'intérieur du monastère, qui sera ensuite attaqué pour libérer la créature. Les personnages pourront être envoyés pour les arrêter.
\subsection{La frontière orientale}
Une des raisons de quitter la capitale : les personnages pourront être envoyés transmettre des ordres et surveiller ce qui se passe sur la frontière orientale avec le Han : le conflit avec l'un des royaumes voisin empire rapidement, et le maréchal Mae-Jun mène les opérations.

Une première possibilité serait d'aller lui transmettre un ordre royal, et en profiter pour enquêter sur des problèmes au sein de l'armée de l'est.

Alors que le conflit empire, les personnages pourraient devoir y retourner afin de continuer une enquête, ou de suivre un fugitif qui tente de passer la frontière.

Egalement, l'un des royaumes successeurs de l'empire Han a envoyé un groupe d'espion. La première piste sera la capture d'un agent de liaison, qui faisait le trajet d'un côté à l'autre des montagnes pour rendre compte à ses maîtres. Il faudra remonter sa piste, pour finalement arriver jusqu'à un petit groupe d'administrateurs qui reçoivent un paiement en échange d'informations, notamment sur les places-fortes de la frontière.


\subsection{Réarmement}
Les personnages sont envoyés enquêter sur un chariot d'armes qui a été intercepté par la garde de Maegu. Il faudrait remonter la piste de celui-ci pour savoir à quoi il correspond, et vers qui il se dirige.

Une enquête en plusieurs parties, essentiellement en acte 1.

Les commanditaires de ces marchandises sont les groupes criminels de la capitale, qui cherchent un avantage sur leurs rivaux. Evidemment, ils ne souhaitent pas voir la garde regarder de trop près dans leurs affaires.

Première étape d'enquête : déterminer vers qui vont ces armes, avec deux indices principaux, que sont le chariot lui-même, ainsi que son conducteur. L'interrogatoire peut se faire de différentes manières, et inspecter le chariot et surtout ce que les gardes peuvent dire, peut donner une idée d'où il allait. Il faut ensuite poser des questions sur place (le quartier près du fleuve), pour être guidé, plus ou moins volontairement, vers un chef de gang local, Pak Megujin. Celui-ci ira sans doute à la confrontation contre les PJs, qui pourront trouver une seconde cargaison d'armes aux mains de ses hommes.

Rencontre 1 : interrogatoire : épreuve d'influence; Chun Woo-Sung, conducteur de charrette; diplomatie 17, intimidation 14, duperie 15, autres compétences 18. découvrir DC 15, permet de remarquer qu'il porte des marques de fouet, probablement un esclave en fuite. Cela permet de réduire de 2 les tests d'intimidation contre lui. Seuil 6, doivent réussir en 4 rounds. Réussite pour savoir qu'il devait livrer à quelqu'un avec un bandana rouge près du navire 'la mouette'. En cas d'échec, il faudra discuter avec la garde, qui aura un avis sur le fait que cela devait aller vers les quais. xp = 100 en cas de réussite, 80 sinon. 6 réussites

Rencontre 2 : affrontement avec une première bande de brigands; 3 adversaires utilisant les règles des ruffians (p208 GMG); XP = 120xp; butin : 1 objet niveau 1 (pied-de-biche), 2 consommables niveau 1 (1 griffe d'ours-hibou et potion de soin mineure), 15 po

Rencontre 3 : le chef de bande : objectif:150XP au total, rencontre sévère (120 par joueur); 2 grave robbers et 1 bandits. Le bandit essaie de démoraliser ses adversaires, les grave robbers commencent par lancer leurs bombes avant d'avancer vers les PJs. butin = 1 objet niveau 2 (Parchemin sacré d'arme +1), 1 consommable niveau 2 (bronze bull pendant) et 10 po. Aucun butin gagné, rencontre évitée.

province de Daejong +trouver le nom du village où aller.

Rencontre optionnelle : un gamin des rues voit les PJs et s'enfuie en courant. Il est au courant que le gang sera intéressé par l'info et cours rapidement : (p156GMG), en utilisant les obstacles : foule, fruit cart, et wooden fence. XP = 50.

Il devrait y avoir deux pistes différentes pour la suite : la première consisterait à remonter la piste jusqu'au fournisseur, celui-ci faisant parti d'un groupe criminel ayant un contact dans l'armée du centre à la Forteresse de Hwaseong.

La seconde piste serait de suivre le chemin des armes précédentes, jusqu'à une province voisine où elles étaient acheminées et vendues. Le trajet vers le village (à 5 jours dans les montagnes), se fait sans problèmes de logistique, mais un groupe de bandit va tenter sa chance à attaquer le convoi non loin de la zone. RENCONTRE A FAIRE(rencontre 4) Une fois sur place, quelques questions pourront sans doute amener les personnages à discuter avec la patriarche de la famille Paek, les nobles locaux. Ceux-ci initialement nient, voire tentent de mettre cela sur le dos des bandits : il explique qu'il a tenté de former une milice, étant lui-même un ancien militaire, mais que l'opposition du magistrat local, peut-être corrompu, a mis un terme a cette démarche. Une observation attentive peut montrer des cibles et poteaux d'entrainements non loin de chez lui. Cela devrait se terminer sur un assaut contre les bandits, potentiellement avec un autre contre la magistrat local, et un défi de compétence pour trouver les bandits.

Village d'imje : lieu infesté par les bandits

 Rencontre 4 : prisonner (1, 40xp), 3 commoners (-1, 20xp) : rencontre à 80xp (normale).
Rencontre 5 (chef des bandits) : tracker (niveau 3, 80xp), 3 commoners (-1, 20xp), prisonnier(40xp). Rencontre à 120xp.
Rencontre 6 (magistrat local et ses gardes) : noble (niveau3, 80xp), bodyguard (40) :rencontre à 90xp.
\section{Arcs tertiaires}
\subsection{Les statues du sanctuaire d'ibarae}
Un sanctuaire en ville semble avoir de plus en plus de statues. Mais personne ne semble trop savoir d'où elles viennent, et puisque personne n'est vraiment dérangé, la question de ce dont il s'agit reste entière.

Il s'agit d'un esprit plutôt bénin (une fée en termes de règles), qui se façonne de la compagnie, et les anime parfois pour s'amuser. 
\subsection{La partie de chasse}
Les personnages sont invités à participer à une partie de chasse : l'idée est de passer une ou deux journée à traquer un tigre, et d'en profiter pour discuter tranquillement avec la personne qui les invite.

\subsection{Le festival du dragon}
Les festivités du nouvel ans se préparent dans la plus grande effervescence : il s'agit pour les grandes villes de s'attirer la présence et les faveurs des dragons !

Les jeunes dragon, ou imugi sont assez communs, mais en ces temps de crises, la présence d'un grand dragon (yong), pourrait rassurer la population, ou peut-être ajouter un nouveau joueur à l'échiquier politique.
\subsection{Le mort sans repos}
Le bureau d'investigation est chargé d'une affaire privée, mais sérieuse : ils doivent se débarrasser du mari d'une puissante ministre. Toutefois, celui-ci est déjà mort il y a deux semaines, mais s'est relevé pendant la nuit suivante, et rôde maintenant dans leu demeure.

\section{Arcs personnels}
\subsection{Shin myun-woo : une guenaude au palais}
Une des nobles du royaume est en réalité une guenaude, qui essaie de mettre en place un couvent au sein de l'administration et de l'aristocratie de Choseon.

Il s'agit de la mère de Shin, et elle profitera du chaos de la succession pour lancer l'Appel sur certaines de ses filles, qu'elle compte bien placer au mieux par la suite.

Shin myun-woo est le personnage de Tara.
\subsection{Gang Ye-mong : le clan Yi}
Ayant été au service de la famille Yi, on pourra lui faire confiance pour protéger les jumeaux royaux au milieu du chaos, à défaut de protéger la reine elle-même. Elle va donc être très liée à l'intrigue principale, avec un intérêt personnel dans celui-ci. On peut même imaginer un affrontement entre elle et un membre de sa famille qui garderait toujours les Yi.

Au-delà de ça, un trajet au niveau du quatrième vieillard pourrait bien se révéler une aventure intéressante suivant la saison au cours duquel il est entreprit et du chemin choisit. Cela pourrait se révéler être une pause détente agréable pour le reste de l'équipe, notamment après une inspection du nord, où un risque de rébellion gronde.


\subsection{Kil-ae : le symbole du chaos}
En premier lieu, l'arc principal de ce personnage sera sans doute lié au monastère de Kaejong, ainsi qu'à ce qu'il enferme. On peut imaginer que l'influence du démon qui y est scellé a pu se répandre, que ce soit dans une famille qui aurait un lien avec son enfermement, ou une autre qui serait lié à son culte (qui se réveille et tente de libérer son maître).

Dans tous les cas, cela pourrait passer par une marque étrange, que quelqu'un au courant de cette histoire pourrait reconnaître comme étant le symbole de ce démon : un bien sombre présage évidemment. Cela pourrait mener les moines à surveiller de près la tieffeline, voir à lui poser directement des questions. Cet arc reste à détailler, mais va inclure la révélation de ses origines, et de l'influence qu'elle porte en elle, pour probablement finir par un affrontement avec la créature.

De manière plus secondaire, il est fort possible que certaines intrigues se déroulent dans la maison de thé où elle réside : ????
\subsection{Lim Jun-yon : L'ambition familliale}
La défense de sa famille risque d'être un moteur important pour ce personnage : celle-ci a de grands projets pour monter dans les sphères sociales, en particulier sa belle-mère. Il y aura notamment le mariage de sa soeur, militaire au sein d'une famille importante (le clan Yi, le clan song, un grand général ????), mais aussi des plans d'achats et d'expansion en général (avec sa demi-soeur déjà mariée et sa belle-mère). 

Tout ceci devenant bien évidemment de plus en plus risqué alors que le chaos va se répandre dans le royaume avec la maladie royale : s'allier à une faction deviendra alors un danger mortel, et Lim ne va pas manquer de problèmes pour les protéger.

Sur un ton plus léger, certains de ces évènements familiaux pourraient servir à détendre l'atmosphère pour une ou deux sessions.
\subsection{Nam Ji-hyo : ???}
\chapter{Règles maisons}
\section{Armes à feu}
Deux types d'armes à feu sont surtout présentes à Choseon :La lance de feu (seungja), constituée d'un canon au bout d'un long manche, tirant une charge de shrapnels dans une direction. Mortelle à courte portée .L'arquebuse,ou jochong, permettant un tir plus précis à longue portée, est un ajout plus récent venant des armées Hans.

\flushleft
\begin{tabular}{c c c c c c c p{0.1\textwidth} p{0.2\textwidth}}
Nom & Prix & dégâts & portée & recharge & enc & mains & groupe & traits \\
Seungja & 15po & d8P & ligne 35ft & 2 & 2 & 2 & gun martial & attachée (bâton), mortel(d10), misfire \\
Jochong & 20po & d10P & 60ft & 2 & 2 & 2 & gun martial & mortel(d10), misfire\\
\end{tabular}


\paragraph{Nouveau trait d'arme : misfire}
Lors d'une attaque avec une arme pouvant misfire, si le dé est un 1 naturel, le tir fait long feu et ne part pas. Cela implique de nettoyer l'arme avant de la recharger. Mécaniquement, cela oblige à utiliser l'action 'gérer un long-feu' avant de pouvoir tirer ou recharger.
Action :
Gérer un long feu : 1 action
Si le personnage réussi un test de crafting DD15 ou de lore(armes à feu) DD10, il peut à nouveau recharger celle-ci et tirer avec. Sinon, il faudra retenter cette action.

\section{Langues}

Les langues utilisées dans le livre de base ne seront pas utilisée dans la campagne : pas de nain, d'elfe, de commun, ou autre langue dépendant de l'ascendance.
En remplacement les règles suivantes:
\begin{itemize}
\item Tout les personnages parlent le Choseon à la place du commun, et remplacent toutes les langues qu'ils devraient obtenir par une langue de la liste ci-dessous.
\item En terme de règle, savoir parler une langue et savoir l'écrire nécessite de dépenser deux langues : une pour l'oral, une pour l'écrit. 
\end{itemize}


Liste des langues parlées : Choseon, Han, Junkan, Ihlan, Jeju (dialecte du nord de Choseon, assez différent)

Liste des langues écrites : Choseon, Choseon classique, Han, Han classique (langue des érudits), Junkan, Ihlan, Jeju, et différents codes secrets(à définir).

\chapter{PNJs}
\section{La cour royale}
\subsection{Le roi Song Su Yun}
Le roi de Choseon depuis 23 ans. Il est gravement malade, mais pour le moment n'en a pas conscience
\subsection{La reine Yi Il Sho}
La seconde épouse du roi Yun, issue du puissant clan Sho
\subsection{Le prince héritier Song Cheol Shin}
Militaire, a passé plusieurs années dans le sud du pays à diriger des opérations de la flotte Bleue
\subsection{Le prince Song Cheol Ahn}
Un érudit plutôt calme, il s'entend très bien avec son père
\subsection{Capitaine Kyo Yong-Chol de la garde royale}
Un soldat de métier, un ancien de l'armée de l'ouest, recruté dans la garde royale il y a 15 ans. Il sert fidèlement le roi depuis.
\subsection{Administratice On Ji-Hye du bureau royal d'investigation}
Récemment nommée à ce poste, après des postes plus subalternes dans l'administration de la capitale. Elle a envie de faire ses preuves à ce poste très proche du roi. Elle a été notée pour son ascension rapide dans les rangs de l'administration, après un concours particulièrement bien réussi. Elle vient de la famille On, soutiens traditionnels du roi, et une des plus anciennes familles aristocratique du royaume.

\section{Les rues de Daegu}
\subsection{Kan Jung-Nam, vampire ancien}
Un vampire relativement ancien, qui a changé à plusieurs reprises d'identités pour brouiller les pistes. Il est relativement subtil dans son approche, et limite pour sa part ses repas. Ses deux rejetons, encore très jeunes, eux ne se contrôlent que peu par contre.

Il ne se considère pas comme mauvais, et pourra peut-être chercher un accord avec les PJs si il est cerné (de part ses services rendus au royaume notamment). Il dispose d'informations importantes sur ce qui se passe au palais de part son âge, notamment il a des soupçons voire des informations sur les agissement d'autres créatures moins maîtrisées que lui : il soupçonne la présence d'une guenaude, voire de plusieurs, et connait l'histoire du démon du monastère de Kaejong.


Après la première rencontre avec les PJs, il est possible qu'il s'intéresse de plus près à eux, et contrôle de beaucoup plus près ses deux rejetons. 


\chapter{Éléments de setting}
\section{Familles nobles}
\subsection{Le clan Yi}
En vérité un groupe de familles nobles apparentées, il est relativement puissant, ayant fourni au cours des derniers siècles plusieurs maréchaux et de nombreux administrateurs, ainsi qu'entre autre, la reine actuelle, Yi il sho. C'est une influence importante dans le le royaume, contrôlant de nombreuses terres dans les ports du sud.
\subsection{la famille Jien}
Famille noble récente, d'origine Han, elle n'a fournit qu'un seul administrateur, qui est toujours à sa tête : Jien jeong-hun. Celui-ci est un réformateur assumé, et le domaine de sa famille est une villa plutôt modeste, en bordure de la capitale.
\subsection{ la famille Paek}
Famille plus ancienne que fortunée, elle ne fournit pas d'officiel à chaque génération, et a plusieurs fois été proche de repasser au sein du peuple à cause de cela. Elle reste une famille plutôt ancienne dans la capitale, ayant fournit notamment de nombreux officiers à l'armée (aujourd'hui plusieurs des fils de la famille servent au nord et à l'ouest).
\chapter{Déroulement de campagne}
\section{Acte 1 : niveau 1-5 ?}
\subsection{niveau 1}
Les personnages, pour leur première enquête, vont être envoyés sur une histoire de trafic d'armes : la garde d'une des portes a trouvé un chariot rempli d'arme quittant la ville, et a arrêté son conducteur. Le responsable du BRI demande de trouver rapidement pour qui étaient ces armes et quel était l'objectif de leur trafic. Trouver leur origine sera l'étape suivante, pour éviter que ce genre d'objets ne se retrouve partout (mais il devrait manquer un indice important à ce moment pour ce faire). 
\emph{Arc secondaire réarmement, suite à prévoir avec de nouveaux indices qui arrivent au BRI}
Après cette introduction, alors que les gens iront se reposer par exemple, Lim va avoir des nouvelles de sa famille : sa belle-mère ayant finalisé l'achat de plusieurs propriétés, notamment une seconde fabrique de soie juste en-dehors de la ville. 

Pour ce niveau 1, la suite devrait être la suivante : 3 combats et 2 obstacles autres. Ceci se répartira entre une session de transition à la capitale puis une session de trajet jusqu'au village où les mène leur première enquête, ou alors la totalité de leur expédition dans les montagnes, suivant les besoins.

Cette session de repos verra donc plusieurs informations:
\begin{itemize}
\item En premier lieu pour Lim des achats par sa famille de nouvelles propriétés industrielles (arc perso)
\item Pour Gang Ye-Mong, une discussion avec le seigneur Yi Il-Sung, frère de la reine et intendants des messagers royaux. Celui-ci déplore que Gang n'ait pas été affecté à la sécurité de sa soeur, malgré ses demandes (lien avec l'arc principal).
\item Quelques disparitions sont signalées dans le quartier de Minae. Les personnages peuvent décider d'y aller avant ou après leur enquête en cours. (arc secondaire des disparitions mystérieuses).
\item Une grande victoire est annoncée dans l'ouest : les troupes du royaume de Wei de l'ancien empire Han ont été écrasées par le général Namgung Sung-Min et son armée de l'ouest. La capitale commence à s'attendre à accueillir le général victorieux, probablement dans quelques mois, si le Wei demande la paix (arc secondaire de la frontière orientale, et anticipation du pivot acte 1/2 de l'arc principal, au cours de cette cérémonie).
\end{itemize}

loot niveau 1 :40 po,healing potion (conso 1), comprehension elixir (conso 2), torche éternelle, chien d'onyx, arme +1, crying angel pendant (conso 2). A fournir dans les 3 prochaines rencontres:
10-15po, 1 objet permanent et un conso par rencontre.

déjà obtenu : 10po, crowbar, healing potion, owlbear claw
\section{Acte 2 : niveau 6-10?}

\section{Acte 3 : niveau 11 - 15 ?}
\chapter{Notes de séance}
\section{Session 0}
Présentation du setting, et réflexion sur le groupe. Sont choisis :
\begin{itemize}
\item Un tengu alchimiste
\item Une barde tieffeline
\item Une duelliste changeline
\item Une elfe aasimar oracle
\item Un guerrier kitsune
\end{itemize}
La création des détails mécaniques et des backgrounds sont laissés aux joueuses en attendant la session 1.
\section{Session 1}
\paragraph{Date}12.12.2020
\paragraph{Déroulement} Introduction par personnage sur leur début de journée, dans le milieu de l'automne. Elles reçoivent leur première enquête : du trafic d'armes intercepté par la garde dans la capitale (à la porte princière). Elles y arrivent dans l'aprem pour y interroger le conducteur, qui révèle ce qu'il sait (un contact et employeur à Hwaseong, et un lieu de dépôt sur les quais du yeongsam). Elles se dirigent ensuite vers les quais, où elles sont attaquées par un groupe de truands, qu'elles surclassent avec difficulté. Gain total : 220xp.
\paragraph{Améliorations}
\begin{itemize}
\item améliorer la préparation des combats : profils des PNJs et préparation des rencontres plus sérieuse, avec le calcul d'xp notamment. Ici, l'équilibrage était douteux
\item trouver plus de musique pour tenir 2-3h au total, l'ost de my country new age étant trop courte.
\item continuer à essayer de plus narrer la situation
\item modifier l'arc secondaire réarmement pour le rendre plus important que juste un groupe de truands essayant de dépasser ses concurrents.
\item préparer les PNJs plus sérieusement : portraits et noms de la capitaine et de la supérieure des investigratices.
\end{itemize}
\section{Session 2}
\paragraph{Date}11.01.2021
\paragraph{Déroulement}
Interrogatoire du brigand s'étant rendu, celui-ci leur indiquant rapidement la direction de son chef de gang. Les PJs se préparent et se rendent à son repaire avec quelques gardes pour obtenir des réponses. Une discussion un peu tendue (et pleine de mauvaise foi), leur donne les informations qu'ils étaient venues chercher : les armes devaient être livrées dans un village de la province de Daejong, à quelques jours de marche vers le nord-est. total xp : 340.
\paragraph{Axes d'améliorations}
\begin{itemize}
\item accélérer le mouvement : anticipation des problèmes techniques, meilleure préparation de l'enchainement et des descriptions.
\item musiques à tester plus sérieusement d'ici la prochaine séance
\item mieux préparer les rencontres hors combat
\end{itemize}

\section{Session 3}
\paragraph{Date}01.02.2021
\paragraph{Déroulement}
Les personnages ont commencé par récupérer de leur enquête précédente et des combats qui en avaient découlé. Ils sont ensuite retourné au palais, pour obtenir plus d'informations sur le village d'Imje et la famille Paek, et signaler leur probable départ prochain.

Leur supérieure les informe à ce moment d'une seconde affaire, à traiter à leur convenance, des personnes ayant disparus dans un quartier du nord de la ville, le long des remparts (arc secondaire des disparitions inquiétantes).

Ils ont donc enquêtés rapidement sur cette seconde affaire, dans l'idée de voir ce qu'il en était avant de partir pour les montagnes. Une journée d'enquête leur a permis d'obtenir pluseurs informations : une créature semble chasser la nuit et se nourrir de sang, à l'intérieur des murs de la ville. Celle-ci a une odeur étrange, ce qui leur a permis de trouver un tunnel de contrebande utilisé par la créature pour se déplacer (le spawn n'en a pas besoin, mais cela lui évite de devoir faire attention aux rondes sur les murs). Ils ont aussi pris contact avec un noble local, habitant à l'extérieur et semblant connaître bien le quartier (Kan Jung-Nam, le maître vampire?).

La session se termine alors que les personnages envisagent de partir, la créature de la ville devant faire plus attention à cause de la garde, plus vigilante que jamais, et connaissant l'emplacement du tunnel.

xp total : 490 (+150 pour l'avancée sur l'enquête).
\paragraph{Améliorations}
La plus notable à faire serait de disposer (et fournir) une carte de la ville de Maegu, afin que les joueuses s'y retrouvent plus facilement.

En dehors de cela, les règles étaient peu nécessaires sur cette session, mais la préparation était bonne, malgré les deux embranchements possibles de la campagne.

Au retour des personnages de la montagne, il va falloir faire rentrer les PJs dans le jeu politique, peut-être au cours d'une soirée ou d'un autre évènement officiel, ou simplement en les faisant rencontrer certains des protagonistes de l'arc principal.
\chapter{Profils de la campagne}
\chapter*{Annexes}
\section*{Listes aléatoires}
\paragraph{Noms masculins (à garder à jour)}
\begin{verbatim}

Hwan Song-Jin
Ch'o Kyung-Mo
Man Jun-Yeong
Ok Woo-Sung
Nae Ji-Hu
Mun Byung-Ho
Sung Joon-Ho

Tongbang Seon
\end{verbatim}
\paragraph{noms féminins (à garder à jour}
\begin{verbatim}
T'ae Han-Bi
Uh Ah-Hyun
Ka Ha-Sun
Paek Sujin
P'yo Hee-Ae
Mi Eun-Ju
Ryong Gri-Na
Sung Ji-Yung
Pi Young-Ah
An Yong-Suk
\end{verbatim}
\paragraph{Musiques utilisées}
\begin{itemize}
\item \begin{verbatim}https://www.youtube.com/watch?v=72JKkL6qSeg&list=PLQxxaVZxxtV3IftEeLXr_9Lul6Yi7iujh \end{verbatim}
\item \begin{verbatim}https://www.youtube.com/playlist?list=PL0c26ZhRmZb0TiUl169gUCrnUDDk7XKN8\end{verbatim} valide, peut-être trop de chants
\item \begin{verbatim}https://www.youtube.com/playlist?list=PLQeZIeLOTDJJ0cZccfI9vHFSgHF98qpCE\end{verbatim} valide, peut-être trop de chants
\end{itemize}

\end{document}
