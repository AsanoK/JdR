\documentclass[10pt,a4paper]{book}
\usepackage[utf8]{inputenc}
\usepackage[french]{babel}
\usepackage[T1]{fontenc}
\usepackage{amsmath}
\usepackage{amsfonts}
\usepackage{amssymb}
\author{ Antoine Robin}
\title{Garde Royale - manuel du joueur}

\newcommand{\nomroyaume}{Choseon}
\begin{document}
\chapter{Région}
\section{Le royaume de \nomroyaume}
Il s'agit du royaume dans lequel se dérouleront les évènement de cette campagne. Ce royaume est fortement mais librement inspiré de la Corée historique, mais contient également de forts éléments de médiéval fantastique.
\subsection{Géographie}
\nomroyaume est un royaume relativement isolé par sa géographie : le nord-est est protégé par la chaîne de montagnes des quatre vieillards qui se continue sur l'axe nord-sud du royaume. Les plaines de l'ouest sont elles coupées par le fleuve Maejong, servant de frontière avec les Junkans, et ses rives sont parsemées de nombreux forts bien protégés et sont régulièrement patrouillées par l'armée. La forteresse de Kaibek est la plus importante du dispositif, et un des rares ouvrages de l'autre côté du Maejong

Enfin, les côtes au sud sont protégées par une marine forte, et la pression des pirates Ihlan, très présente au cours des siècles précédent, est maintenant beaucoup plus faible.

La capitale, Daegu, est nichée au pied des montagnes, et, comme le veut l'expression consacrée par l'art, 'sous le regard du troisième vieillard', le mont Hayan. 

Au-delà de sa capitale, \nomroyaume est divisée en 9 provinces dont les gouverneurs sont nommés par le roi parmi les administrateurs du royaume. Chacune de ces provinces maintient une faible force militaire, et gère surtout les impôts, les examens administratifs ainsi que la gestion de certaines réserves, chaque grande ville étant tenue de conserver des réserves de riz en cas de famine ou de catastrophe.
\subsection{Histoire}
Le royaume a été fondé il y  a environ trois cent ans, par le général Song Mae, qui renversa son souverain, avant d'unifier en une dizaine d'années les nombreux petits royaumes qui précédaient \nomroyaume. Il a mis en place une administration efficace, et la dynastie Song dirige toujours aujourd'hui le pays.

Avant cela, il y avait plusieurs royaumes beaucoup moins puissants, en conflit fréquent, ceux du sud raidés par les pirates, ceux de l'ouest par les Junkans, et enfin, ceux de l'est étaient souvent sous protectorat des Han, dont le royaume puissant est de l'autre côté des montagnes.

Depuis l'unification, l'administration a étendu son contrôle sur le pays, amenant une direction informée et éclairée sur de nombreuses décisions. A sa tête, le roi, dans son palais de Daegul. Les souverains ont également développé le commerce avec les Han, ainsi qu'avec les barbares d'Ilhan, dans les îles du sud.

Il y a eu peu d'expansion territoriale, à part la prise de l'île de Basae, au sud-est. Par contre, une invasion des Junkans a été vaincue en 157 (compté depuis la fondation de Daegul), après de rudes combats sur la frontière. Des expéditions militaires ont été menées contre leurs terres, le plus souvent en représailles.

Aujourd'hui, en 203, le royaume est dans une période de prospérité et de paix relative : quelques raids Ilhan ou Junkans se produisent sur les côtes et les frontières, mais cela est resté très limité au cours de la dernière décennie. L'économie se porte très bien ,avec un commerce intérieur et extérieur florissant. La seule ombre au tableau est la stabilité politique : si la dynastie Song semble solide, les conflits de faction à la cour royale font souvent rage.
\subsection{Culture de \nomroyaume}
La société de \nomroyaume est très hiérarchisée, théoriquement sur un principe de mérite. Ainsi, les lettrés sont ceux qui ont pu passer avec succès le concours de l'administration, un concours, principalement basé sur la connaissance des classiques littéraires et de leur interprétation. Une autre administration se basant au moins partiellement sur le mérite est l'administration militaire, où les promotions tombent en priorité sur l'expérience, et les actions notées par les supérieurs. 

Au sommet de la structure sociale, on trouve le roi et la famille royale. Juste en-dessous sont les nobles héréditaires, ou Yangban qui profitent de leur statut pendant quatre générations après avoir donné à l'état un administrateur de haut rang. On trouve ensuite les citoyens, ou Jungin, disposant d'un bon accès à l'éducation, et travaillant souvent aux niveaux subalternes de l'administration. Les paysans, ou Sangmin, sont plus bas, passant l'essentiel ed leur vie à travailler physiquement. On y retrouve les paysans, marins, artisans.... Les cheonmin sont les impurs, travaillant à des tâches telles que la boucherie, la prostitution etc. Enfin, tout en bas, les nobi, ou esclaves sont considérés comme des propriétés de nobles ou de notables importants.

Sous la dynastie Song, la culture a pu se développer grandement : l'écriture s'est grandement répandue, au oint qu'aujourd'hui, la totalité des marchands puissent lire et écrire, ainsi qu'une part non négligeable des paysans. La classe des lettrés dispose évidemment d'une plus grande maîtrise de ce domaine, et des classiques de littérature, tant Han que \nomroyaume.

En termes d'art, la chanson est très populaire parmi le peuple, la peinture et la poésie sont plus communs parmi la noblesse et les lettrés. Les artisans du royaumes eux sont réputés pour leur magnifiques produits, en particulier des meubles et de la poterie de très grande qualité.

En matière de nourriture, la solution privilégiée est souvent de griller les aliments, sauf le riz, en particulier les produits de la mer et des rivières, très communs dans le royaume. Le gibier est ajouté en fonction du statut social. La cuisine est très variée à part cela, avec de nombreuses recettes, et de nombreux épices pour changer celles-ci. Le misu est une boisson épaisse et nourrissante faite d'un mélange de céréales broyées mélangée à de l'eau.

La magie y est peu réglementée, à l'exception notable de la nécromancie. Celle-ci est à l'inverse très surveillée. En effet, si la création de morts-vivants sans conscience est tolérée sous conditions, la création de morts-vivants conscients est considérée comme une abomination qui mérite la mise à mort immédiate.

De toute manière, la nécromancie est rendue relativement difficile par les traditions funéraires, qui impliquent la crémation des corps. Seuls les plus pauvres vendent leurs corps à des nécromanciens après leur mort, pour aider leur famille malgré cela (ou sont vendus par leur proches).
\subsection{Les examens royaux}
Les examens royaux sont le chemin le plus direct, si ce n'est le seul, vers des postes à haute responsabilité pour l'état, qui sont généralement mieux considérés que les entreprises privées.

Il s'agit d'examens difficiles, auquel chacun peut participer après avoir payé un prix d'entré relativement faible. Les règles y sont draconiennes, le candidat devant passer la journée isolé dans une sorte de cabane dédié, pour répondre aux nombreux problèmes posés.

Il en existe trois types : les examens littéraires, les examens militaires et les examens divers. Le résultat d'un étudiant à cet examen détermine le rang du poste qui peut lui être proposé à la suite de son acceptation (si l'examen est réussi).

Les examens littéraires sont les plus prestigieux, et se basent sur les écrits de grands auteurs \nomroyaume ou Han, généralement en terme de philosophie, de science politique. Il s'agit de certains des plus grands travaux de littérature du continent, qui ont été très commentés par de nombreux lettrés depuis leur écriture. Les grands commentaires sont d'ailleurs également étudiés, et se baser sur eux est considéré comme un élément essentiel d'une réponse réussie. La plupart des très hauts postes administratifs ne peuvent être atteint que par cette voie. Certains de ces postes donnent droit à un statut de noblesse, héréditaire sur trois générations.

Les examens militaires sont souvent passés soit par de jeunes futurs officiers instruits, soit par des soldats jugés prometteurs, et à qui leurs supérieurs laissent du temps pour étudier. La première épreuve est une épreuve pratique, l'appliquant devant notamment montrer sa compétence dans les trois armes. La seconde épreuve est une épreuve orale, dans laquelle l'applicant doit montrer sa connaissance de certains classiques littéraires, ainsi que des classiques militaires. La troisième épreuve teste les compétences de cavaliers et d'archer de cavalerie.

Les examens divers portent sur la traduction, la médecine, les sciences naturelles ainsi que le fait de réaliser des chroniques. Etant plus 'manuel', il est souvent moins bien considéré que les deux autres formes d'examens.

Dans tous les cas, il s'agit d'examens particulièrement difficiles, et s'ils sont ouverts à la majeure partie de la population, les enfants de la noblesse, ayant la possibilité d'apprendre ces notions dès le plus jeune âge, sont considérablement avantagés. 
\subsection{Religion et divinités}
La religion est généralement une chose privée, les prières étant souvent réalisée dans l'enceinte de la maison, ou dans de petits sanctuaires que l'on trouve un peu partout. Ceux-ci sont souvent maintenus par les habitants. Des monastères sont également assez courant, généralement financés par de riches familles, ou des communautés spécifiques. Ceux-ci ont de fait une grande influence sur la vie spirituelle du pays, mais aussi sur sa politique. Enfin, beaucoup maintiennent une force en arme pour se protéger des bandits, en particulier près des frontières. Enfin, les moines itinérants sont très communs, proposant leurs conseils et écoutant les problèmes contre un bol de riz ou un lieu ou dormir.

Les ancêtres de chaque famille sont très respectés, et de nombreuses prières leurs sont adressées par tous. Il s'agit surtout d'une religion privée, mais certains sanctuaires et de nombreuses fêtes font vivre ces croyances.

Maegu est le premier des quatre vieillards. Il est le juge impartial, dieu de la justice, mais aussi responsable des âmes de morts et de leur destinée. Il est très respecté dans une société aussi stricte. Il est représenté avec son livre, dans lequel il note ce qui attends chacun.

Sedang est le second vieillard, il est le dieu des philosophes, des savants et de la connaissance. Il inspire également les artistes, en particulier les peintres et les poètes. Il est représenté avec une grande plume d'écriture, et une tenue d'érudit.

Hayan, le troisième vieillard, est le souverain céleste, dont le souverain est la représentation. Il est chargé de l'ordre, de la civilisation, et d'arbitrer les conflits entre divinités. On le représente avec la coiffe royale, généralement assis sur un trône.

Guang, le quatrième vieillard est le défenseur du royaume, le maréchal céleste. Il est représenté en armure, portant une coutille ainsi que son grand bouclier de pierres. Toutes les fortifications du royaume contiennent un petit autel à son effigie. 

Jayeon est la nature, le renouveau, et est souvent représentée sous les traits d'une femme, un renard et un bambou à ses côtés. Elle est très priée par les paysans pour s'assurer d'une bonne récolte.

Chi Di : un dieu venu de Han il y a longtemps, il est appelé le dieu rouge, est peut être représenté par un dragon de cette couleur, ou un phénix, et aurait donné les connaissances en médecine, en commerce, ainsi que le feu aux hommes. Sa représentation humaine est un homme un tein rouge, parfois écarlate.

Simianshen est le dieu aux mille visages. Il est le dieu des secrets, de la lune et de la magie, mais aussi e la tromperie. On le représente sous la forme d'un être androgyne, avec une multitude de visages changeant.

Canmu est la déesse de la beauté, et la marieuse céleste. Elle protège les amants, déclenche les passions, et a fait don aux hommes des vers à soie. Ele est représentée soit sous sa forme de marieuse, en tant que grand-mère sympathique, soit en tant que tisserande, avec son métier à tisser.

Xiwangnu est la régente du cycle, chargée de la création et de la destruction de tout. Sa face sombre règne également sur les non-morts, sa face plus claire sur l'immortalité. Elle est représentée sous les traits d'une femme âgée, souvent avec une roue.

Heixian est peu vénéré, la corneille à trois patte étant annonciatrice de malheur. Ses sanctuaires sont le plus souvent éloignés dans la nature, et on espère l'apaiser pour limiter son arrivée.

En plus de ces divinités, une religion venue de Han a été introduite récemment en choseon : le Tongbulgyo. Les suivants de cette religion affirment qu'atteindre l'illumination, par l'éloignement des possessions et plaisirs terrestres, permettra à l'âme de s'élever au-dessus de toutes les autres. Relativement epu nombreux, ils se développent de manière relativement agressive.
\subsection{Esprits et créatures de \nomroyaume}
Le Fenghuang, ou phénix, est le roi des oiseaux. Il représente la vertu et la grâce, et on dit qu'il niche sur les plus hauts sommets des montagnes. Un nid de Fenghuang sur le mont Hayan est dit signifier la prospérité pour tout le royaume.

Les renards sont souvent vus comme des messagers divins, ou du moins, sont liés aux créatures surnaturelles.

Les gwisin sont les esprits sans repos des morts. Ils se forment quand quelqu'un meurt sans avoir pu accomplir une dernière tâche. On dit que les plus puissants peuvent rester sur terre même après la complétion de cette tâche, et que certains deviendraient incroyablement puissants.

Les dragons de \nomroyaume sont des créatures puissantes, mais généralement sages et peu agressives. Leur forme mineure par contre, les grands serpents sont beaucoup plus dévastateurs, manquant de l'intelligence des dragons.

Le qilin est une créature possédant un corps de cheval, des bois de cerfs, et des écailles de dragon ou de poisson. Ils ne sont pas agressifs, mais peuvent voir ceux qui sont coupables de méfaits.

Ce n'est bien évidemment qu'une petite sélection de certaines des créatures surnaturelles de la région.
\section{La capitale, Daegu}
Daegu a été fondée par le roi Song Mae, premier roi de \nomroyaume, afin de se démarquer des royaumes précédents, en disposant d'un nouveau centre du pouvoir, qui ne soit lié à aucun d'eux.

Il s'agit de la plus grande ville du royaume, et un carrefour culturel majeur : la haute administration y réside bien souvent, de même que la noblesse. Les arts et artisanats sont aussi très présents.

Elle est nichée dans l'ombre du mont Hayan, et profite ainsi de paysages magnifiques.
\subsection{Le palais royal}
Le palais de Changdeok, palais royal des Song est le siège du pouvoir de la dynastie, et le monument le plus important de la ville, sur une haute colline au pied du mont Hayan. 

C'est ici que travaillent un grand nombre d'administrateurs, compilant des informations sur la totalité du royaume, calculant les taxes, s'assurant de disposer d'approvisionnement pour les différentes bannières militaires.

C'est aussi le centre politique du royaume avec la cour royale, où les décisions importantes sont prises. Les administrateurs commencent leur carrière politique au neuvième rang junior et peuvent monter jusqu'au premier rang senior, en passant par les 9 rangs des deux niveaux. Le roi tient les sessions de sa cour en présence des conseillers de haut rang. Dans ces sessions, les conseillers peuvent soumettre les problèmes relevant de leur juridiction ou des problèmes de politique générale.

Plusieurs commissions et bureaux existent pour diriger le pays. Le bureau royal d'investigation est rattaché à la garde royale et est chargé de lutter contre la trahison, la corruption ainsi que d'autres crimes majeurs. Le bureau de la capitale est chargé de l'administration spéciale de la ville de Daegul. L'office des archives maintient des chroniques détaillées des règnes des souverains de \nomroyaume. Une demi-douzaine d'autres fournissent des conseils au roi, ainsi que le fonctionnement administratif du palais, de la capitale, et du royaume.

Les citoyens de \nomroyaume ont le droit de solliciter leur souverain, en battant de tambours le long de son chemin pour attirer son attention. les audiences de ce type sont faite au bon vouloir du souverain, la plupart maintiennent au moins des sessions plusieurs fois par semaine.

Bien qu'étant le coeur administratif, le palais a également été pensé à sa construction comme une fortification importante, pouvant tenir un siège. Cette capacité a été érodée par différents agrandissements réalisés depuis, mais le palais reste capable d'abriter la totalité de la garde royale ainsi que d'autre unités militaires, et dispose de réserves importantes de vivres.

Aujourd'hui la famille royale est composée du roi Song Su Yun, de sa femme, Yi Il Sho et de ces cinq enfants. La reine Sho est la seconde épouse du roi, et lui a donné son dernier fils ainsi que sa dernière fille. Le fils aîné, Song Cheol Shin  est un militaire, ayant passé plusieurs années à diriger les actions de la flotte Bleue. Son frère cadet, Song Cheol Ahn a administré pendant cinq ans la province de Mingae et est un lettré reconnu pour ses talents. Leur soeur Song Cheol Min est mariée à un des princes de Han. Ils ont également un demi-frère et une demi-soeur, jumeaux, issus du second mariage du roi. Ceux-ci fêtent leurs dix ans dans les prochains mois.
\subsection{Les fortifications et troupes de Daegul}
En premier lieu, le palais royal est un élément important de fortification, abritant la garde royale, et pouvant être défendu par ce petit nombre de défenseurs. La garde royale comporte 300 hommes sélectionnés par le roi, et portant des robes rouges pour les différencier. Ils sont chargés de protéger les bâtiments, mais aussi les membres de la famille royale et les ministres importants. En plus de cela, le roi dispose de 20 gardes rapprochés, en robe noire.

La ville elle-même est protégée par un épais mur d'enceinte, ouvert en 5 portes massives : la porte royale, la porte d'Hayan, la porte De Yeongsam, la porte de Hwaseong et la porte princière. Chacune est un véritable bastion qui serait difficile à prendre pour un assaillant.

Enfin, la forteresse de Hwaseong, à environ un jour de marche au sud de la ville. abrite la puissante armée du centre, sous le contrôle direct du souverain. Les unités qui la compose tournent avec les autres armées du royaume pour maintenir sa capacité martiale, et limiter les risques de loyauté personnelle. Environ 50000 hommes appartiennent à cette armée du centre, en uniforme vert.
\subsection{Maisons de plaisir}
Les maisons de plaisir, sont un élément important de la vie à la capitale : ils sont très courant, en particulier dans le quartier de Jongno-gu, au pied du palais royal. 

Les plus modestes sont un lieu de détente et de rencontre pour la classe moyenne : artisans, médecins, étudiants, militaires.... On peut y assister à de la danse, écouter de la musique, et discuter en profitant de diverses boissons et repas. 

Les plus prestigieux sont de véritables salons ou les administrateurs du royaume y parlent de politiques, mais aussi des arts, de la culture. On y trouve les meilleurs danseurs et danseuses, et la musique suis la dernière mode royale. Évidemment, de tels services sont extrêmement chers, et les clients les plus nobles font l'étalage de leur statut en sponsorisant les plus grands artistes pour réaliser leurs oeuvres dans ces lieux. 

Les kisaeng, les courtisanes (et courtisans) qui travaillent dans ces lieux, sont très souvent vendu(e)s par leur famille qui ne peut plus les nourrir. La carrière commence au plus bas range de samsu vers 15 ans, après un entraînement de plusieurs années, et se termine bien souvent après 25 ans, même si les plus influent(e)s peuvent garder un rôle important jusqu'à une retraite obligatoire à 50 ans. Les kisaeng ne sont pas nécessairement des prostituées, leur travail consistant à offrir des distractions, principalement la musique et la danse, mais aussi une conversation intéressante sur de nombreux sujets. Certains et certaines quittent cette vie en devenant le concubin ou la concubine de quelqu'un disposant des moyens de racheter son contrat et de l'entretenir.

De par leur rôle, ces maisons sont donc de hauts lieux d'intrigues, et les gestionnaires des plus grandes maisons ont un rôle majeure dans la vie de la cité. Les scandales n'y sont pas non plus inconnus : l'intimité que l'on peut obtenir dans ces maisons permet d'y retrouver un amant ou une amante par exemple, et les épopées alcoolisées de certains ont pu faire l'objet de légende. Celles du roi dans sa jeunesse par exemple, arrivent régulièrement à passer outre la censure et le travail de la garde. Les courtisanes et courtisans sont également le sujet de nombreuses histoires, au cours desquelles ils utilisent leur astuce et leur grande connaissance des milieux nobles pour éviter de grandes intrigues politiques, ou sauver quelqu'un d'important.

Les deux plus connues de la ville sont Les Trois Pinsons, et L'inspiration du Poète. La première de ces maisons est possédée par une ancienne Kisaeng, qui maintient un impressionnant réseau de relations jusqu'au sein du palais. La seconde est également connue pour son équipe de cuisiniers, de nombreux fonctionnaires s'y rendant pour les mets qui y sont servis.
\subsection{Les parcs de la ville}
De nombreux parcs de toute taille sont logés entre les bâtiments de la ville, mais trois d'entre eux sont plus grands ou connus que les autres.

Le premier est le jardin du sanctuaire de Bulguksa, un gand sanctuaire proche du centre-ville. Il est connu pour être un îlot de calme au milieu de l'agitation, que les moines maintiennent ainsi. Il est soigneusement entretenu grâce aux dons de riches citadins, et représente par son organisation la totalité de \nomroyaume.

Le second est un peu à l'extérieur de la ville, non loin du Palais de Changdeok. Il s'agit du parc d'Ichon, du nom du lac qui en couvre une partie. Il s'agit d'un lieu de balade pour la classe moyenne, très apprécié des amoureux pour ses paysages magnifiques, et sa situation excentrée.

Enfin, le jardin de Changdeok est sis au pied des murs du palais, et sert aux fonctionnaires, ainsi qu'à la noblesse du royaume. Si la bannière rouge de la garde flotte à son entrée, c'est que le roi s'y rend pour en profiter.

De manière générale, les parcs de Daegul sont fait dans le style de \nomroyaume : le parc doit 'être plus naturel que la nature', en limitant à tout prix les éléments artificiels ou purement représentatifs.
\subsection{Le grand marché}
Le grand marché est sur la plus grande place de Daegul, et il s'agit du principal lieu d'échange et de négoce. Au pied de la porte royale, on trouve les marchands d'épices et de fleurs, puis on descend vers les quartiers populaires avec les artisans, puis les produits alimentaires.

Les rues adjacentes sont également pleines de marchands de rue, qui proposent tous types de produits : nourriture, cerfs-volants, jouets, couteaux, bois de chauffage....

Le grand marché est très souvent patrouillé par la garde de la ville, qui reste souvent impuissante face au voleurs, malgré des peines assez strictes à ce sujet.
\subsection{Les quais du fleuve Yeongsam}
Hors des murs de la ville, le long du Yeongsam, on trouve de nombreux docks, par lesquels transitent une bonne part des marchandises entrant en ville. 

Si certains de ceux-ci sont très bien entretenus et soigneusement patrouillés par la garde. Plus loin de la route du Yeongsam par contre, les quais sont moins bien famés, étant également le quartier avec le plus de tavernes, et d'établissement de plaisirs bas de gamme. 

Plusieurs groupes criminels y font leurs affaires : paris, prostitution, contrebande, vols, et le bon vieux racket. Les familles les plus pauvres de la ville s'y entassent dans des maisons étroites et souvent mal entretenues. Ils sont généralement les premières victimes des groupes criminels qui peuvent profiter facilement de leur faiblesse.
\subsection{Le bureau royal d'investigation}
Le bureau royal d'investigation est un organe administratif chargé d'enquêter sur de nombreux crimes contre l'état. La cinquantaine de fonctionnaires qui le compose est dirigé par un officiel du troisième rang sénior, qui prend ses ordres directement du roi. Ce bureau est considéré comme une section de la garde royale, même si ses membres ne portent pas toujours la robe rouge.

Le premier des crimes traqué par cette unité est la trahison, qui est définie comme 'une action hostile au roi ou à sa position, avec intention et préméditation, ou un complot visant à réaliser une telle action'. Cela a historiquement concerné des membres de la famille royale, mais aussi des officiels influents, et l'accusation a pu être utilisée par certains souverains pour viser un opposant dangereux.

Le second crime traqué est le non-respect de la loi sur les armées privées : les nobles ont le droit à une garde rapprochée, d'une taille maximale de 20 personnes. Le non-respect de cette règle peut être vu comme le premier signe d'une rébellion, et tombe donc sous l'accusation de trahison.

La corruption fait également partie de la juridiction de l'unité. Le fait de recevoir des pots-de-vin est évidemment la plus courante, mais également toute forme de triche aux examens administratifs. D'autres cas de figures existent, mais sont moins courants.

Enfin, le bureau est en charge de régler les enquêtes les plus étranges et inexpliquées se déroulant dans la capitale, et dans certains cas, dans d'autres provinces du royaume. Cela inclut les suspicions de nécromancie illégale, mais aussi les enquêtes trop complexe pour les gardes locaux, mais trop importantes pour être abandonnées.
\section{Au-delà des montagnes}
De l'autre côté des montagnes des quatre vieillards, on trouve l'empire de Han, ou plus précisément, par sept royaumes revendiquant tous le titre d'empereur des Hans.

Deux d'entre eux, les royaumes de Qin et Wan, ont des représentations officielles à la cour de \nomroyaume.

C'est la minorité culturelle sans doute la plus importante dans le royaume, du fait de nombreux échanges, passant soit par les passes de la chaîne des quatre vieillards, soit par la mer orientale. Leur réputation dans le royaume est celle d'être sûrs d'eux, parfois arrogants, mais aussi de respecter l'autorité.
\section{De l'autre côté du Maejong}
Les tribus Junkans sont des tribus nomades vivant dans les grandes forêts de l'est, au rude climat, mais aux ressources abondantes.

Ces tribus forment des confédérations mal comprises par les \nomroyaume, avec des allégeances parfois floues.

Les junkans sont surtout connus pour leurs raids légers le long de la frontière, et plusieurs tentatives d'invasion au cours des siècles. la dernière en date, un siècle plus tôt a été brisée, mais après trois ans de combat qui laissèrent l'est du royaume en ruine.

On trouve de nombreuses personnes d'origine Junkan dans le nord-est du royaume, la dernière conquête territoriale en date, et un peu partout ailleurs du fait d'échanges entre les deux peuples, notamment commerciaux en dehors des périodes de conflit.

Ils sont considérés comme des pisteurs et chasseurs doués, d'excellents archers et sont adeptes à survivre dans le climat hostile de leurs forêts natales.
\section{Les mers}
Les 'pirates' Ilhan sont un peuple de marins des îles du même nom. Ils parcourent les mers pour commercer ou lancer des raids, suivant ce qui est le plus facile ou rapporte le plus. 

Leurs navires sont considérés comme les meilleurs du monde ne matière de vitesse et de capacités de navigation en haute mer, et on dit que leurs capitaine savent toujours où se trouve leur navire.

Dans le royaume, on en trouve essentiellement dans les ports du sud, avec une délégation à la capitale, représentant officiellement la reine Tayakto des neufs îles, souveraine des Ilhans.

Ils ont la réputation dans le royaume d'être débrouillards, bruyants et agitateurs, mais aussi d'avoir les poches pleines et de manier bien leurs longs couteaux de marins.
\chapter{Options de personnages}
Les personnages de cette campagne appartiendront au bureau royal d'investigation, ou seront des gardes royaux qui y seront détachés. Il est possible d'imaginer qu'un 'barbare' (un non-\nomroyaume) soit accepté dans la garde royale, mais il devrait faire face à une résistance certaine, et ses faux-pas ne seraient pas acceptés. Il est possible d'avoir un personnage qui ne soit pas membre d'une de ces deux organisations, si celui-ci est par exemple un serviteur d'un autre personnage, appartenant à une classe sociale très inférieure.

Une des questions importantes pour les personnages sera donc de savoir comment ont-ils réussi à arriver à leur position actuelle : si ils sont dans la garde royale, comment le roi les a-t-ils remarqués? Si ils sont dans le bureau d'investigation, est-ce une volonté d'origine, ou un hasard des suites de l'examen? Comment s'est passé cet examen d'ailleurs ?

Afin de garantir une campagne qui se passe dans de bonne conditions, je déconseille les alignements mauvais, sauf bien justifiés dans le background, et après une discussion avec moi sur le sujet, ou un accord global de l'ensemble des joueuses.

Les règles utilisées sont celles provenant du livre de base, du livre des joueurs avancées et de Gods and magic. Toute autre option (notamment d'ascendance et de dons) est soumise à approbation.
\section{Ascendances}
\subsection{Nains}
Les nains de fantasy classique : petits et robustes. Pas de soucis particulier entre eux et les elfes par contre, ni le côté chiant du nain de naheulbeuk.

Les nains sont relativement courant dans le royaume, en particulier dans l'est. Ils n'ont généralement aucun problème d'intégration. On les retrouve à tous les niveaux de la société, des esclaves à la noblesse.
\subsection{Elfes}
Des elfes plutôt classiques de fantasy. Leur durée de vie peut être plus longue, mais les maladies, guerres et blessures, font que celle-ci dépasse rarement les 150 ans en pratique.

Les elfes sont beaucoup plus rares, la plupart ayant des origines Han. Leur longue vie leur donne un avantage certain en matière d'apprentissage, ce qui a permis à certains de réussir les examens royaux.
\subsection{Humains}
Je ne vous fait pas de dessins, internet en a déjà, certains dans des tenues décentes. Inclut également les demi-elfes et demi-orques.

Les humains forment la grande majorité des habitants du royaume, et la majorité de sa noblesse.Les demi-sangs sont généralement mal vus par contre, et restent assez rares dans à \nomroyaume
\subsection{Gnomes}
A part une petite communauté de gnomes dans le port de Boseong, il en existe peu dans le royaume, et ils disposent d'un statut spécial, leur permettant de bénéficier de certains privilèges dans cette ville.
\subsection{Gobelins}
Les gobelins sont petits, généralement verdâtres, et avec une peau presque ridée. Ils se reproduisent rapidement, et sont capable de consommer presque tout ce qui est organique.

Les gobelins sont assez nombreux, mais généralement mal vus, de par leur réputation à créer du désordre. Ils sont très communs dans les couches les plus basses de la société, même si certains ont pu s'élever jusqu'à des postes importants. Ils sont plus nombreux dans les ports du sud, où leur petite taille en fait des marins appréciés.
\subsection{Halfelins}
Des humanoïdes de petite taille, mais plus fluets que les nains. Ils ont une particularité physique : leurs pieds sont particulièrement poilus.

Les halfelins sont peu nombreux dans le royaume, et la plupart ont une origine Junkan, ce qui les rend suspects aux yeux de beaucoup de monde dans le royaume.
\subsection{Tengu}
Les tengu sont des hommes-oiseaux, la plupart ressemblant partiellement à des corbeaux, mais il semblerait que suivant les groupes, beaucoup de rapaces, grues et autres oiseaux sont possibles.

Les tengu sont peu nombreux, mais hautement respectés dans la société \nomroyaume. Plusieurs communautés plutôt importantes existent dans les montagnes des quatre vieillards, où ils ont plusieurs fois donné l'alerte face à une invasion venue de Han.
\subsection{Kitsune}
Les kitsune sont des hommes-renard, et la légende veut qu'ils servent ou servaient de messager à certains dieux. Cela pourrait venir des renards, à qui l'on prête ce rôle également. Utilisent les règles des catfolk.

Les kitsune sont présent sur tout le continent, mais en petite quantité. Ils sont généralement très respectés, de par leur lien avec la nature notamment.
\subsection{Orques}
Les orques sont de grands humanoïdes, généralement verdâtres, avec des canines proéminentes, que certains utilisent comme arme à part entière. 

Les orques sont moyennement nombreux, beaucoup trouvant un emploi dans l'artisanat ou l'armée de \nomroyaume. Peu ont réussi l'examen royal, mais cela change ces dernières années, avec la famille Jinju qui a été anobli il y a six ans.
\subsection{Aasimars}
Les aasimars appartiennent à une autre ascendance, à laquelle s'est mêlée une influence d'extérieurs bénéfiques : ange, azata, archon... Cette ascendance est parfois subtile, mais peut se manifester avec des ailes ou une halo doré dans les cas les plus visibles.

Les aasimars sont vus comme des bénédictions divines, leur présence signifiant que les cieux sont satisfaits. Ils bénéficient souvent de traitements de faveurs, et sont fortement représentés dans la noblesse.
\subsection{Tiefelins}
Les tiefelins sont proches des aasimars, mais la seconde influence de leur ascendance vient d'extérieurs maléfiques : diable, démon, daemon, oni... Ils ont presque tous des cornes, une queue, et souvent la peau d'une couleur étrange.

A l'inverse de leurs cousins, les tiefelins sont vus comme le signe d'une forme de corruption, et leur présence peut coûter son statut à une famille noble. Ils sont donc fréquemment abandonnés et doivent apprendre à se débrouiller seuls.
\subsection{Dhampirs}
Les dhampirs sont les rejetons mortels de vampires. Cela peut arriver par exemple lorsqu'une femme enceinte est infectée par le vampirisme, quand un mortel et un vampire font un enfant. Ils ont souvent un teint pâle pour leur ascendance, et des crocs plus développé. Cet état se cumule avec une autre ascendance.

Rarissimes, les dhampirs de la région restent le plus souvent discrets sur leur nature, celle-ci étant rarement bien comprise par la population.
\subsection{Marchelimbes}
Les marchelimbes ont été créés par les serviteurs de Maegu et de Xiwangnu, à l'origine pour aider ces deux divinités dans leur rôle de préservation du cycle de toute chose. Ce sont des âmes réincarnées, sans grands souvenirs de leurs vies passées.

Malgré leur rareté, ils bénéficient d'un grand statut, en particulier auprès des religieux du royaume. Le peuple ne les comprend pas forcément, mais sait qu'ils furent créés par les dieux eux-mêmes, et leur témoignent souvent un grand respect.
\subsection{Les changelins}
Dans les campagnes, ont parle de changelins qui remplaceraient des bébés normaux par l'intervention de fées de quelques sortes. Ces enfants sont les rejetons d'une guenaude (une ignoble créature ayant donné légende aux pires histoires de sorcières), et d'un membre d'une autre ascendance. Le rejeton ressemble à l'ascendance de son père, mais l'influence de sa mère est importante. La seule vraie distinction est la présence d'un oeil de la couleur de celui de leur père, et d'un de celui de leur mère (mais évidemment, cela peut avoir d'autres explications). Dans leur vie, il peuvent ressentir l'Appel, qui, si il est répondu, amènera le rejeton à subir l'atroce transformation en guenaude à son tour. Cet appel est beaucoup plus commun et fort chez les femmes.

Paysans comme nobles se méfient des changelins, mais leur nature les rend difficile à identifier. On dit qu'ils ont le mauvais oeil, et on leur attache facilement tout affaire vaguement occulte. La plupart sont donc très discrets quand à leur ascendance, et essaient de dissimuler leur regard autant que possible.
\section{Classes}
Les classes les plus emblématiques sont le duelliste, le guerrier, l'investigateur, le mage, le moine ou l'alchimiste, mais la plupart des classes peuvent être jouées sans aucune difficulté à \nomroyaume.
\subsection{Alchimiste}
Les trois spécialités de l'alchimie peuvent exister dans à \nomroyaume: les alchimistes de la court (et leurs apprentis) sont fréquemment des mutagénistes, l'armée forme et a toujours besoin de bombardiers, et les chirurgiens sont relativement fréquents en ville. Les plus prestigieux des alchimistes ont réussi l'examen royal divers, généralement en médecine ou en sciences naturelles.

Ils ne sont pas commun au sein de la garde royale (sauf les grenadiers), mais peuvent être employés par le bureau royal d'investigation, où leurs compétences peuvent être mises à profit.
\subsection{Barbare}
Les barbares sont assez rares dans le royaume, où la mesure et la retenue sont souvent favorisée. L'armée forme toutefois quelques troupes de chocs dans ce genre de techniques, et la garde royale en maintient une petite unité.. Les barbares totémiques sont pratiquement inexistant au sein des frontières, mais peuvent être trouvés, notamment parmi les tribus Junkans.

La garde royale peut en employer, suivant les désirs du roi, mais cela reste assez rare, la retenue étant bien souvent privilégiée dans la société de \nomroyaume.
\subsection{Guerrier}
Les quatre armes font parties de la tradition militaire de \nomroyaume . Il s'agit de la coutille (appelée guandao), de l'épée courte courbe et bouclier, de l'épée longue droite et de l'arc, arme de la noblesse par excellence. Ce sont les  plus grandes traditions martiales, mais de nombreuses autres existent également, parfois limitées à une ou deux écoles, ou à un corps d'armée spécifique.

Les expert dans le maniement des armes et de la tactique militaire sont très recherchés par la garde royale, qui essaie de se maintenir en tant que force militaire de choc. Ils sont donc très nombreux, avec une bonne variété sur les armes maniées, même si tous ou presque maîtrisent les quatre armes.
\subsection{Champion}
Les champions ne sont pas très communs au sein de \nomroyaume, du fait de la présence de la religion. Toutefois, certains monastère importants maintiennent une force en armes, et certains militaires se dédient à un idéal.

Les plus courant au sein de la garde royale sont ceux qui ont prêté serment face à Hayan ou Guang, mais d'autres ont pu y arriver.
\subsection{Clerc}
Les clercs sont généralement peu nombreux, à part dans les monastères et les quelques grands sanctuaires du pays. Le peuple vient souvent leur demander conseil, car leur sagesse est respectée. Le type de conseil demandé dépend évidemment 

 La garde royale dispose de quelques aumôniers, et il est possible que l'un d'eux soit détaché au bureau d'investigation. Les plus courant au sein de la garde vénèrent Hayan ou Guang, mais les ancêtres et Maegu sont d'autres vraies possibilités.
\subsection{Ensorceleur}
Les ensorceleurs ne sont généralement pas très bien considérés par la noblesse, mais peuvent l'être un peu plus par le peuple. L'armée les empêchera souvent de monter dans les rangs, mais leur propose un emploi stable dans lequel leurs talents peuvent être mis à profit. Les sources de pouvoir les moins reluisantes restent peu appréciée toutefois, même dans ce contexte.

La garde royale peut apprécier d'en employer, de par leur polyvalence et bonne capacité de réaction. Encore une fois, il vaut mieux que l'ensorceleur soit capable de garder les apparences.
\subsection{Mage}
Les mages sont des membres respectés de la société : leur apprentissage de la magie est généralement considéré comme proche du système d'apprentissage pour les examens royaux, et peut fournir de bonnes perspectives sociales.

Les mages peuvent tout à fait appartenir au bureau d'investigation, en particulier si ils ont passé l'examen, ou à la garde royale, qui peut utiliser leurs capacités spécifiques.
\subsection{Ranger}
Les rangers sont peu communs dans les grandes villes du royaume, mais restent appréciés tant dans les campagnes que parmi les unités militaires. Certains des meilleurs viennent des tribus Junkans, et peuvent espérer trouver un emploi comme éclaireurs ou pisteurs. Les nobles peuvent également en employer pour mener leurs parties de chasse, un passe-temps populaire, en particulier pour les futurs officiers, et qui permet de limiter les dangers courus par le peuple dans les régions sauvages.

Plus d'un ranger a déjà été sélectionné dans la garde royale pour avoir montré son efficacité lors d'une partie de chasse royale, et d'autres ont pu partager cette tradition au sein du bureau d'investigation.
\subsection{Roublard}
Les roublards ne sont pas respectés dans la société de \nomroyaume, qui privilégie l'avancement légal, basé théoriquement sur le mérite et les connaissances. La débrouillardise des roublards, si elle est efficace, n'en est pas moins à l'antithèse de la réflexion qui est privilégiée. Cela n'empêche pas ce genre de techniques d'exister, notamment dans les quartiers criminels, ou dans certaines unités militaires.

Un d'eux aurait pu arriver jusqu'au sein de la garde royale, mais le plus probable reste d'arriver dans le bureau d'investigation après l'examen royal.
\subsection{Moine}
Les moines forment un ensemble de traditions très respectée au sein de \nomroyaume : la plupart des monastères forment des combattants pour travailler autant l'esprit que le corps, l'armée considère le combat à main nue comme un parfait entraînement moral et physique, et de nombreux jeunes gens apprennent des rudiments pour se défendre et pratiquer une activité.

La garde royale a recruté de tels individus à de nombreuses reprises, afin de profiter de leurs grandes capacités martiales, et de leur capacité à se battre dans de nombreuses situations.
\subsection{Druide}
Les druides sont étrangement courant à \nomroyaume, le shamanisme étant très souvent la norme dans le pays. Ils interprètent la volonté de la nature et essaient de calmer ses fureurs. Même la noblesse peut les consulter sur de nombreux points.

Ils sont plutôt rares dans l'administration, et donc dans le bureau d'investigation, mais la garde royale a pu en accueillir quelques uns au cours de son histoire.
\subsection{Barde}
Les bardes sont relativement communs : tout le monde, du peuple au roi, peut apprécier une bonne représentation. Évidemment, les bardes de la noblesse ne chantent pas les mêmes airs que ceux du peuple, mais les deux traditions sont souvent plus proches qu'on ne pourrait le penser. De nombreux kisaeng peuvent d'ailleurs être considérés comme des bardes.

Il est arrivé dans au moins un cas qu'une ancienne kisaeng réussisse l'examen royal pour devenir administratrice, et certains musiciens militaires ont été nommés à la garde royale pour leur héroïsme.
\subsection{Bretteur}
Une des traditions les plus prestigieuses dans l'art militaire est sans conteste le duel, pratiqué pour s'entraîner avec de nombreuses armes, en particulier par la noblesse. On attends des officiers qu'ils puissent défendre leur honneur par ce biais dans certains cas d'ailleurs. Cela a donné naissance à de nombreuses écoles centrées sur cet art bien spécifique.

La garde royale est connue pour disposer et former de certains des meilleurs bretteurs du royaume, donnant des sueurs froides à de nombreux nobles enclins à la trahison. Ils s'entraînent régulièrement avec les maîtres d'armes royaux, parmi les meilleurs de leur génération.
\subsection{Investigateur}
Il s'agit vraiment d'une profession de niche, généralement reliée à des postes spécifiques au sein de l'administration. En particulier au sein du bureau royal d'investigation, ou des gardes des grandes villes. Leur faible nombre ne signifie pas toutefois que leurs techniques ne sont pas bonnes, les expériences passées étant souvent transmises par différentes archives et rapports. Bon nombre d'entre eux auront passé un des examens royaux, sans doute le littéraire, ou ceux de médecine ou de sciences naturelles (suivant leur spécialité).

Il s'agit de la classe la plus répandue par les enquêteurs du bureau royal d'investigation.
\subsection{Sorcier}
Les sorciers sont rarement bien vus dans la société de \nomroyaume : on leur prête bien souvent de sombres desseins de par leur pacte. Ils agissent donc le plus souvent cachés, ou dans l'armée, celle-ci pouvant fermer les yeux si les résultats sont présents.

Les tâches de la garde royale ont pu l'amener à recruter quelques sorciers dans son histoire, de même que le bureau d'investigation.
\subsection{Oracle}
Les oracles, comme les clercs, sont généralement très bien perçus. Parmi les plus communs, nombre d'hermites ou de moines mendiants peuvent potentiellement être des oracles. En particulier, les oracles des ancêtres, ou du cosmos, sont très bien considérés par tous.

De part sa responsabilité d'enquêtes étranges, il est arrivé au bureau d'investigation de recruter, ne serait-ce que temporairement des oracles. La garde royale n'a pas d'exemple connu, mais cela ne serait pas absolument impossible.
\section{Religions et domaines}
\subsection{Ancêtres}
\paragraph{Edits}Apporter honneur et fierté à sa famille, prolonger sa lignée
\paragraph{Anathème}Déshonorer son nom et sa famille
\paragraph{Alignement des fidèles} LN, LB, N
\paragraph{Type de canalisation}soin
\paragraph{Compétence et arme favorites} société et épée longue
\paragraph{Domaines}ambition, rêves, famille, richesse
\paragraph{Sorts de clercs}true strike (1er niveau), see invisibility (2eme niveau), creation (4eme niveau)
\subsection{Maegu}
\paragraph{Edits} Faire régner la justice, et amener chacun à recevoir ce qu'il mérite
\paragraph{Anathème}Accepter la corruption
\paragraph{Alignement des fidèles} NB, LN, N
\paragraph{Type de canalisation}Soin ou blessure
\paragraph{Compétence et arme favorites} diplomatie et masse d'arme
\paragraph{Domaines}Mort, destin, secrets, vérité.
\paragraph{Sorts de clercs}mindlink (niveau 1), see invisible (niveau 2), suggestion (niveau 4)
\subsection{Hayan}
\paragraph{Edits}Maintenir l'ordre et la loi, respecter le roi.
\paragraph{Anathème}Désobéir à un ordre de son supérieur
\paragraph{Alignement des fidèles}LB, LN, LM
\paragraph{Type de canalisation}soin ou blessure
\paragraph{Compétence et arme favorites}société et arc long
\paragraph{Domaines}ambition, cités, confiance et perfection
\paragraph{Sorts de clercs}true strike (niveau 1), haste (niveau 3), fire shield (niveau 4)
\subsection{Sedang}
\paragraph{Edits}tenter d'obtenir la meilleure oeuvre, répandre la connaissance
\paragraph{Anathème}détruire des connaissances, censurer un artiste
\paragraph{Alignement des fidèles} N, CB, CN
\paragraph{Type de canalisation}soin
\paragraph{Compétence et arme favorites}performance et éventail de guerre
\paragraph{Domaines}Création, rêves, destin et connaissance
\paragraph{Sorts de clercs}color spray (niveau 1), dream message (niveau 3), creation (niveau 4)
\subsection{Guang}
\paragraph{Edits}Défendre le royaume et les siens
\paragraph{Anathème}Fuir ou refuser un ordre militaire
\paragraph{Alignement des fidèles} LN, N, CN
\paragraph{Type de canalisation}Soin ou blessure
\paragraph{Compétence et arme favorites}athlétisme, et coutille
\paragraph{Domaines}Force, destruction, terre et protection
\paragraph{Sorts de clercs}true strike (niveau 1), earthbind (niveau 3), wall of fire (niveau 4)
\subsection{Jayeon}
\paragraph{Edits}Profiter des dons de la nature, aider ceux qui en vivent
\paragraph{Anathème}Détruire un lieu naturel
\paragraph{Alignement des fidèles}CB,CN, N
\paragraph{Type de canalisation}soin
\paragraph{Compétence et arme favorites}nature et trident
\paragraph{Domaines}air, nature, soleil et eau
\paragraph{Sorts de clercs}gust of wind (niveau 1), lightning bolt (niveau 3), control water (niveau 5)
\subsection{Chi Di}
\paragraph{Edits}Aider son prochain en cas de besoin, améliorer ses connaissances et compétences
\paragraph{Anathème}refuser d'aider quelqu'un quand on en a la possibilité
\paragraph{Alignement des fidèles}LB, NB, CB
\paragraph{Type de canalisation}soin
\paragraph{Compétence et arme favorites}soin et bâton
\paragraph{Domaines}feu, soin, voyage et richesse
\paragraph{Sorts de clercs}burning hand (niveau 1), dream message (niveau 3), creation (niveau 4)
\subsection{Simianshen}
\paragraph{Edits}Ne pas trahir un secret, propager la magie
\paragraph{Anathème}dénoncer un coreligionnaire
\paragraph{Alignement des fidèles}CN, CB, CM
\paragraph{Type de canalisation}soin ou blessure
\paragraph{Compétence et arme favorites}dague, roublardise ou arcane
\paragraph{Domaines}magie, lune, secret, tromperie
\paragraph{Sorts de clercs}illusory disguise (niveau 1), invisibility (niveau 2), phantasmal killer (niveau 4)
\subsection{Canmu}
\paragraph{Edits}célébrer l'amour, encourager la création
\paragraph{Anathème}détruire la beauté
\paragraph{Alignement des fidèles}CB, NB, CN
\paragraph{Type de canalisation}Soin
\paragraph{Compétence et arme favorites}siplomatie et épée courte
\paragraph{Domaines}liberté, excès, perfection et passion
\paragraph{Sorts de clercs}color spray (niveau 1), enthrall (niveau 3), hallucination (niveau 5)
\subsection{Xiwangnu}
\paragraph{Edits}Penser dans la durée, aider le cycle naturel des choses
\paragraph{Anathème}détruire un vieil objet, ou un savoir ancestral
\paragraph{Alignement des fidèles} LN,N,LM
\paragraph{Type de canalisation}soin ou blessure
\paragraph{Compétence et arme favorites}occultisme et cimeterre
\paragraph{Domaines}Création ,destruction ,chance mort-vivants
\paragraph{Sorts de clercs}
\subsection{Heixian}
\paragraph{Edits}Ne cachez pas les malheurs à venir, mais n'abandonnez pas
\paragraph{Anathème}Accepter une catastrophe sans rien faire
\paragraph{Alignement des fidèles}CN, LN, N
\paragraph{Type de canalisation}soin ou blessure
\paragraph{Compétence et arme favorites}intimidation et faux
\paragraph{Domaines}Ténèbres, cauchemar, douleur, tyrannie
\paragraph{Sorts de clercs}phantom pain (niveau 1), false life (niveau 2), mask of terror (niveau 7)
\subsection{Tongbulgyo}
\paragraph{Edits}Se détacher des plaisirs et pouvoirs temporels
\paragraph{Anathème}Eloigner un fidèle de son illumination
\paragraph{Alignement des fidèles} CN, LN, N
\paragraph{Type de canalisation}soin ou blessure
\paragraph{Compétence et arme favorites}religion et bâton
\paragraph{Domaines}Destin, perfection, vérité, zèle
\paragraph{Sorts de clercs}jump (niveau 1), haste (niveau 3), stoneskin (niveau 4)
\section{Les dix questions}
Voici une dizaine de questions qui seraient intéressant de considérer pour votre background.
\begin{enumerate}
\item Comment s'appelle votre personnage?
\item Quel est le statut social de la famille de votre personnage ?
\item Qui sont les membres de la famille du personnage ?
\item Comment votre personnage a-t-il acquis ses compétences ?
\item Comment votre personnage a-t-il été nommé au bureau d'investigation ou choisi par le roi pour sa garde ?
\item pourquoi votre personnage sert-il dans le bureau d'investigation ou la garde royale?
\item Pourquoi êtes-vous à travailler pour le bureau d'investigation
\item A qui est loyal votre personnage ?
\item Quelle ambition a le personnage, quel projet ?
\item Le personnage a-t-il un passe-temps ou un secret particulier ?
\end{enumerate}

\end{document}