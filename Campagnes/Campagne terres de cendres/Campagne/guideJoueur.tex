\documentclass[letterpaper,10pt,twoside,twocolumn,openany]{book}
\usepackage[english]{babel}
\usepackage[utf8]{inputenc}
\usepackage{hang}
\usepackage{lipsum}
\usepackage{listings}

\usepackage{dnd}

\lstset{%
  basicstyle=\ttfamily,
  language=[LaTeX]{TeX},
}
\title{\nomcampagne : guide du joueur}
\date{}
\author{Antoine ROBIN}
% Start document
\begin{document}
\maketitle
\tableofcontents
\chapter{Introduction}
\nomcampagne est une campagne de gestion de royaume : les héros de cette histoire seront amenés à fonder une nouvelle cité, et la diriger afin qu'elle se développe et prospère.
\chapter{La sklarkari: de Korvosa à Vigil}
\section{La route en 10 points}
\begin{enumerate}
    \item La Skalarkarie est une route commerciale qui relie l'est varisien au royaume de Dernier Rempart, en passant par les terres de cendres et la horde de Belkzen. Elle suit la rivière yondabakari jusqu'aux monts d'esprit, qu'elle contourne, avant suivre la rivière Esk jusqu'à Vigil, capitale de dernier rempart.
    \item La varisie est grande, et dirigée par plusieurs cité-état indépendantes : Korvosa, Magnimar et Port-énigme. On y trouve des barbares shoantis, des nomades(à l'origine) varisiens, des colons chélaxiens, et des voyageurs du monde entier.
    \item La citadelle de Janderhoff est une puissante forteresse bâtie il y a des millénaires par les nains. Ses forges en font toujours sa richesse. Elle est une des anciennes forteresses célestes, et abrite toujours un passage, lourdement fortifié, vers l'outreterre.
    \item Les terres de cendres sont des plaines désolées, ravagées par de fréquents incendies qui balaient ces grandes étendues en brûlant la végétation sèche. Elles sont arides, mais pas désertiques. 
    \item Sur les terres de cendre et le plateau de Storval vivent les tribus shoantis, des barbares chassés du sud varisien par les colons chélaxiens il y a bien longtemps. Ils sont divisés en sept quahs (clans) aux territoires et traditions bien spécifiques.
    \item à l'est des terres de cendre vivent les tribus de Belkzen, du nom d'un ancien conquérant orque ayant réussi à unifier les tribus sous sa bannière. Le chaos y règne, les tribus se livrant plus souvent la guerre entre elle qu'avec le reste du monde.
    \item La culture varisienne met en valeur le voyage, et certains ne tiennent pas en place plus de quelques années. Ils sont parfois mal considérés dans les anciennes colonies chélaxiennes associés pour beaucoup aux criminels de la Sczarni.
    \item Korvosa est la plus grande des anciennes colonies chélaxienne, ayant son indépendance depuis la mort d'Aroden et la guerre civile qui a suivi. Les non-chélaxiens y sont toujours moins bien vus, mais le commerce est plus important. La ville ressort d'une période de chaos suite à la mort du roi, puis de la reine, dans des circonstances toujours floues. [nouveau gouvernement ?]
    \item Au nord des terres de cendres, on trouve les monts Kodar, une haute chaîne de montagnes, mal explorées et très hostile. Au sud, les monts de l'esprit, dont la façade occidentale ainsi que le sud sont bien connus, le nord est plus hostile, en raison de géants, d'orques et de nombreuses autres créatures.
    \item Toute la région faisait partie de l'ancien empire du Thassilon, un empire s'étant effondré il y a des millénaires de cela. Au sommet de sa puissance, les 7 royaumes qui le formaient, dirigés par des puissants seigneurs des runes, dominaient des millions d'hommes, de géants et d'autres créatures. La capitale de l'un de ses royaumes, Xin-Shalast aurait récemment été retrouvée par un groupe d'aventuriers dans les monts Kodar.
\end{enumerate}
\section{Korvosa, la porte de la Varisie}
Korvosa est la plus grande ville de Varisie, et accueille de nombreuses organisations, comme sa célèbre Academae de magie, la garde de sable, ainsi qu'un chapitre des Eclaireurs. 

Il s'agit d'une ancienne colonie de l'empire du Chéliax, qui a déclaré son indépendance au moment de la grande guerre civile qui a secoué le pays, et a vu l'avènement de la dynastie régnante actuelle. Depuis, la monarchie de Korvosa a développé son propre réseau de colonies, le long de la côte varisienne et vers l'intérieur des terres. Il s'agit avant tout d'une ville commerçante,avec un grand port de commerce, et de nombreuses routes en parte vers le reste de la Varisie et au-delà. 

Après la chute de la reine Ileosa dans des circonstances encore troubles, la ville se remet du chaos environnant, et s'attache plus que jamais à sécuriser ses propres colonies. La perte d'une des routes commerciales, même s'il ne s'agit pas de la plus profitable, est donc une menace pour la stabilité déjà diminuée de la ville.

\section{Les terres de cendre, terres des shoantis}
Les shoantis sont, à l'origine, les descendants des esclaves de l'ancien Thassilon. Ils contrôlaient autrefois la majeure partie des terres au sud de la Yondabakari, mais ont été repoussés par les colons chélaxiens ayant fondé Korvosa. La ville elle-même a été fondée sur un ancien lieu sacré.

A cause de cela, leurs terres se limitent maintenant aux terres de cendres et au reste du plateau de Storval. Ce sont des terres difficiles, peuplées de nombreuses créatures hostiles et où les ressources sont rares.

Ils sont divisés en 7 Quahs, ou clans, eux-même divisés en plusieurs tribus. L'élément le plus distinctif des shoantis est leurs tatouages, le plus souvent liés à leur quah, leur épreuve ou à leur totem : dans les tribus, les jeunes arrivant à l'âge adultes passent des épreuves afin de marquer cette occasion, et de déterminer si les esprits les favorisent. Certains de ces Quahs, notamment celui du feu, sont connus pour leurs raids sur les terres du sud : ils rasent des fermes, tuent quelques colons, et repartent dans les terres de cendre chargés de leur butin.

Hors des tribus, on trouve des shoantis, ou des gens d'origine shoanti sur toute la côte varisienne, même s'ils sont rarement appréciés par les colons chélaxiens. Ceux qui sont bannis des tribus n'ont souvent pas d'autre choix, à moins de vivre de banditisme près des frontières.
\section{belkzen, terre des hordes}
Nommées d'après un ancien chef de guerre orque ayant unifié les tribus sous sa bannière, il ne s'agit plus depuis longtemps d'un territoire unifié. Si Belkzen avait réussi à imposer sa loi, ses successeurs ne contrôlent généralement que leur propre tribus, en conflit perpétuel avec les autres, ainsi qu'avec les shoanti et les chevaliers de Dernier Rempart.

Il s'agit, la majeure partie de l'année, d'une terre aussi aride qu'hostile, à l'exception de la fonte des neiges sur les montagnes environnantes. Dans ce dernier cas, des inondations aussi rapides que violentes ravages les plaines centrales : aussi hostile, mais dans un style différent. Cette période est aussi l'occasion d'une trêve générale entre les tribus, ce qui leur permet de régler certains différents, d'échanger des ressources, ou de négocier des accords.


\section{Kaer Maga, la cité des falaises}
Kaer Maga est un bâtie au sommet du mur de Storval, une falaise haute à cet endroit de près d'un kilomètre, et jamais sur toute sa longueur, de moins de 300m. 

Il s'agit d'une cité très ancienne, déjà vieille au moment de l'ancien Thassilon. Protégée par son monstrueux mur d'enceinte hexagonal, il s'agit du dernier bastion civilisé au nord du mur de Storval.

Il s'agit également d'un des marchés les plus hétéroclites du nord d'Avsitan (le continent), et la rumeur veut que tout ce que l'on cherche finisse forcément par apparaître sur les étals de la ville. On peut aussi y trouver de nombreux fugitifs, qui peuvent ici faire profil bas loin de toute autorité.

Il n'y a pas de gouvernement central, chacun des onze districts de la ville se gérant à sa façon : l'un est contrôlé par une famille d'enchanteurs, un autre n'impose aucune loi, un troisième est géré par un conseil de marchands.
\section{Factions non-étatiques}
\subsection{Sczarni}
La sczarni est un ensemble de groupes criminels, presque exclusivement constitué de varisiens d'origines. Contrebande, paris (truqués), courses de chevaux, jeux d'argent.... Ils sont plutôt spécialisé dans le crime avec peu de risque. Cela ne veut pas dire qu'ils sont sympathiques : essayer de les doubler est souvent synonyme de passage à tabac dans une ruelle sombre, ou d'un coup de couteau discret.

Ce n'est pas une organisation unifiée, mais plutôt un ensemble de petits groupes rivaux. La plupart n'apprécient pas les étrangers, sauf pour les délester de toutes leurs richesses. Ils sont souvent basé sur un seul clan varisien, allant de quelques individus à plusieurs centaines de voleurs, arnaqueurs et brutes.
\subsection{Ordre des clous}
Les chevaliers infernaux de l'ordre des clous sont basés dans la forteresse Vraid, non loin de Korvosa. De là, les chevaliers et signifer de l'ordre opèrent dans toutes la région.

Le credo des chevaliers infernaux en général peut se résumer à "la loi et l'ordre, quel qu'en soit le prix". Celui des chevaliers de cet ordre en particulier, consiste à dire que la plus grande menace contre l'ordre vient de l'extérieur de la civilisation : les tribus barbares, et leurs pillards sont une menace plus grande que les criminels intérieurs. Ou du moins, d'autres chevaliers infernaux servent déjà à chasser ces menaces.

Un aspirant chevalier doit prouver sa valeur lors de son initiation. Le point culminant de cette cérémonie est le combat à mort entre l'aspirant et un diable invoqué par les signifier de l'ordre. Il est aussi possible de devenir chevalier si l'on défait un diable puissant et qu'un chevalier existant témoigne que le combat était au moins à mesure de l'épreuve officielle.
\subsection{Consortium Aspis}
Le consortium Aspis est une des plus puissantes organisations commerciales du monde.
\subsection{Chevaliers d'Ozem}
Les chevaliers d'Ozem ont fondé le royaume de dernier rempart pour empêcher le retour de Tar-Baphon, un mort-vivant si puissant qu'il ne put qu'être enfermé dans les ruines de sa citadelle.

Aujourd'hui, ces chevaliers défendent leur royaume contre les morts et ceux qui les servent, mais aussi contre les hordes de Belkzen, qui repoussent progressivement leurs lignes défensives. La ligne de front est de plus en plus difficile à tenir pour ces chevaliers, dont les pertes augmentent. Et la croisade pour sauver le Mendev, au nord-est, accapare une bonne partie des donations potentielles pour leur croisade.

Récemment, la frontière nord-est a vu ses défenses tomber rapidement, les chevaliers ont été forcé de se replier sur une nouvelle ligne, construite à la hâte. Toutefois, ce repli les a forcé à abandonner la route commerciale passant par les terres de cendres, la fin de la route est aujourd'hui sous le contrôle des tribus orques.
\chapter{Gérer un domaine}
Il faut différents rôles pour diriger un domaine
\chapter{Personnages de \nomcampagne}
\section{Races et origines}
Toutes les races et origines sont possibles, certaines sont plus communes dans l'est Varisien.
\section{Classes et archétypes}
Toutes les classes et tous les archétypes sont possibles, mais certains collent plus à la campagne que d'autre.
\section{Alignement et divinités}
Tous les alignement et divinités sont possibles, mais il vaut mieux s'assurer que l'alignement du groupe soit cohérent. Ou que tout le monde soit OK avec cette différence.
\section{Traits de campagne}
TODO : créer des traits de campagne pour celle-ci
\chapter{Règles utilisées}
Toutes les règles officielles seront acceptées. Si un supplément en particulier semble intéressant, n'oubliez pas de le signaler, afin que le MJ puisse en prendre connaissance plus détaillée. 

En particulier, les règles d'\textit{Ultimate Campaign} seront centrales à la campagne, du moins les sections détaillant la gestion d'un royaume et d'une communauté.

Au sujet des règles maisons classiques : 
\begin{itemize}
    \item Leadership est autorisé, même si l'abus de ce don ne le sera pas. Discutez-en avec le MJ.
    \item 
\end{itemize}

Le matériel créé par d'autres que Paizo ne sera pas utilisé, sauf après demande d'un joueur, et validation par le MJ, afin d'éviter ce qui semblerait soit incompatible avec le thème de la partie, ou posant des problèmes d'équilibrage.
\end{document}