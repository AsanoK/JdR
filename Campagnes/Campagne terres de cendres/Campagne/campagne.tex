\documentclass[letterpaper,10pt,twoside,twocolumn,openany]{book}
\usepackage[english]{babel}
\usepackage[utf8]{inputenc}
\usepackage{hang}
\usepackage{lipsum}
\usepackage{listings}

\usepackage{dnd}

\lstset{%
  basicstyle=\ttfamily,
  language=[LaTeX]{TeX},
}
\title{\cftchapfont   \nomcampagne}
\date{}
\author{}
% Start document
\begin{document}
\maketitle
\tableofcontents
\chapter{Introduction}
\chapter{Départ pour les terres de cendres}
%Où les personnages partent de Korvosa, avec la bénédiction d'une faction à définir, pour se diriger vers les terres de cendre. 
\section{Introduction et vue générale}
Ce premier chapitre permet d'introduire les personnages aux terres de cendre, ainsi qu'à leurs premiers objectifs à long terme.

En effet, il recevront leur première mission : celle d'explorer le long de la yondabakari afin d'y trouver un lieu pour établir un avant-poste. Le chapitre dure le temps de ce trajet, et jusqu'à ce que les personnages trouvent un emplacement adéquat, en en chassant les occupants précédents, créatures ou humanoïdes.
\begin{commentbox}{Montées de niveau}
Les personnages devraient arriver niveau 2 en arrivant à la limite des terres de cendres. Ils devraient ensuite arriver niveau 3 après quelque temps d'exploration et de découverte de la région. Leur niveau 4 devrait arriver 
\end{commentbox}
\section{Nos héros, et leur mission}
Les personnages commencent dans le manoir de la maison Orsano, une maison mineure récemment arrivée depuis le Chéliax. Ils ont répondus à une annonce ou ont été contacté par des soldats portant l'emblème de la maison.

C'est Dame Valeria Orsano qui leur indique son offre : face à la menace des orques de Belkzen et des tribus shoanti, la route commerciale jusqu'à Vigil, capitale de Dernier Rempart, n'est pas sûre, et pourtant plus critique (et profitable) que jamais. Dame Orsano cherche donc à améliorer la sécurité de cette route qu'elle exploite grandement, mais la garde de Korvosa ne s'aventurera pas dans les terres de cendres pour ce genre de tâches. Elle aurait besoin d'aventuriers comme les personnages pour trouver où installer un avant-poste, puis pour défendre la route contre les différents dangers qui menacent les caravanes.

Les personnages pourront profiter d'une caravane marchande de la maison pour se rendre dans les terres de cendres, où ils exploreront seuls les abords de la Yondabakari.

Cette caravane partira le lendemain par la porte nord.
\section{Le trajet pour les terres de cendres}
\section{Le long de la yondabakari}
\end{document}