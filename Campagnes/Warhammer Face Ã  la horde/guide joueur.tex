\documentclass[10pt,a4paper]{book}
\usepackage[utf8]{inputenc}
\usepackage[french]{babel}
\usepackage[T1]{fontenc}
\usepackage{amsmath}
\usepackage{amsfonts}
\usepackage{amssymb}
\author{ Antoine Robin}
\title{Warhammer - guide du joueur}
\begin{document}
\chapter{L'empire}
\section{Géographie}
L'empire est composé de 11 provinces relativement indépendantes, même si toutes reconnaissent l'autorité de l'empereur. Chacune de ces provinces est dirigée par un comte-électeur, et l'empereur, premier d'entre eux.

Les comtes-électeurs sont ainsi nommé car la succession impériale se fait au vote des électeurs. Chaque titre de comte est héréditaire, mais la couronne impériale ne l'est pas. Depuis environ un siècle, celle-ci est revenue dans le giron des princes d'Altdorf, de la province du Reikland, l'une des provinces les plus puissantes de l'Empire.

Chaque province a sa propre culture, façonnée par son histoire, sa géographie, et sa position dans l'empire.

L'averland est une province pour le moment en plein chaos, suite à la mort de son électeur sans successeur clair. Il s'agit d'une province du sud de l'empire, relativement riche, et connue tant pour ses chevaux vigoureux, que ses chevaliers de talents, que pour les tenues et coutumes parfois excentriques de ses habitants. Les couleurs de l'Averland sont le jaune et le noir.

Le Hochland est très majoritairement couvert de forêt, et est probablement la province la plus rurale de l'empire. Ses tireurs d'élite et chasseurs sont connus dans tous l'empire. La province était autrefois une partie du Talabecland. Les couleurs du Hochland sont le rouge et le vert.

Le middenland est une des plus grandes provinces de l'empire, et en compose une bonne part du nord. Il s'agit du coeur historique du culte d'Ulric, qui est plus souvent respecté que Sigmar. Une des deux provinces rivales du Reikland en terme de puissance et d'influence au cours de l'histoire. Ses couleurs sont le bleu et le blanc. Sa capitale est Middeheim, anciennement Carroburg.

Le Nordland est une terre relativement pauvre le long de la côte de la mer des crocs. Souvent raidée par les norses, la province dispose d'une armée relativement puissante, et accueille la flotte impériale. Ses couleurs sont le jaune et le bleu.

L'ostland forme une part de la frontière nord avec le Kislev, et de ce fait, est souvent sur la route des invasions du chaos. Leur réputation est de boire énormément, et la mode locale pour les hommes est de porter un grande barbe fournie, plutôt pratique au vu du climat. Ses couleurs sont le blanc et le noir, son symbole étant une tête de taureau noire.

L'ostermark est une autre province frontalière de Kislev, et de ce fait, est également en première ligne quand le chaos descend vers le sud. Cela donne au Ostermarkers une certaine proximité avec la mort. Ses couleurs sont le violet et le jaune, mais la province étant relativement pauvre, la teinte exacte de violet peut varier grandement suivant les régiments.

Le reikland est le coeur historique de l'empire, c'est de là où Sigmar a unifié les tribus humaine. Il s'agit traditionnellement d'une des provinces les plus riches et puissantes. Sa couleur est le blanc.

Le Stirland est une province de l'est, une des plus grande par son territoire, même si une partie de celui-ci, la Sylvanie, n'est en pratique pas contrôlée : il s'agit de la terre des comtes vampires, où les morts ne se reposent pas. Ses couleurs sont le vert et le jaune.

Le Talabecland est une province centrale, principalement rurale de l'empire. Sa capitale, Talabheim est bâtie dans un grand cratère lourdement fortifié. Comme dans le Hochland Taal est la divinité la plus importante, devant Sigmar. Ses couleurs sont le rouge et le jaune.

Le Wissenland est la province la plus au sud de l'empire, et a souvent dû se défendre face aux peaux-vertes. Il s'agit également d'une province très puissante grâce à son industrie : la forge, les armes, et surtout, le collège des ingénieurs de sa capitale, Nuln. Les couleurs de la province sont le rouge et le noir, ou juste le noir brûlé pour ses artilleurs et ingénieurs.

Enfin, le mootland est une curiosité impériale : il s'agit de la province des halfelins, dont le grand Moot a le statut d'électeur impérial. Sa population est presque entièrement composée de halfelins, et sa gastronomie est réputée dans tout l'empire. Ses couleurs sont le vert et le rouge.
\section{Histoire}
L'empire a été fondé il y a un peu plus de 2500 ans par le premier Empereur : Sigmar Heldenhammer. Celui-ci a rassemblé les nombreuses tribus humaines au nord des montagnes grises et au sud de la mer des griffes. Il a repoussé les nordiques, adorateurs des dieux noirs ainsi que les peaux-vertes à la bataille du feu noir. Il  a bâti l'alliance entre les impériaux et les nains, alliance toujours aujourd'hui respectée.

A sa disparition, les anciens chefs de tribus se mirent d'accord pour élire son successeur parmi eux, système qui perdure encore aujourd'hui par le système des comtes électeurs. 

L'empire a toujours résisté aux crises majeures : invasions du chaos, guerres civiles, l'âge des trois Empereurs, les guerres vampiriques, de nombreuses attaques de peaux-vertes, et d'aurtes dangers. Souvent, les impériaux ont accompli cela seuls, ou avec l'aide des nains. Pendant les incursions du chaos les plus importantes, les elfes d'Ulthuan (ou haut-elfes) ont également pu prêter main forte sur le champ de bataille.

Au cours des dernières années, une invasion du chaos, menée par Archaon, le Héraut de la Fin des Temps a été repoussée, mais avec de lourdes pertes :Valten, la réincarnation de Sigmar a été tué après sa victoire par des assassins étranges, et le nord de l'empire est constellé de ruines encore fumantes. L'empereur Karl Franz a su rallier les troupes impériales pour repousser les plus grandes bandes de guerre, mais de nombreuses plus petites ont sû échapper aux forces impériales épuisées. Et Archaon n'a été que repoussé, et continue de rassembler de nouvelles forces sous sa bannière noire. 

\section{Ennemis extérieurs}
Les ennemis extérieurs de l'empire sont nombreux et acharnés, et sont toujours à ses frontières, attendant leur heure pour frapper.

En premier lieu, les forces venues du nord, vénérant les dieux noirs du Chaos, au nombre de quatre. Ils descendent souvent vers le sud pour raider, ou plus rarement, en une véritable force d'invasion menée par les élus des dieux, ravageant le nord de l'empire.

Quand les vents du chaos soufflent fort, ses enfants répondent aussi : les hommes-bêtes, qui grouillent dans les sombres forêts de l'empire, en sortent en hardes sauvages, qui s'en prennent aux villes et villages mal protégées. Un chef de harde puissant peut rassembler de nombreuses créatures corrompues sous sa pierre de harde, et toutes ont faim de chair humaine.

Les peaux-vertes sont présents en petites bandes dans l'empire, mais sous l'impulsion d'un chef puissant, ces bandes peuvent se rassembler en masse, jusqu'à former une Waaaagh : une horde d'orques et de gobelins, détruisant tout sur son passage au nom de Gork et Mork, leurs dieux impies.

Parfois, lorsque Morrslieb, la lune gibeuse est à son zénith, d'autres créatures sortent des profondeurs : on parle à voix basse de rats marchant comme des hommes, mais ceux qui colportent de telles rumeurs disparaissent souvent, et pour la majeure partie de l'empire, il ne s'agit que de rumeurs ridicules, ou une sorte d'hommes-bêtes un peu étrange.


\section{Menaces intérieures}
Pratiquement depuis sa fondation ,l'empire se méfie des ennemis internes : les dangereux cultistes du chaos, mais aussi les sorciers, et depuis l'arrivée des Von Carstein en Sylvanie, les morts qui marchent.

Les dieux du chaos attirent les faibles d'esprits ou les ambitieux par leurs promesses de pouvoirs et de puissance. De nombreux cultes existent ainsi dans les villes de l'empire, cachés de tous, et pourrissant de l'intérieur la structure de l'empire. Les répurgateurs, liés à l'église de Sigmar, traquent sans relâche ces hérésies pour les exterminer à la racine.

Il existe une catégorie plus dangereuse encore, car ils deviennent souvent à leur insu les jouets des dieux noirs : les sorciers, en particulier ceux qui n'ont pas reçu l'entrainement des collèges de magie impériaux. Ceux-ci, par manque d'entrainement et de volonté, peuvent servir de porte utilisés par les démons pour se répandre dans l'empire. Ils sont recherchés par les répurgateurs, mais aussi par l'Untersuchung, une organisation hors de l'église, qui doivent déterminer si un sorcier est un danger ou s'il peut encore être entraîné par les collèges.

Depuis l'arrivée de Vlad Von Carstein à sa tête, la Sylvanie n'a plus été la même : les morts n'y trouvent plus le repos, et de puissants vampires mènent ses armées. Vlad lui-même a fait valoir ses droits au trône impérial lors des guerres vampiriques, au cours desquelles Vlad et ses deux successeurs tentèrent de s'emparer de la couronne. Aujourd'hui la Sylvanie est plus calme, mais la rumeur veut que les fenêtres du Château Drakenhof luisent d'une lueur malsaine certaines nuit, et les paysans de la province racontent que la sylvanie a un nouveau comte électeur.

\chapter{Le Middenland}
%Reikland ou une autre province ?
Le middenland est une des plus importantes provinces impériales, et une des plus riches. Elle se situe dans une position relativement centrale, au nord du Reikland.

Aujourd'hui, le grand-duché du middenland est dirigé par le Graf Boris Todbringer, comte électeur de l'empire.

La province étant la seule où le culte d'Ulric surpasse celui de Sigmar, il est par exemple interdit de porter une fourrure de loup sans l'avoir tué soi-même avec une arme que l'on a créé de sa main. Les plus grands guerriers du culte sont les loups blancs, qui servent de garde personnelle à Ar-Ulric.
\section{Middenheim et la tempête du chaos}
Middenheim est le centre du culte d'Ulric, où brûle sa flamme éternelle.

La ville en elle-même est bâtie au sommet d'un rocher escarpée, qui lui confère une protection importante, renforcée de murs et de tours. Sigmar lui-même n'a soumis les Teutogens à l'aube de l'empire qu'en escaladant avec quelques compagnons les falaises vertigineuses (environ 150 mètres de haut) du rocher.

Lors de la dernière invasion du chaos, c'est ici que les hordes des dieux noirs, menées par Archaon, héraut de la fin des temps ont été brisées. Il a fallu l'intervention de Valten, réincarnation de Sigmar, et toutes les forces de l'empire et de ses alliés pour repousser l'assaut, et la ville est en train de se reconstruire après le siège destructeur.

Le graf Todbringer, mais aussi Ar-Ulric (le haut prêtre du culte d'Ulric) sont présentement en train de chasser dans toute la province les restes de la horde, qui continuent à s'en prendre à de nombreuses villes et villages.
\section{Waldenheim}
Waldenheim est aussi appelée la 'porte de la Drakwald'. Il s'agit d'une ville de taille moyenne, bâtie sur la vieille route du nord, entre la cité-état de Marienburg et jusqu'en Ostland, en passant par Middenheim.

La ville est bâtie le long de la rivière Bode, qui traverse les terres désolée pour arriver dans les marais maudits, ce qui permet également de rejoindre Marienburg. En amont, la Bode traverse un partie du middenland, mais n'est pas toujours praticable.

Aujourd'hui, Waldenheim est dirigée par le Reiter Marius von Breickenau, comte de Waldenheim. Celui-ci a mené des troupes lors de la bataille de Middenheim et y a perdu une main.
\section{La forêts de la Drakwald}
La drakwald est une des plus grandes et sombres forêts de l'empire, couvrant une bonne partie du middenland. Elle est connue pour abriter de nombreuses hardes d'hommes-bêtes qui lancent des assauts fréquents contre les villages de la région. En plus de cette plaie, des gobelins et des araignées géantes peuvent représenter un autre danger dans la Drakwald.

La drakwald est parcourue de plusieurs routes, comme la Grande Route du Nord, la vielle route impériale.... Celle-ci sont régulièrement patrouillées, tant par une garde spécifique que par des unités militaires si la menace devient trop importante. La plupart des voyageurs devraient faire comme eux et privilégier de s'arrêter autant que possible dans les auberges, souvent plus sûres, voire fortifiées pour se protéger des créatures.
\section{Les terres désolées}
Ces terres appartiennent à la cité de Marienburg. Elfes et nains racontent qu'elles furent fertiles avant l'avènement de l'empire, mais furent ravagées par les dangereuses magies employées par deux autres races en conflit.

La ville de Marienburg elle-même a obtenu son indépendance de l'empire, et prospère de son commerce avec de nombreuses factions du monde connu : des vins de bretonnie sont négociés à côté d'épices de Cathay ou de gemmes venues des principautés frontalières. 
\chapter{Hauts Elfes}
Les hauts-elfes sont rares dans le vieux monde, étant plus généralement isolés sur leur île ancestrale d'Ulthuan.
\section{Ulthuan}
Ulthuan est l'ancien berceau de toute la race des elfes, située au milieu de l'océan. De part la présence du Maëlstrom en son centre, elle est particulièrement baignée dans les vents de magie.

Certains des plus grands érudits, mages et guerriers du monde y perfectionnent leurs arts. Les différentes provinces ont chacune leurs spécificités, Lothern est la porte de l'île pour les étrangers par exemple, Chrace est une terre sauvage, tandis que Caledor est la province des princes-dragons.

L'histoire du lieu n'est toutefois pas si belle : avant l'avènement de l'empire, le prince de la province de Nagarythe, Malékith déclencha une violente guerre civile, qui finit par mener à l'engloutissement de sa province sous les eaux et l'exil de son peuple de l'autre côté de la mer. C'est ainsi que démarra la Longue Guerre entre les Asurs d'Utlhuan et les Druchii de Naggaroth.

Les haut-elfes avaient autrefois des colonies de part le monde, nombre d'actuelles villes impériales et bretonniennes étant d'ailleurs construites sur leurs ruines. Celles-ci furent détruites ou abandonnées après le conflit connu par les nains et impériaux comme la guerre de la Barbe. Un conflit extrêmement violent entre Asurs et nains, aux origines encore troubles.

Les Asurs se nomment d'après leur principale divinité, Asuryan, le phénix qui choisit leur nouveau souverain dans ses flammes. Ils vénèrent également Isha, la déesse de la nature et de nombreux autres moins communs.
\section{Dans l'Empire}
Les Asurs sont peu nombreux dans l'empire : ils disposent évidemment d'ambassadeurs et de diplomates auprès de leurs alliés de Marienburg et de l'empire, et quelques maisons envoient des agents, mais ils restent extrêmement rares dans les terres des hommes.
\chapter{Elfes sylvains}
Les elfes sylvains sont les seuls elfes à être resté sur le vieux monde, en particulier dans la grande forêts d'Athel Loren.
\section{Athel Loren}
Après la guerre de la barbe, et suite à l'abandon des colonies, quelques elfes, se sentant plus chez eux sur ces terres, décidèrent de ne pas aller sur Ulthuan. Ils découvrirent la forêt d'Athel Loren, d'abord en restant près de sa lisière, puis, alors que leurs ennemis devenaient plus nombreux, en y trouvant abri.

Si la forêt elle-même commença par leur être hostile, ils défendirent leur nouveau foyer face à ses ennemis, et parvinrent à un accord avec les nombreux esprits des bois. Si certains ne l'ont toujours pas accepté, des millénaires plus tard, la majeure partie respecte ce pacte.

Les elfes vivent donc au sein de la forêt, en relative harmonie avec celle-ci. Symboles de ce pacte sont le roi et la reine de la forêt. La reine Arielle est l'incarnation d'Isha, déesse de la nature, est est plus une fée qu'une elfe. Le dieu-roi Kurnous s'incarne à chaque printemps dans un nouvel individu trouvé par sa redoutable chasse sauvage, qui parcourt la forêt et les terres bretoniennes lors des nuits d'été, et malheur à qui les croise, car leur chasse considère que tous sont des proies, hommes ou elfes.

Cet accord et la forêt elle-même rende la vie rude sous les frondaisons d'Athel Loren, que ce soit pour les clans de cavaliers aigles, les groupes de cavaliers sylvestres, ou les redoutables tireurs qui font la réputation des sylvains.
\section{Dans l'Empire}
Plus rares encore que leurs cousins Asurs dans les terres des hommes, les sylvains ne sortent de leurs terres que pour d'excellente raisons : un ennemi à chasser, un objet important à retrouver, ou pour s'assurer qu'une prophétie importante se réalise.
\chapter{Nains}
Les nains sont d'anciens alliés de l'empire, Sigmar heldenhammer étant lui-même un mai des nains. Et les nains ont la mémoire longue, tant pour leurs promesses que pour leurs rancunes.
\section{Histoire des nains}
Pendant des millénaires l'empire nain était la seule puissance notable sur le vieux monde, jusqu'à une grande série de catastrophe qui ravagea leurs tunnels et plusieurs forteresses. 

Vint ensuite la guerre de la barbe, qui, d'après les nains, vient de l'injure ultime infligée à leurs émissaires chez les elfes; émissaires étant venus se plaindre d'une embuscade tendue par des elfes à un convoi nain. Les combats furent féroces pendant des années, les meilleurs guerriers des deux camps mourant dans le conflit.

Les elfes finirent par quitter le continent, et avec leur départ, les hommes prospérèrent. Les nains sont les plus anciens alliés de l'empire, leur ayant fait découvrir la poudre à canon, entre autre.
\section{Dans l'Empire}
On trouve des nains un peu partout dans l'empire, du moins dans les villes. Contrairement aux artisans humains, ils n'ont pas obligation de rejoindre les guildes humaines, et forment bien souvent les leurs.

On trouve de toutes les professions : artisans, citadins, marchands, combattants.... Toutefois, certains des nains les plus notés sont les pariah de leur culture : les tueurs. Ce sont des nains ravagés par un échec personnel, qui préfèrent se raser la barbe, se teindre les cheveux en orange, et qui partent trouver la mort la plus glorieuse possible. Les nains entre eux parlent peu des tueurs, mais de temps en temps, certains sont trop doués pour trouver une mort au combat. Ils errent alors de combat en combat, cherchant désespérément une créature suffisamment dangereuse pour enfin les tuer.
\chapter{Les halfelins}
Les halfelins font partie de la population de l'empire, étant originaire du Mootland. 

Ils sont connus pour leur roublardise et leurs talents culinaires, et ceux qui ne cuisinent pas sont souvent de grands consommateurs de nourriture également. Les gardes des villes s'en méfient souvent, car les objets ont la réputation de disparaître à proximité de certains d'entre eux. Certains privilégient d'arnaquer les grandes-jambes sinon.

Le mootland est une province impériale depuis longtemps, fournissant peu de troupes, mais des récoltes importantes, malgré la proximité avec la sylvanie. La contribution militaire des halfelins se limite bien souvent à des cuistots, bien que certaines rumeurs parlent d'un 'canon à soupe' que le mootland aurait utilisé pour se défendre.
\chapter{Notre campagne}
\section{Point de départ}
Vos personnages vont commencer dans le village d'Utingen, à environ une journée de marche de Waldenheim le long de la vieille route du nord. Ils peuvent y être pour différentes raisons, et se sont probablement arrêtés à l'auberge locale : le crâne bleu. Ce village (et cette auberge) sont une étape relativement courante sur la route vers Waldenheim, et en ce moment, de nombreux réfugiés du Middenland et du Nordland s'y arrêtent, sur la route vers Marienburg, ou en passant par une route détournée pour aller vers le Reikland.
\section{Vos personnages}
A priori, tout personnage généré grâce aux règles de base doit pouvoir convenir, quelque soit sa race, classe et profession. Le livre de base est le seul livre utilisé pour cette campagne.

Idéalement, lors de la rédaction de son background, il faudrait répondre à certaine question:
\begin{itemize}
\item Décrire votre personnage physiquement
\item Pourquoi êtes-vous dans le village d'Utingen ?
\item Pour quelle raison seriez-vous prêt à accomplir des aventures ?
\item A quel(s) divinité(s) votre personnage fait-il appel ?
\item Connaissez-vous les autres personnages ? Pourquoi ?
\item Quelle est la famille de votre personnage ?
\end{itemize}
N'hésitez pas à poser des questions plus détaillée pour réaliser votre background, cet univers étant très vaste, et compliqué à rendre en un simple petit livret.
\end{document}