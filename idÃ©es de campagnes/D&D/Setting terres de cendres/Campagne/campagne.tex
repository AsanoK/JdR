\documentclass[letterpaper,10pt,twoside,twocolumn,openany]{book}
\usepackage[english]{babel}
\usepackage[utf8]{inputenc}
\usepackage{hang}
\usepackage{lipsum}
\usepackage{listings}

\usepackage{dnd}

\lstset{%
  basicstyle=\ttfamily,
  language=[LaTeX]{TeX},
}
\title{\cftchapfont   \nomcampagne}
\date{}
\author{}
% Start document
\begin{document}
\maketitle
\tableofcontents
\chapter{Introduction}
\chapter{Départ pour les terres de cendres}
%Où les personnages partent de Korvosa, avec la bénédiction d'une faction à définir, pour se diriger vers les terres de cendre. 
\section{Introduction et vue générale}
Ce premier chapitre permet d'introduire les personnages aux terres de cendre, ainsi qu'à leurs premiers objectifs à long terme.

En effet, il recevront leur première mission : celle d'explorer le long de la yondabakari afin d'y trouver un lieu pour établir un avant-poste. Le chapitre dure le temps de ce trajet, et jusqu'à ce que les personnages trouvent un emplacement adéquat, en en chassant les occupants précédents, créatures ou humanoïdes.
\begin{commentbox}{Montées de niveau}
Les personnages devraient arriver niveau 2 en arrivant à la limite des terres de cendres. Ils devraient ensuite arriver niveau 3 après quelque temps d'exploration et de découverte de la région. Leur niveau 4 devrait arriver 
\end{commentbox}
\section{Nos héros, et leur mission}
Les personnages commencent dans le manoir de la maison Orsano, une maison mineure récemment arrivée depuis le Chéliax. Ils ont répondus à une annonce ou ont été contacté par des soldats portant l'emblème de la maison.

C'est Dame Valeria Orsano qui leur indique son offre : face à la menace des orques de Belkzen et des tribus shoanti, la route commerciale jusqu'à Vigil, capitale de Dernier Rempart, n'est pas sûre, et pourtant plus critique (et profitable) que jamais. Dame Orsano cherche donc à améliorer la sécurité de cette route qu'elle exploite grandement, mais la garde de Korvosa ne s'aventurera pas dans les terres de cendres pour ce genre de tâches. Elle aurait besoin d'aventuriers comme les personnages pour trouver où installer un avant-poste, puis pour défendre la route contre les différents dangers qui menacent les caravanes.

Les personnages pourront profiter d'une caravane marchande de la maison pour se rendre dans les terres de cendres, où ils exploreront seuls les abords de la Yondabakari.

Cette caravane partira le lendemain par la porte nord.
\section{Le trajet pour les terres de cendres}
La caravane part effectivement le lendemain matin, dirigée par Miria Malenko. Celle-ci a été mise au courant par Dame Orsano des nouveaux venus, et essaie de leur trouver un rôle dans sa caravane : pas question de les laisser trop se reposer. Suivant leur spécialité et leurs compétences, ils pourront se voir confier différentes tâches : soigner, servir d'éclaireur, aider Miria à gérer les différents chariots, ou encore, fournir de l'aide de manière générale.

La caravane comprend une dizaine de chariots de marchandises, chacun avec un conducteur et trois chevaux, auxquels s'ajoutent une dizaine de gardes et une dizaine d'aides diverses : guide, cuisinier, etc. Deux chariots supplémentaires transportent les vivres pour la caravane elle-même, principalement du biscuit et de la viande séchée, mais aussi quelques fruits.


\section{Le long de la yondabakari}
\section{La vallée de Tivergar}
Miria a reçu pour instruction de transporter le petit groupe d'aventuriers jusqu'à la vallée de Tivergar, ceux-ci pouvant quitter la caravane quand ils le souhaitent entre la rive ouest du lac, et le nord du col des trois lames.
\subsection{Présentation de la région}
La vallée de Tivergar est un haut plateau, dont le point le plus marquant est le lac du même nom, possiblement le lac le plus haut de Varisie. Il est plus petit que l'abysse de Storval, et probablement moins profond, mais ses eaux noires ne sont que peu troublées, et qui sait ce qui s'y trouve ?

Au fond de la vallée, menant et sortant du lac, on retrouve la Yondabakari, et sa route commerciale, toujours sur la rive nord. La route quitte la rivière assez haut dans la vallée, pour passer par le col des trois lames, qui mène vers l'est de la trouée d'Urglin.


\chapter{PNJs importants de la campagne}
\subsection{Dame Valeria Orsano}
Cette femme d'une trentaine d'année dirige depuis cinq ans la maison Orsano, une maison relativement mineure de Korvosa. Elle a à coeur de bâtir la richesse de sa famille sur le commerce, et envisage de renforcer la route entre Vigil et Korvosa pour cela. Elle cherche donc à sécuriser cette route, et si possible, à créer des liens avec Kaer Maga.
\subsection{Miria Malenko}

\end{document}