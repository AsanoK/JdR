\documentclass[10pt,a4paper]{book}
\usepackage[utf8]{inputenc}
\usepackage[french]{babel}
\usepackage[T1]{fontenc}
\usepackage{amsmath}
\usepackage{amsfonts}
\usepackage{amssymb}
\author{ Antoine Robin}
\newcommand{\nom}{Bannières et Bannerets}
\newcommand{\diminutif}{B\& B}
\title{\nom}

\begin{document}
\maketitle
\tableofcontents
\chapter*{Introduction}
\chapter{Background}

\chapter{Règles}
\section{Règles des unités}
\subsection{Profil}
\subsection{Figurines}
\subsection{Unités}
\section{Structure d'un tour de jeu}
Au début d'un tour de jeu, chaque jour lance un dé pour déterminer qui a l'initiative. En cas d'égalité, le joueur n'ayant pas eu l'initiative au tour précédent l'emporte. Au premier tour, les joueurs relancent jusqu'à ce qu'il n'y ait plus d'égalité.

Ensuite, et en commençant par le joueur ayant l'initiative et en alternant, les deux joueurs vont activer successivement leurs unités. S'il ne reste plus aucune unité à activer pour un joueur, son adversaire peut activer toutes ses unités restantes tour à tour.
\section{Activer une unité}
Pour activer une unité, on détermine quel ordre lui est donné :
\begin{description}
\item[Manoeuvre] On peut déplacer les figurines de l'unité d'une distance égale à la valeur de déplacement de l'unité. Si l'unité n'est pas engagée en mêlée avec des troupes adverses, elle peut à la place se déplacer de deux fois cette valeur. L'unité ne peut pas finir son mouvement engagé en mêlée.
\item[Affrontement] L'unité peut réaliser une manoeuvre, et doit finir son déplacement en mêlée avec une unité adverse. Si l'unité se déplace au moins de sa valeur de déplacement, elle compte comme ayant réalisé une charge.
\item[Repli] L'unité se déplace vers la \emph{Bannière} la plus proche et en ligne de vue de son camp : chaque figurine doit si possible finir son mouvement hors de mêlée et plus proche de la bannière la plus proche et en ligne de vue de son camp. Si aucune bannière n'est en vue, alors l'unité se replie vers le bord de table le plus proche à la place.
\end{description}
\section{Mêlées}

\chapter{Armées}

\chapter{Glossaire}
\begin{description}
\item[Bannière]
\item[Charge]
\item[Engagée]
\item[Formation]
\item[Ligne de vue]
\item[Manoeuvre]
\end{description}
\end{document}