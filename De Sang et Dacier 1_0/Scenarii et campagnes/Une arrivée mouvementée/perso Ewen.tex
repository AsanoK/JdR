\documentclass[10pt,a4paper]{article}
\usepackage[utf8]{inputenc}
\usepackage{amsmath}
\usepackage{amsfonts}
\usepackage{amssymb}
\title{???}
\date{}
\begin{document}
\maketitle
\section{Histoire}
Après une enfance assez normale au sud de Menide, dans une petite ville, au sein d'une famille d'artisan, elle a quitté sa famille pour faire quelque chose de sa vie. 

Après avoir côtoyé une compagnie de mercenaires, et au grand dam de certains de ses membres, elle finit par s'y engager. Elle y passât une bonne partie de sa vie d'adulte : marche, combat, pillage, attente, tavernes, elle a tout fait. Elle s'y est révélée relativement douée, atteignant finalement le rang (très inhabituel pour une femme), d'homme d'arme. Certains ont parlé de favoritisme de la part du capitaine, d'autres ont fait des remarques plus blessantes encore, mais cela n'avait pas beaucoup d'importance.

Jusqu'à la mort dudit capitaine, il y a quelques mois : cette excentricité n'était pas du goût du nouveau capitaine, qui passait d'une attitude lourde à un mépris franc envers cette femme.

Elle a donc rapidement quitté la compagnie, au départ sans grand but, avant d'entendre parler de l'Alianais : si ils ne semblent guère embaucher des mercenaires, elle entendait qu'ils appréciaient un bon combattant, ou même une bonne combattante sur cette frontière.
\section{Personnalité}
C'est un personnage assez franc, qui n'hésite pas à agir plutôt que de discuter pendant longtemps. Elle considère également que la méthode la plus directe est souvent la meilleure, et en tout cas, celle qui évite de perdre trop de temps à tergiverser.

Cette impulsivité se reflète également dans sa manière de gérer sa vie : elle est partie de la maison sur un coup de tête, et envisage de s'installer quelque part, loin au nord, sur un autre coup de tête.
\section{Mécanique}
enfance:urbain
jeunesse:estafette
adulte : femme d'arme
\subsection{Attributs}
\begin{itemize}
\item Agressivité :5
\item Caractère :3
\item Coordination :3
\item Empathie :2
\item Logique :1
\item Résilience :3
\item Vigilance : 2
\item Forme : 4
\item Mémoire :2
\item Destin :1
\end{itemize}
\subsection{Compétences}
Langue natale : Menidien
athlétisme 3, négociation 1, défense 3, corps à corps 4, culture générale (Menide) 3, éducation 1, discussion 1, culture générale (guerre) 4, arme blanche (couteaux) 3, arme blanche (au choix) 4, langue (Ascellien) 2, art(flute) 1, poignarder 3, intimidation 3, commandement 2, organisation 1, culture générale (Ascellie) 2.
\subsection{Matériel}
Arme au choix, armure à voir, provisions (3 jours), 3 sous, bonnes bottes, vêtements usés, dague fine, couteau de chasse.
\end{document}