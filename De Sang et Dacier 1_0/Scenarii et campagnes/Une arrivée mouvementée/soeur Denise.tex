\documentclass[10pt,a4paper]{article}
\usepackage[utf8]{inputenc}
\usepackage{amsmath}
\usepackage{amsfonts}
\usepackage{amssymb}
\title{Sœur Denise, prêtresse}
\date{}
\begin{document}
\maketitle
\section{Histoire}
"Denise" est une prêtresse, vénérant la trinité, et aidant ses ouailles. Elle a été envoyée vers l'Alianais pour ancrer l'église dans les moeurs des colons, voire peut-être convertir les païens à la seule vraie religion.

C'est du moins ce qu'elle raconte quand on lui pose la question. En vérité, il s'agit de Blanche, une jeune femme qui a jusqu'ici vécu de vol et de contrebande en Ascellie. 

Un de ses receleurs (un prêteur sur gage assez riche du nom de Renaud) a essayé de la doubler, et elle l'a poignardé. Elle a pris la fuite au plus vite pour éviter la justice, et a décidé de quitter complètement la région pour aller en Alianais, sous un déguisement lui permettant de passer relativement au-dessus de tout soupçon.
\section{Personnalité}
Blanche est une femme pragmatique, dont la sécurité passe avant tout : elle a déjà tué, et n'hésitera pas à le refaire au besoin. 

Elle essaie par ailleurs de garder son secret le plus possible, prenant dans les faits le rôle de prêtresse, quand bien même elle n'en a guère l'éducation ni les habitudes.

Quand elle est surprise, elle peut jurer assez violemment, ce qui pourrait la trahir.
\section{Mécanique}
enfance : enfant des rues
jeunesse : voleuse
adulte : assassin
\subsection{Attributs}
\begin{itemize}
\item Agressivité :3
\item Caractère :4
\item Coordination 4
\item Empathie :1
\item Logique :1
\item Résilience :2
\item Vigilance : 3
\item Forme : 2
\item Mémoire :1
\item Destin :1
\end{itemize}

\subsection{Compétences}
Langue natale : Ascellien

athlétisme 2, défense 3, arme blanche(couteau) 3, corps à corps 3, culture générale (Valleaux) 3, culture générale (crime) 5, passe-passe 2, fouille 2, crochetage 1, jeu 1, arme blanche (bâtons) 1, athlétisme 2, escalade 2, intimidation 2, négociation 1, mensonge 3, poignarder 3, culture générale (Ascellie) 2, mensonge 4, manipulation 3.
\subsection{Matériel}
tenue de prêtresse, dague, 4 sous, provisions (2 jours), sac à dos, chapelet.
\end{document}