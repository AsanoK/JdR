\documentclass[10pt,a4paper]{article}
\usepackage[utf8]{inputenc}
\usepackage{amsmath}
\usepackage{amsfonts}
\usepackage{amssymb}
\title{Elène Risieux}
\begin{document}

\section{Histoire}
Elène est orpheline, elle n'a jamais connue son père, et pratiquement pas sa mère. 

Depuis qu'elle est petite, elle travaille au Prêtre Défroqué, une taverne de Valleaux. Elle y a appris à gérer des clients pas toujours facile, à courir dans tous les sens, et à gérer tant la cuisine que le service.

Elle est partie sans prévenir ni se retourner après que plusieurs marchands lui ait parlé de l'Alianais, où il était possible de se bâtir une vie meilleure, si on en avait la volonté. Elle a préparé son départ avec soin : provisions, un couteau, des vêtements chauds, une direction, et est parti un matin avant l'aube, pour ne jamais revenir.
\section{Personnalité}
Elène peut sembler assez naïve au premier abord : elle espère toujours que tout ira pour le mieux, ce qui explique entre autre son propre départ.

Elle est très déterminée dans ce qu'elle entreprend, et ira jusqu'au bout de ses idées.

Elle est généralement gentille avec les gens, mais ne se laisse pas manipuler.
\section{Mécanique}
personnalité : décidée
origine: urbain, bas-fonds
profession: servante
secondaire: voyages
\subsection{Attributs}
\begin{itemize}
\item Agressivité :3
\item Caractère :4
\item Coordination :2
\item Empathie :4
\item Logique :3
\item Résilience :4
\item Vigilance : 4
\item Forme : 1
\item Mémoire :3
\item Destin :3
\end{itemize}
\subsection{Compétences}
langue natale : Ascellien 
culture générale (Valleaux) 6, culture générale (Ascellie) 4, culture générale (Alianais) 2, langue (ménédien) 2,culture générale (Menedie)4, culture générale (crime) 1; défense 1, corps à corps 1, athlétisme 1, discrétion 3, passe-passe 2, négociation 6,discussion 7, intimidation 1, manipulation 3, calme 4, profession (aubergiste) 4, fouille 3, observation 6,éducation 2, organisation 2, art(chant) 3.
\subsection{Equipements}
couteau, provisions (2 semaines), tenue du dimanche, tenue propre, sac à dos, vêtements chauds, matériel de bivouac.
\end{document}