\documentclass[10pt,a4paper,twocolumn]{book}
\usepackage[utf8]{inputenc}
\usepackage[T1]{fontenc}
\usepackage{amsmath}
\usepackage{amsfonts}
\usepackage{amssymb}
\usepackage[tight]{shorttoc}
\newcommand{\sommaire}{\shorttoc{Sommaire}{1}}

\author{Antoine Robin}
\title{De Sang et d'Acier}
\date{}
\begin{document}
\title{De Sang et d'Acier}
\maketitle
\sommaire
\part*{Introduction}
\part{Règles de base}
\chapter{Lancers de dés}
\section{Dés utilisés}
Dans S\&A les dés utilisés sont tous des dés à six faces, ou hexaèdres, ou plus simplement des d6. Ainsi, si le MJ demande de lancer 5d6 dans le cadre d’un test, il s’agit de lancer 5 hexaèdres.

Un dé donnant pour résultat 5 ou 6 est appelé un succès, ou une réussite. Les autres faces du dé sont globalement sans intérêt dans le cas général.
%TODO : reprendre cette partie pour être cohérent avec la 1.0
\section{Tests simples}
La majeure partie des tests du jeu impliquent soit une caractéristique seule, soit un couple caractéristique+compétence.
Dans le cadre d’un test simple, on lance un nombre de d6 égal au score testé (score de caractéristique ou somme des scores de caractéristique et de compétence utilisés). Puis on compte le nombre de succès, avant de comparer avec le seuil nécessaire pour réussir (déterminé par le MJ). Ce seuil dépend de la difficulté de la tâche entreprise, mais le MJ peut également se référer à la table \ref{tableDiffTests} pour s’aider.

\begin{table}
\caption{ Difficulté des tests :}
\label{tableDiffTests}
\begin{center}
\begin{tabular}{cc}
\textbf{Difficulté} & \textbf{nombre de succès} \\
   facile & 1  \\
   normal & 2  \\
   délicat & 3 \\
   difficile & 4 \\
   très difficile & 5 \\
   surhumain & 6 et + \\
\end{tabular}
\end{center}
\end{table}

La marge de réussite (ou d’échec), est la différence entre le seuil à obtenir et le nombre de succès.

Le test simple s’utilise pour déterminer si un personnage réussit une action difficile en elle-même, mais sans opposition active.
\section{Tests en opposition}
Le test en opposition implique que deux personnages effectuent un test simple, puis, comparent leurs nombres de succès respectifs. Le gagnant est alors celui qui en a le plus.

La marge de réussite est la différence entre les deux résultats.
Un test en opposition s’utilise pour tester le résultat d’une opposition directe entre deux personnages ou entités.

Il est possible d’obtenir une égalité dans un test en opposition, auquel cas, le MJ peut décider (suivant le cas), de considérer cela comme une double victoire, ou un double échec par exemple.
\section{Tests liés}
Des tests liés sont des tests dont finalement, seul le dernier est vital, les autres servant de soutien à celui-ci. Pour se faire, on réalise le premier test, dont le nombre de succès est un bonus au nombre de dés du test suivant.

Cela représente par exemple un test de déguisement permettant de faciliter un test de bluff pour se faire passer pour quelqu'un.
\section{Tests en équipe}
Il est possible d'aider d'autres personnages au cours de leurs aventures. Dans ce cas, on choisit un personnage qui réalisera le test principal. Les personnages réalisant un test pour l'aider réalisent alors leur test : le personnage réalisant le test principal reçoit un nombre de dés égal au nombre de réussite des tests de soutien. Le travail en équipe fonctionne ainsi de la même façon que pour des tests liés.
\section{Tests étendus}
Les tests étendus sont caractérisés par trois valeurs : leur objectif, leur seuil et leur intervalle.
L’objectif est la marge de réussite totale à obtenir pour réussir, l’intervalle est le temps passé par le personnage pour chaque test. Le seuil est la difficulté de chaque test effectué.
Tant que l’objectif n’est pas atteint par la valeur actuelle du test (ou que le personnage n’a pas abandonné), le personnage effectue des tests simples. On ajoute la marge de succès de chaque test à la valeur actuelle du test, et de même, on soustrait les marges d’échec de cette valeur. 
Ce genre de test est utilisé pour déterminer le temps nécessaire à réaliser une tâche.
%\section{Notion de maîtrise}
%Un personnage disposant d’une maîtrise(n) sur une compétence, peut, aux cours de tout test de cette compétence, relancer les dés ayant obtenus moins de n. Le plus courant est maîtrise (1), qui permet de relancer les dés ayant obtenus un 1 sur les tests.
\section{Acheter des réussites}
Afin d’accélérer certaines phases impliquant de nombreux lancers de dés, ou de réaliser des tests à la fois très faciles et non critiques, il est possible “d’acheter” des réussites : au lieu d’effectuer un test simple, on calcule un nombre de réussite égal au quart du nombre de dé que l’on aurait utilisé pour le test (arrondi à l’inférieur).

Cette méthode peut également servir parfois pour déterminer la valeur de l'efficacité passive d'un personnage.
\chapter{Attributs}
En matière de chiffres, on définit principalement un personnage par deux ensembles de valeurs : les attributs, et les compétences.

Les attributs représentent la "base" du personnage, des éléments très généraux sur lesquels le personnage se repose en permanence. Ils sont au nombre de 7 attributs majeurs ou principaux, auxquels s'ajoutent trois attributs mineurs ou secondaires.
\section{Les Attributs principaux}
\subsection{Agressivité, AGR}
L'agressivité d'un personnage représente sa propension à prendre l'initiative, à s'imposer, et évidemment, la vitesse à laquelle il peut devenir agressif. Il est intéressant de noter qu'un personnage agressif n'est pas forcément physiquement violent, il peut être agressif verbalement par exemple. L'agressivité sert mécaniquement principalement en combat, où elle est l'attribut de l'attaque, mais aussi dans certaines circonstances sociales (ou de même, il s'agit de l'attribut offensif).

Un personnage avec une faible valeur d'agressivité risque de ne pas chercher la confrontation directe, vocale ou physique, mais plutôt de s'effacer devant celle-ci. Peut-être le personnage est-il timide. A l'inverse un personnage agressif n'aura pas de problème à provoquer, à aller chercher le conflit. cela peut être lié à une certaine expansivité également.
\subsection{Caractère, CAR}
Le caractère est la façon dont la personnalité d'un personnage est affirmée : sa volonté, sa résolution, mais aussi sa solidité mentale face aux coups durs. Il s'agit d'un attribut défensif, en particulier contre les situations stressantes, et également d'un attribut intéressant pour les mystiques. Il peut aussi s'agir d'un attribut défensif dans un contexte social.

Un personnage avec un caractère peu affirmé sera prompt à changer d'avis et à être influençable, quand un personnage avec un fort caractère sera probablement têtu, borné, et assez résistant aux tentatives de manipulation et au stress.
\subsection{Coordination, COO}
La coordination est l'aptitude d'un personnage à bouger de manière précise et rapide. Il s'agit d'un attribut indispensable au combat, étant la base de la défense, mais aussi du tir. En dehors de cela, c'est la compétence des artisans, et des activités nécessitant de l'adresse.

Un personnage avec une faible coordination ne sera pas très vif, ni très adroit, quand un personnage bien coordonné aura un très bon contrôle sur ses actions, avec peu de maladresses, peut-être même un côté félin dans ses déplacements.
\subsection{Empathie, EMP}
L'empathie est l'attribut qui permet de comprendre les autres, de se mettre à leur place. Il s'agit principalement d'un attribut social, étant utilisé par exemple pour toute tentative de manipulation, mais aussi pour maintenir une discussion plaisante par exemple. C'est également l'attribut le plus utilisé pour interagir avec des animaux.

Un personnage avec peu d'empathie sera probablement assez froid, et d'assez mauvaise compagnie, ne se sentant que peu concerné par les autres. A l'inverse un personnage très empathique sera souvent vu comme agréable et compréhensif. A noter cependant, compréhensif ne veut pas dire naïf, du moins pas tout le temps.
\subsection{Logique, LOG}
La logique est l'attribut représentant la capacité de raisonnement d'un personnage, sa façon de réfléchir et de prendre une décision basée sur une réflexion. Elle est nécessaire dans la plupart des profession "intellectuelles" : moines, certains mystiques, marchands, artisans, mais aussi dirigeants de toutes sortes.

Un personnage peu logique sera probablement assez instinctif, tendant à suivre sa première impression plutôt qu'un raisonnement précis. A l'inverse, un personnage logique sera peut-être vu comme calculateur ou érudit, n'hésitant pas à émettre des hypothèses solidement basées.
\subsection{Résilience, RES}
La résilience est la capacité d'un personnage à faire abstraction des douleurs et problèmes physiques : inconfort, fatigue, douleur, etc. C'est cela qui permet à un personnage déjà fatigué de se battre, ou à un malade de marcher malgré tout. C'est un des attributs les plus utiles en campagne militaire par exemple.

Un personnage avec une faible résilience sera probablement assez douillet ou maladif : le manque de confort ou la fatigue seront des épreuves assez importantes. A l'inverse, un personnage avec une forte résilience sera plus "rustique", ou du moins, sera capable de plus ignorer ces problèmes, s’accommodant parfaitement d'un bout de sol pour dormir en cas de besoin.
\subsection{Vigilance, VIG}
La vigilance représente à la fois l'attention d'un personnage envers son environnement et l'acuité de ses sens. C'est un attribut important pour tout aventurier : dans cette profession, le danger peut venir sans prévenir, et il faut pouvoir réagir rapidement.

Un personnage peu vigilant sera peut-être étourdi et peu réactif, peut-être même flegmatique par moment. Un personnage vigilant à l'opposé sera peut-être méfiant, sur le qui-vive, prêt à réagir à tout problème qui arriverait.
\section{Attributs secondaire}
\subsection{Forme, FOR}
La forme est une mesure des capacités physiques d'un personnage : on l'utilise par exemple pour savoir ce que peut porter un personnage ou pour déterminer ses capacités athlétiques.

Un personnage avec une faible forme sera probablement peu sportif, ou maladif. Les efforts sont difficiles à entreprendre. Un personnage en forme aura lui plus de facilité à entreprendre des tâches physiques.
\subsection{Mémoire, MEM}
La mémoire est la capacité d'un personnage à retenir des informations : noms, données diverses, lieux. Elle est souvent bien développé chez des personnages relativement érudits ou sages, ou encore chez les conteurs, musiciens et acteurs, par déformation professionnelle.
\subsection{Destin, DES}
Le destin représente la chance du personnage : parfois une carrière d'aventurier tient plus au fait qu'un autre souffre d'une flèche perdue plutôt que soit-même. On lance le destin lorsque quelque chose risque d'arriver de manière totalement aléatoire.
\chapter{Compétences}
Les compétences () sont des compétences à domaine : chaque domaine est considéré comme une compétence à part.
Il est possible d’effectuer un test d’une compétence proche avec l’accord du MJ (et un malus au test).
Tenter d’effectuer un test sans aucune formation (aucun point) est toujours accompagné d’un malus de -2.
Les compétences marquées d’un astérisque * nécessitent une formation : il est impossible de tenter un test sans avoir au moins un point dans cette compétence.
%pour une compétence : description rapide du contenu, quelques exemples de jets, domaines, éventuelles règles spéciales
%règles spéciales potentielles : formation, utilisation d'un autre domaine.
%les détails des compétences sont à trouver dans le chapitre correspondant
%types de compétences : combat, social, mystique, connaissance, aventure, déplacement

\section{Combat}
Les compétences de combat sont utilisées quand un conflit devient violent : échanges de coups, esquives, saisies... Elles peuvent également avoir des applications dans certains jeux (qui sont souvent des entrainements au combat).
\subsection{Arme blanche()}
\paragraph{Description:}La compétence arme blanche recouvre le maniement des armes de mêlée de tout type : lances, épées, haches, masses... On en trouve de toutes qualités, de toutes formes, pour toutes les bourses. Il s'agit du principal moyen que l'homme a trouvé pour s’entre tuer.
\paragraph{Exemples:}On teste arme blanche(lames)+AGR pour attaquer une cible à l'épée; un test d'arme blanche(lance)+VIG peut fournir une indication du niveau d'un combattant.
\paragraph{Domaines:}Lames, haches, armes d'hast, armes contondantes, lances de cavalerie, boucliers, improvisé.
\subsection{Arme de jet()}
\paragraph{Description:}Arme de jet couvre l'utilisation des armes à distance lançant un projectile : arc, arbalètes, frondes... Il s'agit souvent d'équipements de chasse en premier lieu, bien que bon nombre d'entre eux soient également utilisés avec succès sur les champs de bataille. Leur utilisation est moins courante en escarmouche, principalement en raison de leur temps de rechargement.
\paragraph{Exemples:}On teste Arme de jet(arc)+COO pour tirer sur une cible éloignée avec un arc. 
\paragraph{Domaines:}Arcs, arbalètes, frondes.
\subsection{Arme de traits()}
\paragraph{Description:}Les armes de traits sont les armes que l'on lance : du rocher au javelot en passant par diverses formes de couteaux par exemple. Cela recouvre également le lancer de divers objets qui ne sont pas spécialement fait pour cela.
\paragraph{Exemples:}Pour jeter un javelot, on lance arme de traits(javelots)+COO; pour jeter un tabouret, il s'agit d'un test d'armes de traits(improvisé)+COO; lancer un ballon est un autre test d'armes de traits(improvisé)+COO.
\paragraph{Domaines:}Javelots, couteaux, improvisé.
\subsection{Corps à  corps}
\paragraph{Description:}Les techniques de corps à corps sont l'ensemble des techniques permettant de projeter, maîtriser, frapper, voire tuer un adversaire, sans arme. Elles permettent de se sortir de situation assez désespérées, ou à deux personnes en armure de s'entre tuer.
\paragraph{Exemples:}On lance un test de Corps à corps+AGR pour projeter un adversaire, et on tente d'y résister avec un test de Corps à corps+COO. Un coup de point nécessite un test de Corps à corps+AGR contre défense+COO. 
\subsection{Défense}
\paragraph{Description:}La défense permet de se protéger face à la plupart des attaques qui sont réalisées contre soi. On s'en sert donc très fréquemment pour peu que l'on soit en combat, ou dans le cadre de certains jeux.
\paragraph{Exemples:}Pour éviter un coup ou ne pas heurter quelqu'un qui court dans une foule, on réalise un test de défense+COO.
\subsection{Poignarder}
\paragraph{Description:}Poignarder est une compétence qui a deux utilisations principales : frapper une cible sans défense ou viser une vulnérabilité de l'armure d'un adversaire dans le cadre d'un combat au corps à corps.
\paragraph{Exemples:}Pour attaquer une sentinelle dans le dos (visez la gorge ou les reins...), réaliser un test de Poignarder+COO.

\section{Déplacements}
Les compétences de déplacement permettent sot utilisées pour aller d'un point A à un point B, que ce soit pour y aller vite, ou si le déplacement est difficile physiquement.
\subsection{Athlétisme}
\paragraph{Description:}L'athlétisme correspond à la course, à la marche, au saut (en longueur ou hauteur). Ce qui regroupe donc la totalité des déplacements à pieds qu'un personnage peut effectuer. Évidemment, marcher normalement ne requiert aucun test. Ceux-ci sont demandés lors d'un long déplacement ou un sprint (ou un saut).
\paragraph{Exemples:}Pour endurer des heures de marche forcée, il s'agit d'un test de RES+athlétisme; Pour sauter sur un cheval en course, c'est un test de FOR+athlétisme.
\subsection{Attelage}
\paragraph{Description:}Cette compétence permet de gérer, entretenir et manœuvrer un véhicule hippomobile (charrette, carriole...). Un test n'est normalement pas nécessaire pour diriger l'attelage le long d'une route sans difficulté, mais un terrain difficile à négocier ou un évènement qui effraie les bêtes peut en demander un.
\paragraph{Exemples:}Pour reprendre le contrôle d'une carriole dont les chevaux sont paniqués, on peut tenter un test d'Attelage+EMP.
\subsection{Discrétion}
\paragraph{Description:}La discrétion représente la capacité d'un personnage à ne pas être vu ou remarqué : se glisser dans une foule, se cacher dans un fourré, ou rester hors de l'attention dans un bal sont des utilisations légitimes de cette compétence.
\paragraph{Exemples:}Pour se dissimuler derrière un obstacle, on teste Discrétion+COO; pour ne pas être remarqué par quelqu'un lors d'une soirée, il s'agirait plutôt de discrétion+EMP.
\subsection{Escalade}
\paragraph{Description:}L'escalade est la compétence permettant de passer par-dessus un obstacle, plutôt qu'au travers. Peut-être pas la plus fréquemment utilisée, mais elle peut offrir quelques possibilités. Indispensable sur un navire par exemple.
\paragraph{Exemples:}Pour grimper à un arbre ou à un mur décrépi, on teste escalade+COO. Pour rester accrocher longtemps à une corde, ce serait plutôt escalade+RES.
\subsection{Équitation}
\paragraph{Description:}L'équitation est la compétence pour diriger une monture, le plus souvent un cheval, mais potentiellement une mule ou tout autre animal de monte. Avoir un cheval est cher, mais peut se révéler très utile pour un aventurier.
\paragraph{Exemples:}Reprendre le contrôle d'une bête paniquée se fera au moyen d'un test d'équitation+EMP, la diriger ensuite au travers d'une forêt sans se tuer contre une branche est un test d'équitation+COO. Ne pas trop subir de courbature se fait au moyen d'un test d'équitation+RES.
\subsection{Natation}
\paragraph{Description:}La natation est le fait de savoir nager. Si ne pas couler est assez facile quand il n'y a pas de vague ou de courant, savoir vraiment nager permet en revanche d'aller beaucoup plus vite, et de réussir à nager dans des conditions moins qu'idéales.
\paragraph{Exemples:}On réalise un test de natation+FOR pour tenter de nager rapidement d'un point à un autre. Pour nager longtemps en revanche (et ne pas trop subir les effets de la fatigue), il s'agit d'un test de Natation+RES.
\subsection{Navigation}
\paragraph{Description:}La navigation recouvre les techniques permettant de diriger une embarcation : gérer les voiles et le gouvernail, prendre en compte les courants, réaliser les innombrables nœuds nécessaires et se déplacer malgré les mouvements du sol.
\paragraph{Exemples:}Diriger un navire requiert un test de navigation+LOG pour aller au plus vite. Navigation+VIG peut servir à repérer les problèmes potentiels à la navigation. Navigation+COO permet de réaliser des nœuds efficaces.

\section{Érudition}
Les compétences d'érudition sont les compétences les plus intellectuelles d'un personnage, et ses savoirs.
\subsection{Connaissance()}
\paragraph{Description:}Les connaissances sont les savoirs les plus académiques d'un personnages, appris auprès d'un enseignant ou grâce à des livres. Il s'agit de savoirs relativement théoriques voire purement intellectuels.
\paragraph{Exemples:}Pour reconnaître un symbole religieux, on teste connaissances(théologie)+MEM, pour débattre d'un point de détail juridique, il s'agit d'un test de LOG+connaissances(droit)
\paragraph{Domaines:}Droit, théologie, sciences naturelles, ingénierie, anatomie, stratégie, politique, héraldique, économie, alchimie, théurgie.
\subsection{Culture générale()}
\paragraph{Description:}La culture générale représente un savoir acquis par contact et par habitude : il ne s'agit pas de connaissances académiques et théoriques, mais bien des détails pratiques.
\paragraph{Exemples:}Identifier un animal peut se faire avec un test de MEM+culture générale (faune) par exemple, discuter de l'état de la région avec un test de LOG+culture générale (région)
\paragraph{Domaines:}Régions, faune(par région), forêt, flore(par région), commerce, contes et légendes.
\subsection{Herboristerie}
\paragraph{Description:}L'herboristerie est la connaissance des plantes et des ingrédients qui rentrent dans la composition de nombreux remèdes, onguents et cataplasmes, utilisés par ceux qui peuvent se les payer. C'est un élément essentiel en matière de santé.
\paragraph{Exemples:}pour savoir quel remède appliquer à un mal identifié, il s'agit d'un test de MEM+herboristerie, pour traiter des symptômes sans connaître l'origine, on peut tenter un test de LOG+herboristerie. Enfin, réaliser un remède délicat nécessite un test d'herboristerie+COO.
\subsection{Organisation}
\paragraph{Description:}La compétence organisation couvre les méthodes permettant d'organiser de nombreuses tâches : répartition du travail, logistique, anticipations de problèmes, planification.... C'est la compétence des intendants, contremaîtres et quartiers-maîtres, mais aussi de nombreux marchands, artisans....
\paragraph{Exemples:}Pour planifier la logistique d'un voyage, ou préparer des patrouilles, on réalise un test d'organisation+LOG.
\subsection{Lecture}
\paragraph{Description:}La lecture n'est pas une compétence courante dans la société : seuls des religieux, un certains nombres de nobles, et une part non négligeables des marchands. Un test n'est normalement pas nécessaire pour lire une lettre simple, mais pour saisir un message sous-jacent, et certaines références plus discrètes, un test peut être demandé. Il faut au moins un point dans cette compétence pour pouvoir lire un texte.
\paragraph{Exemples:}Déchiffrer une lettre mal écrite nécessite un test de LOG+lecture, de même pour interpréter une parabole dans un vieux grimoire.
\subsection{Médecine}
\paragraph{Description:}Médecine correspond à l'ensemble des connaissances liées aux maladies, blessures et divers maux qui affectent les hommes. Elle permet d'interpréter des symptômes pour essayer d'en deviner l'origine, et éventuellement de trouver un remède qui paraîtrait adapté.
\paragraph{Exemples:}Interpréter les symptômes d'une maladie pour en deviner l'origine nécessite un test de LOG+Médecine, pour y ajouter une idée de remède, il s'agit d'un test de MEM+médecine.
\subsection{Langue()}
\paragraph{Description:}Cette compétence recouvre l'usage d'une langue particulière : sa grammaire, son alphabet, ses idiomes.... Parler de manière normale ne nécessite pas de test (par exemple, pour prendre une chambre à l'auberge), par contre, comprendre un texte délicat ou participer à une discussion intellectuelle, ou encore comprendre un patois particulièrement difficile, sont toutes des occasions de réaliser un test. Il faut au moins un point dans la compétence pour pouvoir tenter un test.
\paragraph{Exemples:}Pour comprendre une discussion difficile pour une raison ou une autre, on réalise un test de LOG+Langue.
\paragraph{Domaines:}chaque langue est un domaine différent.

\section{Mysticisme}
\subsection{Divination}
\paragraph{Description:}La divination est l'art de lire divers signes afin de prédire le futur, ou au moins d'en trouver quelques indices. C'est un art dont l'utilité est débattue par certains, mais en qui d'autres placent tous leurs espoirs. Il n'est pas possible de réaliser un test de divination sans y avoir de points.
\paragraph{Exemples:}Pour lire des runes, ou interpréter le vol des oiseaux (suivant sa culture), un personnage teste divination+LOG.
\subsection{Liturgie}
\paragraph{Description:}La liturgie est l'ensemble des codes et éléments des cérémonies d'une religion : la façon dont elles s'organisent, les symboles utilisés, les textes sacrés cités.
\paragraph{Exemples:}Officier à une cérémonie religieuse est un test de liturgie+EMP.
\subsection{Prières()}
\paragraph{Description:}La compétence de prières recouvre les paroles à utiliser, les mots qui sonnent justes pour réaliser une prière, mais aussi la foi qui les anime. Là où liturgie couvre une cérémonie, publique, une prière est quelque chose d'individuel.
\paragraph{Exemples:}On réalise un test de CAR+prières pour réaliser une de celles-ci.
\paragraph{Domaines:}Bénédictions, malédictions.
\subsection{Sorcellerie()}
\paragraph{Description:}La sorcellerie est un art peu compris, dans lequel un homme (ou une femme), tente de manipuler la réalité, que cela soit par sa propre volonté, ou par le biais de divers artifices.
\paragraph{Exemples:}
\paragraph{Domaines:}amulettes, liens, envoûtements.
\subsection{Symbolisme}
\paragraph{Description:}Le symbolisme est la capacité d'interpréter des symboles mystiques ou religieux, mais aussi de trouver ceux qui seraient nécessaires à son propre usage. Par exemple, pour affecter un noble avec un envoûtement, un symbole intéressant peut être ce qui est représenté sur son blason, ou peut-être que quelque chose lié à son surnom sera plus adapté.
\paragraph{Exemples:}Pour interpréter les symboles d'un rêve, ou trouver ce qui pourrait le mieux symboliser son adversaire, on réalise un test de LOG+symbolisme.

\section{Savoirs-faire}
\subsection{Chirurgie}
\paragraph{Description:}La chirurgie est la capacité à réaliser une opération sur un patient. Il est utile de rappeler qu'il n'existe par nécessairement d'anesthésiant efficaces ou sans risques, attacher un patient peut être pertinent, comme disposer de solides assistants. On peut aussi noter que les compétences sont souvent communes avec celles de dentiste, et plus surprenamment, de barbier.
\paragraph{Exemples:}Retirer une flèche de chasse de la cuisse d'un personnage nécessite un test de Chirurgie+COO. De même pour extraire une dent cariée.
\subsection{Crochetage}
\paragraph{Description:}Le crochetage est l'art d'ouvrir une serrure verrouillée sans en posséder la clé ni la détruire. Il s'agit de manier divers outils plus ou moins improvisés de manière à activer le mécanisme.
\paragraph{Exemples:}Ouvrir un coffre verrouiller par un cadenas nécessite un test de Crochetage+COO.
\subsection{Dressage()}
\paragraph{Description:}Le dressage est la compétence qui couvre la plupart des interactions avec les animaux : les dresser à proprement parler, mais aussi s'en faire obéir, les calmer.... Les animaux sont très présents dans la vie de tout les jours : chiens, mules, chats, chevaux sont les plus courant, mais la noblesse adore également chasser grâce à des oiseaux de proie, et de certains spectacles sont agrémentés par la présence d'un montreur d'ours.
\paragraph{Exemples:}Lancer un chien sur une piste nécessite un test de Dressage(chiens)+LOG, quand calmer un faucon agité est un test de Dressage(oiseaux)+EMP.
\paragraph{Domaines:}chiens, oiseaux, équidés, bétail, ours;
\subsection{Fouille}
\paragraph{Description:}La fouille est la compétence utilisée pour chercher spécifiquement quelque chose : trouver un objet dans une salle, ou chercher quelqu'un dans une foule. C'est une compétence appréciée dans la profession d'aventurier.
\paragraph{Exemples:}Chercher des armes sur un personnage est un test de COO+fouille, trouver une note dans une chambre d'auberge est un test de LOG+fouille. Enfin, chercher à remarquer quelqu'un dans une foule est un test de VIG+fouille.
\subsection{Jeu()}
\paragraph{Description:}Depuis l'aube de l'humanité, de nombreux jeux permettent de se détendre et de s'occuper : dés, adresse, stratégie, les possibilités sont nombreuses, et souvent, paris et enjeux suivent.On peut noter que les cartes sont peu courante, étant fragiles et chères.
\paragraph{Exemples:}Pour déterminer le vainqueur d'une partie d'échecs, on peut réaliser un test de jeu(stratégie)+LOG, de même pour gagner dans une partie de dés, même s'il peut être plus facile de tricher, avec un test de COO+jeu(dés)
\paragraph{Domaines:}par type de jeu : stratégie, adresse, dés...
\subsection{Passe-passe}
\paragraph{Description:}La compétence passe-passe représente la capacité de certains personnages à manipuler finement et très rapidement leurs mains, souvent pour tromper leur monde.
\paragraph{Exemples:}Un test de passe-passe+COO peut servir à subtiliser discrètement une bourse en vue, ou dissimuler une arme à un garde. Alternativement, il est possible de tenter un test identique pour réaliser un tour de prestidigitation.
\subsection{Pistage}
\paragraph{Description:}Le pistage permet de suivre des traces, en particulier dans le contexte de la chasse. En effet, en se déplaçant, la plupart des individus laissent des traces, d'une forme ou d'une autre, que quelqu'un de doué peut ensuite potentiellement suivre et remonter.
\paragraph{Exemples:}Suivre la piste d'un gibier requiert un test de VIG+pistage.
\subsection{Premiers soins}
\paragraph{Description:}Avant d'envisager des soins, encore faut-il que le patient ne décède pas d'hémorragie. C'est à cela que servent les premiers soins: bander les plaies ou les recoudre suivant les besoins, mais aussi immobiliser une fracture par exemple.
\paragraph{Exemples:}Appliquer les premiers soins nécessite un test de COO+Premiers soins
\subsection{Profession()}
\paragraph{Description:}Au delà d'être aventuriers, certains personnages peuvent disposer de compétences professionnelles très diverses. Elles peuvent, parfois, se révéler utiles, même pour des aventuriers.
\paragraph{Exemples:}Réparer une vieille cotte de maille perforée nécessite quelques matériaux, et un test de profession(armurier)+COO. Réaliser une présentation de table impeccable pour un banquet est un test de profession(page)+LOG.
\paragraph{Domaines:}par profession : armurier, forgeron, charpentier, page, scribe, héraut, garde, ...

\section{Social}
\subsection{Art()}
\paragraph{Description:}L'humanité aime l'art, et ce depuis des millénaires. Il en existe de nombreuses formes, dont la popularité varie avec les modes et les mœurs. 
\paragraph{Exemples:}réaliser une superbe interprétation d'un chant connu est un test d'Art(chant)+EMP. Peindre un portrait flatteur du bourgmestre, est un test d'art(peinture)+COO.
\paragraph{Domaines:}chant, instrument de musique, peinture, jonglage, poésie, danse...
\subsection{Calme}
\paragraph{Description:}Parfois, il est plus difficile de garder son calme que quoique ce soit d'autre, en particulier quand une bonne éducation implique de garder la tête froide.
\paragraph{Exemples:}Pour ne pas coller une gifle à un courtisan particulièrement acide dans ses commentaires, il faut réussir un test de CAR+calme.
\subsection{Commandement}
\paragraph{Description:}Le commandement est l'art de se faire obéir avec efficacité et diligence. C'est une compétence importante pour tout dirigeant, mais aussi pour les sous-officiers qui doivent transmettre les ordres aux troupes par exemple.
\paragraph{Exemples:}Beugler des ordres à un groupe de recrues jusqu'à les faire entrer dans le rang est un test de commandement+AGR. Diriger des cuisines pendant un banquet est un test de commandement+CAR.
\subsection{Discussion}
\paragraph{Description:}Tenir une discussion peut être important, pour être perçu comme un hôte ou un invité agréable, et passer de manière générale pour quelqu'un qui gagne à être connu.
\paragraph{Exemples:}occuper un seigneur pendant que son autre voisin discute d'un élément important est un test de discussion+EMP. Mettre en avant sa personne auprès de la fille de l'aubergiste est également un test de discussion+EMP. Imposer sa discussion au sein d'un groupe de nobles est un test d'AGR+discussion.
\subsection{Éducation}
\paragraph{Description:}Certains milieux nécessitent, pour rester crédible, d'en connaître les codes : un banquet est ainsi un évènement codifié et encadré par des règles plus ou moins tacites, tout comme une cérémonie officielle. La compétence éducation représente cette connaissance et leur utilisation.
\paragraph{Exemples:}Présenter un hommage en bonne et due forme à son seigneur est un test d'éducation+LOG; le faire avec des manières volontairement anciennes pour montrer sa culture est un test de MEM+education.
\subsection{Intimidation}
\paragraph{Description:}Parfois, le plus rapide pour faire passer son point de vue est la menace, plus ou moins évidente. Ce n'est pas élégant, et risque de ne pas être oublié, mais cela a souvent le mérite d'être rapide.
\paragraph{Exemples:}Expliquer à un usurier louche que s'il n'arrête pas de vous arnaquer il aura besoin d'un garde supplémentaire est un test d'AGR+intimidation.
\subsection{Mensonge}
\paragraph{Description:}Le mensonge est une activité qui consiste à ne pas dire ce que l'on pense, et cela est aussi vieux que la parole. On parle ici évidemment de mensonges volontaires, et non d'oublis ou d'erreurs.
\paragraph{Exemples:}Expliquer à un garde que l'on est un paysan amenant du bois alors que des soldats sont cachées entre les fagots nécessite un test d'EMP+mensonge.
\subsection{Manipulation}
\paragraph{Description:}Proche du mensonge, il s'agit ici d'amener quelqu'un à parler d'un sujet particulier, ou de lui suggérer une idée, comme si elle était la sienne.
\paragraph{Exemples:}Planter l'idée que peut-être son intendant n'est pas si honnête auprès d'un seigneur est un test de manipulation+EMP.
\subsection{Négociation}
\paragraph{Description:}Négocier est la compétence recouvrant le fait de convaincre directement, par divers arguments un personnage de quelque chose.
\paragraph{Exemples:}Pour obtenir un prix plus bas sur un achat, c'est un test de négociation+CAR; De même pour obtenir l'autorisation d'un seigneur d'enquêter sur ses terres à propos d'un criminel.

\chapter{Social}
%TODO
\chapter{Aventure}
%TODO
\section{Voyages et déplacements}
\section{Statuts}
Un statut est une jauge représentant un problème que subit le personnage. Les plus courant sont les blessures, le stress et la fatigue, mais des personnages peuvent également rencontrer la faim, la soif, les maladies....
\chapter{Mysticisme}
%TODO
\section{Divination}
\section{Liturgie}
\section{Prières}
\section{Sorcellerie}

\chapter{Combat}
%TODO : finir les détails, les dégâts notamment
\section{Échanges de coups}
La première et la plus fréquente des formes de combat est le combat classique en mêlée, au cours duquel les deux combattants cherchent à mettre un coup à leur adversaire, que ce soit avec leur poings ou une arme.
\subsection{Toucher}
La première étape avant de faire quoique ce soit d'autre, est de réussir à toucher son adversaire. Pour cela, on réalise un test opposé entre l'AGR+compétence de l'attaquant et la COO+Défense du défenseur.

L'attaquant utilise pour son attaque la compétence adaptée au type de coup qu'il tente : arme(lames) pour frapper à l'épée, arme(couteaux) pour une frappe vicieuse à la dague ou encore corps à corps pour mettre un coup à main nue.

En cas de réussite de l'attaquant, celui-ci passe aux dégâts, son attaque ayant touchée, en cas de réussite du défenseur, celui-ci peut contre-attaquer. Dans le cas d'un ex-aequo, l'attaque touche, et le défenseur peut contre-attaquer.

Se défendre plusieurs fois est difficile, et un personnage subit un malus de -2 au nombre de dés lancés pour se défendre pour chaque test de défense réalisé entre deux de ses tours de combats : le premier se fait normalement, le second à -2, le troisième à -4, etc.

Pour attaquer une cible qui ne se défend pas (par choix ou car elle ne se rend pas compte de l'attaque) est un test simple, avec une difficulté de 0 : il faut au moins un succès pour toucher, mais la marge de réussite est égale au nombre total de succès.
\subsection{Contre-attaque}
Entre chacun de ses rounds de combat, un personnage peut réaliser une contre-attaque (et une seule). Il s'agit d'une attaque normale, à l'exception du fait que la cible ne peut pas elle réaliser de contre-attaque.
\section{Infliger des dégâts}

\section{Tirs}
Toucher une cible à distance est un test simple, dont la difficulté dépend de la distance, et de la portée de l'arme. Le test à lancer est un test de compétence+COO. La compétence utilisée dépend de l'arme voulue.

Si le test pour toucher est réussi, le défenseur peut essayer d'éviter ou bloquer le tir avec un test de COO+Défense dont la difficulté dépend de la distance. Un personnage ne peut pas essayer de se défendre contre un tir qu'il n'a pas vu venir, mais peut tenter un test de CHA contre la marge de réussite du tir, peut-être que le hasard fera bien les choses....


\section{Corps à corps}
Cette section est à propos du combat rapproché, où les adversaires se saisissent pour mieux s'affronter.
\subsection{Lutte}
Pour arriver en combat rapproché, on parle de lutte. Il est possible pour un combattant de tenter un test de Corps à corps+AGR contre la défense de son adversaire pour arriver à le saisir. Le défenseur peut contre-attaquer en cas d'ex-aequo ou de victoire de sa part. 

Si un attaquant sans arme tente de saisir un adversaire armé sans que celui-ci ne soit distrait par quoique ce soit, la contre-attaque est toujours autorisée, même si l'attaquant gagne le test.

Une fois les deux combattants en lutte, ils peuvent tenter de frapper (avec un malus pour toute arme non-courte) ou de maîtriser leur adversaire.
\subsection{Projections et clés}
Une fois un adversaire en lutte, il est possible de tenter de projeter ou maîtriser l'adversaire. Pour cela, on réalise un test opposé de Corps à corps+AGR, contre le corps à corps + COO du défenseur.

En cas de réussite de l'attaquant, il peut choisir de projeter son adversaire au sol ou de le maîtriser.En cas d'échec, le défenseur peut tenter lui-même de réaliser une des deux manœuvres. En cas d'ex æquo, rien ne se passe.

Il est possible de ne pas passer par l'étape lutte, l'adversaire ayant alors automatiquement droit à une contre-attaque armée résolue avant le corps à corps. Par ailleurs, en cas d'échec au test de corps à corps, le défenseur peut encore tenter un contre, comme en temps normal.
\subsection{Frapper en combat rapproché}
Une autre option une fois en lutte est de poignarder son adversaire avec une arme courte: pour cela, on réalise un test d'AGR+poignarder contre la défense+COO de l'adversaire.

En cas de réussite au test.....
\section{Dépasser le duel : batailles, sièges, escarmouches}
\subsection{Le combat de masse}
\subsection{Le cas des sièges}


\part*{Annexes}
\section*{Inspiration}
\subsection*{Jeux vidéo}
\begin{itemize}
\item Battle Brothers
\item Mount \& Blade
\item The Banner Saga
\end{itemize}
\subsection*{Livres}
\begin{itemize}
\item The Witcher
\item Le chevalier rouge
\end{itemize}
\subsection*{Films et séries}
\begin{itemize}
\item Ironclad
\item Kingdom of Heaven
\end{itemize}
\subsection*{Jeux de rôle}
\begin{itemize}
\item The Burning Wheel
\item Dungeon World
\item Eclipse Phase
\item Dungeon World
\end{itemize}
\tableofcontents


\end{document}