\documentclass[10pt,a4paper]{article}
\usepackage[utf8]{inputenc}
\usepackage{amsmath}
\usepackage{amsfonts}
\usepackage{amssymb}
\title{Cyberpunk : Vladivostok}
\begin{document}
\maketitle
\tableofcontents

\chapter{Histoire de Vladivostok}
Vladivostok a été, depuis sa recapture par l'armée rouge en 1945, un port majeur pour la flotte soviétique, militaire tant que marchande. C'était aussi la plus grande ville de tout l'orient russe, à une extrémité du transsibérien.

\section{1990-2008 : réformes et NPS}
Alors que l'union soviétique s'affaiblissait, un groupe de membres du parti réussi à prendre le contrôle d'une bonne part des organes d'état, et à imposer un nouveau courant de réforme : le NPS (nouveau parti Soviétique). 

Sous cette impulsion, l'union soviétique rattrapa une bonne part de son retard, notamment sur l'amérique, et les frontières du bloc de l'est furent également ouvertes.

A Vladivostok, cela réorienta de nombreux développement : pour augmenter la capacité commerciale du port, il fut notamment agrandi, ce qui ouvrit une période faste pour la ville, avec de lourds investissement dans les infrastructures de la ville.
\section{2008-2010 : seconde guerre corpo}
La seconde guerre corpo se déroula entre les forces de la corpo américaine Petrochem et celles du groupe soviétique SovOil, et ravagea le Pacifique. Le sujet de l'affrontement était initialement les possibilités commerciales en mer de chine du sud, mais cela faisait quelques années que les deux groupes s'observaient et se testaient.

SovOil remporta les premiers combats, remportant rapidement une série de victoires, mais sans gagner décisivement la guerre.

La contre-offensive de Petrochem ne se fit pas attendre, avec des opérations commando dans de nombreux ports dont Vladivostok. A ce moment, les capacités des deux belligérants à continuer leur activité était déjà bien diminuée par les lourds dégâts aux installations : plate-formes, puits, oléoducs, raffineries et installations portuaires étaient fréquemment visés pour affaiblir la puissance de l'autre.

Enfin, une seconde offensive majeure par les troupes de SovOil (et leurs mercenaires) renversa définitivement la balance en leur faveur, en mettant à genou les capacités de leur adversaire américain.

Ce conflit s'est distingué de la première guerre corpo par l'absence d'effets des sanctions internationales envers les deux protagonistes, qui ont chacun de plus utilisé les états du Pacifique comme des prêtes-noms, allant parfois jusqu'à remplacer la totalité de leur administration. La frappe de Petrochem contre le président de SovOil ayant été conduite par exemple par des Mirages 111 'prêtés' par la Malaisie. Cela a également conduit à des pollutions dramatiques en mer de Chine du sud, objet initial du conflit.

Vladivostok a joué un rôle central dans ce conflit, étant la principale base d'opération entre le QG de SovOil et le théâtre d'opérations. La ville a également subi des opérations de Petrochem, notamment sur son terminal pétrolier.
\section{2010-2021 : crise grave}
A la fin de la seconde guerre corpo, Vladivostok en a subi le contrecoup : après 4 ans, son économie était très centré sur ce conflit, beaucoup plus que sur sa place dans le commerce de l'union soviétique : d'autres ports comme Sébastopol et Arkhangelsk en ayant profité pour prendre une part plus importante du commerce extérieur.

Tout ceci a amené une crise économique locale, malgré le fait d'être clairement du côté des vainqueurs : l'accroissement du trafic de SovOil par le terminal pétrolier (qu'il a fallut réparer) ne compensant pas les pertes dans les autres secteurs.

De cette époque, la principale marque dans le paysage reste les squelettes vides des grandes usines créées pendant la guerre, et les équipements militaires rouillant dans les eaux du port.

\section{2021-2025 : quatrième guerre corpo}
La quatrième guerre corpo fut beaucoup moins présente que la seconde, le conflit étant entre les américains de Militech et les japonais d'Arasaka. Ces deux corporations se livrèrent une guerre sauvage sur l'ensemble du globe, libérant des armes de plus en plus dangereuses pendant quatre ans, avant la destruction du QG américain d'Arasaka par une tête nucléaire tactique déployée par un groupe terroriste toujours inconnu.

La ville a été le théâtre de quelques opérations, visant les filliales régionales de deux corpos, mais peu de véritables batailles eurent lieu ici. Les deux camps firent appel aux usines de la région, et les attaquèrent, mais le plus rude coup pour la ville fut la quasi-destruction du commerce international, dont son port dépendait.

%Changements dans la politique de l'URSS, qui devient la NURSS
Au niveau politique, le NSP vieillissant fut remplacé par un nouveau genre d'hommes politiques, élevés aux 'valeurs' corporatistes et aux dents longues. Si ils ont lutté contre l'influence des deux belligérants, c'était plus pour maintenir leur propre pouvoir que par sympathie avec les états ignorés ou écrasés par eux. Ce changement de régime fut officialisé en 2025 par la fondation de la NURSS : la Nouvelle union des républiques soviétiques socialistes.
\section{2026 - 2039 : l'époque du RED}
Après la guerre, les cendres dans l'atmosphère donnèrent au ciel une teinte rouge, et certaines pluies recouvrent tout d'une sorte de 'jus' rouge rappelant le sang. C'est le temps du RED : Arasaka a été fragmenté en de multiples branches plus ou moins rivales, et les anciens états ont pour beaucoup été disloqués.

La nouvelle union soviétique est ainsi plus un ensemble de villes pillées par leurs administrateurs au milieu des ruines de la guerre. Cet état déplorable causa plusieurs états à s'effondrer ou s'affaiblir : la Corée du nord a dégénéré en plusieurs états de seigneur de guerre, la Chine a perdu une partie de la mongolie intérieure face à la NURSS.

Vladivostok profita partiellement de cette période : les opérations militaires dans la région, notamment les compagnies de mercenaires formées pendant la guerre apportèrent de bons revenus, et l'afflux de réfugiés venus de la région permit à la ville de s'étendre grâce à une main d'oeuvre dans l'ensemble exploitée : coréens, japonais, chinois, et russes évidemment.
\section{2039-2045 : Zhirafa Technical Manufacturing}
L'arrivée à Vladivostok d'un tech de Moscou, enrichi comme mercenaire pendant la guerre, et avec de bons contacts fut un évènement relativement peu noté en 2039, mais qui changea radicalement la ville dans les 6 années suivantes. Artyom Sokolov fonda Zhirafa Technical Manufacturing avec un produit phare : le GRAF3, un nouveau modèle de robot de construction, parfait pour nettoyer les séquelles du conflit, et suffisamment simple, robuste et peu cher pour être à la fois présent en masse, et résistant aux abus. En quelques années, la plupart des communautés russes en disposèrent, et les usines Zhirafa recrutèrent rapidement de plus en plus de personnel. 

Depuis, la corpo a étendu son catalogue à de nombreux modèles de drones, notamment dans le milieu de la sécurité. En raison du niveau technologie en NURSS, ce ne sont pas des drones à la pointe de la modernité, mais ils sont peu coûteux, et surtout, robustes.

Les mauvaises langues parlent de soutien de la puissante Bratva (le crime organisé), et de leurs amis nomades en russie pour expliquer la montée fulgurante de la corp, et il est vrai que certains détails de la vie de son fondateur sont toujours flous.

Dans tous les cas, Vladivostok héberge maintenant son QG global, et est facilement au centre des enjeux économiques et politiques de la ville, avec de nombreuses usines et sous-traitants. Les services de sécurité sont soutenus par un important contingent de drones, les GRAF3 et 4 sont des visions normales sur tous les chantiers, et le logo de Zhirafa est une vision commune.
\chapter{Géographie}
\section{Le port pétrolier}
\section{Le port de commerce}
\section{La flottille}
\section{La forteresse}
\section{L'île russkiy}
\section{Little Pyongyang - Slobodka}
\section{Baie d'Ussuri}
\section{Baie d'Amur}
\section{Millionka - chinatown}
\section{Arcologie Vhi-7}
%
\section{Arcologie Ka-110}
%Arco pénale, extrêmement dangereuse, digne héritière des goulag staliniens. Plus ouverte ces dernières années (mais pas pour de bonnes raisons).

\chapter{Milieux criminels}
\chapter{Politique et gouvernement}
\chapter{Corporations}
\section{Zhirafa Technical manufacturing}
\chapter{Gens importants}
\section{Artyom Sokolov}

\end{document}