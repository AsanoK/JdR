\documentclass[10pt,a4paper]{book}
\usepackage[utf8]{inputenc}
\usepackage{amsmath}
\usepackage{amsfonts}
\usepackage{amssymb}
\title{Cyberpunk : Vladivostok}
\begin{document}
\maketitle
\tableofcontents

\chapter{Histoire de Vladivostok}
Vladivostok a été, depuis sa recapture par l'armée rouge en 1945, un port majeur pour la flotte soviétique, militaire tant que marchande. C'était aussi la plus grande ville de tout l'orient russe, à une extrémité du transsibérien.

\section{1990-2008 : réformes et NPS}
Alors que l'union soviétique s'affaiblissait, un groupe de membres du parti réussi à prendre le contrôle d'une bonne part des organes d'état, et à imposer un nouveau courant de réforme : le NPS (nouveau parti Soviétique). 

Sous cette impulsion, l'union soviétique rattrapa une bonne part de son retard, notamment sur l'amérique, et les frontières du bloc de l'est furent également ouvertes.

A Vladivostok, cela réorienta de nombreux développement : pour augmenter la capacité commerciale du port, il fut notamment agrandi, ce qui ouvrit une période faste pour la ville, avec de lourds investissement dans les infrastructures de la ville.
\section{2008-2010 : seconde guerre corpo}
La seconde guerre corpo se déroula entre les forces de la corpo américaine Petrochem et celles du groupe soviétique SovOil, et ravagea le Pacifique. Le sujet de l'affrontement était initialement les possibilités commerciales en mer de chine du sud, mais cela faisait quelques années que les deux groupes s'observaient et se testaient.

SovOil remporta les premiers combats, remportant rapidement une série de victoires, mais sans gagner décisivement la guerre.

La contre-offensive de Petrochem ne se fit pas attendre, avec des opérations commando dans de nombreux ports dont Vladivostok. A ce moment, les capacités des deux belligérants à continuer leur activité était déjà bien diminuée par les lourds dégâts aux installations : plate-formes, puits, oléoducs, raffineries et installations portuaires étaient fréquemment visés pour affaiblir la puissance de l'autre.

Enfin, une seconde offensive majeure par les troupes de SovOil (et leurs mercenaires) renversa définitivement la balance en leur faveur, en mettant à genou les capacités de leur adversaire américain.

Ce conflit s'est distingué de la première guerre corpo par l'absence d'effets des sanctions internationales envers les deux protagonistes, qui ont chacun de plus utilisé les états du Pacifique comme des prêtes-noms, allant parfois jusqu'à remplacer la totalité de leur administration. La frappe de Petrochem contre le président de SovOil ayant été conduite par exemple par des Mirages 111 'prêtés' par la Malaisie. Cela a également conduit à des pollutions dramatiques en mer de Chine du sud, objet initial du conflit.

Vladivostok a joué un rôle central dans ce conflit, étant la principale base d'opération entre le QG de SovOil et le théâtre d'opérations. La ville a également subi des opérations de Petrochem, notamment sur son terminal pétrolier.
\section{2010-2021 : crise grave}
A la fin de la seconde guerre corpo, Vladivostok en a subi le contrecoup : après 4 ans, son économie était très centré sur ce conflit, beaucoup plus que sur sa place dans le commerce de l'union soviétique : d'autres ports comme Sébastopol et Arkhangelsk en ayant profité pour prendre une part plus importante du commerce extérieur.

Tout ceci a amené une crise économique locale, malgré le fait d'être clairement du côté des vainqueurs : l'accroissement du trafic de SovOil par le terminal pétrolier (qu'il a fallut réparer) ne compensant pas les pertes dans les autres secteurs.

De cette époque, la principale marque dans le paysage reste les squelettes vides des grandes usines créées pendant la guerre, et les équipements militaires rouillant dans les eaux du port.

\section{2021-2025 : quatrième guerre corpo}
La quatrième guerre corpo fut beaucoup moins présente que la seconde, le conflit étant entre les américains de Militech et les japonais d'Arasaka. Ces deux corporations se livrèrent une guerre sauvage sur l'ensemble du globe, libérant des armes de plus en plus dangereuses pendant quatre ans, avant la destruction du QG américain d'Arasaka par une tête nucléaire tactique déployée par un groupe terroriste toujours inconnu.

La ville a été le théâtre de quelques opérations, visant les filliales régionales de deux corpos, mais peu de véritables batailles eurent lieu ici. Les deux camps firent appel aux usines de la région, et les attaquèrent, mais le plus rude coup pour la ville fut la quasi-destruction du commerce international, dont son port dépendait.

%Changements dans la politique de l'URSS, qui devient la NURSS
Au niveau politique, le NSP vieillissant fut remplacé par un nouveau genre d'hommes politiques, élevés aux 'valeurs' corporatistes et aux dents longues. Si ils ont lutté contre l'influence des deux belligérants, c'était plus pour maintenir leur propre pouvoir que par sympathie avec les états ignorés ou écrasés par eux. Ce changement de régime fut officialisé en 2025 par la fondation de la NURSS : la Nouvelle union des républiques soviétiques socialistes.
\section{2026 - 2039 : l'époque du RED}
Après la guerre, les cendres dans l'atmosphère donnèrent au ciel une teinte rouge, et certaines pluies recouvrent tout d'une sorte de 'jus' rouge rappelant le sang. C'est le temps du RED : Arasaka a été fragmenté en de multiples branches plus ou moins rivales, et les anciens états ont pour beaucoup été disloqués.

La nouvelle union soviétique est ainsi plus un ensemble de villes pillées par leurs administrateurs au milieu des ruines de la guerre. Cet état déplorable causa plusieurs états à s'effondrer ou s'affaiblir : la Corée du nord a dégénéré en plusieurs états de seigneur de guerre, la Chine a perdu une partie de la mongolie intérieure face à la NURSS.

Vladivostok profita partiellement de cette période : les opérations militaires dans la région, notamment les compagnies de mercenaires formées pendant la guerre apportèrent de bons revenus, et l'afflux de réfugiés venus de la région permit à la ville de s'étendre grâce à une main d'oeuvre dans l'ensemble exploitée : coréens, japonais, chinois, et russes évidemment.
\section{2039-2045 : Zhirafa Technical Manufacturing}
L'arrivée à Vladivostok d'un tech de Moscou, enrichi comme mercenaire pendant la guerre, et avec de bons contacts fut un évènement relativement peu noté en 2039, mais qui changea radicalement la ville dans les 6 années suivantes. Artyom Sokolov fonda Zhirafa Technical Manufacturing avec un produit phare : le GRAF3, un nouveau modèle de robot de construction, parfait pour nettoyer les séquelles du conflit, et suffisamment simple, robuste et peu cher pour être à la fois présent en masse, et résistant aux abus. En quelques années, la plupart des communautés russes en disposèrent, et les usines Zhirafa recrutèrent rapidement de plus en plus de personnel. 

Depuis, la corpo a étendu son catalogue à de nombreux modèles de drones, notamment dans le milieu de la sécurité. En raison du niveau technologie en NURSS, ce ne sont pas des drones à la pointe de la modernité, mais ils sont peu coûteux, et surtout, robustes.

Les mauvaises langues parlent de soutien de la puissante Bratva (le crime organisé), et de leurs amis nomades en russie pour expliquer la montée fulgurante de la corp, et il est vrai que certains détails de la vie de son fondateur sont toujours flous.

Dans tous les cas, Vladivostok héberge maintenant son QG global, et est facilement au centre des enjeux économiques et politiques de la ville, avec de nombreuses usines et sous-traitants. Les services de sécurité sont soutenus par un important contingent de drones, les GRAF3 et 4 sont des visions normales sur tous les chantiers, et le logo de Zhirafa est une vision commune.
\chapter{Géographie}
\section{Le port pétrolier}
Ravagé par une frappe de Petrochem pendant la seconde guerre corpo, le terminal pétrolier a depuis été reconstruit, pour être presque immédiatement obsolète avec la montée en puissance du COOH2 comme carburant par opposition aux dérivés du pétrole, un point faible important de l'économie soviétique, dont une bonne part des exports étaient justement ce pétrole. 

Le terminal est donc partiellement abandonné , avec pas mal de squats dans les anciennes raffineries. Ces endroits sont mal famés, où la règle la plus commune est celle du chacun pour soi : les habitants possèdent ce qu'ils peuvent porter et pas grand-chose de plus. C'est en particulier vrai pour l'ancien terminal C. On ne va pas dans cet endroit, on y échoue.

Les terminaux A et B eux sont toujours en usage, avec une sécurité relativement modeste pour se défendre contre ceux qui souhaiteraient voler du carburant comme combustible. SovOil est la principale corpo présente, avec une petite installation d'entraînement récemment construite par Faction Rouge.
\paragraph{Niveau de sécurité:} zone de combat/moyen
\section{Le port de commerce}
Le port de commerce a perdu un peu en envergure avec la chute du commerce international, mais plus récemment, des accords avec les nomades de la mer du japon ont permis de relancer l'activité. Toutes sortes d'activités même : de nombreux immigrés japonais notamment, ou d'anciens nomades ont installé toute sortes d'établissement liés au divertissement en utilisant tant les containers que les anciens entrepôts. 

Les nouvelles salles de concert, bordels et casino côtoient donc les jetées, les portiques à container et les entrepôts en activité. C'est un quartier plutôt calme, même la nuit : les corpos tiennent à leurs installations dans le secteur, et les établissements de loisir maintiennent des relations importantes avec les gangs locaux.

Les habitants viennent de partout, avec une forte présence d'immigrés d'origine japonaise (suite à la chute d'Arasaka), parfois d'anciens nomades de la mer du japon.
\paragraph{Niveau de sécurité:}moyen
\section{La flottille}
On désigne par 'la flottille' l'ensemble des nombreuses embarcations et plates-formes flottantes qui opèrent dans les baies d'Ussuri et d'Amur. Si ses habitants considèrent qu'il s'agit pratiquement d'un quartier à part de Vladivostok, il n'existe pas officiellement, et tombe sous le coup de la juridiction des gardes-côtes.

Il s'agit donc de centaines d'embarcations, souvent habitées, parfois mobiles, qui existent dans les deux baies, à bonne distance des chemins des navires plus gros par contre : ces derniers, corpos, ne s'arrêtent pas devant les coquilles de noix des locaux, et tous les mois, il y a des morts, trop près des chenaux.

Il n'y a que peu d'organisation dans cette zone, à part celle, individuelle, des connaissances que l'ont se fait, et les gangs locaux, avec leurs embarcations rapides mais fragiles.
\paragraph{Niveau de sécurité:}Bordures
\section{La forteresse}
Section de la vieille ville abritant entre autre la forteresse protégeant la ville depuis des siècles, il s’agit aujourd'hui du quartier administratif, où se trouve notamment le gouvernement de l'oblast de Vladivostok (la ville, et officiellement la région alentour.

De plus sombre réputation, en face de la porte de l'ancienne forteresse, on trouve le 'cube', un monstrueux bloc de béton aux fenêtres teintées, un crime contre le bon goût : le QG des services de sécurité de la ville. On évite de mentionner son esthétique de crainte qu'un agent ou informateur n'en ait trop vent et trouve que cela fait un bon prétexte pour arrêter quelqu'un. La mention du 'cube' fait frémir tout criminel dont les activités pourraient tomber sous le coup de la trahison, le sabotage ou la dissidence.

Le crime est donc rare dans ce secteur, la présence menaçante du cube et de ses agents étant pour beaucoup bien assez dissuasive.
\paragraph{Niveau de sécurité:}Corpo
\section{L'île russkiy}
Développée plutôt récemment sur une ancienne réserve naturelle, cette île est aujourd'hui le lieu de résidence des ultra-riches de la ville : Artyom Sokolov y a une villa, tout comme plusieurs membres importants de sa corporation.

Après cet exemple, les célébrités de la région y ont également fait bâtir leurs demeures, et plus récemment encore, ont été rejoints par plusieurs membres du gouvernement local, qui avaient en premier lieu autorisé les constructions sur ces terrains inoccupés.

Le pont qui y mène, l'un des plus longs du monde, est soigneusement protégé, et les embarcations de la flottille savent qu'il vaut mieux éviter la proximité des plages de Russkiy : la sécurité tire pour tuer pour préserver les lieux de villégiatures et la vue des résidents.
\paragraph{Niveau de sécurité:}Exec
\section{Little Pyongyang - Slobodka}
Avec l'effondrement de la Corée du nord, engagée dans la quatrième guerre corpo avec l'espoir de soutiens d'Arasaka, plusieurs domaines de seigneurs de guerre ont été formés et se sont effondrés, souvent dirigés par d'anciens cadres du parti communiste ou de l'armée populaire de corée.

Ce chaos a amené de nombreux habitants de la région à fuir, certains se dirigeant vers Vladivostok dans l'espoir d'un travail, ou pour certains, d'un autre pays sous la bannière familière du communisme.

Le quartier de Slobodka étaient il y a dix ans un bidonville, avec beaucoup de gens employés par Zhirafa. Cela a un peu changé depuis, avec des constructions plus définitives,  et l'emploi par la sécurité corpo et les forces de l'ordre des techniques bien connues à base d'informateurs pour limiter les troubles. Certaines zones restent sous influence de divers gangs, mais ce n'est pas le cas de toutes, loin de là : des résidences ouvrières immenses ont été bâties pour mieux exploiter leurs habitants, dans la plus grande tradition soviétique.
\paragraph{Niveau de sécurité:}Zone de guerre/moyen
\section{Quartier Leninskiy - chinatown}
Plus ancien que Slobodka, le quartier de Leninskiy s'est récemment agrandi avec l'afflux de réfugiés du nord de la Chine et de mongolie intérieure, suite au conflit avec la NURSS. 

Le quartier a connu pas mal de troubles pendant cette période, les habitants pouvant être vus avec suspicions par d'autres citoyens de Vladivostok, quand à leur engagement pour la NURSS.

C'est aujourd'hui un quartier plutôt riche et sûr, avec de nombreux commerces et établissement de loisirs, souvent fréquentés par les employés corpo des quartiers voisins. Quelques gangs existent, mais la plupart sont peu ambitieux peu dangereux par rapport à ce qui existe par ailleurs.
\paragraph{Niveau de sécurité:}moyen/corpo
\section{Arcologie Vhi-7}
Une arcologie partiellement sous-marine, principal composant principal des Vodorosli (algues). La totalité du quartier est considéré comme insalubre et dangereux. Ses habitants, quand ils ont un emploi, travaillent souvent pour des sous-traitants de Zhirafa, ou dans les fermes d'algues qui fournissent la base de l'alimentation de la région. 

Ces fermes collectives sont les points les plus sécurisés du quartier, et leur récolte est transportée sous bonne escorte par les forces de sécurité.

Pas mal de gangs recrutent les jeunes du coin, les attirant en leur promettant un but et de l'action, et peut-être même un espoir de se barrer.
\paragraph{Niveau de sécurité:} Zone de guerre/bordures
\section{Arcologie Ka-110, 'Le cratère'}
Une arcologie à ciel ouvert à proximité de la ville, à la base de la péninsule, Ka-110 est connu comme l'antichambre de l'enfer. Il s'agit à l'origine d'une arcologie pénale, où sont envoyés en 'redressement' les citoyens de la NURSS jugés coupable de sédition ou de comportements antisociaux. 

Ses 'habitants' ne peuvent sortir qu'avec un permis, soigneusement contrôlé par le NKSB, les extérieurs peuvent rentrer, et sont également fouillés et notés.

A l'intérieur, le travail 'productif' est obligatoire : différentes corporations profitent de cette main d'oeuvre à bas prix pour accomplir les tâches dangereuses ou ingrates. Cela n'empêche pas un niveau de violence dépassant tout les autres quartiers de la ville : meurtres et agressions sont monnaie courante dans les bidonvilles entre les usines, tant par les forces de sécurité que par les détenus.

Pour les pires criminels de la ville, le cratère est la pire menace dont disposent les forces de l'ordre. Même les petites frappes avec plus de chrome que de bon sens ont des sueurs froides à l'idée.
\paragraph{Niveau de sécurité:}Zone de guerre
%Arco pénale, extrêmement dangereuse, digne héritière des goulag staliniens. Plus ouverte ces dernières années (mais pas pour de bonnes raisons).

\section{Quartier Fruzenskiy}
%quartier d'habitation/vieille ville
Autour de la forteresse se trouve le quartier de Fruzenskiy, la vieille ville, construite principalement pendant la période Stalinienne. Beaucoup de béton, en mauvais état depuis le temps.

C'est surtout un quartier d'habitation, avec une population essentiellement ethniquement Russe. On y trouve certains des clubs et magasins les plus connus de la ville, notamment le Pradva.

Le quartier est plutôt tranquille, certains diront que la Bratva (le crime organisé) n'aime pas le désordre sur son terrain, mais ce ne sont que des rumeurs....
\paragraph{Sécurité :}Moyenne/corpo
\section{Quartier Pervorechenskiy}
%Quartier industriel refait récemment par ZhirafaTM
Cet ancien quartier a été pratiquement rasé il y a cinq ans pour faire de la place aux installations de Zhirafa Technical Manufacturing : de nouvelles usines, de nouveaux bureaux, des entrepôts et un aéroport dédié le long de la côte.

Le quartier est fortement patrouillé par les troupes de sécurité de Zhirafa et notamment ses drones. Il est donc bien sécurisé, même si des runners ambitieux tentent régulièrement des jobs contre certaines des installations. Certains arrivent même à survivre à leur tentative !
\paragraph{Sécurité :}corpo
\section{Quartier Pervomayskiy}
%ancienne zone industrielle
La zone industrielle historique, avec des usines et ateliers de toute sorte : des armes à feu avec notamment l'usine Dozhd, qui a repris plusieurs designs plus anciens et commence à travailler sur ses propres designs. Plusieurs usines de munitions et de missiles existent aussi, ainsi que des usines automobiles, et deux chantiers navals concurrents.

Toutes ces entreprises ne sont pas en pleine forme, et on trouve aussi des usines abandonnées, mais dans l'ensemble le quartier est calme.

On trouve aussi ici la centrale nucléaire G.K.Zhukov, fournissant la quasi-totalité de son énergie à la ville, et qui est beaucoup plus sécurisée que le reste du quartier, les troupes du NKSB la gardant étant connue pour parfois intervenir sur les problèmes voisins, pour 'garder la main'.
\paragraph{Sécurité :}moyenne
\section{Quartier Sovetskiy}
%quartier d'habitation plutôt ouvert vers l'extérieur, avec pas mal de liens avec les nomades terrestres
Le quartier le plus excentré de la ville, bâti autour de l'ancien autoroute. Il sert maintenant d'interface entre les nomades et l'intérieur de la ville. On y trouve pas mal de petits ateliers, et des gangs plutôt agressifs.

L'ordre dans le quartier vient de la présence de la Bratva, mais même eux n'arrivent pas complètement  à calmer les pires têtes brûlées.

Les gangs ici sont souvent très fiers de leur héritage communiste, et mal parler de la NURSS, ou pire de l'URSS peut mener à une vieille tradition : une balle dans la nuque. En revanche, ils s'affrontent pour des histoires de traditions et de territoire tous les jours.
\paragraph{Sécurité :}Zone de guerre/bordures
\section{Les alentours}
Autour de la ville, c'est la sibérie : il fait horriblement froid l'hiver et relativement chaud l'été.

C'est une zone plutôt montagneuse et boisée, percée de l'ancien autoroute, progressivement laissé à l'état d'abandon.

Il y a pas mal de clans nomades qui passent ici, certains poussent jusqu'en Mongolie, d'autres suivent le transsibérien vers le reste la la NURSS. Ils transfèrent beaucoup de marchandises d'une ville à l'autre, légalement ou non, et sont le plus souvent farouchement indépendants. Les plus violents n'hésitent pas à attaquer les convois corpo, et valent à peine plus que de simples bandits.
\paragraph{Sécurité :}Bordures
%ajouter une ou plusieurs zones industrielles corpo, notamment pour Zhirafa TM
%Ajouter une description de l'extérieur de la ville
\chapter{Crimes et loi}
\section{La Bratva}
Le crime organisé, ses membres étant appelés les vory (un Vor au singulier), ou voleurs. Fut une époque où ils respectaient un code, qui a été brisé dans les prisons de l'URSS. 

Auparavant, ils ne travaillaient jamais avec les forces de l'ordre. Cela a changé, et maintenant certains paient grassement leurs 'amis' de la police.

Ils travaillent dans tous les domaines où ils peuvent faire de l'argent rapidement : le cul, la contrebande, les combats illégaux.... 

Ils sont aussi tristement célèbres pour leurs méthodes particulièrement brutales : contrarier un vor est une très mauvaise façon de mourir ou d'être gravement mutilé.
\section{Les nomades}
A proprement parler, il ne s'agit pas systématiquement de criminels, mais ils en sont proches.

Il s'agit de groupes nomades justement, qui voyagent de ville en ville, transportant souvent de nombreuses ressources, parfois légales, souvent de contrebande. Ils s'organisent dans leurs clans qui sont au centre de leur existence.

Ces clans sont connus pour leurs grandes capacités en mécanique et en conduite ou pilotage. Et leurs services sont recherchés par ceux qui doivent sortir des grandes villes : les fugitifs, les trafiquants, parfois les corpos (plus rares, les nomades n'aiment notoirement pas les corpos). 

Certains clans sont terrestres, et passent par la mongolie, ou vers le reste de la russie. D'autre alimentent aussi les seigneurs de guerre de Corée. Ils parcourent les forêts de sibérie et les grandes plaines d'asie centrale, exploitent parfois temporairement leurs ressources naturelles ou les anciens champs de bataille. Certains sont de vrais bandits, d'autres veulent simplement garder leur indépendance vis-à-vis des corpos

D'autres clans sont maritimes, notamment en mer du japon. On y trouve pas mal de pirates plus ou moins permanents, et évidemment, des contrebandiers en tout genre. Ils peuvent aller vers les ports du japon ou de Singapour par exemple, ou transférer personnes et marchandises sur la côte chinoise ou coréenne.
\section{Les gangs}
On trouve pas mal de gangs à Vladivostok, dans beaucoup de quartiers. Souvent des jeunes, qui souhaitent rejoindre un groupe, appartenir à quelque chose.

C'est en particulier vrai dans les populations défavorisées, où la gloire locale des gangsters attire plus que les maigres espoirs de sortir de la misère : en particulier, little Pyongyang, Vhi-7 ou encore la flottille ont vus de nombreux groupes se créer et disparaître, et de nouveaux continuent de se former.

Beaucoup de ces gangs sont en fait une bande 'appartenant' à un groupe plus grand, dont ils portent les couleurs et les symboles.

A little PyongYang, les plus durables ont jusqu'ici été les Gardes Rouges, même si les 38th sont à peine plus récents.

A Vhi-7 aucun groupe n'a pu se maintenir longtemps : les alliances changent très vite, et de nouveaux groupes se forment et délogent les plus anciens.

La flotille, plus ancienne que les deux autres quartiers, héberge plusieurs gangs plutôt vieux aussi : les White Ronins, les abysses ou encore la Brigade. 
\section{Maintien de l'ordre}
\subsection{NKSB et gardes-frontières}
%la vraie menace, apparemment omniprésente. Disposent de troupes d'élite comme les spteznaz
Le NKSB cumule les rôles de service de renseignement de la NURSS, de surveillance des frontières, et de police politique.

Les gardes-frontières, sous sa juridiction, n'essaient même pas de bloquer la totalité des frontières, et sont maintenant plus en charge de limiter le désordre en dehors des villes importantes. Dans la région de Vladivostok, cela consiste essentiellement à surveiller le transsibérien, les eaux de la région, ainsi que les principaux axes de communication et d'échanges. Ils sont assez fréquemment en conflit avec les nomades de la région. Officiellement, ils sont sous la juridiction du NKSB.

Le NKSB lui-même s'assure de limiter le chaos politique : les gangs et autres groupes ne sont pas jugés comme problématiques, mais l'opposition politique au système en place, oui. Pour cela, ils se basent sur un important réseau d'informateurs et une surveillance électronique d'un excellent niveau : le cube est l'endroit le plus proche de toute la ville, vu qu'on y entends tous les murmures.

Quand une 'menace à la stabilité' est repérée par le NKSB, ce sont leurs forces spéciales, les spesnatz qui réalisent les actions de terrain : ils sont connus pour leur brutalité et leurs méthodes expéditives. 

Aujourd'hui ils agissent officiellement de manière analogue à une corpo, mais la collusion entre ce service et le gouvernement de l'union, ou même le gouvernement local, rendent toute concurrence potentielle très complexe. En plus de leurs rôles traditionnels, ils fournissent également une expertise aux autres corpo en matière de sécurité.

A Vladivostok, le NKSB est dirigé par le colonel Piotr Buranin, un ancien officier de la Faction Rouge, qui s'est reconverti à la fin de la quatrième guerre corpo, et a fait depuis carrière au sein du renseignement, en se liant au passage aux politiciens locaux. Il a de nombreux 'amis' riches et influents.
\subsection{Sécurité corpo}
%deux corpos principales en lutte ici
La sécurité corpo est gérée par plusieurs groupes et corporations plus ou moins importantes, les deux principales étant Arasaka Russia et la Faction Rouge.

Arasaka Russia est la filliale russe de la fameuse corporation en lambeaux, et a commencé tout récemment à être compétitive. Elle a absorbé l'ancienne corpo mercenaire 'troupe de saint-georges' en 2043, et travaille depuis à redorer son image.

La Faction Rouge a beaucoup travaillé en soutien des troupes de la NURSS et des autres factions communistes des environs, et auraient, selon la légende, leurs origines comme un camouflage facile pour des opérations non-déclarées de la NURSS. Dans tous les cas, ces mercenaires se trouvent dans toute l'asie centrale, et leurs labos, mis en place au début de la quatrième guerre corpo, commencent à proposer de nouveaux matériels à leurs troupes et à ceux qui recherchent ce genre de chose, notamment avec certaines des usines d'armes de la ville leurs appartiennent.
\subsection{Police de Vladivostok}
%force de police/milice de base, corrompue et brutale
La Militsiya est la principale force de maintient de l'ordre en ville, s'occupant des affaires courantes de sécurité et de police.

Ils sont notoirement mal payés et brutaux, et essayent rarement de comprendre la situation devant eux. La vieille blague de l'époque stalinienne vaut toujours : 'ils vont toujours par trois : un qui sait lire, un qui sait écrire, et le dernier pour surveiller ces deux dangereux intellectuels'.

Cela ne veut pas dire qu'ils ne sont pas dangereux : leur matériel est plutôt bon, et la plupart savent bien se battre. Et ils ont tout à fait le droit d'utiliser la force, y compris létale, sans grande supervision.
 hiérarchique.
 
Ils sont aussi notoirement mal payés, et prompts à accepter des dessous-de-table. Cela n'est toutefois pas systématique, non par honnêteté, mais parfois, leurs supérieurs insistent pour obtenir des résultats quantifiables. Dans ces cas-là, il ne fait pas bon leur tomber dessus : ils arrêtent, passent à tabac et tirent sur toute forme de résistance.
\chapter{Politique et gouvernement}
\chapter{Corporations}
%Quatre ou cinq locales + trois ou quatre grandes
\section{Zhirafa Technical manufacturing}
La principale corpo de la ville depuis sa fondation en 2039, car son QG y est installé. Elle embauche des dizaines de milliers de personne directement, et plus encore pour ses divers sous-traitants.

Elle est donc extrêmement présente, ses affaires et contrats rythmant bien souvent l'économie et les informations locales
\section{Faction Rouge}
Un groupe mercenaire basé au départ sur d'anciens cadres de l'armée rouge, et qui a fournit beaucoup d'appui aux troupes de la NURSS dans le nord de la Chine. On les retrouve également en Afrique, dans le pacifique et dans toute l'asie centrale jusqu'au moyen orient.

Ils sont connus pour leur matériel plutôt robuste et leur mentalité agressive. Leur principaux concurrents sont les restes d'Arasaka et le plus gros poisson qu'est Militech. Ils disposent de plusieurs laboratoires et usines d'armements notamment pour fournir à leurs clients toute une panoplie de services dans le domaine de la défense : des patrouilles renforcées à la défense anti-aérienne.

La dirigeante de Faction Rouge est Mariya Olyanova, la fille d'un puissant oligarque moscovite, avec qui elle est en froid notable.
\section{Arashina Entertainment}
Une entreprise japonaise spécialisée dans toute sortes  de divertissement : des danses sensorielles aux vidéos plus classiques, en passant par de nombreuses émissions sur les réseaux (radios, TV, NET). Ils contrôlent également plusieurs casino plutôt riches à destination d'une clientèle corporatiste.

Ils sont arrivés dans la région au moment de l'établissement de la NURSS, et ont su attirer les spectateurs avec de nombreuses émissions devenues classiques, notamment Armiya par leur présentateur vedette Jo Hirashi, qui met en scène des véhicules de combat en action dans les décors sibériens. Parfois en entraînement, parfois les uns contre les autres, parfois enfin en opération contre un adversaire, souvent les nomades ou l'état chinois par exemple.

Ils gèrent aussi la lotterie morbide, où chaque semaine, les habitants peuvent tenter de deviner le nombre de morts dans la colonie pénale Kha-110. Si aucun vainqueur n'est là une semaine, les gains sont doublés pour la suivante !

Le directeur de la branche régionale est Ichiro Abe, un exécutif plutôt classique, mais notoirement discret.
\section{SovOil}
L'ancien mastodonte régional, lourdement atteint par trois crises successives : la seconde guerre corpo qu'elle a remporté, la montée en puissance du CHOOH2 en tant que carburant et l'effondrement du commerce international.

Malgré cela, elle reste la principale puissance économique en NURSS, et dispose des réserves pour encaisser un tel choc. Ainsi ces vingt dernières années ont vu une grande diversification de ses activités : minage de diverses ressources, agriculture pour la production de CHOOH2, mais aussi système énergétique, chantier naval, ou encore agriculture classique.

Leur commité central n'est pas connu du grand public, et est basé à Moscou.
\section{N54}
Une firme médiatique américaine récemment arrivée à Vladivostok, et aux méthodes de marketing agressives. Leurs publicités notamment sont volontairement très provocatrice, pour déclencher des réactions.

Leur spécialité toutefois reste l'information, de préférence croustillante sur des gens 'importants'. Cela peut autant les opposer que les faire travailler avec le NKSB suivant les affaires.

La directrice de leur branche locale est Katelyn Sarayeva, une exec corpo venue des US, et qui a épousé un des acteurs vedette de son entreprise il y a deux mois.
\section{Kolkoz}
Une autre corpo d'état de la NURSS, celle-ci orientée en biotechnologie et agroalimentaire.

La majeure partie de ce qui se mange ou se boit à Vladivostok vient d'une usine Kolokoz, et les amtières premières viennent de leurs diverses fermes, généralement bien protégées.

Leur concurrence principale vient de la firme américaine 'All Food', dont les produits sont recherchés pour leur nouveauté à Vladivostok, et qui se trouvent au marché noir.
\section{Trauma team}
Trauma team est une entreprise américaine dont les activités se sont multipliées depuis la guerre.

les clients de l'entreprise paient une assurance santé, à différent niveaux de couverture. En cas de problème de santé, les équipes d'interventions de Trauma team arrivent rapidement pour sécuriser leur client et lui prodiguer des soins de qualité dans leurs cliniques.

Leur passé militaire et leur implication dans la guerre a façonné leur culture et leurs méthodes :les équipes d'interventions sont lourdement protégées et armées, et n'hésitent pas à engager de violentes fusillades pour récupérer leurs clients premium.
\section{Barrikady}
Barrikady est un fabricant soviétique de microtechnologie, et notamment d'implants cybernétiques. En comparaison des implants occidentaux, ils sont souvent considérés comme de mauvaise qualité et grossiers, mais ils sont souvent aussi très accessibles, étant facilement produits en masse.

A part cela, ils réalisent également des armes, des véhicules, et disposent d'une petite branche aéronautique.

En ville leur présence est surtout due à leurs implants et cliniques, peu chères mais avec des soins de qualité parfois limité.

\chapter{Lieux notables}
\section{Le Pradva}
%un club mythique du centre ville
\section{Matrioshka}
%un club assez exlusif proposant entre autre les services de poupées. Plutôt classe, et surtout cher.
\section{La rue de la Moskowa}
%La rue chaude de la ville, avec de nombreux bordels etc
\section{Le marché de Tonbak}
%un marché permanent où on trouve de tout, les flics regardant souvent ailleurs
\end{document}